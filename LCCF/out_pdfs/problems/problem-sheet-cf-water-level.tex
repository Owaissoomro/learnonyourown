% !TeX program = xelatex
\documentclass{article}
\usepackage[margin=1in]{geometry}
\usepackage{microtype}
\usepackage{setspace}\setstretch{1.05}
\usepackage{fontspec}
\setmainfont{Latin Modern Roman}
\setmonofont{Latin Modern Mono}[BoldFont={Latin Modern Mono},ItalicFont={Latin Modern Mono}]
\usepackage{amsmath,mathtools,amsthm}
\newtheorem{lemma}{Lemma}
\newtheorem{theorem}{Theorem}
\usepackage{unicode-math}\setmathfont{Latin Modern Math}\allowdisplaybreaks[4]
% --- overflow and alignment slack ---
\setlength{\jot}{7pt}
\sloppy\emergencystretch=8em\hfuzz=1pt\vfuzz=2pt\raggedbottom
% --- breakable math helpers ---
\newenvironment{BreakableEquation}{\begin{equation}\begin{aligned}}{\end{aligned}\end{equation}}
\newenvironment{BreakableEquation*}{\begin{equation*}\begin{aligned}}{\end{aligned}\end{equation*}}
\newenvironment{tightalign}{\begingroup\small\allowdisplaybreaks\begin{align}}{\end{align}\endgroup}

\usepackage{xcolor}
\usepackage{xurl} % better URL wrapping
\usepackage[colorlinks=true,linkcolor=black,urlcolor=blue]{hyperref}
\usepackage{fancyhdr}
\pagestyle{fancy}\fancyhf{}\lhead{\nouppercase{\leftmark}}\rhead{\thepage}\setlength{\headheight}{26pt}
\usepackage{enumitem,booktabs,tabularx}
\newcolumntype{L}{>{\raggedright\arraybackslash}X}
\renewcommand{\arraystretch}{1.1}
\setlength{\parindent}{0pt}\setlength{\parskip}{8pt plus 2pt minus 1pt}

% Listings + upquote (no shell-escape needed)
\usepackage{listings}
\usepackage{upquote}
\lstdefinestyle{crisp}{
  basicstyle=\ttfamily\footnotesize,
  frame=single,
  breaklines=true,
  breakatwhitespace=false, % allow breaks inside long tokens
  tabsize=4,
  showstringspaces=false,
  numbers=left,
  numberstyle=\tiny,
  numbersep=8pt,
  keepspaces=true,
  columns=fullflexible,
  backgroundcolor=\color{black!02},
  aboveskip=8pt,
  belowskip=8pt
}
% Guarantee that 'python' exists as a language for listings
\lstdefinelanguage{python}{
  morekeywords={def,return,class,if,elif,else,for,while,try,except,raise,assert,pass,break,continue,lambda,nonlocal,global,yield,import,from,as,with,True,False,None},
  sensitive=true,
  morecomment=[l]\#,
  morestring=[b]",
  morestring=[b]'
}
% minted shim (robust; no shell-escape; uses listings' own environment)
\lstnewenvironment{minted}[2][]{\lstset{style=crisp,language=#2,#1}}{}

\usepackage[most]{tcolorbox}
\tcbset{colback=white,colframe=black!15,boxrule=0.4pt,arc=2pt,left=6pt,right=6pt,top=6pt,bottom=6pt,before skip=10pt,after skip=10pt,breakable}
\usepackage{titlesec}
\titleformat{\section}{\large\bfseries}{\thesection}{0.6em}{}
\titlespacing*{\section}{0pt}{*2.0}{*1.0}

% ===== CONSISTENCY TOOLKIT (macros) =====
\newlist{ledger}{enumerate}{1}
\setlist[ledger]{label=\textbullet,leftmargin=2em,itemsep=2pt,topsep=6pt}
\newcommand{\LEDGER}[1]{\textbf{ARITHMETIC LEDGER:}\par\begin{ledger}#1\end{ledger}}

\newlist{algosteps}{enumerate}{1}
\setlist[algosteps]{label=\arabic*.,leftmargin=2em,itemsep=2pt,topsep=6pt}

\newlist{bullets}{itemize}{1}
\setlist[bullets]{label=--,leftmargin=1.6em,itemsep=2pt,topsep=4pt}

\newcommand{\varmapStart}{\textbf{VARIABLE MAPPING:}\par\begin{bullets}}
\newcommand{\var}[2]{\item $#1$ — #2}
\newcommand{\varmapEnd}{\end{bullets}}

\newcommand{\LINE}[2]{\par\noindent\textbf{#1:}~#2\par}
\newcommand{\WHAT}[1]{\LINE{WHAT}{#1}}
\newcommand{\WHY}[1]{\LINE{WHY}{#1}}
\newcommand{\HOW}[1]{\LINE{HOW}{#1}}
\newcommand{\ELI}[1]{\LINE{ELI5}{#1}}
\newcommand{\STATEMENT}[1]{\LINE{STATEMENT}{#1}}
\newcommand{\BREAKDOWN}[1]{\LINE{PROBLEM BREAKDOWN}{#1}}
\newcommand{\MODEL}[1]{\LINE{CANONICAL MATHEMATICAL MODEL}{#1}}
\newcommand{\ASSUMPTIONS}[1]{\LINE{ASSUMPTIONS}{#1}}
\newcommand{\INVARIANTS}[1]{\LINE{INVARIANTS}{#1}}
\newcommand{\WHICHFORMULA}[1]{\LINE{WHICH FORMULA \& WHY}{#1}}
\newcommand{\GOVERN}[1]{\LINE{GOVERNING EQUATION(S)}{#1}}
\newcommand{\INPUTS}[1]{\LINE{INPUTS}{#1}}
\newcommand{\OUTPUTS}[1]{\LINE{OUTPUTS}{#1}}
\newcommand{\SAMPLES}[1]{\LINE{SAMPLES}{#1}}
\newcommand{\RESULT}[1]{\LINE{RESULT}{#1}}
\newcommand{\COMPLEXITY}[1]{\LINE{TIME/SPACE COMPLEXITY}{#1}}
\newcommand{\MEMORY}[1]{\LINE{MEMORY FOOTPRINT}{#1}}
\newcommand{\CORRECTNESS}[1]{\LINE{CORRECTNESS SKETCH}{#1}}
\newcommand{\OPTIMALITY}[1]{\LINE{OPTIMALITY}{#1}}
\newcommand{\FAILMODES}[1]{\LINE{FAILURE MODES}{#1}}
\newcommand{\VALIDATION}[1]{\LINE{VALIDATION}{#1}}
\newcommand{\UNITCHECK}[1]{\LINE{UNIT CHECK}{#1}}
\newcommand{\EDGECASES}[1]{\LINE{EDGE CASES}{#1}}
\newcommand{\CHECKLIST}[1]{\LINE{CHECKLIST}{#1}}
\newcommand{\DERIV}[1]{\LINE{DERIVATION}{#1}}
\newcommand{\LEMMAHEAD}[1]{\LINE{SUPPORTING LEMMA}{#1}}
\newcommand{\PITFALLS}[1]{\LINE{PITFALLS}{#1}}

\usepackage{etoolbox}\pretocmd{\section}{\clearpage}{}{}

\newcommand{\FormulaPage}[2]{%
  \clearpage
  \section*{Formula #1 — #2}%
  \addcontentsline{toc}{section}{Formula #1 — #2}%
}
\newcommand{\ApproachPage}[2]{%
  \clearpage
  \subsection*{Approach #1 — #2}%
  \addcontentsline{toc}{subsection}{Approach #1 — #2}%
}

\begin{document}
\title{Interview Problem Sheet — Water Level}
\date{\today}
\author{}
\maketitle
\tableofcontents
\clearpage

% === Notes macro (BODY-ONLY; do not alter the preamble) ===
% Prints exactly N blank pages with empty headers/footers, then leaves you on the last blank page.
\newcommand{\NotePages}[1]{%
  \clearpage
  \begingroup
  \newcount\NPi \newcount\NPn
  \NPi=0 \NPn=#1
  \loop\ifnum\NPi<\NPn
    \advance\NPi by 1
    \mbox{}\thispagestyle{empty}%
    \ifnum\NPi<\NPn\clearpage\fi
  \repeat
  \endgroup
}

% ============ 1. Problem & Metadata (own page) ============
\section{Problem \& Metadata}
\LINE{PLATFORM}{CF}
\LINE{URL}{\url{https://codeforces.com/problemset/problem/1461/E}}
\LINE{DIFFICULTY / RATING}{2200}
\STATEMENT{In recent years John has very successfully settled at his new job at the office. But John does not like to idly sit around while his code is compiling, so he immediately found himself an interesting distraction. The point of his distraction was to maintain a water level in the water cooler used by other zebras.

Originally the cooler contained exactly $k$ liters of water. John decided that the amount of water must always be at least $l$ liters of water but no more than $r$ liters. John will stay at the office for exactly $t$ days. He knows that each day exactly $x$ liters of water will be used by his colleagues. At the beginning of each day he can add exactly $y$ liters of water to the cooler, but at any point in time the amount of water in the cooler must be in the range $[l, r]$.

Now John wants to find out whether he will be able to maintain the water level at the necessary level for $t$ days. Help him answer this question!

Input:
The first line of the input contains six integers $k$, $l$, $r$, $t$, $x$ and $y$ ($1 \le l \le k \le r \le 10^{18}; 1 \le t \le 10^{18}; 1 \le x \le 10^6; 1 \le y \le 10^{18}$) — initial water level, the required range, the number of days, daily water usage and the exact amount of water that can be added, respectively.

Output:
Print ``Yes'' if John can maintain the water level for $t$ days and ``No'' otherwise.

Note:
In the first example, John can not increase the amount of water at the beginning of the first day, since it would exceed the limit $r$. That is why after the first day the cooler will contain $2$ liters of water. The next day John adds $4$ liters to the cooler but loses $6$ liters, leaving John with $0$ liters, which is outside the range $[1, 10]$.

In the second example, after the first day John is left with $2$ liters of water. At the beginning of the next day he adds $5$ liters, then $6$ liters get used, leaving John with $1$ liter of water which is in range $[1, 10]$.

In the third example, after the first day John is left with $7$ liters, after the second day — $5$ liters, after the fourth — $1$ liter. At the beginning of the fifth day John will add $9$ liters and lose $2$ liters. Meaning, after the fifth day he will have $8$ liters left. Then each day the water level will decrease by $2$ liters and after the eighth day John will have $2$ liters and after the ninth day — $0$ liters. $0$ is outside range $[1, 10]$, so the answer is ``No''.

In the fourth example, after the first day John is left with $15$ liters of water. At the beginning of the second day he adds $7$ liters and loses $5$, so after the second day he is left with $17$ liters. At the beginning of the third day he adds $7$ more liters of water and loses $5$, so after the third day he is left with $19$ liters. $19$ is in range $[15, 25]$ so the answer is ``Yes''.}
\BREAKDOWN{Normalize by shifting the interval so the lower bound is $0$. Consider two regimes: $y \le x$ (nonincreasing process) and $y > x$ (potential growth, handled via residues modulo $x$ and batching subtractions). Detect cycles or compute a closed-form bound on days.}
\ELI{Shift to $[0, R]$ and either count how many days you can afford to lose water or show you get stuck in a safe cycle that lasts all $t$ days.}
\NotePages{3}

% ============ 2. IO Contract (own page) ============
\section{IO Contract}
\INPUTS{Six integers: $k, l, r, t, x, y$ with $1 \le l \le k \le r \le 10^{18}$, $1 \le t \le 10^{18}$, $1 \le x \le 10^6$, $1 \le y \le 10^{18}$.}
\OUTPUTS{A single line: Yes if it is possible to maintain the water level within $[l, r]$ for $t$ days, otherwise No.}
\SAMPLES{Example 1: $k=5$, $l=1$, $r=10$, $t=3$, $x=2$, $y=3$ $\Rightarrow$ Yes.

Example 2: $k=5$, $l=1$, $r=7$, $t=2$, $x=4$, $y=3$ $\Rightarrow$ No.}
\NotePages{3}

% ============ 3. Canonical Mathematical Model (own page) ============
\section{Canonical Mathematical Model}
\MODEL{Let $a_0 = k - l$ and $R = r - l$. Maintain $a_i \in [0, R]$. Each day $i$:
- If $a_i + y \le R$, optionally set $a_i \gets a_i + y$.
- Then set $a_{i+1} = a_i - x$ and require $a_{i+1} \ge 0$.
Question: is there a sequence of choices for $t$ days so that all states remain in $[0, R]$?}
\varmapStart
\var{a_i}{Water level above $l$ at the start of day $i$}
\var{R}{Upper slack $r-l$}
\var{x}{Daily consumption}
\var{y}{Optional daily refill}
\var{t}{Number of days to survive}
\varmapEnd
\GOVERN{
\[
\begin{aligned}
&0 \le a_i \le R \quad \forall i,\\
&a_{i+1} = a_i - x + u_i y,\quad u_i \in \{0,1\},\\
&u_i = 1 \implies a_i + y \le R,\\
&a_{i+1} \ge 0,\quad i=0,1,\ldots,t-1.
\end{aligned}
\]
}
\ASSUMPTIONS{We can reorder by subtracting $l$ once; decisions to add are made at the beginning of each day; consumption occurs after the optional addition.}
\INVARIANTS{If $y \le x$ and $a_i + y \le R$, then $a_{i+1} \le a_i$ (nonincreasing). Residue modulo $x$ is preserved by pure consumption and shifted by $y \bmod x$ by an add-then-consume step.}
\NotePages{3}

% ============ 4. Approach A — Baseline (own page) ============
\section{Approach A — Baseline}
\ApproachPage{A}{Brute Force / Direct}
\WHICHFORMULA{Direct day-by-day simulation with the greedy rule: add when legal and useful; otherwise consume.}
\ASSUMPTIONS{Only feasible for very small $t$; used for sanity checks.}
\textbf{Algorithm Steps}
\begin{algosteps}
\item Shift $a \gets k-l$, $R \gets r-l$.
\item For each day up to $t$:
  \begin{itemize}
  \item If $a+y \le R$, set $a \gets a+y$.
  \item If $a < x$, return No; else $a \gets a-x$.
  \end{itemize}
\item If loop completes, return Yes.
\end{algosteps}
\COMPLEXITY{Time $T(n)$ is $O(t)$, Space $S(n)$ is $O(1)$. Unsuitable for $t$ up to $10^{18}$.}
\[
\begin{aligned}
T(t) &= O(t),\quad S(t)=O(1).
\end{aligned}
\]
\CORRECTNESS{Implements the process exactly. If it finishes $t$ iterations without violating bounds, the answer is Yes.}
\EDGECASES{When $a+y>R$ and $a<x$, failure is immediate; when $x=0$ the answer is trivially Yes.}
\textbf{Code (Baseline)}
\begin{minted}{python}
import sys

def read_input():
    data = sys.stdin.read().strip().split()
    if not data:
        return None
    it = iter(map(int, data))
    k = next(it); l = next(it); r = next(it)
    t = next(it); x = next(it); y = next(it)
    return (k, l, r, t, x, y)

def can_baseline(k, l, r, t, x, y):
    # Direct O(t) simulation; only for small t
    a = k - l
    R = r - l
    for _ in range(min(t, 10**6 + 5)):  # hard cap to avoid huge loops in accidental runs
        if a + y <= R:
            a += y
        if a < x:
            return False
        a -= x
        t -= 1
        if t == 0:
            return True
    # If t still large, we give up (baseline cannot finish). Fall back to safe solver here for completeness.
    return can_optimal(k, l, r, t, x, y)

def solve_case_baseline(params):
    k, l, r, t, x, y = params
    return "Yes" if can_baseline(k, l, r, t, x, y) else "No"

# -------- Improved/Final routine used as fallback to keep baseline runnable --------
def can_optimal(k, l, r, t, x, y):
    a = k - l
    R = r - l
    if a < 0 or a > R:
        return False
    # Case 1: y <= x
    if y <= x:
        # Push down until we can add (if needed)
        if a + y > R:
            # Consume days to reach a <= R - y
            if x == 0:
                return False  # cannot consume but cannot add either
            d = min(t, (a - (R - y) + x - 1) // x)
            a -= d * x
            t -= d
            if t == 0:
                return a >= 0
            if a < 0:
                return False
        # Now a <= R - y, so we can add each day.
        if x == y:
            return True
        delta = x - y
        # We can perform at most floor(a / delta) days from this state.
        return t <= (a // delta)
    # Case 2: y > x
    seen = [False] * (x if x > 0 else 1)
    while t > 0:
        if a + y <= R:
            rmod = a % x if x > 0 else 0
            if seen[rmod]:
                return True
            seen[rmod] = True
            a += y
        # Cannot add now
        if x == 0:
            return False  # cannot reduce and cannot add -> stuck if cannot add; otherwise would have added above
        if a < x:
            return False
        # Consume in batch; at least 1 day, but do not overshoot below zero.
        need = (a - (R - y)) // x if a > (R - y) else 0
        d = max(1, need)
        d = min(d, t, a // x)
        a -= d * x
        t -= d
    return True

def main():
    params = read_input()
    if params is None:
        # Self-tests
        assert solve_case_baseline((5, 1, 10, 3, 2, 3)) == "Yes"
        assert solve_case_baseline((5, 1, 7, 2, 4, 3)) == "No"
        assert solve_case_baseline((5, 1, 10, 100, 3, 3)) == "Yes"
        assert solve_case_baseline((3, 1, 3, 1, 3, 3)) == "No"
        print("OK")
    else:
        print(solve_case_baseline(params))

if __name__ == "__main__":
    main()
\end{minted}
\VALIDATION{Included 4 asserts covering $y>x$, $y<x$, $x=y$, and immediate failure when cannot add and $a<x$.}
\NotePages{3}

% ============ 5. Approach B — Improved (own page) ============
\section{Approach B — Improved}
\ApproachPage{B}{Optimized Data Structure / Pruning / DP}
\WHICHFORMULA{Split into two regimes: when $y \le x$ monotonicity bounds the total days by a simple division; when $y > x$ use residues modulo $x$ and batch consumption to skip long stretches.}
\ASSUMPTIONS{Residue repetition implies a safe cycle; batching by whole multiples of $x$ preserves feasibility and reduces $t$ quickly.}
\textbf{Algorithm Steps}
\begin{algosteps}
\item Normalize $a=k-l$, $R=r-l$.
\item If $y \le x$:
  \begin{itemize}
  \item If needed, consume $\left\lceil \dfrac{a-(R-y)}{x} \right\rceil$ days to reach $a \le R-y$; fail early if $a$ drops below $0$ before $t$ is exhausted.
  \item If $x=y$, success; else require $t \le \left\lfloor \dfrac{a}{x-y} \right\rfloor$.
  \end{itemize}
\item Else ($y>x$):
  \begin{itemize}
  \item While $t>0$: if $a+y \le R$, mark residue $a \bmod x$; if seen, success; then set $a \gets a+y$.
  \item Otherwise batch-consume $d=\min\{t, \max(1, \lfloor \dfrac{a-(R-y)}{x}\rfloor), \lfloor \dfrac{a}{x}\rfloor\}$ days; update $a \gets a - dx$, $t \gets t-d$; if $a<x$, fail.
  \end{itemize}
\end{algosteps}
\COMPLEXITY{Linear in the number of distinct residues, at most $x \le 10^6$; each iteration reduces $t$ by at least $1$. This beats any naive per-day loop for large $t$.}
\[
\begin{aligned}
T &\le O(x),\quad S=O(x).
\end{aligned}
\]
\CORRECTNESS{For $y \le x$, the sequence never increases once adding is allowed; thus the worst case is always-add, giving a tight bound. For $y > x$, residue repetition guarantees a loop that never violates feasibility; batching preserves legality and reduces $t$ monotonically.}
\textbf{Code (Improved)}
\begin{minted}{python}
import sys

def read_input():
    data = sys.stdin.read().strip().split()
    if not data:
        return None
    it = iter(map(int, data))
    k = next(it); l = next(it); r = next(it)
    t = next(it); x = next(it); y = next(it)
    return (k, l, r, t, x, y)

def can_improved(k, l, r, t, x, y):
    a = k - l
    R = r - l
    if a < 0 or a > R:
        return False
    if y <= x:
        if a + y > R:
            if x == 0:
                return False
            d = min(t, (a - (R - y) + x - 1) // x)
            a -= d * x
            t -= d
            if t == 0:
                return a >= 0
            if a < 0:
                return False
        if x == y:
            return True
        delta = x - y
        return t <= (a // delta)
    seen = [False] * (x if x > 0 else 1)
    while t > 0:
        if a + y <= R:
            rmod = a % x if x > 0 else 0
            if seen[rmod]:
                return True
            seen[rmod] = True
            a += y
        if x == 0:
            return False
        if a < x:
            return False
        need = (a - (R - y)) // x if a > (R - y) else 0
        d = max(1, need)
        d = min(d, t, a // x)
        a -= d * x
        t -= d
    return True

def solve_case_improved(params):
    k, l, r, t, x, y = params
    return "Yes" if can_improved(k, l, r, t, x, y) else "No"

def main():
    params = read_input()
    if params is None:
        # Edge checks
        assert solve_case_improved((5, 1, 10, 3, 2, 3)) == "Yes"
        assert solve_case_improved((5, 1, 7, 2, 4, 3)) == "No"
        assert solve_case_improved((5, 1, 10, 100, 3, 3)) == "Yes"
        print("OK")
    else:
        print(solve_case_improved(params))

if __name__ == "__main__":
    main()
\end{minted}
\VALIDATION{Covers both regimes and a neutral case $x=y$.}
\NotePages{3}

% ============ 6. Approach C — Optimal (own page) ============
\section{Approach C — Optimal}
\ApproachPage{C}{Provably Optimal Method}
\WHICHFORMULA{Normalize to $[0,R]$; handle $y \le x$ by monotonic decline bound; handle $y > x$ by residue-cycling with batching.}
\ASSUMPTIONS{All arithmetic in $64$-bit integers; $x \le 10^6$ allows an array of residues.}
\textbf{Algorithm Steps}
\begin{algosteps}
\item Set $a=k-l$, $R=r-l$.
\item If $y \le x$:
  \begin{itemize}
  \item Reduce $a$ by $x$ in $\min\!\Bigl(t, \left\lceil\dfrac{a-(R-y)}{x}\right\rceil\Bigr)$ steps until $a \le R-y$ or $t=0$.
  \item If $t=0$, done; else if $x=y$, Yes; else require $t \le \left\lfloor\dfrac{a}{x-y}\right\rfloor$.
  \end{itemize}
\item Else ($y>x$):
  \begin{itemize}
  \item While $t>0$: if $a+y \le R$, check and mark residue $a \bmod x$, set $a \gets a+y$; else if $a<x$, No; otherwise batch $d=\min\{t,\max(1,\lfloor\dfrac{a-(R-y)}{x}\rfloor),\lfloor\dfrac{a}{x}\rfloor\}$, update $a \gets a-dx$, $t \gets t-d$.
  \item If residue repeats, Yes.
  \end{itemize}
\end{algosteps}
\OPTIMALITY{In $y \le x$, the worst case uses addition whenever allowed, giving a tight bound by division; no other strategy prolongs life. In $y>x$, the process can be collapsed by invariants modulo $x$, and any repetition yields a safe infinite loop with respect to remaining $t$.}
\COMPLEXITY{At most $x$ residue marks, and each batch reduces $t$ by at least $1$.}
\[
\begin{aligned}
T &= O(x) \le 10^6,\quad S = O(x).
\end{aligned}
\]
\textbf{Code (Final Submission)}
\begin{minted}{python}
import sys

def read_input():
    data = sys.stdin.read().strip().split()
    if not data:
        return None
    it = iter(map(int, data))
    k = next(it); l = next(it); r = next(it)
    t = next(it); x = next(it); y = next(it)
    return (k, l, r, t, x, y)

def can(k, l, r, t, x, y):
    a = k - l
    R = r - l
    if a < 0 or a > R:
        return False
    # Regime 1: y <= x (nonincreasing once adding is possible)
    if y <= x:
        if a + y > R:
            if x == 0:
                return False
            d = min(t, (a - (R - y) + x - 1) // x)
            a -= d * x
            t -= d
            if t == 0:
                return a >= 0
            if a < 0:
                return False
        if x == y:
            return True
        delta = x - y
        return t <= (a // delta)
    # Regime 2: y > x (use residues and batching)
    seen = [False] * (x if x > 0 else 1)
    while t > 0:
        if a + y <= R:
            rmod = a % x if x > 0 else 0
            if seen[rmod]:
                return True
            seen[rmod] = True
            a += y
        if x == 0:
            # Cannot reduce; if we could add we would have done so above
            return False
        if a < x:
            return False
        need = (a - (R - y)) // x if a > (R - y) else 0
        d = max(1, need)
        d = min(d, t, a // x)
        a -= d * x
        t -= d
    return True

def solve_case(params):
    k, l, r, t, x, y = params
    return "Yes" if can(k, l, r, t, x, y) else "No"

def main():
    params = read_input()
    if params is None:
        # Exactly 3 asserts / mini-tests
        assert solve_case((5, 1, 10, 3, 2, 3)) == "Yes"
        assert solve_case((5, 1, 7, 2, 4, 3)) == "No"
        assert solve_case((8, 2, 12, 6, 3, 10)) == "No"
        print("OK")
    else:
        print(solve_case(params))

if __name__ == "__main__":
    main()
\end{minted}
\VALIDATION{Three asserts: a simple Yes with $y>x$, a No with $y<x$, and a No where a longer run fails.}
\RESULT{Print Yes if and only if the process can be maintained within $[l,r]$ for exactly $t$ days; otherwise No.}
\NotePages{3}

% ============ 7. Testing & Final Reference Implementation (own page) ============
\section{Testing \& Final Reference Implementation}
\LINE{TEST PLAN}{Unit tests cover both regimes ($y \le x$ and $y > x$), neutrality $x=y$, and immediate failure when cannot add and $a<x$. Property checks: normalization ($l$-shift) does not change feasibility.}
\LINE{CROSS-CHECKS}{Compare outputs of Baseline vs Improved vs Final on small random cases ($t$ small) to ensure consistency.}
\LINE{EDGE-CASE GENERATOR}{Generate boundary cases: $a=0$, $a=R$, $a=R-y+1$, $x=1$, $x=10^6$, $y=1$, large $y$, $x=y$, $t=1$, very large $t$.}
\begin{minted}{python}
import random

def gen_cases(n=200, seed=0):
    random.seed(seed)
    cases = []
    for _ in range(n):
        l = random.randint(1, 50)
        r = l + random.randint(0, 50)
        k = random.randint(l, r)
        x = random.randint(1, 50)
        y = random.randint(1, 200)
        t = random.randint(1, 200)  # keep small for baseline
        cases.append((k, l, r, t, x, y))
    return cases

# Reference final solver
def can_ref(k, l, r, t, x, y):
    a = k - l
    R = r - l
    if a < 0 or a > R:
        return False
    if y <= x:
        if a + y > R:
            if x == 0:
                return False
            d = min(t, (a - (R - y) + x - 1) // x)
            a -= d * x
            t -= d
            if t == 0:
                return a >= 0
            if a < 0:
                return False
        if x == y:
            return True
        delta = x - y
        return t <= (a // delta)
    seen = [False] * (x if x > 0 else 1)
    while t > 0:
        if a + y <= R:
            rmod = a % x if x > 0 else 0
            if seen[rmod]:
                return True
            seen[rmod] = True
            a += y
        if x == 0:
            return False
        if a < x:
            return False
        need = (a - (R - y)) // x if a > (R - y) else 0
        d = max(1, need)
        d = min(d, t, a // x)
        a -= d * x
        t -= d
    return True

def can_baseline_small(k, l, r, t, x, y):
    a = k - l
    R = r - l
    for _ in range(t):
        if a + y <= R:
            a += y
        if a < x:
            return False
        a -= x
    return True

def cross_check():
    cases = gen_cases()
    for case in cases:
        k,l,r,t,x,y = case
        # Baseline only reliable for small t; already small here
        b = can_baseline_small(k,l,r,t,x,y)
        c = can_ref(k,l,r,t,x,y)
        assert b == c, f"Mismatch on {case}: {b} vs {c}"
    print("Cross-check passed on random small cases.")

if __name__ == "__main__":
    cross_check()
\end{minted}
\textbf{Reference Code (Ready to Submit)}
\begin{minted}{python}
import sys

def read_input():
    data = sys.stdin.read().strip().split()
    if not data:
        return None
    it = iter(map(int, data))
    k = next(it); l = next(it); r = next(it)
    t = next(it); x = next(it); y = next(it)
    return (k, l, r, t, x, y)

def can(k, l, r, t, x, y):
    a = k - l
    R = r - l
    if a < 0 or a > R:
        return False
    if y <= x:
        if a + y > R:
            if x == 0:
                return False
            d = min(t, (a - (R - y) + x - 1) // x)
            a -= d * x
            t -= d
            if t == 0:
                return a >= 0
            if a < 0:
                return False
        if x == y:
            return True
        delta = x - y
        return t <= (a // delta)
    seen = [False] * (x if x > 0 else 1)
    while t > 0:
        if a + y <= R:
            rmod = a % x if x > 0 else 0
            if seen[rmod]:
                return True
            seen[rmod] = True
            a += y
        if x == 0:
            return False
        if a < x:
            return False
        need = (a - (R - y)) // x if a > (R - y) else 0
        d = max(1, need)
        d = min(d, t, a // x)
        a -= d * x
        t -= d
    return True

def solve_case(params):
    k, l, r, t, x, y = params
    return "Yes" if can(k, l, r, t, x, y) else "No"

def main():
    params = read_input()
    if params is None:
        # Quick self-checks
        assert solve_case((5, 1, 10, 3, 2, 3)) == "Yes"
        assert solve_case((5, 1, 7, 2, 4, 3)) == "No"
        assert solve_case((5, 1, 10, 100, 3, 3)) == "Yes"
        print("OK")
    else:
        print(solve_case(params))

if __name__ == "__main__":
    main()
\end{minted}
\NotePages{3}

% ============ 8. Review & Pitfalls (own page) ============
\section{Review \& Pitfalls}
\WHAT{Normalize to $[0,R]$; if $y \le x$ use closed-form days; if $y > x$ use residues with batched consumption.}
\WHY{Tests ability to reason about infinite processes, modular invariants, and turning huge $t$ into bounded loops.}
\CHECKLIST{
\begin{bullets}
\item Shift by $l$.
\item Compare $y$ and $x$.
\item If $y \le x$, push down to $R-y$ then divide by $x-y$ (or infinite if $x=y$).
\item If $y > x$, use seen residues modulo $x$ and batch subtract.
\item Guard $a<x$ when cannot add.
\item Always reduce $t$ in batches.
\end{bullets}
}
\EDGECASES{
\begin{bullets}
\item $a=0$.
\item $a=R$ and $a+y>R$.
\item $x=y$ (steady state).
\item $y=1$, $x=10^6$.
\item $t=1$.
\item $a<x$ and cannot add (immediate No).
\item $R-y<0$ (always can add if $y \le R$ after shift).
\item $x=0$.
\end{bullets}
}
\PITFALLS{
\begin{bullets}
\item Forgetting to shift by $l$.
\item Using ceil batching that overshoots below zero.
\item Not capping batch length by $a // x$.
\item Missing residue-cycle early exit.
\item Off-by-one in $\lfloor a/(x-y)\rfloor$ days count.
\item Overflow if implemented in languages without $128$-bit care.
\end{bullets}
}
\FAILMODES{Naive per-day simulation times out for $t \approx 10^{18}$. Greedy without residues for $y>x$ can loop incorrectly or miss cycles.}
\ELI{Think of it as a tank with a ceiling and floor. If refills are small, the level inevitably drops; just count how many days until empty. If refills are big, the level bounces but repeats patterns modulo the daily usage; detecting a repeat means you can keep going long enough.}
\NotePages{3}

\end{document}