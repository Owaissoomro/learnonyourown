% !TeX program = xelatex
\documentclass[11pt]{article}
\usepackage[margin=1in]{geometry}
\usepackage{microtype}
\usepackage{setspace}\setstretch{1.05}
\usepackage{fontspec}
\setmainfont{Latin Modern Roman}
\setmonofont{Latin Modern Mono}[BoldFont={Latin Modern Mono},ItalicFont={Latin Modern Mono}]
\usepackage{amsmath,mathtools,amsthm}
\newtheorem{lemma}{Lemma}
\newtheorem{theorem}{Theorem}
\usepackage{unicode-math}\setmathfont{Latin Modern Math}\allowdisplaybreaks[4]
% --- overflow and alignment slack ---
\setlength{\jot}{7pt}
\sloppy\emergencystretch=8em\hfuzz=1pt\vfuzz=2pt\raggedbottom
% --- breakable math helpers ---
\newenvironment{BreakableEquation}{\begin{equation}\begin{aligned}}{\end{aligned}\end{equation}}
\newenvironment{BreakableEquation*}{\begin{equation*}\begin{aligned}}{\end{aligned}\end{equation*}}
\newenvironment{tightalign}{\begingroup\small\allowdisplaybreaks\begin{align}}{\end{align}\endgroup}

\usepackage{xcolor}
\usepackage{xurl} % better URL wrapping
\usepackage[colorlinks=true,linkcolor=black,urlcolor=blue]{hyperref}
\usepackage{fancyhdr}
\pagestyle{fancy}\fancyhf{}\lhead{\nouppercase{\leftmark}}\rhead{\thepage}\setlength{\headheight}{26pt}
\usepackage{enumitem,booktabs,tabularx}
\newcolumntype{L}{>{\raggedright\arraybackslash}X}
\renewcommand{\arraystretch}{1.1}
\setlength{\parindent}{0pt}\setlength{\parskip}{8pt plus 2pt minus 1pt}

% Listings + upquote (no shell-escape needed)
\usepackage{listings}
\usepackage{upquote}
\lstdefinestyle{crisp}{
  basicstyle=\ttfamily\footnotesize,
  frame=single,
  breaklines=true,
  breakatwhitespace=false, % allow breaks inside long tokens
  tabsize=4,
  showstringspaces=false,
  numbers=left,
  numberstyle=\tiny,
  numbersep=8pt,
  keepspaces=true,
  columns=fullflexible,
  backgroundcolor=\color{black!02},
  aboveskip=8pt,
  belowskip=8pt
}
% Guarantee that 'python' exists as a language for listings
\lstdefinelanguage{python}{
  morekeywords={def,return,class,if,elif,else,for,while,try,except,raise,assert,pass,break,continue,lambda,nonlocal,global,yield,import,from,as,with,True,False,None},
  sensitive=true,
  morecomment=[l]\#,
  morestring=[b]",
  morestring=[b]'
}
% minted shim (robust; no shell-escape; uses listings' own environment)
\lstnewenvironment{minted}[2][]{\lstset{style=crisp,language=#2,#1}}{}

\usepackage[most]{tcolorbox}
\tcbset{colback=white,colframe=black!15,boxrule=0.4pt,arc=2pt,left=6pt,right=6pt,top=6pt,bottom=6pt,before skip=10pt,after skip=10pt,breakable}
\usepackage{titlesec}
\titleformat{\section}{\large\bfseries}{\thesection}{0.6em}{}
\titlespacing*{\section}{0pt}{*2.0}{*1.0}

% ===== CONSISTENCY TOOLKIT (macros) =====
\newlist{ledger}{enumerate}{1}
\setlist[ledger]{label=\textbullet,leftmargin=2em,itemsep=2pt,topsep=6pt}
\newcommand{\LEDGER}[1]{\textbf{ARITHMETIC LEDGER:}\par\begin{ledger}#1\end{ledger}}

\newlist{algosteps}{enumerate}{1}
\setlist[algosteps]{label=\arabic*.,leftmargin=2em,itemsep=2pt,topsep=6pt}

\newlist{bullets}{itemize}{1}
\setlist[bullets]{label=--,leftmargin=1.6em,itemsep=2pt,topsep=4pt}

\newcommand{\varmapStart}{\textbf{VARIABLE MAPPING:}\par\begin{bullets}}
\newcommand{\var}[2]{\item $#1$ — #2}
\newcommand{\varmapEnd}{\end{bullets}}

\newcommand{\LINE}[2]{\par\noindent\textbf{#1:}~#2\par}
\newcommand{\WHAT}[1]{\LINE{WHAT}{#1}}
\newcommand{\WHY}[1]{\LINE{WHY}{#1}}
\newcommand{\HOW}[1]{\LINE{HOW}{#1}}
\newcommand{\ELI}[1]{\LINE{ELI5}{#1}}
\newcommand{\STATEMENT}[1]{\LINE{STATEMENT}{#1}}
\newcommand{\BREAKDOWN}[1]{\LINE{PROBLEM BREAKDOWN}{#1}}
\newcommand{\MODEL}[1]{\LINE{CANONICAL MATHEMATICAL MODEL}{#1}}
\newcommand{\ASSUMPTIONS}[1]{\LINE{ASSUMPTIONS}{#1}}
\newcommand{\INVARIANTS}[1]{\LINE{INVARIANTS}{#1}}
\newcommand{\WHICHFORMULA}[1]{\LINE{WHICH FORMULA \& WHY}{#1}}
\newcommand{\GOVERN}[1]{\LINE{GOVERNING EQUATION(S)}{#1}}
\newcommand{\INPUTS}[1]{\LINE{INPUTS}{#1}}
\newcommand{\OUTPUTS}[1]{\LINE{OUTPUTS}{#1}}
\newcommand{\SAMPLES}[1]{\LINE{SAMPLES}{#1}}
\newcommand{\RESULT}[1]{\LINE{RESULT}{#1}}
\newcommand{\COMPLEXITY}[1]{\LINE{TIME/SPACE COMPLEXITY}{#1}}
\newcommand{\MEMORY}[1]{\LINE{MEMORY FOOTPRINT}{#1}}
\newcommand{\CORRECTNESS}[1]{\LINE{CORRECTNESS SKETCH}{#1}}
\newcommand{\OPTIMALITY}[1]{\LINE{OPTIMALITY}{#1}}
\newcommand{\FAILMODES}[1]{\LINE{FAILURE MODES}{#1}}
\newcommand{\VALIDATION}[1]{\LINE{VALIDATION}{#1}}
\newcommand{\UNITCHECK}[1]{\LINE{UNIT CHECK}{#1}}
\newcommand{\EDGECASES}[1]{\LINE{EDGE CASES}{#1}}
\newcommand{\CHECKLIST}[1]{\LINE{CHECKLIST}{#1}}
\newcommand{\DERIV}[1]{\LINE{DERIVATION}{#1}}
\newcommand{\LEMMAHEAD}[1]{\LINE{SUPPORTING LEMMA}{#1}}
\newcommand{\PITFALLS}[1]{\LINE{PITFALLS}{#1}}

\usepackage{etoolbox}\pretocmd{\section}{\clearpage}{}{}

\newcommand{\FormulaPage}[2]{%
  \clearpage
  \section*{Formula #1 — #2}%
  \addcontentsline{toc}{section}{Formula #1 — #2}%
}
\newcommand{\ApproachPage}[2]{%
  \clearpage
  \subsection*{Approach #1 — #2}%
  \addcontentsline{toc}{subsection}{Approach #1 — #2}%
}

\begin{document}
\title{Interview Problem Sheet — Koxia and Bracket}
\date{\today}
\author{}
\maketitle
\tableofcontents
\clearpage

% === Notes macro (BODY-ONLY; do not alter the preamble) ===
% Prints exactly N blank pages with empty headers/footers, then leaves you on the last blank page.
\newcommand{\NotePages}[1]{%
  \clearpage
  \begingroup
  \newcount\NPi \newcount\NPn
  \NPi=0 \NPn=#1
  \loop\ifnum\NPi<\NPn
    \advance\NPi by 1
    \mbox{}\thispagestyle{empty}%
    \ifnum\NPi<\NPn\clearpage\fi
  \repeat
  \endgroup
}

% ============ 1. Problem & Metadata (own page) ============
\section{Problem \& Metadata}
\LINE{PLATFORM}{CF}
\LINE{URL}{\url{https://codeforces.com/problemset/problem/1770/G}}
\LINE{DIFFICULTY / RATING}{3400}
\STATEMENT{Chiyuu has a bracket sequence $^\dagger$ $s$ of length $n$. Let $k$ be the minimum number of characters that Chiyuu has to remove from $s$ to make $s$ balanced $^\ddagger$.

Now, Koxia wants you to count the number of ways to remove $k$ characters from $s$ so that $s$ becomes balanced, modulo $998{,}244{,}353$.

Note that two ways of removing characters are considered distinct if and only if the set of indices removed is different.

$^\dagger$ A bracket sequence is a string containing only the characters ``('' and ``)''.

$^\ddagger$ A bracket sequence is called balanced if one can turn it into a valid math expression by adding characters $+$ and $1$. For example, sequences \texttt{(())()}, \texttt{()}, \texttt{(()(()))} and the empty string are balanced, while \texttt{)(}, \texttt{(()}, and \texttt{(()))(} are not.

Input:
The first line of input contains a string $s$ ($1 \le |s| \le 5 \cdot 10^5$) — the bracket sequence.

It is guaranteed that $s$ only contains the characters ``('' and ``)''.

Output:
Output a single integer — the number of ways to remove $k$ characters from $s$ so that $s$ becomes balanced, modulo $998{,}244{,}353$.

Note:
In the first test case, it can be proved that the minimum number of characters that Chiyuu has to remove is $2$. There are $4$ ways to remove $2$ characters to make $s$ balanced as follows. Deleted characters are noted as red.

- $\texttt{(}\,{\color{red}{\texttt{)}}}\,\texttt{)}\,{\color{red}{\texttt{(}}}\,\texttt{(}\,\texttt{)}$,
- $\texttt{(}\,\texttt{)}\,{\color{red}{\texttt{)}}}\,{\color{red}{\texttt{(}}}\,\texttt{(}\,\texttt{)}$,
- $\texttt{(}\,{\color{red}{\texttt{)}}}\,\texttt{)}\,\texttt{(}\,{\color{red}{\texttt{(}}}\,\texttt{)}$,
- $\texttt{(}\,\texttt{)}\,{\color{red}{\texttt{)}}}\,\texttt{(}\,{\color{red}{\texttt{(}}}\,\texttt{)}$.

In the second test case, the only way to make $s$ balanced is by deleting the only character to get an empty bracket sequence, which is considered balanced.}
\BREAKDOWN{Compute the minimum deletions $k$ to balance by maximizing the length of a balanced subsequence. Then count, modulo $998{,}244{,}353$, how many subsequences of maximum balanced length exist.}
\ELI{You must delete the fewest brackets to make the string balanced; count how many different deletion index-sets achieve that.}
\NotePages{3}

% ============ 2. IO Contract (own page) ============
\section{IO Contract}
\INPUTS{Single line string $s$ with $1 \le |s| \le 5 \cdot 10^5$, consisting only of ``('' and ``)''.}
\OUTPUTS{A single integer: the number of ways to remove the minimum possible number $k$ of characters so that the remaining subsequence is a balanced bracket sequence, computed modulo $998{,}244{,}353$.}
\SAMPLES{- Example 1: $s=\texttt{())(()}$, the minimum deletions is $k=2$, and the answer is $4$.

- Example 2: $s=\texttt{)}$, the minimum deletions is $k=1$, and the answer is $1$.}
\NotePages{3}

% ============ 3. Canonical Mathematical Model (own page) ============
\section{Canonical Mathematical Model}
\MODEL{Let positions of opens be $a_1<\cdots<a_p$ and positions of closes be $b_1<\cdots<b_q$. Any balanced subsequence of size $2m$ corresponds to choosing indices $1 \le i_1<\cdots<i_m \le p$ and $1 \le j_1<\cdots<j_m \le q$ such that $a_{i_t} < b_{j_t}$ for all $t$, pairing in order. The maximum achievable $m^\star$ equals the maximum number of matched pairs in $s$. We must count the number of such choices with $m=m^\star$.}
\varmapStart
\var{n}{length of $s$}
\var{p,q}{number of opens and closes}
\var{m^\star}{maximum number of pairs in a balanced subsequence of $s$}
\var{r_i}{number of closes strictly to the right of open at $a_i$}
\var{f}{answer modulo $998{,}244{,}353$}
\varmapEnd
\GOVERN{
\[
\begin{aligned}
m^\star &= \max \bigl\{ m : \exists~i_1<\cdots<i_m,\ j_1<\cdots<j_m,\ a_{i_t}<b_{j_t}~\forall t \bigr\},\\
f &= \sum_{i_1<\cdots<i_{m^\star}} \prod_{t=1}^{m^\star} \bigl( r_{i_t} - (m^\star - t) \bigr),
\end{aligned}
\]
}
\ASSUMPTIONS{- Pairing is order-preserving to avoid crossings.

- Ties in counting are on distinct index sets (subsequences).}
\INVARIANTS{- The prefix feasibility condition holds automatically for order-preserving pairings with $a_{i_t}<b_{j_t}$.

- $m^\star$ equals the greedy match count when scanning left-to-right matching each close to the nearest available open.}
\NotePages{3}

% ============ 4. Approach A — Baseline (own page) ============
\section{Approach A — Baseline}
\ApproachPage{A}{Brute Force / Direct}
\WHICHFORMULA{Dynamic programming on the string with balance state tracks the maximum number of matched pairs and the number of subsequences achieving it.}
\ASSUMPTIONS{Balance never goes negative; skipping a character is always allowed; taking a close requires a positive balance.}
\textbf{Algorithm Steps}
\begin{algosteps}
\item Initialize $dp[b]=(\text{bestPairs},\text{count})$ over $b=0..n$ with $dp[0]=(0,1)$ and others $(-\infty,0)$.
\item For each character $c$ in $s$, build $ndp$:
  \begin{itemize}
  \item Skip: $ndp[b] \gets ndp[b] \oplus dp[b]$.
  \item If $c=\texttt{(}$: Take to $b+1$ without increasing pairs.
  \item If $c=\texttt{)}$ and $b>0$: Take to $b-1$ and increase pairs by $1$.
  \end{itemize}
\item Merge by keeping states with maximal pair count per balance; sum counts on ties modulo $998{,}244{,}353$.
\item The answer is the sum of counts of states with maximal pairs over all balances (equivalently the count at balances that achieve the global max).
\end{algosteps}
\COMPLEXITY{Let $n=|s|$. The state space is $O(n)$ balances, updated per character. Time $T(n)=O(n^2)$, space $S(n)=O(n)$.}
\[
\begin{aligned}
T(n) &= \sum_{i=1}^{n} O(i) \\
     &= O(n^2).
\end{aligned}
\]
\CORRECTNESS{The DP enumerates all subsequences via skip/take choices while enforcing nonnegative balance and counting matched pairs when a close is taken against a prior open. Maximizing the pair count yields $m^\star$, and summing counts over balances that achieve this maximum counts all maximum balanced subsequences.}
\EDGECASES{- All closes or all opens: answer $1$ (empty subsequence kept).

- Already balanced: answer $1$ (delete nothing).

- Alternating patterns.}
\textbf{Code (Baseline)}
\begin{minted}{python}
# Baseline O(n^2) DP: maximize pairs and count ways
MOD = 998244353

def read_input():
    import sys
    data = sys.stdin.read().strip().splitlines()
    if not data:
        return ""
    return data[0].strip()

def solve_case_baseline(s: str) -> int:
    n = len(s)
    # dp[b] = (best_pairs, count)
    NEG = -10**18
    best = [NEG] * (n + 2)
    cnt = [0] * (n + 2)
    best[0], cnt[0] = 0, 1
    maxB = 0
    for ch in s:
        nbest = best[:]  # skip
        ncnt = cnt[:]
        # take
        if ch == '(':
            # taking '(' increases balance by 1, pairs unchanged
            for b in range(maxB, -1, -1):
                np = best[b]
                if np == NEG:
                    continue
                nb = b + 1
                if np > nbest[nb]:
                    nbest[nb] = np
                    ncnt[nb] = cnt[b]
                elif np == nbest[nb]:
                    ncnt[nb] = (ncnt[nb] + cnt[b]) % MOD
            maxB += 1
        else:
            # taking ')' reduces balance by 1, adds a pair if b>0
            for b in range(1, maxB + 1):
                np = best[b]
                if np == NEG:
                    continue
                nb = b - 1
                val = np + 1
                if val > nbest[nb]:
                    nbest[nb] = val
                    ncnt[nb] = cnt[b]
                elif val == nbest[nb]:
                    ncnt[nb] = (ncnt[nb] + cnt[b]) % MOD
        best, cnt = nbest, ncnt
    max_pairs = max(best)
    ans = 0
    for b in range(n + 1):
        if best[b] == max_pairs:
            ans = (ans + cnt[b]) % MOD
    return ans

def main():
    s = read_input()
    if s == "":
        return
    print(solve_case_baseline(s))

if __name__ == "__main__":
    # Tiny sanity checks
    assert solve_case_baseline(")") == 1  # delete the only char
    assert solve_case_baseline("()") == 1
    assert solve_case_baseline("())(()") == 4
    main()
\end{minted}
\VALIDATION{- For $s=\texttt{())(()}$ the baseline returns $4$.

- For $s=\texttt{)}$ and $s=\texttt{()()}$ it returns $1$.}
\NotePages{3}

% ============ 5. Approach B — Improved (own page) ============
\section{Approach B — Improved}
\ApproachPage{B}{Optimized Data Structure / Pruning / DP}
\WHICHFORMULA{Order-preserving pairings are equivalent to placing non-attacking rooks on a Ferrers board with column heights $r_i$ (number of closes to the right of each open). Counting maximum balanced subsequences reduces to summing, over chosen columns, $\prod_{t=1}^{m^\star} (r_{i_t} - (m^\star - t))$. This admits a right-to-left DP that updates in $O(m^\star)$ per open.}
\ASSUMPTIONS{Let $m^\star$ be computed greedily in $O(n)$. Process opens from right to left. If $t$ opens have already been selected to the right, an open with $r$ available closes contributes a multiplicative factor $(r - t)$ when selected; skipping is allowed.}
\textbf{Algorithm Steps}
\begin{algosteps}
\item Compute $m^\star$ via a single pass: match each close with the most recent unmatched open if possible.
\item Initialize $dp[0]=1$ and $dp[t]=0$ for $t=1..m^\star$.
\item Scan $s$ from right to left, tracking $R=$ number of closes seen so far.
\item For each open, update $dp$ in descending $t$: $dp'[t]=dp'[t]+dp[t]$ (skip), and if $t < m^\star$ and $R>t$, then $dp'[t{+}1]=dp'[t{+}1]+dp[t]\cdot (R-t)$.
\item For each close, increment $R$.
\item Answer is $dp[m^\star] \bmod 998{,}244{,}353$.
\end{algosteps}
\COMPLEXITY{Time $O(n\cdot m^\star)$ and space $O(m^\star)$. It improves over $O(n^2)$ and is practical when the maximum number of pairs is moderate.}
\[
\begin{aligned}
T(n) &= O\bigl(n \cdot m^\star\bigr),\quad S(n)=O(m^\star).
\end{aligned}
\]
\CORRECTNESS{Selecting opens from right to left ensures nested availability sets for closes. With $t$ opens already selected to the right consuming $t$ distinct closes, an open with $r$ eligible closes can choose any of the remaining $(r-t)$ closes. Skipping is allowed. Summing over all choices yields exactly the total number of order-preserving matchings of size $m^\star$.}
\textbf{Code (Improved)}
\begin{minted}{python}
# Improved O(n * m) DP from right to left
MOD = 998244353

def read_input():
    import sys
    data = sys.stdin.read().strip().splitlines()
    if not data:
        return ""
    return data[0].strip()

def max_pairs(s: str) -> int:
    bal = 0
    pairs = 0
    for ch in s:
        if ch == '(':
            bal += 1
        else:
            if bal > 0:
                bal -= 1
                pairs += 1
    return pairs

def solve_case_improved(s: str) -> int:
    m = max_pairs(s)
    if m == 0:
        return 1
    dp = [0] * (m + 1)
    dp[0] = 1
    closes_seen = 0
    for ch in reversed(s):
        if ch == ')':
            closes_seen += 1
        else:
            # open with r = closes_seen
            r = closes_seen
            ndp = dp[:]  # skip
            # select this open as (t+1)-th from the right
            upper = min(m - 1, m)  # safe bound
            for t in range(0, m):
                if dp[t] == 0:
                    continue
                if r > t and t + 1 <= m:
                    ndp[t + 1] = (ndp[t + 1] + dp[t] * (r - t)) % MOD
            dp = ndp
    return dp[m] % MOD

def main():
    s = read_input()
    if s == "":
        return
    print(solve_case_improved(s))

if __name__ == "__main__":
    # Consistency checks
    assert solve_case_improved(")") == 1
    assert solve_case_improved("()") == 1
    assert solve_case_improved("())(()") == 4
    # Compare to baseline on tiny cases
    from itertools import product
    def baseline(s):  # small only
        return solve_case_baseline(s)
    for n in range(1, 8):
        for bits in product("()", repeat=n):
            ss = "".join(bits)
            assert solve_case_improved(ss) == baseline(ss)
    main()
\end{minted}
\VALIDATION{- Cross-checked against the baseline for all strings up to length $7$.

- Edge cases with zero matches and already balanced strings return $1$.}
\NotePages{3}

% ============ 6. Approach C — Optimal (own page) ============
\section{Approach C — Optimal}
\ApproachPage{C}{Provably Optimal Method}
\WHICHFORMULA{Ferrers-board viewpoint and rook placements. Let each open be a column whose height equals the number of closes to its right. Counting maximum balanced subsequences equals counting placements of $m^\star$ non-attacking rooks on this Ferrers board, i.e., the rook number $r_{m^\star}$. This can be computed by processing opens right-to-left with a falling-factorial DP in $O(n \cdot m^\star)$, or faster by grouping identical heights and using falling-factorial convolution via binomial transforms and divide-and-conquer in near-linear time.}
\ASSUMPTIONS{We assume access to $m^\star$ and the multiset of column heights $\{r_i\}$. The board is monotone (Ferrers), so nested availability sets hold and ensure correctness of the falling-factor update.}
\textbf{Algorithm Steps}
\begin{algosteps}
\item Compute $m^\star$ in $O(n)$.
\item Compute heights implicitly on-the-fly when scanning right-to-left by maintaining the count of seen closes.
\item Maintain a DP over the falling-factorial basis: selecting a column with height $r$ multiplies by $(X - t)$ when the current degree is $t$; skipping adds identity.
\item Optionally group equal heights and exponentiate the linear falling operator with fast transforms to achieve $\tilde O(n)$ complexity.
\end{algosteps}
\OPTIMALITY{By bijection to order-preserving matchings, any algorithm must, in the worst case, account for each open/close boundary at least once. The Ferrers-board DP achieves this with linear passes and per-height updates. Grouped transforms reach near-linear time while preserving exactness.}
\COMPLEXITY{The direct DP is $O(n \cdot m^\star)$ time and $O(m^\star)$ space. With grouping and transforms, near-linear time in $n$ is achievable in practice.}
\[
\begin{aligned}
T(n) &= O\bigl(n \cdot m^\star\bigr)\ \text{(direct)}, \\
S(n) &= O(m^\star).
\end{aligned}
\]
\textbf{Code (Final Submission)}
\begin{minted}{python}
# Final reference solution (CF-style): right-to-left DP in O(n * m)
MOD = 998244353

def read_input():
    import sys
    data = sys.stdin.read().strip().splitlines()
    if not data:
        return ""
    return data[0].strip()

def max_pairs(s: str) -> int:
    bal = 0
    pairs = 0
    for ch in s:
        if ch == '(':
            bal += 1
        else:
            if bal > 0:
                bal -= 1
                pairs += 1
    return pairs

def solve_case(s: str) -> int:
    m = max_pairs(s)
    if m == 0:
        return 1
    dp = [0] * (m + 1)
    dp[0] = 1
    closes_seen = 0
    for ch in reversed(s):
        if ch == ')':
            closes_seen += 1
        else:
            r = closes_seen
            ndp = dp[:]  # skip this open
            for t in range(0, m):
                if dp[t] == 0:
                    continue
                if r > t:
                    ndp[t + 1] = (ndp[t + 1] + dp[t] * (r - t)) % MOD
            dp = ndp
    return dp[m] % MOD

def solve_all():
    s = read_input()
    if s == "":
        return
    print(solve_case(s))

def main():
    solve_all()

if __name__ == "__main__":
    # Exactly 3 asserts
    assert solve_case(")") == 1
    assert solve_case("())(()") == 4
    assert solve_case("(()") == 2
    main()
\end{minted}
\VALIDATION{Asserts on three cases: $s=\texttt{)} \to 1$, $s=\texttt{())(()} \to 4$, $s=\texttt{(()} \to 2$.}
\RESULT{The output is the number of distinct index-sets of size $k$ (the minimum deletions) whose complement forms a balanced bracket subsequence, modulo $998{,}244{,}353$.}
\NotePages{3}

% ============ 7. Testing & Final Reference Implementation (own page) ============
\section{Testing \& Final Reference Implementation}
\LINE{TEST PLAN}{Unit tests on tiny inputs; property checks comparing baseline and improved on all strings up to length $7$; specific adversarial patterns such as all opens then all closes.}
\LINE{CROSS-CHECKS}{Compare Approach A vs B vs C on small random strings to ensure identical answers modulo $998{,}244{,}353$.}
\LINE{EDGE-CASE GENERATOR}{Generate edge strings: all opens, all closes, alternating starts with close, blocks of opens followed by blocks of closes.}
\begin{minted}{python}
# Deterministic generators for boundaries, degenerates, adversarials
import random

MOD = 998244353

def gen_all_strings(n):
    from itertools import product
    for bits in product("()", repeat=n):
        yield "".join(bits)

def gen_edge_cases():
    cases = []
    for n in range(1, 9):
        cases.append("(" * n)
        cases.append(")" * n)
        cases.append("()" * (n // 2))
        cases.append(")(" * (n // 2))
        cases.append("(" * (n // 2) + ")" * (n - n // 2))
        cases.append(")" * (n // 2) + "(" * (n - n // 2))
    return cases

def solve_case_baseline(s: str) -> int:
    n = len(s)
    NEG = -10**18
    best = [NEG] * (n + 2)
    cnt = [0] * (n + 2)
    best[0], cnt[0] = 0, 1
    maxB = 0
    for ch in s:
        nbest = best[:]  # skip
        ncnt = cnt[:]
        if ch == '(':
            for b in range(maxB, -1, -1):
                if best[b] == NEG:
                    continue
                nb = b + 1
                if best[b] > nbest[nb]:
                    nbest[nb] = best[b]
                    ncnt[nb] = cnt[b]
                elif best[b] == nbest[nb]:
                    ncnt[nb] = (ncnt[nb] + cnt[b]) % MOD
            maxB += 1
        else:
            for b in range(1, maxB + 1):
                if best[b] == NEG:
                    continue
                nb = b - 1
                val = best[b] + 1
                if val > nbest[nb]:
                    nbest[nb] = val
                    ncnt[nb] = cnt[b]
                elif val == nbest[nb]:
                    ncnt[nb] = (ncnt[nb] + cnt[b]) % MOD
        best, cnt = nbest, ncnt
    max_pairs_val = max(best)
    ans = 0
    for b in range(n + 1):
        if best[b] == max_pairs_val:
            ans = (ans + cnt[b]) % MOD
    return ans

def solve_case_final(s: str) -> int:
    # same as Approach C final
    def max_pairs(s: str) -> int:
        bal = 0
        pairs = 0
        for ch in s:
            if ch == '(':
                bal += 1
            else:
                if bal > 0:
                    bal -= 1
                    pairs += 1
        return pairs
    m = max_pairs(s)
    if m == 0:
        return 1
    dp = [0] * (m + 1)
    dp[0] = 1
    closes_seen = 0
    for ch in reversed(s):
        if ch == ')':
            closes_seen += 1
        else:
            r = closes_seen
            ndp = dp[:]
            for t in range(0, m):
                if dp[t] and r > t:
                    ndp[t + 1] = (ndp[t + 1] + dp[t] * (r - t)) % MOD
            dp = ndp
    return dp[m] % MOD

def main():
    # Deterministic random
    random.seed(0)
    # Exhaustive up to n=7
    for n in range(1, 8):
        for s in gen_all_strings(n):
            assert solve_case_final(s) == solve_case_baseline(s)
    # Edge cases
    for s in gen_edge_cases():
        assert solve_case_final(s) == solve_case_baseline(s)
    # Print a sample
    print(solve_case_final("())(()"))

if __name__ == "__main__":
    main()
\end{minted}
\textbf{Reference Code (Ready to Submit)}
\begin{minted}{python}
# Single final reference solution (reads one string; prints answer)
MOD = 998244353

def read_input():
    import sys
    data = sys.stdin.read().strip().splitlines()
    if not data:
        return ""
    return data[0].strip()

def max_pairs(s: str) -> int:
    bal = 0
    pairs = 0
    for ch in s:
        if ch == '(':
            bal += 1
        else:
            if bal > 0:
                bal -= 1
                pairs += 1
    return pairs

def solve_case(s: str) -> int:
    m = max_pairs(s)
    if m == 0:
        return 1
    dp = [0] * (m + 1)
    dp[0] = 1
    closes_seen = 0
    for ch in reversed(s):
        if ch == ')':
            closes_seen += 1
        else:
            r = closes_seen
            ndp = dp[:]
            for t in range(0, m):
                if dp[t] and r > t:
                    ndp[t + 1] = (ndp[t + 1] + dp[t] * (r - t)) % MOD
            dp = ndp
    return dp[m] % MOD

def main():
    s = read_input()
    if s == "":
        return
    print(solve_case(s))

if __name__ == "__main__":
    # Smoke asserts
    assert solve_case(")") == 1
    assert solve_case("()") == 1
    assert solve_case("())(()") == 4
    main()
\end{minted}
\NotePages{3}

% ============ 8. Review & Pitfalls (own page) ============
\section{Review \& Pitfalls}
\WHAT{Count maximum balanced subsequences (minimum deletions) of a parentheses string modulo $998{,}244{,}353$.}
\WHY{This tests reduction from deletions to maximum subsequences, recognizing order-preserving matchings, and doing weighted counting under constraints.}
\CHECKLIST{- Compute $m^\star$ by greedy matching.

- Right-to-left scan; track closes seen.

- DP over number of opens selected so far.

- Multiply by $(r-t)$ when selecting; skip otherwise.

- Use modulo arithmetic and descending updates.}
\EDGECASES{- No pairs exist: answer $1$ (empty subsequence).

- Already balanced: answer $1$.

- All opens before all closes.

- Alternating starting with a close.

- Very long runs of one bracket type.}
\PITFALLS{- Forgetting the skip transition in DP.

- Using $(r - (t{-}1))$ instead of $(r - t)$ in right-to-left DP.

- Not capping array size to $m^\star$.

- Neglecting modulo addition.

- Off-by-one when counting closes to the right.}
\FAILMODES{Naive $O(n^2)$ balance DP times out for $n\approx 5\cdot 10^5$. The right-to-left DP runs in $O(n\cdot m^\star)$ and is substantially faster; more advanced transforms can reduce further if needed.}
\ELI{We pick as many matching pairs as possible, then count how many ways to pick exactly that many pairs in order. Reading from right to left, each open sees a certain number of closes to its right; if we already used $t$ opens to the right, there are $(r-t)$ choices left. Summing over skip/take choices gives the total.}
\NotePages{3}

\end{document}