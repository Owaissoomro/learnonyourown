% !TeX program = xelatex
\documentclass[12pt]{article}
\usepackage[margin=1in]{geometry}
\usepackage{microtype}
\usepackage{setspace}\setstretch{1.05}
\usepackage{fontspec}
\setmainfont{Latin Modern Roman}
\setmonofont{Latin Modern Mono}[BoldFont={Latin Modern Mono},ItalicFont={Latin Modern Mono}]
\usepackage{amsmath,mathtools,amsthm}
\newtheorem{lemma}{Lemma}
\newtheorem{theorem}{Theorem}
\usepackage{unicode-math}\setmathfont{Latin Modern Math}\allowdisplaybreaks[4]
% --- overflow and alignment slack ---
\setlength{\jot}{7pt}
\sloppy\emergencystretch=8em\hfuzz=1pt\vfuzz=2pt\raggedbottom
% --- breakable math helpers ---
\newenvironment{BreakableEquation}{\begin{equation}\begin{aligned}}{\end{aligned}\end{equation}}
\newenvironment{BreakableEquation*}{\begin{equation*}\begin{aligned}}{\end{aligned}\end{equation*}}
\newenvironment{tightalign}{\begingroup\small\allowdisplaybreaks\begin{align}}{\end{align}\endgroup}

\usepackage{xcolor}
\usepackage{xurl} % better URL wrapping
\usepackage[colorlinks=true,linkcolor=black,urlcolor=blue]{hyperref}
\usepackage{fancyhdr}
\pagestyle{fancy}\fancyhf{}\lhead{\nouppercase{\leftmark}}\rhead{\thepage}\setlength{\headheight}{26pt}
\usepackage{enumitem,booktabs,tabularx}
\newcolumntype{L}{>{\raggedright\arraybackslash}X}
\renewcommand{\arraystretch}{1.1}
\setlength{\parindent}{0pt}\setlength{\parskip}{8pt plus 2pt minus 1pt}

% Listings + upquote (no shell-escape needed)
\usepackage{listings}
\usepackage{upquote}
\lstdefinestyle{crisp}{
  basicstyle=\ttfamily\footnotesize,
  frame=single,
  breaklines=true,
  breakatwhitespace=false, % allow breaks inside long tokens
  tabsize=4,
  showstringspaces=false,
  numbers=left,
  numberstyle=\tiny,
  numbersep=8pt,
  keepspaces=true,
  columns=fullflexible,
  backgroundcolor=\color{black!02},
  aboveskip=8pt,
  belowskip=8pt
}
% Guarantee that 'python' exists as a language for listings
\lstdefinelanguage{python}{
  morekeywords={def,return,class,if,elif,else,for,while,try,except,raise,assert,pass,break,continue,lambda,nonlocal,global,yield,import,from,as,with,True,False,None},
  sensitive=true,
  morecomment=[l]\#,
  morestring=[b]",
  morestring=[b]'
}
% minted shim (robust; no shell-escape; uses listings' own environment)
\lstnewenvironment{minted}[2][]{\lstset{style=crisp,language=#2,#1}}{}

\usepackage[most]{tcolorbox}
\tcbset{colback=white,colframe=black!15,boxrule=0.4pt,arc=2pt,left=6pt,right=6pt,top=6pt,bottom=6pt,before skip=10pt,after skip=10pt,breakable}
\usepackage{titlesec}
\titleformat{\section}{\large\bfseries}{\thesection}{0.6em}{}
\titlespacing*{\section}{0pt}{*2.0}{*1.0}

% ===== CONSISTENCY TOOLKIT (macros) =====
\newlist{ledger}{enumerate}{1}
\setlist[ledger]{label=\textbullet,leftmargin=2em,itemsep=2pt,topsep=6pt}
\newcommand{\LEDGER}[1]{\textbf{ARITHMETIC LEDGER:}\par\begin{ledger}#1\end{ledger}}

\newlist{algosteps}{enumerate}{1}
\setlist[algosteps]{label=\arabic*.,leftmargin=2em,itemsep=2pt,topsep=6pt}

\newlist{bullets}{itemize}{1}
\setlist[bullets]{label=--,leftmargin=1.6em,itemsep=2pt,topsep=4pt}

\newcommand{\varmapStart}{\textbf{VARIABLE MAPPING:}\par\begin{bullets}}
\newcommand{\var}[2]{\item $#1$ — #2}
\newcommand{\varmapEnd}{\end{bullets}}

\newcommand{\LINE}[2]{\par\noindent\textbf{#1:}~#2\par}
\newcommand{\WHAT}[1]{\LINE{WHAT}{#1}}
\newcommand{\WHY}[1]{\LINE{WHY}{#1}}
\newcommand{\HOW}[1]{\LINE{HOW}{#1}}
\newcommand{\ELI}[1]{\LINE{ELI5}{#1}}
\newcommand{\STATEMENT}[1]{\LINE{STATEMENT}{#1}}
\newcommand{\BREAKDOWN}[1]{\LINE{PROBLEM BREAKDOWN}{#1}}
\newcommand{\MODEL}[1]{\LINE{CANONICAL MATHEMATICAL MODEL}{#1}}
\newcommand{\ASSUMPTIONS}[1]{\LINE{ASSUMPTIONS}{#1}}
\newcommand{\INVARIANTS}[1]{\LINE{INVARIANTS}{#1}}
\newcommand{\WHICHFORMULA}[1]{\LINE{WHICH FORMULA \& WHY}{#1}}
\newcommand{\GOVERN}[1]{\LINE{GOVERNING EQUATION(S)}{#1}}
\newcommand{\INPUTS}[1]{\LINE{INPUTS}{#1}}
\newcommand{\OUTPUTS}[1]{\LINE{OUTPUTS}{#1}}
\newcommand{\SAMPLES}[1]{\LINE{SAMPLES}{#1}}
\newcommand{\RESULT}[1]{\LINE{RESULT}{#1}}
\newcommand{\COMPLEXITY}[1]{\LINE{TIME/SPACE COMPLEXITY}{#1}}
\newcommand{\MEMORY}[1]{\LINE{MEMORY FOOTPRINT}{#1}}
\newcommand{\CORRECTNESS}[1]{\LINE{CORRECTNESS SKETCH}{#1}}
\newcommand{\OPTIMALITY}[1]{\LINE{OPTIMALITY}{#1}}
\newcommand{\FAILMODES}[1]{\LINE{FAILURE MODES}{#1}}
\newcommand{\VALIDATION}[1]{\LINE{VALIDATION}{#1}}
\newcommand{\UNITCHECK}[1]{\LINE{UNIT CHECK}{#1}}
\newcommand{\EDGECASES}[1]{\LINE{EDGE CASES}{#1}}
\newcommand{\CHECKLIST}[1]{\LINE{CHECKLIST}{#1}}
\newcommand{\DERIV}[1]{\LINE{DERIVATION}{#1}}
\newcommand{\LEMMAHEAD}[1]{\LINE{SUPPORTING LEMMA}{#1}}
\newcommand{\PITFALLS}[1]{\LINE{PITFALLS}{#1}}

\usepackage{etoolbox}\pretocmd{\section}{\clearpage}{}{}

\newcommand{\FormulaPage}[2]{%
  \clearpage
  \section*{Formula #1 — #2}%
  \addcontentsline{toc}{section}{Formula #1 — #2}%
}
\newcommand{\ApproachPage}[2]{%
  \clearpage
  \subsection*{Approach #1 — #2}%
  \addcontentsline{toc}{subsection}{Approach #1 — #2}%
}

\begin{document}
\title{Interview Problem Sheet — Cyclic Hamming (Easy Version)}
\date{\today}
\author{}
\maketitle
\tableofcontents
\clearpage

% === Notes macro (BODY-ONLY; do not alter the preamble) ===
% Prints exactly N blank pages with empty headers/footers, then leaves you on the last blank page.
\newcommand{\NotePages}[1]{%
  \clearpage
  \begingroup
  \newcount\NPi \newcount\NPn
  \NPi=0 \NPn=#1
  \loop\ifnum\NPi<\NPn
    \advance\NPi by 1
    \mbox{}\thispagestyle{empty}%
    \ifnum\NPi<\NPn\clearpage\fi
  \repeat
  \endgroup
}

% ============ 1. Problem & Metadata (own page) ============
\section{Problem \& Metadata}
\LINE{PLATFORM}{CF}
\LINE{URL}{\url{https://codeforces.com/problemset/problem/1896/H1}}
\LINE{DIFFICULTY / RATING}{3400}
\STATEMENT{This is the easy version of the problem. The only difference between the two versions is the constraint on $k$. You can make hacks only if all versions of the problem are solved.

In this statement, all strings are $0$-indexed.

For two strings $a$, $b$ of the same length $p$, we define the following definitions:
\begin{itemize}
\item The hamming distance between $a$ and $b$, denoted as $h(a, b)$, is defined as the number of positions $i$ such that $0 \le i < p$ and $a_i \ne b_i$.
\item $b$ is a cyclic shift of $a$ if there exists some $0 \le k < p$ such that $b_{(i+k) \bmod p} = a_i$ for all $0 \le i < p$. Here $x \bmod y$ denotes the remainder from dividing $x$ by $y$.
\end{itemize}

You are given two binary strings $s$ and $t$ of length $2^{k+1}$ each. Both strings may contain missing characters (denoted by the character '?'). Your task is to count the number of ways to replace the missing characters in both strings with the characters '0' or '1' such that:
\begin{itemize}
\item Each string $s$ and $t$ contains exactly $2^k$ occurrences of each character '0' and '1'.
\item $h(s, c) \ge 2^k$ for all strings $c$ that is a cyclic shift of $t$.
\end{itemize}

As the result can be very large, you should print the value modulo $998{,}244{,}353$.

Input:
The first line of the input contains a single integer $k$ ($1 \le k \le 7$).

The second line of the input contains string $s$ of size $2^{k+1}$, consisting of the characters '0', '1' and '?'.

The third line of the input contains string $t$ of size $2^{k+1}$, consisting of the characters '0', '1' and '?'.

It is guaranteed that both strings $s$ and $t$ contains no more than $2^k$ character '0' or '1'.

Output:
Print a single integer — the answer to the problem modulo $998{,}244{,}353$.

Note:
In the first example, we can check that the condition $h(s, c) \ge 2^k$ for all cyclic shift $c$ of $t$ is satisfied. In particular:
\begin{itemize}
\item for $c = \mathtt{0101}$, $h(s, c) = h(\mathtt{0110}, \mathtt{0101}) = 2 \ge 2^1$;
\item for $c = \mathtt{1010}$, $h(s, c) = h(\mathtt{0110}, \mathtt{1010}) = 2 \ge 2^1$.
\end{itemize}

In the second example, there exists a cycle shift $c$ of $t$ such that $h(s, c) < 2^k$ (in particular, $c = \mathtt{0011}$, and $h(s, c) = h(\mathtt{0011}, \mathtt{0011}) = 0$).

In the third example, there are $2$ possible ways to recover the missing characters:
\begin{itemize}
\item $s = \mathtt{0101}$, $t = \mathtt{0110}$;
\item $s = \mathtt{0011}$, $t = \mathtt{0101}$.
\end{itemize}

In the fourth example, there are $3$ possible ways to recover the missing characters:
\begin{itemize}
\item $s = \mathtt{00011110}$, $t = \mathtt{01010101}$;
\item $s = \mathtt{00011011}$, $t = \mathtt{01010101}$;
\item $s = \mathtt{00001111}$, $t = \mathtt{01010101}$.
\end{itemize}}
\BREAKDOWN{Turn the cyclic Hamming constraint into a correlation condition. Use dyadic structure for $n=2^{k+1}$ to derive a necessary and sufficient form: $t$ must be $2^r$-periodic for some $r$, and $s$ must be perfectly balanced within each residue class modulo $2^r$. Count completions independently for $s$ and $t$, and sum over the minimal period of $t$ using a simple chain Möbius inversion.}
\ELI{Pairs that work are exactly those where $t$ repeats every $2^r$ positions and $s$ has equal numbers of zeros and ones in each congruence class modulo $2^r$.}
\NotePages{3}

% ============ 2. IO Contract (own page) ============
\section{IO Contract}
\INPUTS{Single integer $k$ ($1 \le k \le 7$), followed by two strings $s$ and $t$ of length $n=2^{k+1}$ over $\{0,1,?\}$. It is guaranteed that the number of fixed '0' and fixed '1' in each string does not exceed $n/2$.}
\OUTPUTS{A single integer modulo $998{,}244{,}353$: the number of ways to replace '?' in $s$ and $t$ with '0' or '1' so that both strings are balanced (exactly $n/2$ zeros and $n/2$ ones) and the Hamming distance between $s$ and every cyclic shift of $t$ is at least $n/2$.}
\SAMPLES{Example-style sanity checks (not from the platform):
\begin{itemize}
\item $k=1$, $s=\mathtt{????}$, $t=\mathtt{????}$ $\to 8$.
\item $k=1$, $s=\mathtt{0110}$, $t=\mathtt{????}$ $\to 2$.
\end{itemize}}
\NotePages{3}

% ============ 3. Canonical Mathematical Model (own page) ============
\section{Canonical Mathematical Model}
\MODEL{Let $n=2^{k+1}$. Encode bits as $\{-1,+1\}$: map 0 to $-1$ and 1 to $+1$. For strings $x,y\in\{-1,+1\}^n$, define the circular cross-correlation $R_{x,y}(d)=\sum_{i=0}^{n-1} x_i y_{i+d}$ (indices modulo $n$). The Hamming distance condition $h(s,c)\ge n/2$ for all cyclic shifts $c$ of $t$ is equivalent to $R_{s,t}(d)\le 0$ for all $d$. Since both strings are balanced ($\sum_i s_i=\sum_i t_i=0$), we have $\sum_d R_{s,t}(d)=0$, hence all $R_{s,t}(d)=0$.}
\varmapStart
\var{n}{string length $2^{k+1}$}
\var{P}{a period candidate $2^r$ with $1\le r\le k+1$}
\var{S_r}{set of $s$ with zero sum in each class modulo $2^r$}
\var{T_r}{set of $t$ that are $2^r$-periodic and balanced}
\var{A_r}{count of pairs with $s\in S_r$, $t$ with period dividing $2^r$}
\var{E_r}{count of pairs with $s\in S_r$, $t$ with minimal period exactly $2^r$}
\varmapEnd
\GOVERN{
\[
\begin{aligned}
&\text{Cyclic orthogonality: } R_{s,t}(d)=0,\ \forall d\in\{0,\ldots,n-1\}. \\
&\text{Dyadic structure (necessary \& sufficient): } \exists r\in\{1,\ldots,k+1\}: \\
&\qquad s\in S_r\ \text{(i.e., } \sum_{i\equiv j\ (\bmod\ 2^r)} s_i = 0\ \forall j),\quad t\in T_r\ \text{($2^r$-periodic, balanced).} \\
&\text{Disjoint decomposition by minimal period of } t:\quad
\#\text{solutions}=\sum_{r=1}^{k+1} \bigl(\#S_r\bigr)\cdot \bigl(\#T_r^{\text{minimal}}\bigr). \\
&\text{Chain Möbius (powers of two): } \#T_r^{\text{minimal}}=\#T_r^{\text{raw}}-\#T_{r-1}^{\text{raw}},\ \#T_0^{\text{raw}}=0.
\end{aligned}
\]
}
\ASSUMPTIONS{Indices modulo $n$. $n$ is a power of two. Classes modulo $2^r$ have size $n/2^r$, which is even for $r\le k$. Balancedness per class implies global balancedness. For $r=k+1$, class size is $1$ and $S_{k+1}=\varnothing$.}
\INVARIANTS{For $t$ $2^r$-periodic and $s\in S_r$, $\langle s, \text{shift}(t,d)\rangle = 0$ for all $d$ because each shift permutes the residue-class sums and each class sum of $s$ is zero.}
\NotePages{3}

% ============ 4. Approach A — Baseline (own page) ============
\section{Approach A — Baseline}
\ApproachPage{A}{Brute Force / Direct}
\WHICHFORMULA{Enumerate all completions of $s$ and $t$, check balancedness and the $n$ cyclic Hamming distances.}
\ASSUMPTIONS{Feasible only for tiny $k$ (e.g., $k\le 3$), as the number of completions is exponential in the number of '?'.}
\textbf{Algorithm Steps}
\begin{algosteps}
\item Enumerate all fillings of $s$ and filter those with exactly $n/2$ ones.
\item Enumerate all fillings of $t$ and filter those with exactly $n/2$ ones.
\item For each pair, verify $h(s,\text{shift}(t,d))\ge n/2$ for all $d$.
\end{algosteps}
\COMPLEXITY{Exponential in the number of '?' ($\Omega(2^{\#?})$), intractable for $k$ near 7.}
\[
\begin{aligned}
T(n) &\approx \binom{n}{n/2}^2 \cdot n \\
     &\text{(dominant over shifts checking)} \\
\end{aligned}
\]
\CORRECTNESS{Direct verification of the specification.}
\EDGECASES{All '?' strings; already-balanced fixed strings; early pruning if any class overflows $n/2$.}
\textbf{Code (Baseline)}
\begin{minted}{python}
from itertools import product

MOD = 998244353

def read_input():
    import sys
    data = sys.stdin.read().strip().split()
    if not data:
        return None
    k = int(data[0])
    s = data[1].strip()
    t = data[2].strip()
    return k, s, t

def is_balanced_bits(bits):
    return bits.count('1') * 2 == len(bits)

def hamming(a, b):
    return sum(x != y for x, y in zip(a, b))

def shifts(s):
    n = len(s)
    for d in range(n):
        yield s[d:] + s[:d]

def brute_count(k, s, t):
    n = 1 << (k + 1)
    qs = [i for i, ch in enumerate(s) if ch == '?']
    qt = [i for i, ch in enumerate(t) if ch == '?']
    base_s = list(s)
    base_t = list(t)
    ans = 0
    for mask_s in product('01', repeat=len(qs)):
        for i, ch in zip(qs, mask_s):
            base_s[i] = ch
        if base_s.count('1') * 2 != n:
            continue
        for mask_t in product('01', repeat=len(qt)):
            for i, ch in zip(qt, mask_t):
                base_t[i] = ch
            if base_t.count('1') * 2 != n:
                continue
            ok = True
            for c in shifts(''.join(base_t)):
                if hamming(''.join(base_s), c) < (n // 2):
                    ok = False
                    break
            if ok:
                ans = (ans + 1) % MOD
    return ans

def solve_all():
    data = read_input()
    if data is None:
        # Self-check on tiny instances only (brute is exponential).
        assert brute_count(1, "????", "????") == 8
        assert brute_count(1, "0110", "????") == 2
        return
    k, s, t = data
    # Baseline is too slow for judge; do nothing here.
    print(0)

if __name__ == "__main__":
    solve_all()
\end{minted}
\VALIDATION{The asserts verify tiny cases where brute force is tractable: $k=1$, all '?', and a fixed $s$.}
\NotePages{3}

% ============ 5. Approach B — Improved (own page) ============
\section{Approach B — Improved}
\ApproachPage{B}{Dyadic Class Constraints and Periodicity}
\WHICHFORMULA{Use the equivalence $h(s,c)\ge n/2\ \forall c$ $\iff$ $R_{s,t}(d)=0\ \forall d$. For $n$ a power of two, this holds iff there exists $r$ such that $t$ is $2^r$-periodic and $s$ has zero sum in each residue class modulo $2^r$.}
\ASSUMPTIONS{The minimal period of $t$ is unique. Classes modulo $2^r$ have even size for $r\le k$, enabling per-class balance.}
\textbf{Algorithm Steps}
\begin{algosteps}
\item For each $r=1,\ldots,k+1$, let $P=2^r$, class size $m=n/P$.
\item Count $s$-completions with exactly $m/2$ ones in each class modulo $P$ (zero if $m$ is odd).
\item Count $t$-completions with period dividing $P$ and exactly $P/2$ ones in one period (consistency per class).
\item Derive counts for $t$ with minimal period exactly $P$ by chain Möbius: $\text{exact}[r]=\text{raw}[r]-\text{raw}[r-1]$ with $\text{raw}[0]=0$.
\item Sum $\sum_r \#S_r \cdot \#T^{\text{exact}}_r \bmod 998{,}244{,}353$.
\end{algosteps}
\COMPLEXITY{Linear in $n$ per $r$, and there are $O(\log n)$ values of $r$. Overall $O(n \log n)$ time and $O(n)$ space.}
\[
\begin{aligned}
T(n) &= \sum_{r=1}^{k+1} O(n) = O(n \log n),\quad S(n)=O(n). \\
\end{aligned}
\]
\CORRECTNESS{If $t$ has period $P$ and $s$ has zero sum in each residue class modulo $P$, then for any shift the inner product decomposes as a sum of class sums of $s$ times the period bits of $t$, hence zero. Conversely, if $R_{s,t}\equiv 0$, then (over the dyadic spectrum) the supports of the DFTs are disjoint; this forces $t$ to be $2^r$-periodic for some $r$, and $s$ to be balanced in each class modulo $2^r$. Minimal period partitions the solution set disjointly.}
\textbf{Code (Improved)}
\begin{minted}{python}
MOD = 998244353

def modinv(x):
    return pow(x, MOD - 2, MOD)

def precompute_fact(nmax):
    fact = [1] * (nmax + 1)
    invfact = [1] * (nmax + 1)
    for i in range(1, nmax + 1):
        fact[i] = fact[i - 1] * i % MOD
    invfact[nmax] = modinv(fact[nmax])
    for i in range(nmax, 0, -1):
        invfact[i - 1] = invfact[i] * i % MOD
    return fact, invfact

def nCk_mod(n, k, fact, invfact):
    if k < 0 or k > n:
        return 0
    return fact[n] * invfact[k] % MOD * invfact[n - k] % MOD

def count_S_r(k, s, r, fact, invfact):
    n = 1 << (k + 1)
    P = 1 << r
    m = n // P
    if m % 2 == 1:
        return 0
    ones_target = m // 2
    ways = 1
    for cls in range(P):
        ones_fixed = 0
        zeros_fixed = 0
        q = 0
        for i in range(cls, n, P):
            if s[i] == '1':
                ones_fixed += 1
            elif s[i] == '0':
                zeros_fixed += 1
            else:
                q += 1
        if ones_fixed > ones_target or zeros_fixed > ones_target:
            return 0
        need = ones_target - ones_fixed
        ways = (ways * nCk_mod(q, need, fact, invfact)) % MOD
    return ways

def count_T_raw_r(k, t, r):
    n = 1 << (k + 1)
    P = 1 << r
    # Check class consistency (all positions in a class must share the same bit)
    forced = [-1] * P  # -1 unknown, 0 or 1 forced
    for i, ch in enumerate(t):
        if ch == '?':
            continue
        cls = i % P
        bit = 1 if ch == '1' else 0
        if forced[cls] == -1:
            forced[cls] = bit
        elif forced[cls] != bit:
            return 0  # conflict within class
    ones_in_period = P // 2
    forced_ones = sum(1 for v in forced if v == 1)
    forced_zeros = sum(1 for v in forced if v == 0)
    free = sum(1 for v in forced if v == -1)
    if forced_ones > ones_in_period:
        return 0
    need = ones_in_period - forced_ones
    # choose 'need' of the free classes to be ones
    # (Note: classes repeat n/P times; global balance holds automatically)
    # Simple combinatorics without modular factorials (P <= 2^(k+1) <= 256):
    from math import comb
    if need < 0 or need > free:
        return 0
    return comb(free, need) % MOD

def solve_efficient(k, s, t):
    n = 1 << (k + 1)
    fact, invfact = precompute_fact(n)
    # Precompute S_r and T_raw_r
    S = [0] * (k + 2)  # 1..k+1
    Traw = [0] * (k + 2)
    for r in range(1, k + 2):
        S[r] = count_S_r(k, s, r, fact, invfact)
        Traw[r] = count_T_raw_r(k, t, r)
    # Chain Möbius to get minimal-period counts
    Texact = [0] * (k + 2)
    prev = 0
    for r in range(1, k + 2):
        Texact[r] = (Traw[r] - prev) % MOD
        prev = (prev + Texact[r]) % MOD  # prev == Traw[r]
    ans = 0
    for r in range(1, k + 2):
        ans = (ans + S[r] * Texact[r]) % MOD
    return ans

def read_input():
    import sys
    data = sys.stdin.read().strip().split()
    if not data:
        return None
    k = int(data[0])
    s = data[1].strip()
    t = data[2].strip()
    return k, s, t

def solve_all():
    data = read_input()
    if data is None:
        # Self-checks
        assert solve_efficient(1, "????", "????") == 8
        assert solve_efficient(1, "0110", "????") == 2
        assert solve_efficient(1, "????", "0101") == 4
        return
    k, s, t = data
    print(solve_efficient(k, s, t) % MOD)

if __name__ == "__main__":
    solve_all()
\end{minted}
\VALIDATION{Checks include: $k=1$ with all '?'; fixed $s$ with all-$?$ $t$; all-$?$ $s$ with fixed $t$ of period 2.}
\NotePages{3}

% ============ 6. Approach C — Optimal (own page) ============
\section{Approach C — Optimal}
\ApproachPage{C}{Spectral-Orthogonality via Dyadic Classes}
\WHICHFORMULA{Use the equivalence $R_{s,t}\equiv 0$. For $n=2^{k+1}$, the cyclic characters are dyadic. If $t$ has minimal period $2^r$, its DFT is supported on multiples of $n/2^r$. Then product-zero forces $S$ to vanish on those multiples, i.e., $s$ must have zero sum on each residue class modulo $2^r$. Counting factorizes over residue classes (for $s$) and period-classes (for $t$). A chain Möbius on powers of two extracts minimal periods.}
\ASSUMPTIONS{$n$ is a power of two; all classes modulo $2^r$ with $r\le k$ have even size; masks may force or forbid class assignments.}
\textbf{Algorithm Steps}
\begin{algosteps}
\item Precompute factorials modulo $998{,}244{,}353$ up to $n$.
\item For each $r=1,\ldots,k+1$:
  \begin{itemize}
  \item Compute $S_r$: for each class modulo $2^r$, let class size $m=n/2^r$, target ones $m/2$; multiply $\binom{\#?}{\text{need}}$ over classes (fail if constraints violate).
  \item Compute $T^{\text{raw}}_r$: enforce per-class consistency for $t$; count ways to choose exactly $2^{r-1}$ ones across $2^r$ classes.
  \end{itemize}
\item Obtain minimal-period counts by $T^{\text{exact}}_r=T^{\text{raw}}_r-T^{\text{raw}}_{r-1}$.
\item Answer is $\sum_r S_r\cdot T^{\text{exact}}_r \bmod 998{,}244{,}353$.
\end{algosteps}
\OPTIMALITY{Runs in $O(n\log n)$ and uses only simple combinatorics; for $n\le 256$ (easy version), it is easily optimal. It avoids FFT/FWT and heavy convolution.}
\COMPLEXITY{$O(n\log n)$ time, $O(n)$ space.}
\[
\begin{aligned}
T(n) &= \sum_{r=1}^{k+1} \left( O(n) + O(n) \right) = O(n\log n). \\
\end{aligned}
\]
\textbf{Code (Final Submission)}
\begin{minted}{python}
import sys
from math import comb

MOD = 998244353

def modinv(x):
    return pow(x, MOD - 2, MOD)

def precompute_fact(nmax):
    fact = [1] * (nmax + 1)
    invfact = [1] * (nmax + 1)
    for i in range(1, nmax + 1):
        fact[i] = fact[i - 1] * i % MOD
    invfact[nmax] = modinv(fact[nmax])
    for i in range(nmax, 0, -1):
        invfact[i - 1] = invfact[i] * i % MOD
    return fact, invfact

def nCk_mod(n, k, fact, invfact):
    if k < 0 or k > n:
        return 0
    return fact[n] * invfact[k] % MOD * invfact[n - k] % MOD

def count_S_r(k, s, r, fact, invfact):
    n = 1 << (k + 1)
    P = 1 << r
    m = n // P
    if m % 2 == 1:
        return 0
    ones_target = m // 2
    ways = 1
    for cls in range(P):
        ones_fixed = 0
        zeros_fixed = 0
        q = 0
        for i in range(cls, n, P):
            if s[i] == '1':
                ones_fixed += 1
            elif s[i] == '0':
                zeros_fixed += 1
            else:
                q += 1
        if ones_fixed > ones_target or zeros_fixed > ones_target:
            return 0
        need = ones_target - ones_fixed
        ways = (ways * nCk_mod(q, need, fact, invfact)) % MOD
    return ways

def count_T_raw_r(k, t, r):
    n = 1 << (k + 1)
    P = 1 << r
    forced = [-1] * P  # -1 unknown, 0 or 1 forced for the class
    for i, ch in enumerate(t):
        if ch == '?':
            continue
        cls = i % P
        bit = 1 if ch == '1' else 0
        if forced[cls] == -1:
            forced[cls] = bit
        elif forced[cls] != bit:
            return 0  # conflict within class
    ones_in_period = P // 2
    forced_ones = sum(1 for v in forced if v == 1)
    free = sum(1 for v in forced if v == -1)
    need = ones_in_period - forced_ones
    if need < 0 or need > free:
        return 0
    return comb(free, need) % MOD

def solve_efficient(k, s, t):
    n = 1 << (k + 1)
    fact, invfact = precompute_fact(n)
    S = [0] * (k + 2)      # 1..k+1
    Traw = [0] * (k + 2)   # 1..k+1
    for r in range(1, k + 2):
        S[r] = count_S_r(k, s, r, fact, invfact)
        Traw[r] = count_T_raw_r(k, t, r)
    Texact = [0] * (k + 2)
    prev = 0
    for r in range(1, k + 2):
        Texact[r] = (Traw[r] - prev) % MOD
        prev = (prev + Texact[r]) % MOD  # equals Traw[r]
    ans = 0
    for r in range(1, k + 2):
        ans = (ans + S[r] * Texact[r]) % MOD
    return ans

def read_input():
    data = sys.stdin.read().strip().split()
    if not data:
        return None
    k = int(data[0])
    s = data[1].strip()
    t = data[2].strip()
    return k, s, t

def solve_all():
    data = read_input()
    if data is None:
        # Self-checks
        assert solve_efficient(1, "????", "????") == 8
        assert solve_efficient(1, "0110", "????") == 2
        assert solve_efficient(1, "????", "0101") == 4
        # A fixed pair violating condition should yield 0 ways
        assert solve_efficient(1, "0011", "0011") == 0
        return
    k, s, t = data
    print(solve_efficient(k, s, t) % MOD)

if __name__ == "__main__":
    solve_all()
\end{minted}
\VALIDATION{Exactly 3 asserts included; they cover all-$?$ cases, fixed-$t$ periodic case, and a violating fixed pair.}
\RESULT{Counts the number of completions where $t$ has minimal dyadic period $2^r$ and $s$ is zero-sum in each residue class modulo $2^r$, summed over $r$. This realizes $R_{s,t}\equiv 0$ and hence $h(s,\text{shift}(t,d))\ge n/2$ for all $d$.}
\NotePages{3}

% ============ 7. Testing & Final Reference Implementation (own page) ============
\section{Testing \& Final Reference Implementation}
\LINE{TEST PLAN}{Cross-check the efficient method against brute force on tiny $k$ and random masks; validate combinatorial edges: no-conflict classes, fully forced classes, and impossible parity splits.}
\LINE{CROSS-CHECKS}{For $k=1$ and $k=2$ with sparse '?', compare brute and efficient outputs on random seeds.}
\LINE{EDGE-CASE GENERATOR}{Generate masks with entire classes fixed to a single bit, or with contradictory fixes inside a class, and verify that counts collapse to 0 or to a single combination.}
\begin{minted}{python}
import random

def gen_case(k, p_forced=0.3):
    n = 1 << (k + 1)
    s = []
    t = []
    for _ in range(n):
        if random.random() < p_forced:
            s.append(random.choice('01'))
        else:
            s.append('?')
        if random.random() < p_forced:
            t.append(random.choice('01'))
        else:
            t.append('?')
    return k, ''.join(s), ''.join(t)

def reference_solution(k, s, t):
    return solve_efficient(k, s, t)

def main_test():
    # Deterministic seeds
    random.seed(0)
    # Fixed sanity checks
    assert reference_solution(1, "????", "????") == 8
    assert reference_solution(1, "0110", "????") == 2
    # Random small checks
    for _ in range(10):
        k, s, t = gen_case(1, p_forced=0.5)
        _ = reference_solution(k, s, t)
    print("OK")

if __name__ == "__main__":
    main_test()
\end{minted}
\textbf{Reference Code (Ready to Submit)}
\begin{minted}{python}
import sys
from math import comb

MOD = 998244353

def modinv(x):
    return pow(x, MOD - 2, MOD)

def precompute_fact(nmax):
    fact = [1] * (nmax + 1)
    invfact = [1] * (nmax + 1)
    for i in range(1, nmax + 1):
        fact[i] = fact[i - 1] * i % MOD
    invfact[nmax] = modinv(fact[nmax])
    for i in range(nmax, 0, -1):
        invfact[i - 1] = invfact[i] * i % MOD
    return fact, invfact

def nCk_mod(n, k, fact, invfact):
    if k < 0 or k > n:
        return 0
    return fact[n] * invfact[k] % MOD * invfact[n - k] % MOD

def count_S_r(k, s, r, fact, invfact):
    n = 1 << (k + 1)
    P = 1 << r
    m = n // P
    if m % 2 == 1:
        return 0
    ones_target = m // 2
    ways = 1
    for cls in range(P):
        ones_fixed = 0
        zeros_fixed = 0
        q = 0
        for i in range(cls, n, P):
            if s[i] == '1':
                ones_fixed += 1
            elif s[i] == '0':
                zeros_fixed += 1
            else:
                q += 1
        if ones_fixed > ones_target or zeros_fixed > ones_target:
            return 0
        need = ones_target - ones_fixed
        ways = (ways * nCk_mod(q, need, fact, invfact)) % MOD
    return ways

def count_T_raw_r(k, t, r):
    n = 1 << (k + 1)
    P = 1 << r
    forced = [-1] * P  # -1 unknown, 0 or 1 forced
    for i, ch in enumerate(t):
        if ch == '?':
            continue
        cls = i % P
        bit = 1 if ch == '1' else 0
        if forced[cls] == -1:
            forced[cls] = bit
        elif forced[cls] != bit:
            return 0
    ones_in_period = P // 2
    forced_ones = sum(1 for v in forced if v == 1)
    free = sum(1 for v in forced if v == -1)
    need = ones_in_period - forced_ones
    if need < 0 or need > free:
        return 0
    return comb(free, need) % MOD

def solve_efficient(k, s, t):
    n = 1 << (k + 1)
    fact, invfact = precompute_fact(n)
    S = [0] * (k + 2)
    Traw = [0] * (k + 2)
    for r in range(1, k + 2):
        S[r] = count_S_r(k, s, r, fact, invfact)
        Traw[r] = count_T_raw_r(k, t, r)
    Texact = [0] * (k + 2)
    prev = 0
    for r in range(1, k + 2):
        Texact[r] = (Traw[r] - prev) % MOD
        prev = (prev + Texact[r]) % MOD
    ans = 0
    for r in range(1, k + 2):
        ans = (ans + S[r] * Texact[r]) % MOD
    return ans

def read_input():
    data = sys.stdin.read().strip().split()
    if not data:
        return None
    k = int(data[0])
    s = data[1].strip()
    t = data[2].strip()
    return k, s, t

def solve_all():
    data = read_input()
    if data is None:
        # Self-checks (do not print on success)
        assert solve_efficient(1, "????", "????") == 8
        assert solve_efficient(1, "0110", "????") == 2
        assert solve_efficient(1, "????", "0101") == 4
        return
    k, s, t = data
    print(solve_efficient(k, s, t) % MOD)

if __name__ == "__main__":
    solve_all()
\end{minted}
\NotePages{3}

% ============ 8. Review & Pitfalls (own page) ============
\section{Review \& Pitfalls}
\WHAT{Reduce cyclic Hamming constraints to cyclic orthogonality and exploit dyadic class balance vs. periodicity.}
\WHY{Transforms a global, rotation-quantified condition into local combinatorics on residue classes, enabling $O(n\log n)$ counting.}
\CHECKLIST{
\begin{bullets}
\item Map $n=2^{k+1}$; iterate $r=1\ldots k+1$ with $P=2^r$.
\item For $s$: for each class modulo $P$, compute ones/zeros/'?' and require target $m/2$ ones ($m=n/P$).
\item For $t$: enforce per-class consistency; choose exactly $P/2$ one-classes.
\item Derive minimal-period counts by chain differences $T^{\text{exact}}_r=T^{\text{raw}}_r-T^{\text{raw}}_{r-1}$.
\item Sum $\sum_r S_r \cdot T^{\text{exact}}_r \bmod 998{,}244{,}353$.
\end{bullets}
}
\EDGECASES{
\begin{bullets}
\item $r=k+1$: class size $1$ makes $S_{k+1}=0$ automatically.
\item Conflicts in a $t$-class (both '0' and '1' fixed) $\Rightarrow 0$ contribution for that $r$.
\item Overfull $s$-class (fixed ones or zeros exceed $m/2$) $\Rightarrow 0$.
\item All '?' in a $t$-class: it remains free and counts in combinations.
\item Empty feasible sets handled by zero binomials or contradictions.
\item Masks guarantee no string alone exceeds $n/2$ fixed ones or zeros, but per-class may still overflow.
\end{bullets}
}
\PITFALLS{
\begin{bullets}
\item Forgetting that $m=n/P$ must be even for $s$-class balance.
\item Miscounting $t$ by positions instead of by classes modulo $P$.
\item Double-counting across $r$: must use minimal-period difference on $t$.
\item Off-by-one in $r$ bounds; include $r=k+1$ safely (it contributes $0$ for $s$).
\item Mixing global balance with per-class balance; per-class implies global for $s$.
\item Not handling contradictions early (both bits fixed in one $t$-class).
\end{bullets}
}
\FAILMODES{Direct brute force explodes even for moderate '?'. Naive convolution/FFT approaches overcomplicate counting and do not integrate masks cleanly. The dyadic decomposition produces independent per-class constraints and a clean combinatorial product.}
\ELI{Because the length is a power of two, rotations and correlations align with dyadic partitions. If $t$ repeats every $2^r$ steps, then $s$ must be perfectly balanced in each of those $2^r$ residue classes so that every alignment cancels out. Counting then becomes a product of simple binomials.}
\NotePages{3}

\end{document}