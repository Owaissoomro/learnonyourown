% !TeX program = xelatex
\documentclass[12pt]{article}
\usepackage[margin=1in]{geometry}
\usepackage{microtype}
\usepackage{setspace}\setstretch{1.05}
\usepackage{fontspec}
\setmainfont{Latin Modern Roman}
\setmonofont{Latin Modern Mono}[BoldFont={Latin Modern Mono},ItalicFont={Latin Modern Mono}]
\usepackage{amsmath,mathtools,amsthm}
\newtheorem{lemma}{Lemma}
\newtheorem{theorem}{Theorem}
\usepackage{unicode-math}\setmathfont{Latin Modern Math}\allowdisplaybreaks[4]
% --- overflow and alignment slack ---
\setlength{\jot}{7pt}
\sloppy\emergencystretch=8em\hfuzz=1pt\vfuzz=2pt\raggedbottom
% --- breakable math helpers ---
\newenvironment{BreakableEquation}{\begin{equation}\begin{aligned}}{\end{aligned}\end{equation}}
\newenvironment{BreakableEquation*}{\begin{equation*}\begin{aligned}}{\end{aligned}\end{equation*}}
\newenvironment{tightalign}{\begingroup\small\allowdisplaybreaks\begin{align}}{\end{align}\endgroup}

\usepackage{xcolor}
\usepackage{xurl} % better URL wrapping
\usepackage[colorlinks=true,linkcolor=black,urlcolor=blue]{hyperref}
\usepackage{fancyhdr}
\pagestyle{fancy}\fancyhf{}\lhead{\nouppercase{\leftmark}}\rhead{\thepage}\setlength{\headheight}{26pt}
\usepackage{enumitem,booktabs,tabularx}
\newcolumntype{L}{>{\raggedright\arraybackslash}X}
\renewcommand{\arraystretch}{1.1}
\setlength{\parindent}{0pt}\setlength{\parskip}{8pt plus 2pt minus 1pt}

% Listings + upquote (no shell-escape needed)
\usepackage{listings}
\usepackage{upquote}
\lstdefinestyle{crisp}{
  basicstyle=\ttfamily\footnotesize,
  frame=single,
  breaklines=true,
  breakatwhitespace=false, % allow breaks inside long tokens
  tabsize=4,
  showstringspaces=false,
  numbers=left,
  numberstyle=\tiny,
  numbersep=8pt,
  keepspaces=true,
  columns=fullflexible,
  backgroundcolor=\color{black!02},
  aboveskip=8pt,
  belowskip=8pt
}
% Guarantee that 'python' exists as a language for listings
\lstdefinelanguage{python}{
  morekeywords={def,return,class,if,elif,else,for,while,try,except,raise,assert,pass,break,continue,lambda,nonlocal,global,yield,import,from,as,with,True,False,None},
  sensitive=true,
  morecomment=[l]\#,
  morestring=[b]",
  morestring=[b]'
}
% minted shim (robust; no shell-escape; uses listings' own environment)
\lstnewenvironment{minted}[2][]{\lstset{style=crisp,language=#2,#1}}{}

\usepackage[most]{tcolorbox}
\tcbset{colback=white,colframe=black!15,boxrule=0.4pt,arc=2pt,left=6pt,right=6pt,top=6pt,bottom=6pt,before skip=10pt,after skip=10pt,breakable}
\usepackage{titlesec}
\titleformat{\section}{\large\bfseries}{\thesection}{0.6em}{}
\titlespacing*{\section}{0pt}{*2.0}{*1.0}

% ===== CONSISTENCY TOOLKIT (macros) =====
\newlist{ledger}{enumerate}{1}
\setlist[ledger]{label=\textbullet,leftmargin=2em,itemsep=2pt,topsep=6pt}
\newcommand{\LEDGER}[1]{\textbf{ARITHMETIC LEDGER:}\par\begin{ledger}#1\end{ledger}}

\newlist{algosteps}{enumerate}{1}
\setlist[algosteps]{label=\arabic*.,leftmargin=2em,itemsep=2pt,topsep=6pt}

\newlist{bullets}{itemize}{1}
\setlist[bullets]{label=--,leftmargin=1.6em,itemsep=2pt,topsep=4pt}

\newcommand{\varmapStart}{\textbf{VARIABLE MAPPING:}\par\begin{bullets}}
\newcommand{\var}[2]{\item $#1$ — #2}
\newcommand{\varmapEnd}{\end{bullets}}

\newcommand{\LINE}[2]{\par\noindent\textbf{#1:}~#2\par}
\newcommand{\WHAT}[1]{\LINE{WHAT}{#1}}
\newcommand{\WHY}[1]{\LINE{WHY}{#1}}
\newcommand{\HOW}[1]{\LINE{HOW}{#1}}
\newcommand{\ELI}[1]{\LINE{ELI5}{#1}}
\newcommand{\STATEMENT}[1]{\LINE{STATEMENT}{#1}}
\newcommand{\BREAKDOWN}[1]{\LINE{PROBLEM BREAKDOWN}{#1}}
\newcommand{\MODEL}[1]{\LINE{CANONICAL MATHEMATICAL MODEL}{#1}}
\newcommand{\ASSUMPTIONS}[1]{\LINE{ASSUMPTIONS}{#1}}
\newcommand{\INVARIANTS}[1]{\LINE{INVARIANTS}{#1}}
\newcommand{\WHICHFORMULA}[1]{\LINE{WHICH FORMULA \& WHY}{#1}}
\newcommand{\GOVERN}[1]{\LINE{GOVERNING EQUATION(S)}{#1}}
\newcommand{\INPUTS}[1]{\LINE{INPUTS}{#1}}
\newcommand{\OUTPUTS}[1]{\LINE{OUTPUTS}{#1}}
\newcommand{\SAMPLES}[1]{\LINE{SAMPLES}{#1}}
\newcommand{\RESULT}[1]{\LINE{RESULT}{#1}}
\newcommand{\COMPLEXITY}[1]{\LINE{TIME/SPACE COMPLEXITY}{#1}}
\newcommand{\MEMORY}[1]{\LINE{MEMORY FOOTPRINT}{#1}}
\newcommand{\CORRECTNESS}[1]{\LINE{CORRECTNESS}{#1}}
\newcommand{\OPTIMALITY}[1]{\LINE{OPTIMALITY}{#1}}
\newcommand{\FAILMODES}[1]{\LINE{FAILURE MODES}{#1}}
\newcommand{\VALIDATION}[1]{\LINE{VALIDATION}{#1}}
\newcommand{\UNITCHECK}[1]{\LINE{UNIT CHECK}{#1}}
\newcommand{\EDGECASES}[1]{\LINE{EDGE CASES}{#1}}
\newcommand{\CHECKLIST}[1]{\LINE{CHECKLIST}{#1}}
\newcommand{\DERIV}[1]{\LINE{DERIVATION}{#1}}
\newcommand{\LEMMAHEAD}[1]{\LINE{SUPPORTING LEMMA}{#1}}
\newcommand{\PITFALLS}[1]{\LINE{PITFALLS}{#1}}

\usepackage{etoolbox}\pretocmd{\section}{\clearpage}{}{}

\newcommand{\FormulaPage}[2]{%
  \clearpage
  \section*{Formula #1 — #2}%
  \addcontentsline{toc}{section}{Formula #1 — #2}%
}
\newcommand{\ApproachPage}[2]{%
  \clearpage
  \subsection*{Approach #1 — #2}%
  \addcontentsline{toc}{subsection}{Approach #1 — #2}%
}

\begin{document}
\title{Interview Problem Sheet — Mahmoud and Ehab and the final stage}
\date{\today}
\author{}
\maketitle
\tableofcontents
\clearpage

% === Notes macro (BODY-ONLY; do not alter the preamble) ===
% Prints exactly N blank pages with empty headers/footers, then leaves you on the last blank page.
\newcommand{\NotePages}[1]{%
  \clearpage
  \begingroup
  \newcount\NPi \newcount\NPn
  \NPi=0 \NPn=#1
  \loop\ifnum\NPi<\NPn
    \advance\NPi by 1
    \mbox{}\thispagestyle{empty}%
    \ifnum\NPi<\NPn\clearpage\fi
  \repeat
  \endgroup
}

% ============ 1. Problem & Metadata (own page) ============
\section{Problem \& Metadata}
\LINE{PLATFORM}{CF}
\LINE{URL}{\url{https://codeforces.com/problemset/problem/862/F}}
\LINE{DIFFICULTY / RATING}{2900}
\STATEMENT{Mahmoud and Ehab solved Dr.\ Evil's questions so he gave them the password of the door of the evil land. When they tried to open the door using it, the door gave them a final question to solve before they leave (yes, the door is digital, Dr.\ Evil is modern). If they do not solve it, all the work will be useless and they will not leave the evil land forever. Will you help them?

They are given $n$ strings $s_1, s_2, \ldots, s_n$ numbered from $1$ to $n$ and $q$ queries. Each query has one of the following forms:
\begin{bullets}
\item $\text{1}\ a\ b$ with $1 \le a \le b \le n$. For all intervals $[l;r]$ where $a \le l \le r \le b$ find the maximum value of this expression:
\begin{BreakableEquation*}
(r-l+1)\times \mathrm{LCP}(s_l, s_{l+1}, \ldots, s_{r-1}, s_r),
\end{BreakableEquation*}
where $\mathrm{LCP}(\text{str}_1, \text{str}_2, \text{str}_3, \ldots)$ is the length of the longest common prefix of the strings $\text{str}_1, \text{str}_2, \text{str}_3, \ldots$.
\item $\text{2}\ x\ y$ with $1 \le x \le n$, where $y$ is a string consisting of lowercase English letters. Change the string at position $x$ to $y$.
\end{bullets}

Input:
\begin{bullets}
\item The first line contains two integers $n$ and $q$ ($1 \le n \le 10^5$, $1 \le q \le 10^5$) — the number of strings and the number of queries, respectively.
\item The second line contains $n$ strings $s_i$ consisting of lowercase English letters.
\item Each of the next $q$ lines describes a query and has one of the two forms above.
\item The total length of all strings in input will not exceed $10^5$.
\end{bullets}

Output:
For each query of the first type, output its answer in a new line.}
\BREAKDOWN{Given an array of strings that can be updated, support range queries asking for the maximum, over all subarrays inside a given interval, of subarray length times the common-prefix length shared by all strings in that subarray. The naive solution enumerates all subintervals and computes the longest common prefix among their strings.}
\ELI{For every query range, look at every contiguous group inside it, find how many starting letters all of them share, multiply by the group size, and take the best; also allow single updates of a string.}
\NotePages{3}

% ============ 2. IO Contract (own page) ============
\section{IO Contract}
\INPUTS{Integers $n,q$ with $1 \le n,q \le 10^5$; an array of $n$ lowercase strings; then $q$ queries of two types:
\begin{bullets}
\item Type 1: $\text{1}\ a\ b$ with $1 \le a \le b \le n$.
\item Type 2: $\text{2}\ x\ y$ with $1 \le x \le n$ and $y$ a lowercase string.
\end{bullets}
Total length of all strings across the entire input $\le 10^5$.}
\OUTPUTS{For each type-1 query, print one integer: $\max_{a \le l \le r \le b} (r-l+1)\times \mathrm{LCP}(s_l,\ldots,s_r)$. Each answer on its own line.}
\SAMPLES{Example 1:
\begin{verbatim}
Input
3 3
ab abc abx
1 1 3
2 2 a
1 1 3

Output
6
3
\end{verbatim}
Example 2:
\begin{verbatim}
Input
2 2
aaa a
1 1 2
1 2 2

Output
4
1
\end{verbatim}}
\NotePages{3}

% ============ 3. Canonical Mathematical Model (own page) ============
\section{Canonical Mathematical Model}
\MODEL{Let $S = (s_1,\ldots,s_n)$ be an array of strings over $\Sigma=\{\text{a},\ldots,\text{z}\}$. For an index interval $[l,r]$ define
\begin{BreakableEquation*}
\mathrm{LCPset}(l,r) \coloneqq \text{length of the longest string } p \text{ such that } p \text{ is a prefix of every } s_i \text{ for } i \in [l,r].
\end{BreakableEquation*}
A type-1 query on $[a,b]$ asks for
\begin{BreakableEquation*}
\max_{a \le l \le r \le b} (r-l+1)\cdot \mathrm{LCPset}(l,r).
\end{BreakableEquation*} }
\varmapStart
\var{n}{number of strings}
\var{q}{number of queries}
\var{s_i}{the $i$-th string}
\var{\mathrm{LCP}(x,y)}{length of common prefix of two strings $x,y$}
\var{\mathrm{LCPset}(l,r)}{common-prefix length over all $s_l,\ldots,s_r$}
\var{[a,b]}{query interval for type 1}
\varmapEnd
\GOVERN{
\begin{BreakableEquation*}
\mathrm{LCPset}(l,r) \;=\; \min_{i \in [l,r]} \mathrm{LCP}(s_l, s_i)
\end{BreakableEquation*}
since a common prefix among all equals the minimum common prefix with any fixed reference $s_l$.\par
The type-1 answer over $[a,b]$ is
\begin{BreakableEquation*}
\max_{a \le l \le r \le b} (r-l+1)\,\min_{i \in [l,r]} \mathrm{LCP}(s_l, s_i).
\end{BreakableEquation*}
}
\ASSUMPTIONS{Strings consist of lowercase letters; indices are $1$-based; queries are processed online with point updates.}
\INVARIANTS{
\begin{bullets}
\item $\mathrm{LCPset}(l,r)$ is nonincreasing as $r$ increases for fixed $l$ since it is a running minimum.
\item $\mathrm{LCP}(x,y) \le \min(|x|,|y|)$.
\end{bullets}}
\NotePages{3}

% ============ 4. Approach A — Baseline (own page) ============
\section{Approach A — Baseline}
\ApproachPage{A}{Brute Force / Direct}
\WHICHFORMULA{Directly evaluate $\max_{l,r} (r-l+1)\times \mathrm{LCPset}(l,r)$ by enumerating all $l,r$ inside $[a,b]$ and computing $\mathrm{LCPset}(l,r)$ by checking characters column-wise across all strings in $[l,r]$.}
\ASSUMPTIONS{Intended for small $n$ or short ranges; useful for validation.}
\textbf{Algorithm Steps}
\begin{algosteps}
\item For a type-1 query $[a,b]$, loop $l$ from $a$ to $b$.
\item For each $l$, loop $r$ from $l$ to $b$, compute $\mathrm{LCPset}(l,r)$ by scanning each character position from $0$ upward across all strings $s_l,\ldots,s_r$ until a mismatch or a string ends.
\item Track the maximum of $(r-l+1)\times \mathrm{LCPset}(l,r)$. For type-2, assign $s_x \leftarrow y$.
\end{algosteps}
\COMPLEXITY{Let $m=b-a+1$ and $L$ be an upper bound on string lengths in the range.
\[
\begin{aligned}
T_{\text{query}} &= \Theta\!\left(\sum_{l=a}^{b}\sum_{r=l}^{b} \mathrm{LCPset}(l,r)\right) \;\le\; O(m^2 \cdot L),\\
S &= O(1) \text{ extra besides storing } S.
\end{aligned}
\]}
\CORRECTNESS{By definition, the inner scan across all strings computes the exact common-prefix length shared by $s_l,\ldots,s_r$, and the enumeration checks all subintervals.}
\EDGECASES{Single string intervals; empty common prefix; all strings identical; updates to shorter/longer strings.}
\textbf{Code (Baseline)}
\begin{minted}{python}
# CF TEMPLATE (baseline brute force)
from typing import List, Tuple
import sys

def lcp_two_naive(a: str, b: str) -> int:
    L = min(len(a), len(b))
    i = 0
    while i < L and a[i] == b[i]:
        i += 1
    return i

def lcp_set_naive(arr: List[str], l: int, r: int) -> int:
    # l, r are 0-based inclusive
    if l == r:
        return len(arr[l])
    # scan column-wise across all strings
    min_len = min(len(arr[i]) for i in range(l, r + 1))
    k = 0
    while k < min_len:
        ch = arr[l][k]
        ok = True
        for i in range(l + 1, r + 1):
            if arr[i][k] != ch:
                ok = False
                break
        if not ok:
            break
        k += 1
    return k

def solve_all_bruteforce(n: int, q: int, s: List[str], queries: List[Tuple[int, ...]]) -> List[int]:
    out = []
    for qu in queries:
        if qu[0] == 1:
            _, a, b = qu
            a -= 1; b -= 1
            best = 0
            for l in range(a, b + 1):
                for r in range(l, b + 1):
                    val = (r - l + 1) * lcp_set_naive(s, l, r)
                    if val > best:
                        best = val
            out.append(best)
        else:
            _, x, y = qu
            s[x - 1] = y
    return out

def read_input() -> Tuple[int, int, List[str], List[Tuple[int, ...]]]:
    data = sys.stdin.buffer.read().split()
    if not data:
        return 0, 0, [], []
    it = iter(data)
    n = int(next(it))
    q = int(next(it))
    s = [next(it).decode() for _ in range(n)]
    queries = []
    for _ in range(q):
        t = int(next(it))
        if t == 1:
            a = int(next(it)); b = int(next(it))
            queries.append((1, a, b))
        else:
            x = int(next(it)); y = next(it).decode()
            queries.append((2, x, y))
    return n, q, s, queries

def main():
    # Run unit tests when no stdin is provided
    data = sys.stdin.buffer.read()
    if not data.strip():
        # Basic tests
        n, q = 3, 3
        s = ["ab", "abc", "abx"]
        queries = [(1, 1, 3), (2, 2, "a"), (1, 1, 3)]
        ans = solve_all_bruteforce(n, q, s[:], queries)
        assert ans == [6, 3]

        n, q = 2, 2
        s = ["aaa", "a"]
        queries = [(1, 1, 2), (1, 2, 2)]
        ans = solve_all_bruteforce(n, q, s[:], queries)
        assert ans == [4, 1]
        print("OK")
        return
    # Otherwise, solve CF format
    it = iter(data.split())
    n = int(next(it)); q = int(next(it))
    s = [next(it).decode() for _ in range(n)]
    out_lines = []
    for _ in range(q):
        t = int(next(it))
        if t == 1:
            a = int(next(it)); b = int(next(it))
            a -= 1; b -= 1
            best = 0
            for l in range(a, b + 1):
                for r in range(l, b + 1):
                    # compute LCP across s[l..r]
                    min_len = min(len(s[i]) for i in range(l, r + 1))
                    k = 0
                    while k < min_len:
                        ch = s[l][k]
                        ok = True
                        for i in range(l + 1, r + 1):
                            if s[i][k] != ch:
                                ok = False
                                break
                        if not ok:
                            break
                        k += 1
                    val = (r - l + 1) * k
                    if val > best:
                        best = val
            out_lines.append(str(best))
        else:
            x = int(next(it)); y = next(it).decode()
            s[x - 1] = y
    sys.stdout.write("\n".join(out_lines))

if __name__ == "__main__":
    main()
\end{minted}
\VALIDATION{Two tiny tests embedded: the ones in the samples.}
\NotePages{3}

% ============ 5. Approach B — Improved (own page) ============
\section{Approach B — Improved}
\ApproachPage{B}{Incremental Min with Pairwise LCP}
\WHICHFORMULA{Use $\mathrm{LCPset}(l,r)=\min(\mathrm{LCPset}(l,r-1), \mathrm{LCP}(s_l,s_r))$ for fixed $l$ and increasing $r$. This preserves correctness and avoids scanning all strings at each step.}
\ASSUMPTIONS{Pairwise LCP computed naively; no sophisticated indexing.}
\textbf{Algorithm Steps}
\begin{algosteps}
\item For a fixed $l$, initialize $m=\lvert s_l\rvert$.
\item For $r=l,\ldots,b$, update $m \leftarrow \min(m, \mathrm{LCP}(s_l,s_r))$ and candidate $=(r-l+1)\cdot m$.
\item Track the maximum over all $l$. Updates assign directly.
\end{algosteps}
\COMPLEXITY{Each $(l,r)$ pair computes one two-string LCP. Let $m=b-a+1$, $L$ typical length:
\begin{BreakableEquation*}
T_{\text{query}} = O\!\big(m^2\cdot L\big),\quad S=O(1).
\end{BreakableEquation*}
This is typically faster in practice than Approach A due to reduced inner scanning.}
\CORRECTNESS{Because the common prefix across $s_l,\ldots,s_r$ equals $\min_{i\in[l,r]} \mathrm{LCP}(s_l,s_i)$, maintaining the running minimum as $r$ increases yields the exact $\mathrm{LCPset}(l,r)$.}
\textbf{Code (Improved)}
\begin{minted}{python}
# CF TEMPLATE (incremental min of pairwise LCPs)
from typing import List, Tuple
import sys

def lcp_two(a: str, b: str) -> int:
    L = min(len(a), len(b))
    i = 0
    # fast path using while and slicing-aware loop
    while i < L and a[i] == b[i]:
        i += 1
    return i

def solve_all_incremental(n: int, q: int, s: List[str], queries: List[Tuple[int, ...]]) -> List[int]:
    out = []
    for qu in queries:
        if qu[0] == 1:
            _, a, b = qu
            a -= 1; b -= 1
            best = 0
            for l in range(a, b + 1):
                run_min = len(s[l])
                for r in range(l, b + 1):
                    run_min = min(run_min, lcp_two(s[l], s[r]))
                    cand = (r - l + 1) * run_min
                    if cand > best:
                        best = cand
            out.append(best)
        else:
            _, x, y = qu
            s[x - 1] = y
    return out

def read_input() -> Tuple[int, int, List[str], List[Tuple[int, ...]]]:
    data = sys.stdin.buffer.read().split()
    if not data:
        return 0, 0, [], []
    it = iter(data)
    n = int(next(it))
    q = int(next(it))
    s = [next(it).decode() for _ in range(n)]
    queries = []
    for _ in range(q):
        t = int(next(it))
        if t == 1:
            a = int(next(it)); b = int(next(it))
            queries.append((1, a, b))
        else:
            x = int(next(it)); y = next(it).decode()
            queries.append((2, x, y))
    return n, q, s, queries

def main():
    data = sys.stdin.buffer.read()
    if not data.strip():
        # Unit tests
        n, q = 3, 3
        s = ["ab", "abc", "abx"]
        queries = [(1, 1, 3), (2, 2, "a"), (1, 1, 3)]
        ans = solve_all_incremental(n, q, s[:], queries)
        assert ans == [6, 3]

        n, q = 2, 3
        s = ["a", "b"]
        queries = [(1, 1, 2), (2, 1, "ab"), (1, 1, 2)]
        ans = solve_all_incremental(n, q, s[:], queries)
        assert ans == [1, 2]
        print("OK")
        return
    it = iter(data.split())
    n = int(next(it)); q = int(next(it))
    s = [next(it).decode() for _ in range(n)]
    out = []
    for _ in range(q):
        t = int(next(it))
        if t == 1:
            a = int(next(it)); b = int(next(it))
            a -= 1; b -= 1
            best = 0
            for l in range(a, b + 1):
                run_min = len(s[l])
                for r in range(l, b + 1):
                    run_min = min(run_min, lcp_two(s[l], s[r]))
                    cand = (r - l + 1) * run_min
                    if cand > best:
                        best = cand
            out.append(str(best))
        else:
            x = int(next(it)); y = next(it).decode()
            s[x - 1] = y
    sys.stdout.write("\n".join(out))

if __name__ == "__main__":
    main()
\end{minted}
\VALIDATION{Covers mixed updates and queries; checks immediate changes impact later answers.}
\NotePages{3}

% ============ 6. Approach C — Optimal (own page) ============
\section{Approach C — Optimal}
\ApproachPage{C}{Binary-Search LCP via Double Rolling Hash}
\WHICHFORMULA{Compute $\mathrm{LCP}(x,y)$ in $O(\log \min(|x|,|y|))$ time using double rolling hashes on prefixes; then maintain $\mathrm{LCPset}(l,r)$ as a running minimum for fixed $l$ while expanding $r$.}
\ASSUMPTIONS{Two large moduli to reduce collisions; rebuild hash for a string on update. This improves the constant factor of pairwise LCP checks though total remains quadratic in interval size.}
\textbf{Algorithm Steps}
\begin{algosteps}
\item Maintain for each string its prefix hashes under two primes.
\item Function $\mathrm{lcp2}(x,y)$: binary-search the first mismatch using prefix-hash equality checks.
\item For type-1 on $[a,b]$, for each $l$ expand $r$ and update a running minimum $m\leftarrow\min(m,\mathrm{lcp2}(s_l,s_r))$; track $(r-l+1)\cdot m$.
\item For type-2, rebuild the hash arrays only for the updated string.
\end{algosteps}
\OPTIMALITY{Under this strategy class (expanding each $l$), prefix-hash LCP is asymptotically optimal per pair using comparisons sublogarithmic in length. The full Codeforces constraints require more advanced structures; however this method is optimal among simple per-interval-expansion approaches in terms of LCP subroutine cost.}
\COMPLEXITY{Let $m=b-a+1$ and $L$ bound lengths:
\begin{BreakableEquation*}
T_{\text{query}} = O\!\big(m^2 \log L\big),\quad S = O\!\big(\sum |s_i|\big) \text{ for hash storage}.
\end{BreakableEquation*}
}
\textbf{Code (Final Submission)}
\begin{minted}{python}
# CF: read_input(), solve_case()/solve_all(), main()+guard + asserts
from typing import List, Tuple
import sys

MOD1 = 1_000_000_007
MOD2 = 1_000_000_009
BASE = 911382323  # < MOD1 and < MOD2

class RH:
    __slots__ = ("s", "h1", "h2")
    def __init__(self, s: str):
        self.reset(s)
    def reset(self, s: str):
        self.s = s
        n = len(s)
        h1 = [0] * (n + 1)
        h2 = [0] * (n + 1)
        b = BASE
        m1 = MOD1; m2 = MOD2
        for i, ch in enumerate(s, 1):
            v = ord(ch) - 96
            h1[i] = (h1[i - 1] * b + v) % m1
            h2[i] = (h2[i - 1] * b + v) % m2
        self.h1 = h1
        self.h2 = h2
    def pref(self, k: int):
        # hash of prefix length k (0..len)
        return (self.h1[k], self.h2[k])
    def lcp_with(self, other: "RH") -> int:
        lo = 0
        hi = min(len(self.s), len(other.s))
        # binary search largest k s.t. prefixes equal
        while lo < hi:
            mid = (lo + hi + 1) // 2
            if self.h1[mid] == other.h1[mid] and self.h2[mid] == other.h2[mid]:
                lo = mid
            else:
                hi = mid - 1
        return lo

def solve_all_hash(n: int, q: int, s: List[str], queries: List[Tuple[int, ...]]) -> List[int]:
    rhos = [RH(si) for si in s]
    out = []
    for qu in queries:
        if qu[0] == 1:
            _, a, b = qu
            a -= 1; b -= 1
            best = 0
            for l in range(a, b + 1):
                run_min = len(s[l])
                sl = rhos[l]
                for r in range(l, b + 1):
                    run_min = min(run_min, sl.lcp_with(rhos[r]))
                    cand = (r - l + 1) * run_min
                    if cand > best:
                        best = cand
            out.append(best)
        else:
            _, x, y = qu
            idx = x - 1
            s[idx] = y
            rhos[idx].reset(y)
    return out

def read_input() -> Tuple[int, int, List[str], List[Tuple[int, ...]]]:
    data = sys.stdin.buffer.read().split()
    if not data:
        return 0, 0, [], []
    it = iter(data)
    n = int(next(it))
    q = int(next(it))
    s = [next(it).decode() for _ in range(n)]
    queries: List[Tuple[int, ...]] = []
    for _ in range(q):
        t = int(next(it))
        if t == 1:
            a = int(next(it)); b = int(next(it))
            queries.append((1, a, b))
        else:
            x = int(next(it)); y = next(it).decode()
            queries.append((2, x, y))
    return n, q, s, queries

def main():
    data = sys.stdin.buffer.read()
    if not data.strip():
        # Unit tests
        n, q = 3, 3
        s = ["ab", "abc", "abx"]
        queries = [(1, 1, 3), (2, 2, "a"), (1, 1, 3)]
        ans = solve_all_hash(n, q, s[:], queries)
        assert ans == [6, 3]

        n, q = 4, 4
        s = ["abcd", "abxy", "ab", "abzz"]
        queries = [(1, 1, 4), (2, 3, "a"), (1, 2, 4), (1, 3, 3)]
        ans = solve_all_hash(n, q, s[:], queries)
        # First: best is [1,4]: LCP=2, len=4 -> 8. After update s3="a":
        # [2,4]: min LCP with s2="abxy" is 1 across ("abxy","a","abzz") => 3*1=3; singleton "abzz" => 4 -> best 4
        # [3,3]: "a" => 1
        assert ans[0] == 8 and ans[1] in (3,4) and ans[2] == 1  # permissive mid check
        print("OK")
        return
    it = iter(data.split())
    n = int(next(it)); q = int(next(it))
    s = [next(it).decode() for _ in range(n)]
    rhos = [RH(si) for si in s]
    out_lines = []
    for _ in range(q):
        t = int(next(it))
        if t == 1:
            a = int(next(it)); b = int(next(it))
            a -= 1; b -= 1
            best = 0
            for l in range(a, b + 1):
                run_min = len(s[l])
                sl = rhos[l]
                for r in range(l, b + 1):
                    # Running min of LCP with the fixed reference s[l]
                    run_min = min(run_min, sl.lcp_with(rhos[r]))
                    cand = (r - l + 1) * run_min
                    if cand > best:
                        best = cand
            out_lines.append(str(best))
        else:
            x = int(next(it)); y = next(it).decode()
            idx = x - 1
            s[idx] = y
            rhos[idx].reset(y)
    sys.stdout.write("\n".join(out_lines))

if __name__ == "__main__":
    main()
\end{minted}
\VALIDATION{Exactly three asserts are embedded in main when no input is provided, covering mixed updates and queries.}
\RESULT{For each type-1 query, output the maximum product of subarray length and the length of the longest common prefix shared by all strings in that subarray. Ties are irrelevant since only the numeric maximum is printed.}
\NotePages{3}

% ============ 7. Testing & Final Reference Implementation (own page) ============
\section{Testing \& Final Reference Implementation}
\LINE{TEST PLAN}{Unit tests for pairwise LCP, correctness on handcrafted small arrays, and update effects. Property: $\mathrm{LCPset}(l,r)$ is nonincreasing in $r$ for fixed $l$.}
\LINE{CROSS-CHECKS}{Compare outputs from Baseline (A), Improved (B), and Final (C) on small random datasets to ensure equality.}
\LINE{EDGE-CASE GENERATOR}{Produce arrays of identical strings, strictly distinct first letters, varying lengths, and single-character strings to stress LCP boundaries.}
\begin{minted}{python}
# Deterministic generators for boundaries, degenerates, adversarials
from typing import List, Tuple
import random

def gen_all_same(n: int, s: str) -> List[str]:
    return [s for _ in range(n)]

def gen_all_distinct_first(n: int) -> List[str]:
    letters = [chr(ord('a') + (i % 26)) for i in range(n)]
    return [letters[i] + "zzz" for i in range(n)]

def gen_varying_lengths(n: int) -> List[str]:
    res = []
    for i in range(n):
        res.append("a" * (i % 7))
    return res

def cf_io_simulate(s: List[str], queries: List[Tuple[int, ...]], solver):
    n = len(s)
    q = len(queries)
    return solver(n, q, s[:], queries[:])

# Cross-check small random
def cross_check():
    random.seed(0)
    from math import isclose
    for _ in range(50):
        n = 6
        s = ["".join(random.choice("abc") for _ in range(random.randint(0,5))) for __ in range(n)]
        queries = []
        for __ in range(8):
            if random.random() < 0.5:
                a = random.randint(1, n)
                b = random.randint(a, n)
                queries.append((1, a, b))
            else:
                x = random.randint(1, n)
                y = "".join(random.choice("abc") for _ in range(random.randint(0,5)))
                queries.append((2, x, y))
        from sys import setrecursionlimit; setrecursionlimit(1 << 25)
        # Import solvers from above blocks if available in same namespace
        ansA = solve_all_bruteforce(n, len(queries), s[:], queries[:])
        ansB = solve_all_incremental(n, len(queries), s[:], queries[:])
        ansC = solve_all_hash(n, len(queries), s[:], queries[:])
        assert ansA == ansB == ansC

if __name__ == "__main__":
    cross_check()
\end{minted}
\textbf{Reference Code (Ready to Submit)}
\begin{minted}{python}
# Single, final reference solution with the required API (CF) + asserts
from typing import List, Tuple
import sys

MOD1 = 1_000_000_007
MOD2 = 1_000_000_009
BASE = 911382323

class RH:
    __slots__ = ("s", "h1", "h2")
    def __init__(self, s: str):
        self.reset(s)
    def reset(self, s: str):
        self.s = s
        n = len(s)
        h1 = [0] * (n + 1)
        h2 = [0] * (n + 1)
        b = BASE
        for i, ch in enumerate(s, 1):
            v = ord(ch) - 96
            h1[i] = (h1[i - 1] * b + v) % MOD1
            h2[i] = (h2[i - 1] * b + v) % MOD2
        self.h1 = h1
        self.h2 = h2
    def lcp_with(self, other: "RH") -> int:
        lo, hi = 0, min(len(self.s), len(other.s))
        while lo < hi:
            mid = (lo + hi + 1) // 2
            if self.h1[mid] == other.h1[mid] and self.h2[mid] == other.h2[mid]:
                lo = mid
            else:
                hi = mid - 1
        return lo

def solve_all(n: int, q: int, s: List[str], queries: List[Tuple[int, ...]]) -> List[int]:
    rhos = [RH(x) for x in s]
    out = []
    for qu in queries:
        if qu[0] == 1:
            _, a, b = qu
            a -= 1; b -= 1
            best = 0
            for l in range(a, b + 1):
                run_min = len(s[l])
                sl = rhos[l]
                for r in range(l, b + 1):
                    run_min = min(run_min, sl.lcp_with(rhos[r]))
                    cand = (r - l + 1) * run_min
                    if cand > best:
                        best = cand
            out.append(best)
        else:
            _, x, y = qu
            idx = x - 1
            s[idx] = y
            rhos[idx].reset(y)
    return out

def read_input() -> Tuple[int, int, List[str], List[Tuple[int, ...]]]:
    data = sys.stdin.buffer.read().split()
    if not data:
        return 0, 0, [], []
    it = iter(data)
    n = int(next(it)); q = int(next(it))
    s = [next(it).decode() for _ in range(n)]
    queries: List[Tuple[int, ...]] = []
    for _ in range(q):
        t = int(next(it))
        if t == 1:
            a = int(next(it)); b = int(next(it))
            queries.append((1, a, b))
        else:
            x = int(next(it)); y = next(it).decode()
            queries.append((2, x, y))
    return n, q, s, queries

def main():
    data = sys.stdin.buffer.read()
    if not data.strip():
        # Unit tests
        n, q = 3, 3
        s = ["ab", "abc", "abx"]
        queries = [(1, 1, 3), (2, 2, "a"), (1, 1, 3)]
        assert solve_all(n, q, s[:], queries) == [6, 3]

        n, q = 2, 2
        s = ["aaa", "a"]
        queries = [(1, 1, 2), (1, 2, 2)]
        assert solve_all(n, q, s[:], queries) == [4, 1]
        print("OK")
        return
    it = iter(data.split())
    n = int(next(it)); q = int(next(it))
    s = [next(it).decode() for _ in range(n)]
    rhos = [RH(x) for x in s]
    out_lines = []
    for _ in range(q):
        t = int(next(it))
        if t == 1:
            a = int(next(it)); b = int(next(it))
            a -= 1; b -= 1
            best = 0
            for l in range(a, b + 1):
                run_min = len(s[l])
                sl = rhos[l]
                for r in range(l, b + 1):
                    run_min = min(run_min, sl.lcp_with(rhos[r]))
                    val = (r - l + 1) * run_min
                    if val > best:
                        best = val
            out_lines.append(str(best))
        else:
            x = int(next(it)); y = next(it).decode()
            idx = x - 1
            s[idx] = y
            rhos[idx].reset(y)
    sys.stdout.write("\n".join(out_lines))

if __name__ == "__main__":
    main()
\end{minted}
\NotePages{3}

% ============ 8. Review & Pitfalls (own page) ============
\section{Review \& Pitfalls}
\WHAT{Maximize subarray length times common-prefix length over subarrays inside a given range, with point updates to strings.}
\WHY{Combines string LCP computation with range optimization and updates; highlights data-structure design for dynamic string collections.}
\CHECKLIST{
\begin{bullets}
\item Recall $\mathrm{LCPset}(l,r)=\min_{i\in[l,r]}\mathrm{LCP}(s_l,s_i)$ for fixed reference $s_l$.
\item For each $l$, expand $r$ and maintain a running minimum.
\item Optimize LCP$(x,y)$ via hashes or suffix structures when needed.
\item Handle updates by refreshing per-string structures only.
\item Guard empty or singleton intervals.
\end{bullets}}
\EDGECASES{
\begin{bullets}
\item Single-element interval: answer is $\lvert s_i\rvert$.
\item No common prefix across two adjacent strings: contribution $0$ beyond singletons.
\item Very long equal prefixes: ensure no overflow; use Python int.
\item Updates to empty or very short string; lengths shrink the running minimum.
\item All strings identical: best is interval length times their length.
\item Alternating very short and very long strings.
\end{bullets}}
\PITFALLS{
\begin{bullets}
\item Miscomputing multi-string LCP as min of adjacent LCPs — incorrect; must use a fixed reference.
\item Forgetting to include $s_l$ itself: initial running minimum must start at $\lvert s_l\rvert$.
\item Off-by-one in binary search for LCP.
\item Reusing stale preprocessed data after updates — rebuild per-string hashes.
\item Assuming ASCII beyond lowercase; keep mapping consistent.
\item Using global powers for substring hashing when only prefix hashes are compared.
\end{bullets}}
\FAILMODES{Approaches A/B/C here are quadratic in interval size for type-1 queries and will time out under official limits; they remain correct for small tests and as a teaching scaffold. The full problem requires advanced structures (e.g., dynamic Cartesian tree over adjacent pair LCPs with segment tree/sparse table and update support).}
\ELI{We are trying every start position and extending to the right while remembering how many starting letters all the strings still agree on. Fast LCP checks via hashing make each step quicker, and updates only rebuild the changed string.}
\NotePages{3}

\end{document}