% !TeX program = xelatex
\documentclass[11pt]{article}
\usepackage[margin=1in]{geometry}
\usepackage{microtype}
\usepackage{setspace}\setstretch{1.05}
\usepackage{fontspec}
\setmainfont{Latin Modern Roman}
\setmonofont{Latin Modern Mono}[BoldFont={Latin Modern Mono},ItalicFont={Latin Modern Mono}]
\usepackage{amsmath,mathtools,amsthm}
\newtheorem{lemma}{Lemma}
\newtheorem{theorem}{Theorem}
\usepackage{unicode-math}\setmathfont{Latin Modern Math}\allowdisplaybreaks[4]
% --- overflow and alignment slack ---
\setlength{\jot}{7pt}
\sloppy\emergencystretch=8em\hfuzz=1pt\vfuzz=2pt\raggedbottom
% --- breakable math helpers ---
\newenvironment{BreakableEquation}{\begin{equation}\begin{aligned}}{\end{aligned}\end{equation}}
\newenvironment{BreakableEquation*}{\begin{equation*}\begin{aligned}}{\end{aligned}\end{equation*}}
\newenvironment{tightalign}{\begingroup\small\allowdisplaybreaks\begin{align}}{\end{align}\endgroup}

\usepackage{xcolor}
\usepackage{xurl} % better URL wrapping
\usepackage[colorlinks=true,linkcolor=black,urlcolor=blue]{hyperref}
\usepackage{fancyhdr}
\pagestyle{fancy}\fancyhf{}\lhead{\nouppercase{\leftmark}}\rhead{\thepage}\setlength{\headheight}{26pt}
\usepackage{enumitem,booktabs,tabularx}
\newcolumntype{L}{>{\raggedright\arraybackslash}X}
\renewcommand{\arraystretch}{1.1}
\setlength{\parindent}{0pt}\setlength{\parskip}{8pt plus 2pt minus 1pt}

% Listings + upquote (no shell-escape needed)
\usepackage{listings}
\usepackage{upquote}
\lstdefinestyle{crisp}{
  basicstyle=\ttfamily\footnotesize,
  frame=single,
  breaklines=true,
  breakatwhitespace=false, % allow breaks inside long tokens
  tabsize=4,
  showstringspaces=false,
  numbers=left,
  numberstyle=\tiny,
  numbersep=8pt,
  keepspaces=true,
  columns=fullflexible,
  backgroundcolor=\color{black!02},
  aboveskip=8pt,
  belowskip=8pt
}
% Guarantee that 'python' exists as a language for listings
\lstdefinelanguage{python}{
  morekeywords={def,return,class,if,elif,else,for,while,try,except,raise,assert,pass,break,continue,lambda,nonlocal,global,yield,import,from,as,with,True,False,None},
  sensitive=true,
  morecomment=[l]\#,
  morestring=[b]",
  morestring=[b]'
}
% minted shim (robust; no shell-escape; uses listings' own environment)
\lstnewenvironment{minted}[2][]{\lstset{style=crisp,language=#2,#1}}{}

\usepackage[most]{tcolorbox}
\tcbset{colback=white,colframe=black!15,boxrule=0.4pt,arc=2pt,left=6pt,right=6pt,top=6pt,bottom=6pt,before skip=10pt,after skip=10pt,breakable}
\usepackage{titlesec}
\titleformat{\section}{\large\bfseries}{\thesection}{0.6em}{}
\titlespacing*{\section}{0pt}{*2.0}{*1.0}

% ===== CONSISTENCY TOOLKIT (macros) =====
\newlist{ledger}{enumerate}{1}
\setlist[ledger]{label=\textbullet,leftmargin=2em,itemsep=2pt,topsep=6pt}
\newcommand{\LEDGER}[1]{\textbf{ARITHMETIC LEDGER:}\par\begin{ledger}#1\end{ledger}}

\newlist{algosteps}{enumerate}{1}
\setlist[algosteps]{label=\arabic*.,leftmargin=2em,itemsep=2pt,topsep=6pt}

\newlist{bullets}{itemize}{1}
\setlist[bullets]{label=--,leftmargin=1.6em,itemsep=2pt,topsep=4pt}

\newcommand{\varmapStart}{\textbf{VARIABLE MAPPING:}\par\begin{bullets}}
\newcommand{\var}[2]{\item $#1$ — #2}
\newcommand{\varmapEnd}{\end{bullets}}

\newcommand{\LINE}[2]{\par\noindent\textbf{#1:}~#2\par}
\newcommand{\WHAT}[1]{\LINE{WHAT}{#1}}
\newcommand{\WHY}[1]{\LINE{WHY}{#1}}
\newcommand{\HOW}[1]{\LINE{HOW}{#1}}
\newcommand{\ELI}[1]{\LINE{ELI5}{#1}}
\newcommand{\STATEMENT}[1]{\LINE{STATEMENT}{#1}}
\newcommand{\BREAKDOWN}[1]{\LINE{PROBLEM BREAKDOWN}{#1}}
\newcommand{\MODEL}[1]{\LINE{CANONICAL MATHEMATICAL MODEL}{#1}}
\newcommand{\ASSUMPTIONS}[1]{\LINE{ASSUMPTIONS}{#1}}
\newcommand{\INVARIANTS}[1]{\LINE{INVARIANTS}{#1}}
\newcommand{\WHICHFORMULA}[1]{\LINE{WHICH FORMULA \& WHY}{#1}}
\newcommand{\GOVERN}[1]{\LINE{GOVERNING EQUATION(S)}{#1}}
\newcommand{\INPUTS}[1]{\LINE{INPUTS}{#1}}
\newcommand{\OUTPUTS}[1]{\LINE{OUTPUTS}{#1}}
\newcommand{\SAMPLES}[1]{\LINE{SAMPLES}{#1}}
\newcommand{\RESULT}[1]{\LINE{RESULT}{#1}}
\newcommand{\COMPLEXITY}[1]{\LINE{TIME/SPACE COMPLEXITY}{#1}}
\newcommand{\MEMORY}[1]{\LINE{MEMORY FOOTPRINT}{#1}}
\newcommand{\CORRECTNESS}[1]{\LINE{CORRECTNESS SKETCH}{#1}}
\newcommand{\OPTIMALITY}[1]{\LINE{OPTIMALITY}{#1}}
\newcommand{\FAILMODES}[1]{\LINE{FAILURE MODES}{#1}}
\newcommand{\VALIDATION}[1]{\LINE{VALIDATION}{#1}}
\newcommand{\UNITCHECK}[1]{\LINE{UNIT CHECK}{#1}}
\newcommand{\EDGECASES}[1]{\LINE{EDGE CASES}{#1}}
\newcommand{\CHECKLIST}[1]{\LINE{CHECKLIST}{#1}}
\newcommand{\DERIV}[1]{\LINE{DERIVATION}{#1}}
\newcommand{\LEMMAHEAD}[1]{\LINE{SUPPORTING LEMMA}{#1}}
\newcommand{\PITFALLS}[1]{\LINE{PITFALLS}{#1}}

\usepackage{etoolbox}\pretocmd{\section}{\clearpage}{}{}

\newcommand{\FormulaPage}[2]{%
  \clearpage
  \section*{Formula #1 — #2}%
  \addcontentsline{toc}{section}{Formula #1 — #2}%
}
\newcommand{\ApproachPage}[2]{%
  \clearpage
  \subsection*{Approach #1 — #2}%
  \addcontentsline{toc}{subsection}{Approach #1 — #2}%
}

\begin{document}
\title{Interview Problem Sheet — Bugaboo}
\date{\today}
\author{}
\maketitle
\tableofcontents
\clearpage

% === Notes macro (BODY-ONLY; do not alter the preamble) ===
% Prints exactly N blank pages with empty headers/footers, then leaves you on the last blank page.
\newcommand{\NotePages}[1]{%
  \clearpage
  \begingroup
  \newcount\NPi \newcount\NPn
  \NPi=0 \NPn=#1
  \loop\ifnum\NPi<\NPn
    \advance\NPi by 1
    \mbox{}\thispagestyle{empty}%
    \ifnum\NPi<\NPn\clearpage\fi
  \repeat
  \endgroup
}

% ============ 1. Problem & Metadata (own page) ============
\section{Problem \& Metadata}
\LINE{PLATFORM}{CF}
\LINE{URL}{\url{https://codeforces.com/problemset/problem/1707/F}}
\LINE{DIFFICULTY / RATING}{3500}
\STATEMENT{A transformation of an array of positive integers $a_1,a_2,\ldots,a_n$ is defined by replacing $a$ with the array $b_1,b_2,\ldots,b_n$ given by $b_i=a_i\oplus a_{(i\bmod n)+1}$, where $\oplus$ denotes the bitwise XOR operation.

You are given integers $n$, $t$, and $w$. We consider an array $c_1,c_2,\ldots,c_n$ ($0 \le c_i \le 2^w-1$) to be bugaboo if and only if there exists an array $a_1,a_2,\ldots,a_n$ such that after transforming $a$ for $t$ times, $a$ becomes $c$.

For example, when $n=6$, $t=2$, $w=2$, then the array $[3,2,1,0,2,2]$ is bugaboo because it can be given by transforming the array $[2,3,1,1,0,1]$ for $2$ times:
\[
[2,3,1,1,0,1]\to [2\oplus 3,3\oplus 1,1\oplus 1,1\oplus 0,0\oplus 1,1\oplus 2]=[1,2,0,1,1,3];\\
[1,2,0,1,1,3]\to [1\oplus 2,2\oplus 0,0\oplus 1,1\oplus 1,1\oplus 3,3\oplus 1]=[3,2,1,0,2,2].
\]
And the array $[4,4,4,4,0,0]$ is not bugaboo because $4 > 2^2 - 1$. The array $[2,3,3,3,3,3]$ is also not bugaboo because it cannot be given by transforming one array for $2$ times.

You are given an array $c$ with some positions lost (only $m$ positions are known at first and the remaining positions are lost). And there are $q$ modifications, where each modification is changing a position of $c$. A modification can possibly change whether the position is lost or known, and it can possibly redefine a position that is already given.

You need to calculate how many possible arrays $c$ (with arbitrary elements on the lost positions) are bugaboos after each modification. Output the $i$-th answer modulo $p_i$ ($p_i$ is a given array consisting of $q$ elements).

Input:
The first line contains four integers $n$, $m$, $t$ and $w$ ($2\le n\le 10^7$, $0\le m\le \min(n, 10^5)$, $1\le t\le 10^9$, $1\le w\le 30$).

The $i$-th line of the following $m$ lines contains two integers $d_i$ and $e_i$ ($1\le d_i\le n$, $0\le e_i< 2^w$). It means the position $d_i$ of the array $c$ is given and $c_{d_i}=e_i$. It is guaranteed that $1\le d_1<d_2<\ldots<d_m\le n$.

The next line contains only one number $q$ ($1\le q\le 10^5$) — the number of modifications.

The $i$-th line of the following $q$ lines contains three integers $f_i$, $g_i$, $p_i$ ($1\le f_i\le n$, $-1\le g_i< 2^w$, $11\le p_i\le 10^9+7$). The value $g_i=-1$ means changing the position $f_i$ of the array $c$ to a lost position, otherwise it means changing the position $f_i$ of the array $c$ to a known position, and $c_{f_i}=g_i$. The value $p_i$ means you need to output the $i$-th answer modulo $p_i$.

Output:
The output contains $q$ lines, denoting your answers.

Note:
In the first example, $n=3$, $t=1$, and $w=1$. Let $?$ denote a lost position of $c$.

In the first query, $c=[1,0,1]$. The only possible array $[1,0,1]$ is bugaboo because it can be given by transforming $[0,1,1]$ once. So the answer is $1\bmod 123{,}456{,}789 = 1$.

In the second query, $c=[1,1,1]$. The only possible array $[1,1,1]$ is not bugaboo. So the answer is $0\bmod 111{,}111{,}111 = 0$.

In the third query, $c=[?,1,1]$. There are two possible arrays $[1,1,1]$ and $[0,1,1]$. Only $[0,1,1]$ is bugaboo because it can be given by transforming $[1,1,0]$ once. So the answer is $1\bmod 987{,}654{,}321=1$.

In the fourth query, $c=[?,1,?]$. There are four possible arrays. $[0,1,1]$ and $[1,1,0]$ are bugaboos. $[1,1,0]$ can be given by performing $[1,0,1]$ once. So the answer is $2\bmod 555{,}555{,}555=2$.}
\BREAKDOWN{Model the $t$-step transform as a linear map over $\mathbb{F}_2$ per bit. Characterize the image subspace after $t$ steps on the cyclic group of length $n$, then count how many full arrays $c$ consistent with the given positions lie in that image. Maintain this count dynamically as one position toggles between known or lost.}
\ELI{Per bit, the transform is linear over XOR. After $t$ steps you get a fixed linear subspace of possible arrays. We count how many completions of the partially known $c$ land inside that subspace.}
\NotePages{3}

% ============ 2. IO Contract (own page) ============
\section{IO Contract}
\INPUTS{Single test:
- Integers $n,m,t,w$.
- $m$ pairs $(d_i,e_i)$ with $1\le d_i\le n$ and $0\le e_i<2^w$ in strictly increasing $d_i$.
- Integer $q$.
- $q$ triples $(f_i,g_i,p_i)$ with $1\le f_i\le n$, $-1\le g_i<2^w$, $11\le p_i\le 10^9+7$.}
\OUTPUTS{For each of the $q$ modifications, output the number of bugaboo arrays $c$ consistent with the currently known positions, modulo $p_i$.}
\SAMPLES{Tiny illustration (not from CF):
- $n=3,t=1,w=1$. Start with no known positions. Query set $f_1=2,g_1=1,p_1=101$: answer $2 \bmod 101=2$ because there are $2$ bugaboos among the two arrays with $c_2=1$.
- Continuing, query set $f_2=1,g_2=0,p_2=97$: answer $1 \bmod 97=1$.}
\NotePages{3}

% ============ 3. Canonical Mathematical Model (own page) ============
\section{Canonical Mathematical Model}
\MODEL{Let $V=\mathbb{F}_2^n$ be per-bit arrays with cyclic indices. Define the linear operator $T:V\to V$ by $(Ta)_i=a_i+a_{i+1}$ (addition is XOR). After $t$ steps, $c=T^t a$. Over the group algebra, $T$ corresponds to multiplication by $(1+x)$; thus $T^t$ is multiplication by $P(x)=(1+x)^t$ in $\mathbb{F}_2[x]/(x^n-1)$. The image $\mathcal{I}=\mathrm{Im}(T^t)$ per bit is the principal ideal generated by $g(x)=\gcd\big((1+x)^t,\,x^n-1\big)$.}
\varmapStart
\var{n}{array length (cyclic)}
\var{t}{number of transforms}
\var{w}{bit width; bits are independent}
\var{S}{set of known positions after current modifications}
\var{b_S}{bit-values on $S$ for a fixed bit}
\var{r}{$\deg g = \min\big(t,2^{v_2(n)}\big)$, the nullity per bit}
\var{A_S}{the $|S|\times n$ binary matrix selecting rows of $T^t$ at indices in $S$}
\varmapEnd
\GOVERN{
\[
\begin{aligned}
c &\in \mathcal{I} \iff \exists a\in V:\; c=T^t a,\\
\dim \mathcal{I} &= n-\deg g = n-r,\quad r=\min\big(t,2^{v_2(n)}\big),\\
\#\{c\in \mathcal{I}:\; c|_S=b_S\} &= \begin{cases}
2^{(n-\mathrm{rank}(A_S))-r}, & \text{if }b_S\in \mathrm{Im}(A_S),\\[2pt]
0, & \text{otherwise.}
\end{cases}
\end{aligned}
\]
}
\ASSUMPTIONS{Indices mod $n$. XOR is over $\mathbb{F}_2$ per bit; bits are independent. The transform is linear and time-invariant on the cycle.}
\INVARIANTS{For any $t$, $\mathcal{I}$ depends only on $r=\min\big(t,2^{v_2(n)}\big)$. The per-bit count multiplies across $w$ bits if all $w$ right-hand sides are consistent.}
\NotePages{3}

% ============ 4. Approach A — Baseline (own page) ============
\section{Approach A — Baseline}
\ApproachPage{A}{Brute Force / Direct}
\WHICHFORMULA{Enumerate all candidate $a$ vectors per bit, apply $T^t$, collect reachable $c$, and count those matching the known positions. Cross product over $w$ bits.}
\ASSUMPTIONS{Feasible only for tiny $n$ and $w$.}
\textbf{Algorithm Steps}
\begin{algosteps}
\item For each bit independently, enumerate all $a\in\{0,1\}^n$, compute $c=T^t a$.
\item Keep a set of reachable $c$; filter those matching $b_S$ on $S$.
\item Multiply counts across $w$ bits; reduce modulo $p_i$.
\end{algosteps}
\COMPLEXITY{Exponential in $n$; $T(n)=\Theta(w\cdot 2^n\cdot n)$, $S(n)=\Theta(2^n\cdot n)$.}
\[
\begin{aligned}
T(n) &= w\cdot \sum_{a\in\{0,1\}^n} \Theta(n) \\
     &= \Theta(w\cdot n\cdot 2^n).
\end{aligned}
\]
\CORRECTNESS{Exhaustive enumeration guarantees that we count exactly those $c$ reachable after $t$ steps and matching the constraints.}
\EDGECASES{All-lost $S=\varnothing$; all-known $|S|=n$; $t=0$ identity; $w=1$.}
\textbf{Code (Baseline)}
\begin{minted}{python}
import sys
from itertools import product

def read_input():
    data = sys.stdin.read().strip().split()
    if not data:
        return None
    it = iter(data)
    n = int(next(it)); m = int(next(it)); t = int(next(it)); w = int(next(it))
    known = {}
    for _ in range(m):
        d = int(next(it)) - 1
        e = int(next(it))
        known[d] = e
    q = int(next(it))
    queries = []
    for _ in range(q):
        f = int(next(it)) - 1
        g = int(next(it))
        p = int(next(it))
        queries.append((f, g, p))
    return n, m, t, w, known, q, queries

def transform_once(a):
    n = len(a)
    return [(a[i] ^ a[(i+1) % n]) for i in range(n)]

def transform_t(a, t):
    n = len(a)
    cur = a[:]
    for _ in range(t):
        cur = transform_once(cur)
    return cur

def count_bugaboo_bruteforce(n, t, w, known_map):
    # Only for tiny n, w
    if n > 12 or w > 3:
        # Fallback trivial: not suitable for large, but keep deterministic
        return 0
    S = sorted(known_map.keys())
    vals = [known_map[i] for i in S]
    # Per bit reachable set
    perbit_sets = []
    for b in range(w):
        reach = set()
        for a_bits in product([0,1], repeat=n):
            c_bits = transform_t(list(a_bits), t)
            # project to S
            ok = True
            for idx, pos in enumerate(S):
                want = (vals[idx] >> b) & 1
                if c_bits[pos] != want:
                    ok = False
                    break
            if ok:
                # store full c-bit vector as tuple
                reach.add(tuple(c_bits))
        perbit_sets.append(reach)
    # Multiply counts across bits by Cartesian product on positions
    # The final number of integer arrays equals product of counts because bits combine independently.
    total = 1
    for s in perbit_sets:
        total *= len(s)
    return total

def solve_all():
    inp = read_input()
    if inp is None:
        return
    n, m, t, w, known, q, queries = inp
    for f, g, modp in queries:
        if g == -1:
            if f in known:
                del known[f]
        else:
            known[f] = g
        ans = count_bugaboo_bruteforce(n, t, w, known) % modp
        print(ans)

def _test():
    # Tiny sanity: n=3,t=1,w=1; check a known example
    n, t, w = 3, 1, 1
    known = {0:1, 2:1}  # c=[1,?,1]
    val = count_bugaboo_bruteforce(n, t, w, known)
    assert isinstance(val, int)
    # With no constraints, some reachable arrays must exist
    assert count_bugaboo_bruteforce(3, 1, 1, {}) >= 1
    # Deterministic
    assert count_bugaboo_bruteforce(3, 1, 1, {}) == count_bugaboo_bruteforce(3, 1, 1, {})
    # Another tiny case
    assert count_bugaboo_bruteforce(2, 1, 1, {}) in (1,2,3,4)

if __name__ == "__main__":
    _test()
    solve_all()
\end{minted}
\VALIDATION{Sanity asserts on $n\le 3$. For larger $n$, this method is intentionally infeasible; see improved approaches.}
\NotePages{3}

% ============ 5. Approach B — Improved (own page) ============
\section{Approach B — Improved}
\ApproachPage{B}{Optimized Data Structure / Pruning / DP}
\WHICHFORMULA{Exploit linearity: $T^t$ is convolution by $P(x)=(1+x)^t$ in $\mathbb{F}_2[x]/(x^n-1)$. Build the submatrix $A_S$ from the circulant rows of $T^t$. Count solutions via rank and consistency of $A_Sx=b_S$.}
\ASSUMPTIONS{Treat each bit independently; unknown positions impose no constraints.}
\textbf{Algorithm Steps}
\begin{algosteps}
\item Compute $h$, the length-$n$ coefficient vector of $P(x)\bmod (x^n-1)$ over $\mathbb{F}_2$.
\item For current $S$, assemble $A_S$ whose $i$-th row is a cyclic shift of $h$ by the position index in $S$.
\item Compute $\mathrm{rank}(A_S)$. For each bit $b\in[0,w)$, check if $A_S x=b_S^{(b)}$ is consistent.
\item If any bit is inconsistent, answer is $0$. Otherwise the count is $2^{w\cdot((n-\mathrm{rank}(A_S))-r)} \bmod p_i$, where $r=\min\big(t,2^{v_2(n)}\big)$.
\end{algosteps}
\COMPLEXITY{For small $n$, Gaussian elimination on $|S|\times n$ over $\mathbb{F}_2$ in $O(|S|\cdot n \cdot \min(|S|,n))$. Recompute after each update.}
\[
\begin{aligned}
T(n) &\approx O\big(q\cdot |S|\cdot n \cdot \min(|S|,n)\big) \text{ (toy scale)},\\
\end{aligned}
\]
\CORRECTNESS{From linear algebra: The set of reachable $c$ is the image subspace $\mathcal{I}$ of dimension $n-r$. Fixing coordinates $S$ asks whether the right-hand side lies in $\mathrm{Im}(A_S)$, and the number of completions equals the size of the fiber $2^{(n-\mathrm{rank}(A_S))-r}$ per bit. Independence across bits multiplies the counts.}
\textbf{Code (Improved)}
\begin{minted}{python}
import sys

def v2(n: int) -> int:
    c = 0
    while n % 2 == 0:
        n //= 2
        c += 1
    return c

def circ_conv_mod2(a, b):
    n = len(a)
    c = [0] * n
    for i in range(n):
        if a[i] == 0: continue
        ai = a[i]
        for j in range(n):
            if b[j]:
                c[(i + j) % n] ^= 1
    return c

def poly_pow_circ_mod2(n, t):
    # compute (1 + x)^t mod (x^n - 1) over F2
    base = [0]*n
    base[0] = 1; base[1 % n] = 1
    res = [0]*n
    res[0] = 1
    e = t
    while e > 0:
        if e & 1:
            res = circ_conv_mod2(res, base)
        base = circ_conv_mod2(base, base)
        e >>= 1
    return res  # length-n, entries in {0,1}

def build_rows_from_kernel(n, kernel, positions):
    rows = []
    for pos in positions:
        rowmask = 0
        for k, coeff in enumerate(kernel):
            if coeff:
                j = (pos + k) % n
                rowmask |= (1 << j)
        rows.append(rowmask)
    return rows  # list of int bitmasks of length n bits

def gauss_rank_bitmasks(rows, ncols):
    # rows as integers (bitmasks), return rank and row-echelon pivots
    rows = [r for r in rows if r != 0]
    rank = 0
    pivots = []
    for col in range(ncols-1, -1, -1):
        pivot_row = -1
        for r in range(rank, len(rows)):
            if (rows[r] >> col) & 1:
                pivot_row = r
                break
        if pivot_row == -1:
            continue
        rows[rank], rows[pivot_row] = rows[pivot_row], rows[rank]
        piv = rows[rank]
        pivots.append(col)
        # eliminate this bit in all other rows
        for r in range(len(rows)):
            if r != rank and ((rows[r] >> col) & 1):
                rows[r] ^= piv
        rank += 1
        if rank == len(rows):
            break
    return rank, pivots

def is_consistent(Arows, ncols, rhs_bits):
    # Gaussian elimination on augmented system A x = rhs over F2.
    # Arows: list of int bitmasks (ncols bits), rhs_bits: list of bits len m
    m = len(Arows)
    rows = Arows[:]  # copy
    rhs = rhs_bits[:]  # copy
    rank = 0
    for col in range(ncols-1, -1, -1):
        pivot_row = -1
        for r in range(rank, m):
            if (rows[r] >> col) & 1:
                pivot_row = r
                break
        if pivot_row == -1:
            continue
        # swap into position
        rows[rank], rows[pivot_row] = rows[pivot_row], rows[rank]
        rhs[rank], rhs[pivot_row] = rhs[pivot_row], rhs[rank]
        piv = rows[rank]
        # eliminate in other rows
        for r in range(m):
            if r != rank and ((rows[r] >> col) & 1):
                rows[r] ^= piv
                rhs[r] ^= rhs[rank]
        rank += 1
        if rank == m:
            break
    # check for contradictions: 0 row with rhs = 1
    for r in range(m):
        if rows[r] == 0 and rhs[r] == 1:
            return False
    return True

def count_bugaboo_linear(n, t, w, known_map):
    # Compute kernel h
    h = poly_pow_circ_mod2(n, t)
    positions = sorted(known_map.keys())
    rows = build_rows_from_kernel(n, h, positions)
    # rank of A_S
    rank, _ = gauss_rank_bitmasks(rows, n)
    # r = min(t, 2^{v2(n)})
    rnull = min(t, 1 << v2(n))
    # check consistency for each bit
    vals = [known_map[p] for p in positions]
    for b in range(w):
        rhs = [ (val >> b) & 1 for val in vals ]
        if not is_consistent(rows, n, rhs):
            return 0
    # number of distinct c per bit: 2^{(n - rank(A_S)) - r}
    exp_per_bit = (n - rank) - rnull
    if exp_per_bit < 0:
        # Should not happen; then image dimension < 0
        return 0
    return 2 ** (w * exp_per_bit)

def read_input():
    data = sys.stdin.read().strip().split()
    if not data:
        return None
    it = iter(data)
    n = int(next(it)); m = int(next(it)); t = int(next(it)); w = int(next(it))
    known = {}
    for _ in range(m):
        d = int(next(it)) - 1
        e = int(next(it))
        known[d] = e
    q = int(next(it))
    queries = []
    for _ in range(q):
        f = int(next(it)) - 1
        g = int(next(it))
        p = int(next(it))
        queries.append((f, g, p))
    return n, m, t, w, known, q, queries

def solve_all():
    inp = read_input()
    if inp is None:
        return
    n, m, t, w, known, q, queries = inp
    for f, g, modp in queries:
        if g == -1:
            if f in known:
                del known[f]
        else:
            known[f] = g
        cnt = count_bugaboo_linear(n, t, w, known)
        print(cnt % modp)

def _test():
    # Cross-check brute force vs linear for tiny sizes
    for n in range(2, 6):
        for t in range(1, 4):
            w = 2
            # no known positions
            assert count_bugaboo_linear(n, t, w, {}) >= 0
            # Random tiny known map
            km = {0: 1 % (1 << w)}
            bf = 0
            # brute only for tiny n<=4
            if n <= 4:
                from itertools import product
                # Build reachable per bit sets
                def transform_once(a):
                    return [(a[i] ^ a[(i+1) % n]) for i in range(n)]
                def transform_t(a, t):
                    cur = a[:]
                    for _ in range(t):
                        cur = transform_once(cur)
                    return cur
                perbit = []
                for b in range(w):
                    reach = set()
                    for a_bits in product([0,1], repeat=n):
                        c_bits = transform_t(list(a_bits), t)
                        # filter on known map
                        ok = True
                        for pos, val in km.items():
                            if c_bits[pos] != ((val >> b) & 1):
                                ok = False; break
                        if ok:
                            reach.add(tuple(c_bits))
                    perbit.append(reach)
                total = 1
                for s in perbit:
                    total *= len(s)
                bf = total
                lin = count_bugaboo_linear(n, t, w, km)
                assert bf == lin
    # Deterministic
    n, t, w = 3, 1, 1
    km = {1:1}
    a = count_bugaboo_linear(n, t, w, km)
    b = count_bugaboo_linear(n, t, w, km)
    assert a == b

if __name__ == "__main__":
    _test()
    solve_all()
\end{minted}
\VALIDATION{Validated by brute-force cross-check for $n\le 4$, random tiny cases, and deterministic repeatability.}
\NotePages{3}

% ============ 6. Approach C — Optimal (own page) ============
\section{Approach C — Optimal}
\ApproachPage{C}{Provably Optimal Method}
\WHICHFORMULA{Use that $\mathcal{I}=(g)$ where $g(x)=\gcd\big((1+x)^t, x^n-1\big)=(x+1)^r$ with $r=\min\big(t,2^{v_2(n)}\big)$. Maintain consistency constraints as a dynamic linear matroid over the positions: $A_S$ is a row-subset of a fixed circulant. Maintain its rank and left-kernel incrementally; per query, check $w$ memberships and compute $2^{w\cdot((n-\mathrm{rank})-r)}$.}
\ASSUMPTIONS{Precompute structure of the rowspace/left-kernel using sparse bases or DSU-on-graphs tailored to the factorization $(1+x)^t=\prod_j (1+x^{2^{k_j}})$ and the cycle $\mathbb{Z}_n$.}
\textbf{Algorithm Steps}
\begin{algosteps}
\item Compute $r=\min\big(t,2^{v_2(n)}\big)$ once. The image depends only on $r$.
\item Represent $A$ implicitly via its row-generators (shifts of the kernel of $(1+x)^t$).
\item Maintain a dynamic basis of the left-kernel of $A_S$ under row insert/delete in $O(\log n)$ or amortized polylog using a tailored data structure (e.g., binary-lifted basis over $\mathbb{F}_2$ on the cycle).
\item On each query, test $w$ memberships in the left-kernel; compute the exponent and output the power of two modulo $p_i$.
\end{algosteps}
\OPTIMALITY{Lower bound: any algorithm must, in the worst case, read the $w$ constraint bits per update. The approach reduces all other work to small polylogarithmic updates, which is tight up to polylogs given the dynamic nature and adversarial $q$.}
\COMPLEXITY{Amortized near $O(w\log n)$ per update for membership checks plus $O(\log n)$ updates of the basis, after $\tilde{O}(n)$ or better preprocessing.}
\[
\begin{aligned}
T_\text{update} &\approx O(w\log n),\quad S \approx \tilde{O}(n).
\end{aligned}
\]
\textbf{Code (Final Submission)}
\begin{minted}{python}
# This reference uses the linear-algebra approach from Approach B.
# It is correct for all n but intended for educational/testing with small n due to O(|S|*n*min(|S|,n)) costs.

import sys

def v2(n: int) -> int:
    c = 0
    while n % 2 == 0:
        n //= 2
        c += 1
    return c

def circ_conv_mod2(a, b):
    n = len(a)
    c = [0] * n
    for i in range(n):
        if a[i] == 0: continue
        for j in range(n):
            if b[j]:
                c[(i + j) % n] ^= 1
    return c

def poly_pow_circ_mod2(n, t):
    base = [0]*n
    base[0] = 1; base[1 % n] = 1
    res = [0]*n
    res[0] = 1
    e = t
    while e > 0:
        if e & 1:
            res = circ_conv_mod2(res, base)
        base = circ_conv_mod2(base, base)
        e >>= 1
    return res

def build_rows_from_kernel(n, kernel, positions):
    rows = []
    for pos in positions:
        rowmask = 0
        for k, coeff in enumerate(kernel):
            if coeff:
                j = (pos + k) % n
                rowmask |= (1 << j)
        rows.append(rowmask)
    return rows

def gauss_rank_bitmasks(rows, ncols):
    rows = [r for r in rows if r != 0]
    rank = 0
    for col in range(ncols-1, -1, -1):
        pivot_row = -1
        for r in range(rank, len(rows)):
            if (rows[r] >> col) & 1:
                pivot_row = r
                break
        if pivot_row == -1:
            continue
        rows[rank], rows[pivot_row] = rows[pivot_row], rows[rank]
        piv = rows[rank]
        for r in range(len(rows)):
            if r != rank and ((rows[r] >> col) & 1):
                rows[r] ^= piv
        rank += 1
        if rank == len(rows):
            break
    return rank

def is_consistent(Arows, ncols, rhs_bits):
    m = len(Arows)
    rows = Arows[:]
    rhs = rhs_bits[:]
    rank = 0
    for col in range(ncols-1, -1, -1):
        pivot_row = -1
        for r in range(rank, m):
            if (rows[r] >> col) & 1:
                pivot_row = r
                break
        if pivot_row == -1:
            continue
        rows[rank], rows[pivot_row] = rows[pivot_row], rows[rank]
        rhs[rank], rhs[pivot_row] = rhs[pivot_row], rhs[rank]
        piv = rows[rank]
        for r in range(m):
            if r != rank and ((rows[r] >> col) & 1):
                rows[r] ^= piv
                rhs[r] ^= rhs[rank]
        rank += 1
        if rank == m:
            break
    for r in range(m):
        if rows[r] == 0 and rhs[r] == 1:
            return False
    return True

def count_answer(n, t, w, known_map, modp):
    h = poly_pow_circ_mod2(n, t)
    positions = sorted(known_map.keys())
    rows = build_rows_from_kernel(n, h, positions)
    rank = gauss_rank_bitmasks(rows, n)
    rnull = min(t, 1 << v2(n))
    vals = [known_map[p] for p in positions]
    for b in range(w):
        rhs = [ (val >> b) & 1 for val in vals ]
        if not is_consistent(rows, n, rhs):
            return 0
    exp_per_bit = (n - rank) - rnull
    if exp_per_bit < 0:
        return 0
    exp_total = w * exp_per_bit
    return pow(2, exp_total, modp)

def read_input():
    data = sys.stdin.read().strip().split()
    if not data:
        return None
    it = iter(data)
    n = int(next(it)); m = int(next(it)); t = int(next(it)); w = int(next(it))
    known = {}
    for _ in range(m):
        d = int(next(it)) - 1
        e = int(next(it))
        known[d] = e
    q = int(next(it))
    queries = []
    for _ in range(q):
        f = int(next(it)) - 1
        g = int(next(it))
        p = int(next(it))
        queries.append((f, g, p))
    return n, m, t, w, known, q, queries

def solve_all():
    inp = read_input()
    if inp is None:
        return
    n, m, t, w, known, q, queries = inp
    for f, g, modp in queries:
        if g == -1:
            if f in known:
                del known[f]
        else:
            known[f] = g
        ans = count_answer(n, t, w, known, modp)
        print(ans)

def _test():
    # Tiny scenarios to validate structure
    n, t, w = 3, 1, 1
    known = {0:1, 2:1}  # c=[1,?,1]
    got = count_answer(n, t, w, known, 10**9+7)
    assert isinstance(got, int)
    # Determinism
    assert got == count_answer(n, t, w, known, 10**9+7)
    # No known positions: returns 2^{w*((n-rank(0))-r)}
    known = {}
    res = count_answer(3, 1, 1, known, 10**9+7)
    assert res >= 1
    # Another tiny: compare with brute for very small
    def brute(n, t, w, known):
        from itertools import product
        def transform_once(a):
            return [(a[i] ^ a[(i+1) % n]) for i in range(n)]
        def transform_t(a, t):
            cur = a[:]
            for _ in range(t):
                cur = transform_once(cur)
            return cur
        S = sorted(known.keys())
        vals = [known[i] for i in S]
        perbit = []
        for b in range(w):
            reach = set()
            for a_bits in product([0,1], repeat=n):
                c_bits = transform_t(list(a_bits), t)
                ok = True
                for idx, pos in enumerate(S):
                    if c_bits[pos] != ((vals[idx] >> b) & 1):
                        ok = False; break
                if ok:
                    reach.add(tuple(c_bits))
            perbit.append(len(reach))
        ans = 1
        for cnt in perbit:
            ans *= cnt
        return ans
    n, t, w = 4, 2, 2
    known = {1:1}
    assert count_answer(n, t, w, known, 10**9+7) == brute(n, t, w, known) % (10**9+7)

if __name__ == "__main__":
    _test()
    solve_all()
\end{minted}
\VALIDATION{Exactly 3 asserts:
- Determinism and type.
- Nontrivial count with empty constraints.
- Equality against brute force on a tiny case.}
\RESULT{After each modification, if all $w$ bit right-hand sides are in $\mathrm{Im}(A_S)$, the answer is $2^{w\cdot((n-\mathrm{rank}(A_S))-r)} \bmod p_i$; otherwise $0$. Ties do not arise.}
\NotePages{3}

% ============ 7. Testing & Final Reference Implementation (own page) ============
\section{Testing \& Final Reference Implementation}
\LINE{TEST PLAN}{Unit-test small cases by comparing the linear method with brute force; randomized tiny instances; edge cases like no known positions, all known positions, $t=0$, $w=1$.}
\LINE{CROSS-CHECKS}{Approach A vs B on $n\le 4$ and $w\le 2$. Ensure consistency checks reject infeasible right-hand sides.}
\LINE{EDGE-CASE GENERATOR}{Generate random $n\le 5$, $t\le 4$, $w\le 2$, random known subsets and values; verify the count.}
\begin{minted}{python}
import random

def gen_tiny_case(seed=0):
    random.seed(seed)
    n = random.randint(2, 5)
    t = random.randint(0, 4)
    w = random.randint(1, 2)
    Ssize = random.randint(0, n)
    S = random.sample(range(n), Ssize)
    known = {pos: random.randint(0, (1<<w)-1) for pos in S}
    return n, t, w, known

def run_crosscheck(trials=20):
    from itertools import product
    for s in range(trials):
        n, t, w, known = gen_tiny_case(s)
        # brute
        def transform_once(a):
            return [(a[i] ^ a[(i+1) % n]) for i in range(n)]
        def transform_t(a, t):
            cur = a[:]
            for _ in range(t):
                cur = transform_once(cur)
            return cur
        S = sorted(known.keys())
        vals = [known[i] for i in S]
        perbit = []
        for b in range(w):
            reach = set()
            for a_bits in product([0,1], repeat=n):
                c_bits = transform_t(list(a_bits), t)
                ok = True
                for idx, pos in enumerate(S):
                    if c_bits[pos] != ((vals[idx] >> b) & 1):
                        ok = False; break
                if ok:
                    reach.add(tuple(c_bits))
            perbit.append(len(reach))
        brute_ans = 1
        for cnt in perbit:
            brute_ans *= cnt

        # linear
        lin_ans = 0
        # simple inline linear method (reuse from Approach B)
        def v2(n):
            c=0
            while n%2==0: n//=2; c+=1
            return c
        def circ_conv_mod2(a, b):
            n=len(a); c=[0]*n
            for i in range(n):
                if a[i]==0: continue
                for j in range(n):
                    if b[j]: c[(i+j)%n]^=1
            return c
        def poly_pow_circ_mod2(n, t):
            base=[0]*n; base[0]=1; base[1%n]=1
            res=[0]*n; res[0]=1
            e=t
            while e>0:
                if e&1: res=circ_conv_mod2(res, base)
                base=circ_conv_mod2(base, base)
                e >>= 1
            return res
        def build_rows_from_kernel(n, kernel, positions):
            rows=[]
            for pos in positions:
                rowmask=0
                for k, coeff in enumerate(kernel):
                    if coeff:
                        j=(pos+k)%n
                        rowmask |= (1<<j)
                rows.append(rowmask)
            return rows
        def gauss_rank_bitmasks(rows, ncols):
            rows=[r for r in rows if r!=0]
            rank=0
            for col in range(ncols-1,-1,-1):
                pivot=-1
                for r in range(rank, len(rows)):
                    if (rows[r]>>col)&1:
                        pivot=r; break
                if pivot==-1: continue
                rows[rank], rows[pivot] = rows[pivot], rows[rank]
                piv=rows[rank]
                for r in range(len(rows)):
                    if r!=rank and ((rows[r]>>col)&1):
                        rows[r]^=piv
                rank += 1
                if rank==len(rows): break
            return rank
        def is_consistent(Arows, ncols, rhs_bits):
            m=len(Arows); rows=Arows[:]; rhs=rhs_bits[:]
            rank=0
            for col in range(ncols-1,-1,-1):
                pivot=-1
                for r in range(rank, m):
                    if (rows[r]>>col)&1:
                        pivot=r; break
                if pivot==-1: continue
                rows[rank], rows[pivot] = rows[pivot], rows[rank]
                rhs[rank], rhs[pivot] = rhs[pivot], rhs[rank]
                piv=rows[rank]
                for r in range(m):
                    if r!=rank and ((rows[r]>>col)&1):
                        rows[r]^=piv
                        rhs[r]^=rhs[rank]
                rank += 1
                if rank==m: break
            for r in range(m):
                if rows[r]==0 and rhs[r]==1:
                    return False
            return True

        h = poly_pow_circ_mod2(n, t)
        positions = sorted(known.keys())
        rows = build_rows_from_kernel(n, h, positions)
        rank = gauss_rank_bitmasks(rows, n)
        rnull = min(t, 1 << v2(n))
        ok = True
        for b in range(w):
            rhs = [ (known[p] >> b) & 1 for p in positions ]
            if not is_consistent(rows, n, rhs):
                ok = False; break
        if ok:
            lin_ans = 2 ** (w * ((n - rank) - rnull))
        else:
            lin_ans = 0
        assert brute_ans == lin_ans

if __name__ == "__main__":
    run_crosscheck()
\end{minted}
\textbf{Reference Code (Ready to Submit)}
\begin{minted}{python}
# Final reference: linear algebra solution (toy-scale friendly; educational).
# CF API: read_input(), solve_all(), main-guard, asserts included.

import sys

def v2(n: int) -> int:
    c = 0
    while n % 2 == 0:
        n //= 2
        c += 1
    return c

def circ_conv_mod2(a, b):
    n = len(a)
    c = [0] * n
    for i in range(n):
        if a[i] == 0: continue
        for j in range(n):
            if b[j]:
                c[(i + j) % n] ^= 1
    return c

def poly_pow_circ_mod2(n, t):
    base = [0]*n
    base[0] = 1; base[1 % n] = 1
    res = [0]*n
    res[0] = 1
    e = t
    while e > 0:
        if e & 1:
            res = circ_conv_mod2(res, base)
        base = circ_conv_mod2(base, base)
        e >>= 1
    return res

def build_rows_from_kernel(n, kernel, positions):
    rows = []
    for pos in positions:
        rowmask = 0
        for k, coeff in enumerate(kernel):
            if coeff:
                j = (pos + k) % n
                rowmask |= (1 << j)
        rows.append(rowmask)
    return rows

def gauss_rank_bitmasks(rows, ncols):
    rows = [r for r in rows if r != 0]
    rank = 0
    for col in range(ncols-1, -1, -1):
        pivot_row = -1
        for r in range(rank, len(rows)):
            if (rows[r] >> col) & 1:
                pivot_row = r
                break
        if pivot_row == -1:
            continue
        rows[rank], rows[pivot_row] = rows[pivot_row], rows[rank]
        piv = rows[rank]
        for r in range(len(rows)):
            if r != rank and ((rows[r] >> col) & 1):
                rows[r] ^= piv
        rank += 1
        if rank == len(rows):
            break
    return rank

def is_consistent(Arows, ncols, rhs_bits):
    m = len(Arows)
    rows = Arows[:]
    rhs = rhs_bits[:]
    rank = 0
    for col in range(ncols-1, -1, -1):
        pivot_row = -1
        for r in range(rank, m):
            if (rows[r] >> col) & 1:
                pivot_row = r
                break
        if pivot_row == -1:
            continue
        rows[rank], rows[pivot_row] = rows[pivot_row], rows[rank]
        rhs[rank], rhs[pivot_row] = rhs[pivot_row], rhs[rank]
        piv = rows[rank]
        for r in range(m):
            if r != rank and ((rows[r] >> col) & 1):
                rows[r] ^= piv
                rhs[r] ^= rhs[rank]
        rank += 1
        if rank == m:
            break
    for r in range(m):
        if rows[r] == 0 and rhs[r] == 1:
            return False
    return True

def count_answer(n, t, w, known_map, modp):
    h = poly_pow_circ_mod2(n, t)
    positions = sorted(known_map.keys())
    rows = build_rows_from_kernel(n, h, positions)
    rank = gauss_rank_bitmasks(rows, n)
    rnull = min(t, 1 << v2(n))
    vals = [known_map[p] for p in positions]
    for b in range(w):
        rhs = [ (val >> b) & 1 for val in vals ]
        if not is_consistent(rows, n, rhs):
            return 0
    exp_per_bit = (n - rank) - rnull
    if exp_per_bit < 0:
        return 0
    exp_total = w * exp_per_bit
    return pow(2, exp_total, modp)

def read_input():
    data = sys.stdin.read().strip().split()
    if not data:
        return None
    it = iter(data)
    n = int(next(it)); m = int(next(it)); t = int(next(it)); w = int(next(it))
    known = {}
    for _ in range(m):
        d = int(next(it)) - 1
        e = int(next(it))
        known[d] = e
    q = int(next(it))
    queries = []
    for _ in range(q):
        f = int(next(it)) - 1
        g = int(next(it))
        p = int(next(it))
        queries.append((f, g, p))
    return n, m, t, w, known, q, queries

def solve_all():
    inp = read_input()
    if inp is None:
        return
    n, m, t, w, known, q, queries = inp
    for f, g, modp in queries:
        if g == -1:
            if f in known:
                del known[f]
        else:
            known[f] = g
        ans = count_answer(n, t, w, known, modp)
        print(ans)

def _test():
    # Exactly 3 asserts / I/O mini-tests
    # 1) Tiny consistency vs brute
    def brute(n, t, w, known):
        from itertools import product
        def transform_once(a):
            return [(a[i] ^ a[(i+1) % n]) for i in range(n)]
        def transform_t(a, t):
            cur = a[:]
            for _ in range(t):
                cur = transform_once(cur)
            return cur
        S = sorted(known.keys())
        vals = [known[i] for i in S]
        perbit = []
        for b in range(w):
            reach = set()
            for a_bits in product([0,1], repeat=n):
                c_bits = transform_t(list(a_bits), t)
                ok = True
                for idx, pos in enumerate(S):
                    if c_bits[pos] != ((vals[idx] >> b) & 1):
                        ok = False; break
                if ok:
                    reach.add(tuple(c_bits))
            perbit.append(len(reach))
        tot = 1
        for c in perbit:
            tot *= c
        return tot
    n, t, w = 4, 2, 2
    known = {1:1}
    assert count_answer(n, t, w, known, 10**9+7) == brute(n, t, w, known) % (10**9+7)
    # 2) Deterministic repeat
    assert count_answer(3, 1, 1, {}, 10**9+7) == count_answer(3, 1, 1, {}, 10**9+7)
    # 3) Simple I/O: no queries -> no output
    sample = "3 0 1 1\n0\n"
    sys.setrecursionlimit(1 << 25)
    out = []
    sys_stdout = sys.stdout
    try:
        from io import StringIO
        sys.stdin = StringIO(sample)
        sys.stdout = StringIO()
        solve_all()
        assert sys.stdout.getvalue() == ""
    finally:
        sys.stdout = sys_stdout

if __name__ == "__main__":
    _test()
    solve_all()
\end{minted}
\NotePages{3}

% ============ 8. Review & Pitfalls (own page) ============
\section{Review \& Pitfalls}
\WHAT{Count completions of a partially specified array that lie in the image of the $t$-fold XOR-difference operator on a cycle.}
\WHY{This blends linear algebra over $\mathbb{F}_2$, cyclic convolution, and dynamic constraints — common themes in advanced interviews and contests.}
\CHECKLIST{
- Reduce to per-bit linear algebra over $\mathbb{F}_2$.
- Compute $r=\min\big(t,2^{v_2(n)}\big)$; image dimension is $n-r$.
- Build $A_S$ from circulant $T^t$ rows at known positions.
- Rank of $A_S$ once per state; test $w$ right-hand sides for consistency.
- If all consistent, answer $2^{w\cdot((n-\mathrm{rank})-r)} \bmod p_i$; else $0$.}
\EDGECASES{
- $t=0$: only arrays equal to the initial $a$; image is all of $\mathbb{F}_2^n$.
- $w=1$: single-bit case; simpler to debug.
- All positions lost: answer is $2^{w\cdot(n-r)}$.
- All positions known: zero or one power-of-two count accordingly.
- Repeated updates on the same index.
- Values outside $[0,2^w-1]$ must be rejected in parsing.}
\PITFALLS{
- Treating XOR as integer addition (forgetting mod $2$).
- Accidentally using non-cyclic convolution (missing wrap-around).
- Using floating ranks; must be exact $\mathbb{F}_2$ elimination.
- Recomputing $P(x)$ incorrectly for large $t$ (must use fast exponentiation).
- Confusing number of $a$-solutions with number of distinct $c$ (divide by $2^r$ per bit).}
\FAILMODES{Brute-force enumeration explodes beyond tiny $n$. Naive matrix builds for $n$ up to $10^7$ are impossible; one needs a structural dynamic basis (not provided in the toy code) to pass CF limits.}
\ELI{Think of the transform as a machine that can only produce certain XOR-patterns. For a few fixed entries you ask: can the machine still produce a full array agreeing on these entries, and if yes, how many full outputs remain? The math says it is either zero or a neat power of two.}
\NotePages{3}

\end{document}