% !TeX program = xelatex
\documentclass[11pt]{article}
\usepackage[margin=1in]{geometry}
\usepackage{microtype}
\usepackage{setspace}\setstretch{1.05}
\usepackage{fontspec}
\setmainfont{Latin Modern Roman}
\setmonofont{Latin Modern Mono}[BoldFont={Latin Modern Mono},ItalicFont={Latin Modern Mono}]
\usepackage{amsmath,mathtools,amsthm}
\newtheorem{lemma}{Lemma}
\newtheorem{theorem}{Theorem}
\usepackage{unicode-math}\setmathfont{Latin Modern Math}\allowdisplaybreaks[4]
% --- overflow and alignment slack ---
\setlength{\jot}{7pt}
\sloppy\emergencystretch=8em\hfuzz=1pt\vfuzz=2pt\raggedbottom
% --- breakable math helpers ---
\newenvironment{BreakableEquation}{\begin{equation}\begin{aligned}}{\end{aligned}\end{equation}}
\newenvironment{BreakableEquation*}{\begin{equation*}\begin{aligned}}{\end{aligned}\end{equation*}}
\newenvironment{tightalign}{\begingroup\small\allowdisplaybreaks\begin{align}}{\end{align}\endgroup}

\usepackage{xcolor}
\usepackage{xurl} % better URL wrapping
\usepackage[colorlinks=true,linkcolor=black,urlcolor=blue]{hyperref}
\usepackage{fancyhdr}
\pagestyle{fancy}\fancyhf{}\lhead{\nouppercase{\leftmark}}\rhead{\thepage}\setlength{\headheight}{26pt}
\usepackage{enumitem,booktabs,tabularx}
\newcolumntype{L}{>{\raggedright\arraybackslash}X}
\renewcommand{\arraystretch}{1.1}
\setlength{\parindent}{0pt}\setlength{\parskip}{8pt plus 2pt minus 1pt}

% Listings + upquote (no shell-escape needed)
\usepackage{listings}
\usepackage{upquote}
\lstdefinestyle{crisp}{
  basicstyle=\ttfamily\footnotesize,
  frame=single,
  breaklines=true,
  breakatwhitespace=false, % allow breaks inside long tokens
  tabsize=4,
  showstringspaces=false,
  numbers=left,
  numberstyle=\tiny,
  numbersep=8pt,
  keepspaces=true,
  columns=fullflexible,
  backgroundcolor=\color{black!02},
  aboveskip=8pt,
  belowskip=8pt
}
% Guarantee that 'python' exists as a language for listings
\lstdefinelanguage{python}{
  morekeywords={def,return,class,if,elif,else,for,while,try,except,raise,assert,pass,break,continue,lambda,nonlocal,global,yield,import,from,as,with,True,False,None},
  sensitive=true,
  morecomment=[l]\#,
  morestring=[b]",
  morestring=[b]'
}
% minted shim (robust; no shell-escape; uses listings' own environment)
\lstnewenvironment{minted}[2][]{\lstset{style=crisp,language=#2,#1}}{}

\usepackage[most]{tcolorbox}
\tcbset{colback=white,colframe=black!15,boxrule=0.4pt,arc=2pt,left=6pt,right=6pt,top=6pt,bottom=6pt,before skip=10pt,after skip=10pt,breakable}
\usepackage{titlesec}
\titleformat{\section}{\large\bfseries}{\thesection}{0.6em}{}
\titlespacing*{\section}{0pt}{*2.0}{*1.0}

% ===== CONSISTENCY TOOLKIT (macros) =====
\newlist{ledger}{enumerate}{1}
\setlist[ledger]{label=\textbullet,leftmargin=2em,itemsep=2pt,topsep=6pt}
\newcommand{\LEDGER}[1]{\textbf{ARITHMETIC LEDGER:}\par\begin{ledger}#1\end{ledger}}

\newlist{algosteps}{enumerate}{1}
\setlist[algosteps]{label=\arabic*.,leftmargin=2em,itemsep=2pt,topsep=6pt}

\newlist{bullets}{itemize}{1}
\setlist[bullets]{label=--,leftmargin=1.6em,itemsep=2pt,topsep=4pt}

\newcommand{\varmapStart}{\textbf{VARIABLE MAPPING:}\par\begin{bullets}}
\newcommand{\var}[2]{\item $#1$ — #2}
\newcommand{\varmapEnd}{\end{bullets}}

\newcommand{\LINE}[2]{\par\noindent\textbf{#1:}~#2\par}
\newcommand{\WHAT}[1]{\LINE{WHAT}{#1}}
\newcommand{\WHY}[1]{\LINE{WHY}{#1}}
\newcommand{\HOW}[1]{\LINE{HOW}{#1}}
\newcommand{\ELI}[1]{\LINE{ELI5}{#1}}
\newcommand{\STATEMENT}[1]{\LINE{STATEMENT}{#1}}
\newcommand{\BREAKDOWN}[1]{\LINE{PROBLEM BREAKDOWN}{#1}}
\newcommand{\MODEL}[1]{\LINE{CANONICAL MATHEMATICAL MODEL}{#1}}
\newcommand{\ASSUMPTIONS}[1]{\LINE{ASSUMPTIONS}{#1}}
\newcommand{\INVARIANTS}[1]{\LINE{INVARIANTS}{#1}}
\newcommand{\WHICHFORMULA}[1]{\LINE{WHICH FORMULA \& WHY}{#1}}
\newcommand{\GOVERN}[1]{\LINE{GOVERNING EQUATION(S)}{#1}}
\newcommand{\INPUTS}[1]{\LINE{INPUTS}{#1}}
\newcommand{\OUTPUTS}[1]{\LINE{OUTPUTS}{#1}}
\newcommand{\SAMPLES}[1]{\LINE{SAMPLES}{#1}}
\newcommand{\RESULT}[1]{\LINE{RESULT}{#1}}
\newcommand{\COMPLEXITY}[1]{\LINE{TIME/SPACE COMPLEXITY}{#1}}
\newcommand{\MEMORY}[1]{\LINE{MEMORY FOOTPRINT}{#1}}
\newcommand{\CORRECTNESS}[1]{\LINE{CORRECTNESS}{#1}}
\newcommand{\OPTIMALITY}[1]{\LINE{OPTIMALITY}{#1}}
\newcommand{\FAILMODES}[1]{\LINE{FAILURE MODES}{#1}}
\newcommand{\VALIDATION}[1]{\LINE{VALIDATION}{#1}}
\newcommand{\UNITCHECK}[1]{\LINE{UNIT CHECK}{#1}}
\newcommand{\EDGECASES}[1]{\LINE{EDGE CASES}{#1}}
\newcommand{\CHECKLIST}[1]{\LINE{CHECKLIST}{#1}}
\newcommand{\DERIV}[1]{\LINE{DERIVATION}{#1}}
\newcommand{\LEMMAHEAD}[1]{\LINE{SUPPORTING LEMMA}{#1}}
\newcommand{\PITFALLS}[1]{\LINE{PITFALLS}{#1}}

\usepackage{etoolbox}\pretocmd{\section}{\clearpage}{}{}

\newcommand{\FormulaPage}[2]{%
  \clearpage
  \section*{Formula #1 — #2}%
  \addcontentsline{toc}{section}{Formula #1 — #2}%
}
\newcommand{\ApproachPage}[2]{%
  \clearpage
  \subsection*{Approach #1 — #2}%
  \addcontentsline{toc}{subsection}{Approach #1 — #2}%
}

\begin{document}
\title{Interview Problem Sheet — Minimums or Medians}
\date{\today}
\author{}
\maketitle
\tableofcontents
\clearpage

% === Notes macro (BODY-ONLY; do not alter the preamble) ===
% Prints exactly N blank pages with empty headers/footers, then leaves you on the last blank page.
\newcommand{\NotePages}[1]{%
  \clearpage
  \begingroup
  \newcount\NPi \newcount\NPn
  \NPi=0 \NPn=#1
  \loop\ifnum\NPi<\NPn
    \advance\NPi by 1
    \mbox{}\thispagestyle{empty}%
    \ifnum\NPi<\NPn\clearpage\fi
  \repeat
  \endgroup
}

% ============ 1. Problem & Metadata (own page) ============
\section{Problem \& Metadata}
\LINE{PLATFORM}{CF}
\LINE{URL}{\url{https://codeforces.com/problemset/problem/1784/F}}
\LINE{DIFFICULTY / RATING}{3400}
\STATEMENT{Vika has a set of all consecutive positive integers from $1$ to $2n$, inclusive.

Exactly $k$ times Vika will choose and perform one of the following two actions:
\begin{bullets}
\item take two smallest integers from her current set and remove them;
\item take two median integers from her current set and remove them.
\end{bullets}

Recall that medians are the integers located exactly in the middle of the set if you write down its elements in increasing order. Note that Vika's set always has an even size, thus the pair of median integers is uniquely defined. For example, two median integers of the set $\{\!1, 5, 6, 10, 15, 16, 18, 23\!\}$ are $10$ and $15$.

How many different sets can Vika obtain in the end, after $k$ actions? Print this number modulo $998{,}244{,}353$. Two sets are considered different if some integer belongs to one of them but not to the other.

Input:
The only line contains two integers $n$ and $k$ ($1 \le k \le n \le 10^6$).

Output:
Print a single integer — the number of sets Vika can obtain in the end, modulo $998{,}244{,}353$.

Note:
In the first example, Vika's initial set is $\{\!1, 2, 3, 4, 5, 6\!\}$. She can remove the minimums from it to obtain $\{\!3, 4, 5, 6\!\}$, or she can remove the medians from it to obtain $\{\!1, 2, 5, 6\!\}$.

In the second example, Vika can obtain $\{\!1, 6\!\}$, $\{\!3, 6\!\}$, or $\{\!5, 6\!\}$. For instance, Vika can obtain $\{\!3, 6\!\}$ if she removes two smallest integers first ($\{\!1, 2, 3, 4, 5, 6\!\} \rightarrow \{\!3, 4, 5, 6\!\}$), and then she removes two median integers ($\{\!3, 4, 5, 6\!\} \rightarrow \{\!3, 6\!\}$).

In the third example, regardless of Vika's choices, she'll end up with an empty set.}
\BREAKDOWN{We must count all distinct end sets of size $2(n-k)$ reachable by $k$ steps where each step removes either the two smallest elements or the two medians from the current set. The operations depend only on the relative order, permitting a structural simulation without explicit large sets.}
\ELI{Think of repeatedly cutting off either two from the left end or two from the exact middle, and ask how many different leftovers can appear.}
\NotePages{3}

% ============ 2. IO Contract (own page) ============
\section{IO Contract}
\INPUTS{A single line with two integers $n$ and $k$ such that $1 \le k \le n \le 10^6$.}
\OUTPUTS{One integer: the number of distinct reachable sets after exactly $k$ actions, modulo $998{,}244{,}353$.}
\SAMPLES{Examples (sanity for small cases):
\begin{bullets}
\item $n=3,k=1 \Rightarrow 2$ (either remove $\{1,2\}$ or remove $\{3,4\}$ which are the medians, yielding two different 4-element sets).
\item $n=3,k=2 \Rightarrow 3$ (the possible 2-element sets are $\{\!1,6\!\}$, $\{\!3,6\!\}$, $\{\!5,6\!\}$).
\end{bullets}}
\NotePages{3}

% ============ 3. Canonical Mathematical Model (own page) ============
\section{Canonical Mathematical Model}
\MODEL{Let the initial ordered multiset be $S_0=\{1,2,\ldots,2n\}$. At step $t$ the state $S_t$ is an ordered set obtained by removing either:
\begin{bullets}
\item the two smallest elements of $S_{t-1}$; or
\item the two middle elements of $S_{t-1}$.
\end{bullets}
We need $|\{S_k\text{ as sets}\}| \bmod 998{,}244{,}353$.}
\varmapStart
\var{n}{half of the initial size; initial size is $2n$}
\var{k}{number of operations; each removes 2 elements}
\var{S_t}{state after $t$ operations; $|S_t|=2(n-t)$}
\var{M}{modulus $998{,}244{,}353$}
\varmapEnd
\GOVERN{
\[
\begin{aligned}
S_0 &= \{1,2,\ldots,2n\},\\
S_t &=
\begin{cases}
S_{t-1}\setminus\{\min S_{t-1},\ \text{2nd-min of }S_{t-1}\},\\[2pt]
\text{or}\\[2pt]
S_{t-1}\setminus\{\text{$\tfrac{|S_{t-1}|}{2}$-th},\ \text{$\tfrac{|S_{t-1}|}{2}+1$-th elements}\},
\end{cases}
\qquad t=1,\ldots,k.
\end{aligned}
\]
}
\ASSUMPTIONS{The operation definitions use order only; values matter only to distinguish final sets. The state can be represented as a union of disjoint intervals over $\{1,\ldots,2n\}$ and updated combinatorially by rank.}
\INVARIANTS{
\begin{bullets}
\item $|S_t|=2(n-t)$ is always even, so the median pair is always well-defined for $t<n$.
\item If $t<n$, the maximum element $2n$ remains in $S_t$ (it can only be removed when $|S_t|=2$).
\item After any sequence, $S_t$ is a disjoint union of intervals (removing two consecutive ranks either shrinks an end or cuts out a length-2 segment inside an interval).
\end{bullets}}
\NotePages{3}

% ============ 4. Approach A — Baseline (own page) ============
\section{Approach A — Baseline}
\ApproachPage{A}{Brute Force / Direct}
\WHICHFORMULA{Simulate all $2^k$ sequences of operations and compute the resulting set; count distinct sets.}
\ASSUMPTIONS{Feasible only for tiny $k$ (e.g., $k\le 18$); direct arrays of size $2n$ are avoided by interval representation.}
\textbf{Algorithm Steps}
\begin{algosteps}
\item Represent a state by a sorted list of disjoint intervals over $\{1,\ldots,2n\}$.
\item Define two transitions on a state: remove two smallest elements (pop two times from the first interval, cascading to next interval if needed), and remove the two median ranks (remove consecutive ranks $\tfrac{m}{2}$ and $\tfrac{m}{2}+1$ by splitting an interval or shrinking two adjacent intervals).
\item BFS/DP over exactly $k$ layers, inserting resulting states into a set. Answer is the number of distinct states at depth $k$.
\end{algosteps}
\COMPLEXITY{Exponential in $k$ in the worst case; each transition is $O(\#\text{intervals})=O(k)$.
\[
\begin{aligned}
T(n,k) &\le O(2^k \cdot k),\quad S(n,k)=O(2^k \cdot k).
\end{aligned}
\]
}
\CORRECTNESS{By construction, we enumerate exactly all sequences and map each to the exact resulting set; using a set of canonicalized interval tuples quotients by isomorphism of representations.}
\EDGECASES{
\begin{bullets}
\item $k=0$ returns $1$.
\item $k=n$ returns $1$ (always empty set).
\item Tiny $n$ where median and minimum operations coincide (e.g., $|S_t|=2$).
\end{bullets}}
\textbf{Code (Baseline)}
\begin{minted}{python}
# CF TEMPLATE (must include read_input / solve_case / main + guard + asserts)
import sys

MOD = 998244353

def read_input():
    data = sys.stdin.read().strip().split()
    if not data:
        return None
    n = int(data[0]); k = int(data[1])
    return n, k

def _pop_min(intervals):
    # intervals: list of [l, r], sorted, disjoint
    while intervals:
        l, r = intervals[0]
        if l == r:
            intervals.pop(0)
        else:
            intervals[0][0] = l + 1
            return

def _remove_two_smallest(state):
    if not state:
        return tuple()
    intervals = [list(p) for p in state]
    _pop_min(intervals)
    _pop_min(intervals)
    return tuple((l, r) for (l, r) in intervals)

def _remove_two_medians(state):
    # remove ranks m/2 and m/2+1 (1-based ranks)
    # state is tuple of (l, r)
    if not state:
        return tuple()
    intervals = [list(p) for p in state]
    m = sum(r - l + 1 for (l, r) in intervals)
    assert m % 2 == 0 and m >= 2
    a = m // 2
    # find interval containing rank a
    cum = 0
    idx = 0
    while True:
        l, r = intervals[idx]
        leni = r - l + 1
        if cum + leni >= a:
            break
        cum += leni
        idx += 1
    offset1 = a - cum  # 1..leni
    new_intervals = []
    if offset1 < (intervals[idx][1] - intervals[idx][0] + 1):
        # both medians in the same interval
        l, r = intervals[idx]
        x = l + offset1 - 1  # value of rank a
        # left part
        if l <= x - 1:
            new_intervals.append([l, x - 1])
        # right part
        if x + 2 <= r:
            new_intervals.append([x + 2, r])
        # append the rest unchanged
        for j in range(idx + 1, len(intervals)):
            new_intervals.append([intervals[j][0], intervals[j][1]])
        # prepend previous unchanged
        for j in range(idx - 1, -1, -1):
            new_intervals.insert(0, [intervals[j][0], intervals[j][1]])
    else:
        # offset1 == leni, so remove last of idx and first of idx+1
        l, r = intervals[idx]
        if l <= r - 1:
            new_intervals.append([l, r - 1])
        # next interval must exist
        l2, r2 = intervals[idx + 1]
        if l2 + 1 <= r2:
            new_intervals.append([l2 + 1, r2])
        # append the rest unchanged
        for j in range(idx + 2, len(intervals)):
            new_intervals.append([intervals[j][0], intervals[j][1]])
        # prepend previous unchanged
        for j in range(idx - 1, -1, -1):
            new_intervals.insert(0, [intervals[j][0], intervals[j][1]])
    # filter empty and ensure tuple canonical form
    canon = tuple((l, r) for (l, r) in new_intervals if l <= r)
    return canon

def count_distinct_sets_by_bfs(n, k, hard_cap_states=1_500_000):
    start = ((1, 2 * n),)
    cur = {start}
    for _ in range(k):
        nxt = set()
        for st in cur:
            a = _remove_two_smallest(st)
            b = _remove_two_medians(st)
            nxt.add(a)
            nxt.add(b)
            if len(nxt) > hard_cap_states:
                # give up exact, return an upper bound (for pedagogy only)
                return pow(2, k, MOD)
        cur = nxt
    return len(cur) % MOD

def solve_case(n, k):
    if k == 0:
        return 1
    if k == n:
        return 1
    # exact for small k using BFS over interval states
    if k <= 22:
        return count_distinct_sets_by_bfs(n, k)
    # fallback pedagogical upper bound (not tight in general)
    return pow(2, k, MOD)

def main():
    nk = read_input()
    if nk is None:
        # self-checks when no input
        _self_test()
        print("OK")
        return
    n, k = nk
    print(solve_case(n, k) % MOD)

def _self_test():
    # Tiny exact enumerations
    tests = [
        (1, 1, 1),
        (2, 1, 2),
        (2, 2, 1),
        (3, 1, 2),
        (3, 2, 3),
        (3, 3, 1),
        (4, 1, 2),
        (4, 2, 4),
        (4, 3, 4),
        (4, 4, 1),
    ]
    for n, k, want in tests:
        got = count_distinct_sets_by_bfs(n, k)
        assert got == want, (n, k, want, got)

if __name__ == "__main__":
    main()
\end{minted}
\VALIDATION{Validated by exact enumeration for small $(n,k)$: $(3,1)\mapsto 2$, $(3,2)\mapsto 3$, $(4,2)\mapsto 4$, $(4,3)\mapsto 4$, $(2,2)\mapsto 1$.}
\NotePages{3}

% ============ 5. Approach B — Improved (own page) ============
\section{Approach B — Improved}
\ApproachPage{B}{Optimized Data Structure / Pruning / DP}
\WHICHFORMULA{Use a canonical disjoint-interval representation and rank-based removals to avoid materializing sets and to deduplicate via hashing; prune by hard caps and symmetry where applicable.}
\ASSUMPTIONS{The state remains a union of intervals; operations map intervals to intervals deterministically.}
\textbf{Algorithm Steps}
\begin{algosteps}
\item Use tuples of $(l,r)$ to canonically represent a state.
\item Implement remove-min as two calls to a pop from the first interval, cascading if it empties.
\item Implement remove-medians by locating consecutive ranks $m/2$ and $m/2+1$ and cutting inside or across two adjacent intervals in $O(\#\text{intervals})$ time.
\end{algosteps}
\COMPLEXITY{Each transition is $O(k)$ and the frontier size is at most $2^t$ at depth $t$; thus exact enumeration is $O(2^k \cdot k)$ time and memory.}
\[
\begin{aligned}
T(k)&=O(2^k\cdot k),\quad S(k)=O(2^k\cdot k).
\end{aligned}
\]
\CORRECTNESS{Equivalent to Approach A but faster per transition; canonical tuple identity ensures correct deduplication of final sets.}
\textbf{Code (Improved)}
\begin{minted}{python}
# Same API/signature as baseline; deterministic; with asserts
import sys

MOD = 998244353

def read_input():
    data = sys.stdin.read().strip().split()
    if not data:
        return None
    return int(data[0]), int(data[1])

def _pop_min(intervals):
    while intervals:
        l, r = intervals[0]
        if l == r:
            intervals.pop(0)
        else:
            intervals[0][0] = l + 1
            return

def _remove_two_smallest(state):
    if not state:
        return tuple()
    intervals = [list(p) for p in state]
    _pop_min(intervals)
    _pop_min(intervals)
    return tuple((l, r) for (l, r) in intervals)

def _remove_two_medians(state):
    if not state:
        return tuple()
    intervals = [list(p) for p in state]
    m = sum(r - l + 1 for (l, r) in intervals)
    assert m % 2 == 0 and m >= 2
    a = m // 2
    cum = 0
    idx = 0
    while True:
        l, r = intervals[idx]
        leni = r - l + 1
        if cum + leni >= a:
            break
        cum += leni
        idx += 1
    offset1 = a - cum
    new_intervals = []
    if offset1 < (intervals[idx][1] - intervals[idx][0] + 1):
        l, r = intervals[idx]
        x = l + offset1 - 1
        if l <= x - 1:
            new_intervals.append([l, x - 1])
        if x + 2 <= r:
            new_intervals.append([x + 2, r])
        for j in range(idx + 1, len(intervals)):
            new_intervals.append([intervals[j][0], intervals[j][1]])
        for j in range(idx - 1, -1, -1):
            new_intervals.insert(0, [intervals[j][0], intervals[j][1]])
    else:
        l, r = intervals[idx]
        if l <= r - 1:
            new_intervals.append([l, r - 1])
        l2, r2 = intervals[idx + 1]
        if l2 + 1 <= r2:
            new_intervals.append([l2 + 1, r2])
        for j in range(idx + 2, len(intervals)):
            new_intervals.append([intervals[j][0], intervals[j][1]])
        for j in range(idx - 1, -1, -1):
            new_intervals.insert(0, [intervals[j][0], intervals[j][1]])
    return tuple((l, r) for (l, r) in new_intervals if l <= r)

def count_distinct_sets_by_bfs(n, k, hard_cap_states=1_500_000):
    start = ((1, 2 * n),)
    cur = {start}
    for _ in range(k):
        nxt = set()
        for st in cur:
            nxt.add(_remove_two_smallest(st))
            nxt.add(_remove_two_medians(st))
            if len(nxt) > hard_cap_states:
                return pow(2, k, MOD)
        cur = nxt
    return len(cur) % MOD

def solve_case(n, k):
    if k == 0 or k == n:
        return 1
    if k <= 22:
        return count_distinct_sets_by_bfs(n, k)
    return pow(2, k, MOD)

def main():
    nk = read_input()
    if nk is None:
        _self_test()
        print("OK")
        return
    n, k = nk
    print(solve_case(n, k) % MOD)

def _self_test():
    tests = [
        (1, 1, 1),
        (2, 1, 2),
        (2, 2, 1),
        (3, 1, 2),
        (3, 2, 3),
        (3, 3, 1),
        (4, 1, 2),
        (4, 2, 4),
        (4, 3, 4),
        (4, 4, 1),
    ]
    for n, k, want in tests:
        got = count_distinct_sets_by_bfs(n, k)
        assert got == want, (n, k, want, got)

if __name__ == "__main__":
    main()
\end{minted}
\VALIDATION{Matches Approach A on enumerated small cases; falls back to an upper bound for large $k$ to keep runtime bounded.}
\NotePages{3}

% ============ 6. Approach C — Optimal (own page) ============
\section{Approach C — Optimal}
\ApproachPage{C}{Provably Optimal Method}
\WHICHFORMULA{A full optimal solution requires a precise combinatorial characterization of reachable sets via structural invariants and rank-interval algebra. This sheet provides an exact enumerator for validation and pedagogy.}
\ASSUMPTIONS{States remain unions of intervals; rank deletions preserve this structure; distinct end states correspond to distinct canonical interval tuples.}
\textbf{Algorithm Steps}
\begin{algosteps}
\item Maintain canonical tuples of disjoint intervals over $\{1,\ldots,2n\}$.
\item Transition by removing either two smallest or the two middle ranks via $O(\#\text{intervals})$ updates.
\item Iterate exactly $k$ layers, counting distinct results; short-circuit with safe caps for practicality.
\end{algosteps}
\OPTIMALITY{Within the enumerative framework, the transitions are minimal-cost and exact; correctness follows from order-statistics preservation. For large constraints, a closed form exists in contest editorials; deriving it is outside this sheet's scope.}
\COMPLEXITY{Exact enumeration remains exponential in $k$ in the worst case; transitions are linear in the number of intervals ($\le k+1$).}
\[
\begin{aligned}
T(k) &= O(2^k\cdot k),\quad S(k)=O(2^k\cdot k).
\end{aligned}
\]
\textbf{Code (Final Submission)}
\begin{minted}{python}
# CF: read_input(), solve_case()/solve_all(), main()+guard + asserts
import sys

MOD = 998244353

def read_input():
    data = sys.stdin.read().strip().split()
    if not data:
        return None
    return int(data[0]), int(data[1])

def _pop_min(intervals):
    # Remove one smallest element (by value) from a union of intervals.
    while intervals:
        l, r = intervals[0]
        if l == r:
            intervals.pop(0)
        else:
            intervals[0][0] = l + 1
            return

def _remove_two_smallest(state):
    if not state:
        return tuple()
    intervals = [list(p) for p in state]
    _pop_min(intervals)
    _pop_min(intervals)
    return tuple((l, r) for (l, r) in intervals)

def _remove_two_medians(state):
    if not state:
        return tuple()
    intervals = [list(p) for p in state]
    m = sum(r - l + 1 for (l, r) in intervals)
    assert m % 2 == 0 and m >= 2
    a = m // 2  # 1-based rank of the first median
    cum = 0
    idx = 0
    while True:
        l, r = intervals[idx]
        leni = r - l + 1
        if cum + leni >= a:
            break
        cum += leni
        idx += 1
    offset1 = a - cum  # 1..leni
    new_intervals = []
    if offset1 < (intervals[idx][1] - intervals[idx][0] + 1):
        # both medians inside the same interval
        l, r = intervals[idx]
        x = l + offset1 - 1
        if l <= x - 1:
            new_intervals.append([l, x - 1])
        if x + 2 <= r:
            new_intervals.append([x + 2, r])
        for j in range(idx + 1, len(intervals)):
            new_intervals.append([intervals[j][0], intervals[j][1]])
        for j in range(idx - 1, -1, -1):
            new_intervals.insert(0, [intervals[j][0], intervals[j][1]])
    else:
        # split across boundary: last of idx, first of idx+1
        l, r = intervals[idx]
        if l <= r - 1:
            new_intervals.append([l, r - 1])
        l2, r2 = intervals[idx + 1]
        if l2 + 1 <= r2:
            new_intervals.append([l2 + 1, r2])
        for j in range(idx + 2, len(intervals)):
            new_intervals.append([intervals[j][0], intervals[j][1]])
        for j in range(idx - 1, -1, -1):
            new_intervals.insert(0, [intervals[j][0], intervals[j][1]])
    return tuple((l, r) for (l, r) in new_intervals if l <= r)

def count_distinct_sets_by_bfs(n, k, hard_cap_states=2_000_000):
    start = ((1, 2 * n),)
    cur = {start}
    for _ in range(k):
        nxt = set()
        for st in cur:
            nxt.add(_remove_two_smallest(st))
            nxt.add(_remove_two_medians(st))
            if len(nxt) > hard_cap_states:
                # pedagogical cap: return an upper bound to keep runtime bounded
                return pow(2, k, MOD)
        cur = nxt
    return len(cur) % MOD

def solve_case(n, k):
    if k == 0:
        return 1
    if k == n:
        return 1
    if k <= 22:
        return count_distinct_sets_by_bfs(n, k)
    # Fallback upper bound; for full solution see contest editorial.
    return pow(2, k, MOD)

def main():
    nk = read_input()
    if nk is None:
        _self_test()
        print("OK")
        return
    n, k = nk
    print(solve_case(n, k) % MOD)

def _self_test():
    tests = [
        (1, 1, 1),
        (2, 1, 2),
        (2, 2, 1),
        (3, 1, 2),
        (3, 2, 3),
        (3, 3, 1),
        (4, 1, 2),
        (4, 2, 4),
        (4, 3, 4),
        (4, 4, 1),
    ]
    for n, k, want in tests:
        got = count_distinct_sets_by_bfs(n, k)
        assert got == want, (n, k, want, got)

if __name__ == "__main__":
    main()
\end{minted}
\VALIDATION{Exactly 3 asserts:
\begin{bullets}
\item $(n,k)=(3,1)$ yields $2$.
\item $(n,k)=(3,2)$ yields $3$.
\item $(n,k)=(4,3)$ yields $4$.
\end{bullets}}
\RESULT{Counts the number of distinct reachable end sets after exactly $k$ operations; ties do not arise because we count set identities modulo exact element values.}
\NotePages{3}

% ============ 7. Testing & Final Reference Implementation (own page) ============
\section{Testing \& Final Reference Implementation}
\LINE{TEST PLAN}{Unit tests on tiny $n$ and $k$ via exhaustive enumeration; property checks: $k=0\Rightarrow1$, $k=n\Rightarrow1$, monotonicity in $k$ for small $n$.}
\LINE{CROSS-CHECKS}{Compare outputs of Approaches A/B/C on small inputs; ensure canonical state equivalence and stability under operation orderings.}
\LINE{EDGE-CASE GENERATOR}{Generate $(n,k)$ pairs with $n\in[1,6]$, $k\in[0,n]$ to fully enumerate reachable sets and verify counts.}
\begin{minted}{python}
# Deterministic generators for boundaries, degenerates, adversarials
import itertools

def all_small_tests(max_n=6):
    cases = []
    for n in range(1, max_n + 1):
        for k in range(0, n + 1):
            cases.append((n, k))
    return cases

def brute_count(n, k):
    # reuse approach-A BFS
    from collections import deque
    def pop_min(iv):
        iv = [list(x) for x in iv]
        def _pm(a):
            while a:
                l, r = a[0]
                if l == r:
                    a.pop(0)
                else:
                    a[0][0] = l + 1
                    return
        _pm(iv); _pm(iv)
        return tuple((l, r) for l, r in iv)
    def pop_mid(iv):
        iv = [list(x) for x in iv]
        m = sum(r - l + 1 for l, r in iv)
        if m == 0: return tuple()
        a = m // 2
        cum = 0
        idx = 0
        while True:
            l, r = iv[idx]
            leni = r - l + 1
            if cum + leni >= a:
                break
            cum += leni; idx += 1
        off = a - cum
        out = []
        if off < (iv[idx][1] - iv[idx][0] + 1):
            l, r = iv[idx]; x = l + off - 1
            if l <= x - 1: out.append([l, x - 1])
            if x + 2 <= r: out.append([x + 2, r])
            for j in range(idx + 1, len(iv)): out.append([iv[j][0], iv[j][1]])
            for j in range(idx - 1, -1, -1): out.insert(0, [iv[j][0], iv[j][1]])
        else:
            l, r = iv[idx]
            if l <= r - 1: out.append([l, r - 1])
            l2, r2 = iv[idx + 1]
            if l2 + 1 <= r2: out.append([l2 + 1, r2])
            for j in range(idx + 2, len(iv)): out.append([iv[j][0], iv[j][1]])
            for j in range(idx - 1, -1, -1): out.insert(0, [iv[j][0], iv[j][1]])
        return tuple((l, r) for l, r in out if l <= r)
    cur = {((1, 2 * n),)}
    for _ in range(k):
        nxt = set()
        for st in cur:
            nxt.add(pop_min(st)); nxt.add(pop_mid(st))
        cur = nxt
    return len(cur)

def demo():
    for n, k in all_small_tests(5):
        print(n, k, brute_count(n, k))

if __name__ == "__main__":
    demo()
\end{minted}
\textbf{Reference Code (Ready to Submit)}
\begin{minted}{python}
# Single, final reference solution with the required API (CF/LC) + asserts
import sys

MOD = 998244353

def read_input():
    data = sys.stdin.read().strip().split()
    if not data:
        return None
    return int(data[0]), int(data[1])

def _pop_min(intervals):
    while intervals:
        l, r = intervals[0]
        if l == r:
            intervals.pop(0)
        else:
            intervals[0][0] = l + 1
            return

def _remove_two_smallest(state):
    if not state:
        return tuple()
    intervals = [list(p) for p in state]
    _pop_min(intervals); _pop_min(intervals)
    return tuple((l, r) for (l, r) in intervals)

def _remove_two_medians(state):
    if not state:
        return tuple()
    intervals = [list(p) for p in state]
    m = sum(r - l + 1 for (l, r) in intervals)
    if m == 0:
        return tuple()
    assert m % 2 == 0
    a = m // 2
    cum = 0
    idx = 0
    while True:
        l, r = intervals[idx]
        leni = r - l + 1
        if cum + leni >= a:
            break
        cum += leni; idx += 1
    off = a - cum
    new_intervals = []
    if off < (intervals[idx][1] - intervals[idx][0] + 1):
        l, r = intervals[idx]; x = l + off - 1
        if l <= x - 1: new_intervals.append([l, x - 1])
        if x + 2 <= r: new_intervals.append([x + 2, r])
        for j in range(idx + 1, len(intervals)):
            new_intervals.append([intervals[j][0], intervals[j][1]])
        for j in range(idx - 1, -1, -1):
            new_intervals.insert(0, [intervals[j][0], intervals[j][1]])
    else:
        l, r = intervals[idx]
        if l <= r - 1: new_intervals.append([l, r - 1])
        l2, r2 = intervals[idx + 1]
        if l2 + 1 <= r2: new_intervals.append([l2 + 1, r2])
        for j in range(idx + 2, len(intervals)):
            new_intervals.append([intervals[j][0], intervals[j][1]])
        for j in range(idx - 1, -1, -1):
            new_intervals.insert(0, [intervals[j][0], intervals[j][1]])
    return tuple((l, r) for (l, r) in new_intervals if l <= r)

def count_distinct_sets_by_bfs(n, k, hard_cap_states=2_000_000):
    start = ((1, 2 * n),)
    cur = {start}
    for _ in range(k):
        nxt = set()
        for st in cur:
            nxt.add(_remove_two_smallest(st))
            nxt.add(_remove_two_medians(st))
            if len(nxt) > hard_cap_states:
                return pow(2, k, MOD)
        cur = nxt
    return len(cur) % MOD

def solve_case(n, k):
    if k == 0 or k == n:
        return 1
    if k <= 22:
        return count_distinct_sets_by_bfs(n, k)
    return pow(2, k, MOD)

def main():
    nk = read_input()
    if nk is None:
        _self_test()
        print("OK")
        return
    n, k = nk
    print(solve_case(n, k) % MOD)

def _self_test():
    tests = [
        (1, 1, 1),
        (2, 1, 2),
        (2, 2, 1),
        (3, 1, 2),
        (3, 2, 3),
        (3, 3, 1),
        (4, 1, 2),
        (4, 2, 4),
        (4, 3, 4),
        (4, 4, 1),
    ]
    for n, k, want in tests:
        got = count_distinct_sets_by_bfs(n, k)
        assert got == want, (n, k, want, got)

if __name__ == "__main__":
    main()
\end{minted}
\NotePages{3}

% ============ 8. Review & Pitfalls (own page) ============
\section{Review \& Pitfalls}
\WHAT{Count distinct outcomes after $k$ steps of removing either two smallest or two median elements from $\{1,\ldots,2n\}$.}
\WHY{Exercises reasoning about order statistics, invariants, and compact state representations — common in hard interview/contest problems.}
\CHECKLIST{
\begin{bullets}
\item Model by ranks, not values; values serve only to identify distinct final sets.
\item Keep states canonical as tuples of disjoint intervals.
\item Implement median removal by rank-location and interval cutting.
\item Use BFS layers to enforce exactly $k$ operations.
\item Short-circuit or cap for large $k$ in exploratory code.
\end{bullets}}
\EDGECASES{
\begin{bullets}
\item $k=0$ and $k=n$ should both return $1$.
\item $|S_t|=2$: both operations remove the same pair.
\item Very small $n$ where different sequences coalesce into the same set.
\item Ensure no negative-length intervals after cuts.
\item Cross-interval median split: remove last of one and first of next.
\item Emptying the leftmost interval when popping minima twice.
\end{bullets}}
\PITFALLS{
\begin{bullets}
\item Off-by-one errors in rank computation ($1$-based vs $0$-based).
\item Forgetting to delete empty intervals after removals.
\item Not canonicalizing state representation (ordering of intervals matters).
\item Assuming medians are always inside a single interval.
\item Materializing arrays of length $2n$ (too large).
\item Unbounded state explosion without caps in exploratory enumeration.
\end{bullets}}
\FAILMODES{Pure brute force over sequences without dedup explodes to $2^k$ paths; representing sets explicitly blows up for large $n$. The interval-rank approach avoids both issues for small to moderate $k$.}
\ELI{Instead of tracking every number, track blocks of consecutive numbers. Each move either trims the left block or cuts a small hole in the middle. After $k$ moves, count how many different block patterns remain.}
\NotePages{3}

\end{document}