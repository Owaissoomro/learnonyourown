% !TeX program = xelatex
\documentclass{article}
\usepackage[margin=1in]{geometry}
\usepackage{microtype}
\usepackage{setspace}\setstretch{1.05}
\usepackage{fontspec}
\setmainfont{Latin Modern Roman}
\setmonofont{Latin Modern Mono}[BoldFont={Latin Modern Mono},ItalicFont={Latin Modern Mono}]
\usepackage{amsmath,mathtools,amsthm}
\newtheorem{lemma}{Lemma}
\newtheorem{theorem}{Theorem}
\usepackage{unicode-math}\setmathfont{Latin Modern Math}\allowdisplaybreaks[4]
% --- overflow and alignment slack ---
\setlength{\jot}{7pt}
\sloppy\emergencystretch=8em\hfuzz=1pt\vfuzz=2pt\raggedbottom
% --- breakable math helpers ---
\newenvironment{BreakableEquation}{\begin{equation}\begin{aligned}}{\end{aligned}\end{equation}}
\newenvironment{BreakableEquation*}{\begin{equation*}\begin{aligned}}{\end{aligned}\end{equation*}}
\newenvironment{tightalign}{\begingroup\small\allowdisplaybreaks\begin{align}}{\end{align}\endgroup}

\usepackage{xcolor}
\usepackage{xurl} % better URL wrapping
\usepackage[colorlinks=true,linkcolor=black,urlcolor=blue]{hyperref}
\usepackage{fancyhdr}
\pagestyle{fancy}\fancyhf{}\lhead{\nouppercase{\leftmark}}\rhead{\thepage}\setlength{\headheight}{26pt}
\usepackage{enumitem,booktabs,tabularx}
\newcolumntype{L}{>{\raggedright\arraybackslash}X}
\renewcommand{\arraystretch}{1.1}
\setlength{\parindent}{0pt}\setlength{\parskip}{8pt plus 2pt minus 1pt}

% Listings + upquote (no shell-escape needed)
\usepackage{listings}
\usepackage{upquote}
\lstdefinestyle{crisp}{
  basicstyle=\ttfamily\footnotesize,
  frame=single,
  breaklines=true,
  breakatwhitespace=false, % allow breaks inside long tokens
  tabsize=4,
  showstringspaces=false,
  numbers=left,
  numberstyle=\tiny,
  numbersep=8pt,
  keepspaces=true,
  columns=fullflexible,
  backgroundcolor=\color{black!02},
  aboveskip=8pt,
  belowskip=8pt
}
% Guarantee that 'python' exists as a language for listings
\lstdefinelanguage{python}{
  morekeywords={def,return,class,if,elif,else,for,while,try,except,raise,assert,pass,break,continue,lambda,nonlocal,global,yield,import,from,as,with,True,False,None},
  sensitive=true,
  morecomment=[l]\#,
  morestring=[b]",
  morestring=[b]'
}
% minted shim (robust; no shell-escape; uses listings' own environment)
\lstnewenvironment{minted}[2][]{\lstset{style=crisp,language=#2,#1}}{}

\usepackage[most]{tcolorbox}
\tcbset{colback=white,colframe=black!15,boxrule=0.4pt,arc=2pt,left=6pt,right=6pt,top=6pt,bottom=6pt,before skip=10pt,after skip=10pt,breakable}
\usepackage{titlesec}
\titleformat{\section}{\large\bfseries}{\thesection}{0.6em}{}
\titlespacing*{\section}{0pt}{*2.0}{*1.0}

% ===== CONSISTENCY TOOLKIT (macros) =====
\newlist{ledger}{enumerate}{1}
\setlist[ledger]{label=\textbullet,leftmargin=2em,itemsep=2pt,topsep=6pt}
\newcommand{\LEDGER}[1]{\textbf{ARITHMETIC LEDGER:}\par\begin{ledger}#1\end{ledger}}

\newlist{algosteps}{enumerate}{1}
\setlist[algosteps]{label=\arabic*.,leftmargin=2em,itemsep=2pt,topsep=6pt}

\newlist{bullets}{itemize}{1}
\setlist[bullets]{label=--,leftmargin=1.6em,itemsep=2pt,topsep=4pt}

\newcommand{\varmapStart}{\textbf{VARIABLE MAPPING:}\par\begin{bullets}}
\newcommand{\var}[2]{\item $#1$ — #2}
\newcommand{\varmapEnd}{\end{bullets}}

\newcommand{\LINE}[2]{\par\noindent\textbf{#1:}~#2\par}
\newcommand{\WHAT}[1]{\LINE{WHAT}{#1}}
\newcommand{\WHY}[1]{\LINE{WHY}{#1}}
\newcommand{\HOW}[1]{\LINE{HOW}{#1}}
\newcommand{\ELI}[1]{\LINE{ELI5}{#1}}
\newcommand{\STATEMENT}[1]{\LINE{STATEMENT}{#1}}
\newcommand{\BREAKDOWN}[1]{\LINE{PROBLEM BREAKDOWN}{#1}}
\newcommand{\MODEL}[1]{\LINE{CANONICAL MATHEMATICAL MODEL}{#1}}
\newcommand{\ASSUMPTIONS}[1]{\LINE{ASSUMPTIONS}{#1}}
\newcommand{\INVARIANTS}[1]{\LINE{INVARIANTS}{#1}}
\newcommand{\WHICHFORMULA}[1]{\LINE{WHICH FORMULA \& WHY}{#1}}
\newcommand{\GOVERN}[1]{\LINE{GOVERNING EQUATION(S)}{#1}}
\newcommand{\INPUTS}[1]{\LINE{INPUTS}{#1}}
\newcommand{\OUTPUTS}[1]{\LINE{OUTPUTS}{#1}}
\newcommand{\SAMPLES}[1]{\LINE{SAMPLES}{#1}}
\newcommand{\RESULT}[1]{\LINE{RESULT}{#1}}
\newcommand{\COMPLEXITY}[1]{\LINE{TIME/SPACE COMPLEXITY}{#1}}
\newcommand{\MEMORY}[1]{\LINE{MEMORY FOOTPRINT}{#1}}
\newcommand{\CORRECTNESS}[1]{\LINE{CORRECTNESS SKETCH}{#1}}
\newcommand{\OPTIMALITY}[1]{\LINE{OPTIMALITY}{#1}}
\newcommand{\FAILMODES}[1]{\LINE{FAILURE MODES}{#1}}
\newcommand{\VALIDATION}[1]{\LINE{VALIDATION}{#1}}
\newcommand{\UNITCHECK}[1]{\LINE{UNIT CHECK}{#1}}
\newcommand{\EDGECASES}[1]{\LINE{EDGE CASES}{#1}}
\newcommand{\CHECKLIST}[1]{\LINE{CHECKLIST}{#1}}
\newcommand{\DERIV}[1]{\LINE{DERIVATION}{#1}}
\newcommand{\LEMMAHEAD}[1]{\LINE{SUPPORTING LEMMA}{#1}}
\newcommand{\PITFALLS}[1]{\LINE{PITFALLS}{#1}}

\usepackage{etoolbox}\pretocmd{\section}{\clearpage}{}{}

\newcommand{\FormulaPage}[2]{%
  \clearpage
  \section*{Formula #1 — #2}%
  \addcontentsline{toc}{section}{Formula #1 — #2}%
}
\newcommand{\ApproachPage}[2]{%
  \clearpage
  \subsection*{Approach #1 — #2}%
  \addcontentsline{toc}{subsection}{Approach #1 — #2}%
}

\begin{document}
\title{Interview Problem Sheet — Perfect Encoding}
\date{\today}
\author{}
\maketitle
\tableofcontents
\clearpage

% === Notes macro (BODY-ONLY; do not alter the preamble) ===
% Prints exactly N blank pages with empty headers/footers, then leaves you on the last blank page.
\newcommand{\NotePages}[1]{%
  \clearpage
  \begingroup
  \newcount\NPi \newcount\NPn
  \NPi=0 \NPn=#1
  \loop\ifnum\NPi<\NPn
    \advance\NPi by 1
    \mbox{}\thispagestyle{empty}%
    \ifnum\NPi<\NPn\clearpage\fi
  \repeat
  \endgroup
}

% ============ 1. Problem & Metadata (own page) ============
\section{Problem \& Metadata}
\LINE{PLATFORM}{CF}
\LINE{URL}{\url{https://codeforces.com/problemset/problem/986/D}}
\LINE{DIFFICULTY / RATING}{CF: 986/D, Rating: 3100}
\STATEMENT{You are working as an analyst in a company working on a new system for big data storage. This system will store $n$ different objects. Each object should have a unique ID.

To create the system, you choose the parameters of the system — integers $m \ge 1$ and $b_{1}, b_{2}, \ldots, b_{m}$. With these parameters an ID of some object in the system is an array of integers $[a_{1}, a_{2}, \ldots, a_{m}]$ where $1 \le a_{i} \le b_{i}$ holds for every $1 \le i \le m$.

Developers say that production costs are proportional to $\sum_{i=1}^{m} b_{i}$. You are asked to choose parameters $m$ and $b_{i}$ so that the system will be able to assign unique IDs to $n$ different objects and production costs are minimized. Note that you do not have to use all available IDs.

Input: In the only line of input there is one positive integer $n$. The length of the decimal representation of $n$ is no greater than $1.5 \cdot 10^{6}$. The integer does not contain leading zeros.

Output: Print one number — minimal value of $\sum_{i=1}^{m} b_{i}$.}
\BREAKDOWN{We must minimize $\sum b_i$ subject to $\prod b_i \ge n$, where $m$ and $b_i \in \mathbb{Z}_{\ge 1}$ are free. This reduces to finding the least integer sum $S$ for which the maximum achievable product with parts summing to $S$ is at least $n$.}
\ELI{Break the budget $S$ into chunk sizes to maximize product; use the fewest total chunks $S$ whose best possible product reaches $n$.}
\NotePages{3}

% ============ 2. IO Contract (own page) ============
\section{IO Contract}
\INPUTS{Single line with a positive decimal integer $n$ with up to $1.5\cdot 10^{6}$ digits, no leading zeros.}
\OUTPUTS{A single integer equal to the minimal achievable $\sum b_i$ such that $\prod b_i \ge n$.}
\SAMPLES{
Input: 6; Output: 5. Explanation: choose $b=(3,2)$, product $6$, sum $5$.

Input: 9; Output: 6. Explanation: choose $b=(3,3)$, product $9$, sum $6$.}
\NotePages{3}

% ============ 3. Canonical Mathematical Model (own page) ============
\section{Canonical Mathematical Model}
\MODEL{Let $S=\sum_{i=1}^m b_i$ with $m \ge 1$ and $b_i \in \mathbb{Z}_{\ge 1}$. Define $M(S)$ as the maximum product achievable by partitioning $S$ into positive integers whose sum is $S$. The optimization is to find the minimal $S$ such that $M(S) \ge n$.}
\varmapStart
\var{n}{required number of distinct IDs}
\var{m}{number of dimensions}
\var{b_i}{upper bound in dimension $i$}
\var{S}{total cost $\sum b_i$}
\var{M(S)}{maximum product for a partition of $S$}
\varmapEnd
\GOVERN{
\[
  M(S)=
  \begin{cases}
    1, & S=0 \text{ or } 1,\\
    2, & S=2,\\
    3, & S=3,\\
    3^{S/3}, & S \equiv 0 \pmod{3},\\
    4 \cdot 3^{(S-4)/3}, & S \equiv 1 \pmod{3},\ S\ge 4,\\
    2 \cdot 3^{(S-2)/3}, & S \equiv 2 \pmod{3},\ S\ge 2.
  \end{cases}
\]
}
\ASSUMPTIONS{We never use parts equal to $1$ in an optimal partition for maximizing product. For $S\ge 2$, optimal partitions use only $2$ and $3$, with at most two $2$'s and otherwise $3$'s.}
\INVARIANTS{
- For fixed $S$, $M(S)$ is achieved by partitioning $S$ into $\{3,3,\ldots\}$, replacing a trailing $(3,1)$ by $(2,2)$ when $S \equiv 1 \pmod{3}$.\\
- $M(S)$ is strictly increasing in $S$ with multiplicative step ratios among consecutive $S$ in $\{4/3, 3/2, 3/2\}$.}
\NotePages{3}

% ============ 4. Approach A — Baseline (own page) ============
\section{Approach A — Baseline}
\ApproachPage{A}{Brute Force / Direct}
\WHICHFORMULA{Enumerate $S$ from $0$ upward, compute $M(S)$ exactly by the closed form, and stop at the first $S$ with $M(S)\ge n$. For this baseline, we parse $n$ as a native integer; it only works for small inputs.}
\ASSUMPTIONS{Suitable only when $n$ fits in native integers.}
\textbf{Algorithm Steps}
\begin{algosteps}
\item Parse $n$ as an integer.
\item For $S=0,1,2,\ldots$: compute $M(S)$ by the closed form and compare to $n$.
\item Return the first $S$ where $M(S)\ge n$ (with $S=1$ for $n=1$).
\end{algosteps}
\COMPLEXITY{For small $n$: $T$ roughly $O(S^\star \log n)$ due to exponentiation by squaring; $S^\star \approx \tfrac{3\log n}{\log 3}$. Space $O(1)$ besides big-integer temporaries.}
\[
\begin{aligned}
S^\star &\approx \frac{3\log n}{\log 3},\quad
M(S) \text{ via pow in } O(\log S) \text{ mults.}
\end{aligned}
\]
\CORRECTNESS{Follows since $M(S)$ is the maximal achievable product at fixed $S$ and is monotone increasing; the first $S$ with $M(S)\ge n$ is minimal.}
\EDGECASES{$n=1 \Rightarrow S=1$; $n\in\{2,3,4\}$ handled by the base cases of $M(S)$.}
\textbf{Code (Baseline)}
\begin{minted}{python}
import sys
import math

def max_product_at_sum(S: int) -> int:
    if S <= 1:
        return 1
    if S == 2:
        return 2
    if S == 3:
        return 3
    r = S % 3
    if r == 0:
        return pow(3, S // 3)
    if r == 1:
        # Use 4 + 3 + ... instead of 3 + 1
        return 4 * pow(3, (S - 4) // 3)
    # r == 2
    return 2 * pow(3, (S - 2) // 3)

def solve_case_small(n_int: int) -> int:
    if n_int == 1:
        return 1
    S = 0
    while True:
        mp = max_product_at_sum(S)
        if mp >= n_int:
            return S if n_int > 1 else 1
        S += 1

def read_input():
    data = sys.stdin.read().strip().split()
    return data[0] if data else ""

def main():
    # Tiny self-checks with brute force
    expected = {
        1: 1, 2: 2, 3: 3, 4: 4,
        5: 5, 6: 5, 7: 6, 8: 6, 9: 6, 10: 7, 11: 7, 12: 7,
        27: 9, 28: 10, 36: 10
    }
    for k, v in expected.items():
        assert solve_case_small(k) == v

    s = read_input()
    if not s:
        return
    # Baseline only supports small inputs
    if len(s) <= 18:
        n_int = int(s)
        print(solve_case_small(n_int))
    else:
        # Fallback: rough estimate via logs to avoid hang on huge input.
        L = len(s)
        K = min(100, L)
        a = int(s[:K])
        logn = (L - K) + math.log10(a)
        S_est = int(math.ceil(3.0 * logn / math.log10(3.0)))
        print(S_est)

if __name__ == "__main__":
    main()
\end{minted}
\VALIDATION{Checked a dictionary of small $n$ values and base cases.}
\NotePages{3}

% ============ 5. Approach B — Improved (own page) ============
\section{Approach B — Improved}
\ApproachPage{B}{Optimized Data Structure / Pruning / DP}
\WHICHFORMULA{Use monotonicity of $M(S)$ and binary search on $S$, but still compare using integers when $n$ is small. This separates search from product evaluation.}
\ASSUMPTIONS{Works well when $n$ fits in native integers; falls back to approximate when large.}
\textbf{Algorithm Steps}
\begin{algosteps}
\item If $n$ fits in machine integer, binary search minimal $S$ with $M(S)\ge n$ using integer arithmetic.
\item Otherwise, estimate $S$ by continuous approximation $S \approx \tfrac{3\log n}{\log 3}$.
\item Output the found/estimated $S$.
\end{algosteps}
\COMPLEXITY{Binary search uses $O(\log S^\star)$ steps; each step computes $M(S)$ in $O(\log S)$ multiplications. Space is $O(1)$.}
\[
\begin{aligned}
T(n) &= O(\log S^\star \cdot \log S^\star) \approx O((\log\log n)^2).
\end{aligned}
\]
\CORRECTNESS{Binary search on a monotone predicate finds the minimal $S$. Product maximality is by the known optimal integer partition into $3$'s and a bounded number of $2$'s.}
\textbf{Code (Improved)}
\begin{minted}{python}
import sys
import math

def max_product_at_sum(S: int) -> int:
    if S <= 1:
        return 1
    if S == 2:
        return 2
    if S == 3:
        return 3
    r = S % 3
    if r == 0:
        return pow(3, S // 3)
    if r == 1:
        return 4 * pow(3, (S - 4) // 3)
    return 2 * pow(3, (S - 2) // 3)

def solve_case_int(n_int: int) -> int:
    if n_int == 1:
        return 1
    # upper bound via continuous approximation + slack
    logn = math.log(n_int, 10.0)
    hi = int(math.ceil(3.0 * logn / math.log10(3.0))) + 10
    lo = 0
    while lo < hi:
        mid = (lo + hi) // 2
        if max_product_at_sum(mid) >= n_int:
            hi = mid
        else:
            lo = mid + 1
    return lo

def read_input():
    data = sys.stdin.read().strip().split()
    return data[0] if data else ""

def main():
    # Self-checks
    for n, ans in [(1,1),(2,2),(3,3),(4,4),(5,5),(6,5),(7,6),(8,6),(9,6),(10,7)]:
        assert solve_case_int(n) == ans
    # Cross-check small range
    for n in range(1, 60):
        # brute
        s = 0
        while True:
            if max_product_at_sum(s) >= n:
                brute = s if n > 1 else 1
                break
            s += 1
        assert solve_case_int(n) == brute

    s = read_input()
    if not s:
        return
    if len(s) <= 18:
        print(solve_case_int(int(s)))
    else:
        # estimate for huge inputs
        L = len(s)
        K = min(100, L)
        a = int(s[:K])
        logn = (L - K) + math.log10(a)
        S_est = int(math.ceil(3.0 * logn / math.log10(3.0)))
        print(S_est)

if __name__ == "__main__":
    main()
\end{minted}
\VALIDATION{Verified exactness on $n \le 59$ by brute force and several hand-picked values.}
\NotePages{3}

% ============ 6. Approach C — Optimal (own page) ============
\section{Approach C — Optimal}
\ApproachPage{C}{Provably Optimal Method}
\WHICHFORMULA{We need the minimal $S$ such that $M(S)\ge n$. Using the known optimal partition, $M(S)$ has the closed form in terms of $3$ and possibly $2$ or $4$. For extremely large $n$ given as a string, directly constructing $M(S)$ is infeasible, but $\log_{10} M(S)$ can be compared to $\log_{10} n$ using only the first $K$ digits of $n$ and floating-point logs, exploiting the fact that consecutive $S$ steps change $M(S)$ by factors $\ge \tfrac{4}{3}$, giving a wide safety margin.}
\ASSUMPTIONS{Float precision is sufficient because the minimal multiplicative gap across successive $S$ is $4/3$ (about $0.1249$ in $\log_{10}$), far exceeding floating error.}
\textbf{Algorithm Steps}
\begin{algosteps}
\item If $n=1$, return $1$.
\item Compute $L=\lvert n\rvert$ and $K=\min(100,L)$. Let $a$ be the integer from the first $K$ digits. Approximate $\log_{10} n \approx (L-K) + \log_{10}(a)$.
\item Define $\ell_2=\log_{10}2$, $\ell_3=\log_{10}3$, $\ell_4=\log_{10}4$.
\item Define $\log_{10} M(S)$ by cases: $S\equiv 0\Rightarrow \tfrac{S}{3}\ell_3$; $S\equiv 1\Rightarrow \tfrac{S-4}{3}\ell_3 + \ell_4$; $S\equiv 2\Rightarrow \tfrac{S-2}{3}\ell_3 + \ell_2$ (with small-$S$ guards).
\item Binary search the minimal $S$ with $\log_{10} M(S) + \varepsilon \ge \log_{10} n$, using $\varepsilon=10^{-12}$ and an upper bound $S_0=\left\lceil \tfrac{3\log_{10} n}{\ell_3} \right\rceil + 10$.
\end{algosteps}
\OPTIMALITY{From integer partition theory, for fixed $S$ the maximum product is obtained by using only $3$'s and at most two $2$'s. Hence $M(S)$ is the true upper envelope; the minimal $S$ with $M(S)\ge n$ is the optimal cost.}
\COMPLEXITY{All operations are on short floats and strings: $O(\log S^\star)$ iterations, each $O(1)$. Space $O(1)$.}
\[
\begin{aligned}
T(n) &= O(\log S^\star) = O(\log\log n),\quad S^\star \approx \frac{3\log n}{\log 3}.
\end{aligned}
\]
\textbf{Code (Final Submission)}
\begin{minted}{python}
import sys
import math

L2 = math.log10(2.0)
L3 = math.log10(3.0)
L4 = math.log10(4.0)
EPS = 1e-12

def logM(S: int) -> float:
    if S <= 1:
        return 0.0
    if S == 2:
        return L2  # log10(2)
    if S == 3:
        return math.log10(3.0)
    r = S % 3
    if r == 0:
        return (S / 3.0) * L3
    if r == 1:
        # Use 4 and the rest 3's; guard S>=4 handled by small cases above
        return ((S - 4) / 3.0) * L3 + L4
    # r == 2
    return ((S - 2) / 3.0) * L3 + L2

def log10_of_big_decimal_str(s: str, K: int = 100) -> float:
    L = len(s)
    K = min(K, L)
    a = int(s[:K])
    return (L - K) + math.log10(float(a))

def solve_case(s: str) -> int:
    # s is a decimal string, no leading zeros
    if s == "1":
        return 1
    logn = log10_of_big_decimal_str(s, 100)
    # Upper bound with slack
    hi = int(math.ceil(3.0 * logn / L3)) + 10
    if hi < 2:
        hi = 2
    lo = 0
    # Binary search minimal S with logM(S) >= logn (with tolerance)
    while lo < hi:
        mid = (lo + hi) // 2
        if logM(mid) + EPS >= logn:
            hi = mid
        else:
            lo = mid + 1
    return lo if s != "1" else 1

def read_input():
    data = sys.stdin.read().strip().split()
    return data[0] if data else ""

def solve_all():
    s = read_input()
    if not s:
        return
    ans = solve_case(s)
    print(ans)

def main():
    # Internal tests
    small_expected = {
        "1": 1, "2": 2, "3": 3, "4": 4,
        "5": 5, "6": 5, "7": 6, "8": 6, "9": 6,
        "10": 7, "11": 7, "12": 7, "27": 9, "36": 10
    }
    for k, v in small_expected.items():
        assert solve_case(k) == v
    # Random spot checks on small ints vs exact search
    for n in range(1, 200):
        # exact via M(S) integers
        S = 0
        while True:
            # construct exact M(S) using integers
            if S <= 1:
                mp = 1
            elif S == 2:
                mp = 2
            elif S == 3:
                mp = 3
            else:
                r = S % 3
                if r == 0:
                    mp = pow(3, S // 3)
                elif r == 1:
                    mp = 4 * pow(3, (S - 4) // 3)
                else:
                    mp = 2 * pow(3, (S - 2) // 3)
            if mp >= n:
                brute = S if n > 1 else 1
                break
            S += 1
        assert solve_case(str(n)) == brute

    solve_all()

if __name__ == "__main__":
    main()
\end{minted}
\VALIDATION{Exactly 3 asserts or I/O mini-tests:
- Unit map for specific values: $\{1\mapsto 1, 2\mapsto 2, 6\mapsto 5\}$.
- Equality thresholds: $n=27\Rightarrow S=9$, $n=36\Rightarrow S=10$.
- Range check: all $n\in[1,199]$ cross-validated against exact integer $M(S)$.}
\RESULT{Print the minimal total cost $S=\sum b_i$. When multiple parameter sets $(m,b_i)$ achieve this minimal cost, any is acceptable; only $S$ is required.}
\NotePages{3}

% ============ 7. Testing & Final Reference Implementation (own page) ============
\section{Testing \& Final Reference Implementation}
\LINE{TEST PLAN}{Validate on base cases, near-boundary values where $S \bmod 3$ changes, and equality cases $n \in \{3^k, 2\cdot 3^k, 4\cdot 3^k\}$.}
\LINE{CROSS-CHECKS}{Compare the float-based search (optimal) against exact integer computations for small $n$ to ensure no rounding-caused off-by-one.}
\LINE{EDGE-CASE GENERATOR}{Generate $n$ as strings: tiny numbers $1\ldots 200$, powers $3^k$, and mixed forms $2\cdot 3^k$, $4\cdot 3^k$.}
\begin{minted}{python}
# Deterministic generators for boundaries, degenerates, adversarials
import math

def gen_small():
    return [str(i) for i in range(1, 201)]

def gen_powers3(limit_k=20):
    out = []
    x = 1
    for k in range(limit_k+1):
        out.append(str(x))
        x *= 3
    return out

def gen_mixed(limit_k=20):
    arr = []
    x = 1
    for k in range(limit_k+1):
        arr.append(str(2*x))
        arr.append(str(4*x))
        x *= 3
    return arr

if __name__ == "__main__":
    tests = gen_small() + gen_powers3() + gen_mixed()
    # Unique and sorted by numeric value (safe for small tests)
    tests = sorted(set(tests), key=int)
    print(len(tests), "generated test inputs.")
\end{minted}
\textbf{Reference Code (Ready to Submit)}
\begin{minted}{python}
import sys
import math

L2 = math.log10(2.0)
L3 = math.log10(3.0)
L4 = math.log10(4.0)
EPS = 1e-12

def logM(S: int) -> float:
    if S <= 1:
        return 0.0
    if S == 2:
        return L2
    if S == 3:
        return math.log10(3.0)
    r = S % 3
    if r == 0:
        return (S / 3.0) * L3
    if r == 1:
        return ((S - 4) / 3.0) * L3 + L4
    return ((S - 2) / 3.0) * L3 + L2

def log10_of_big_decimal_str(s: str, K: int = 100) -> float:
    L = len(s)
    K = min(K, L)
    a = int(s[:K])
    return (L - K) + math.log10(float(a))

def solve_case(s: str) -> int:
    if s == "1":
        return 1
    logn = log10_of_big_decimal_str(s, 100)
    hi = int(math.ceil(3.0 * logn / L3)) + 10
    if hi < 2:
        hi = 2
    lo = 0
    while lo < hi:
        mid = (lo + hi) // 2
        if logM(mid) + EPS >= logn:
            hi = mid
        else:
            lo = mid + 1
    return lo

def read_input():
    data = sys.stdin.read().strip().split()
    return data[0] if data else ""

def solve_all():
    s = read_input()
    if not s:
        return
    print(solve_case(s))

def main():
    # Asserts
    assert solve_case("1") == 1
    assert solve_case("2") == 2
    assert solve_case("6") == 5
    # Cross-check small range with exact integer search
    for n in range(1, 120):
        S = 0
        while True:
            if S <= 1:
                mp = 1
            elif S == 2:
                mp = 2
            elif S == 3:
                mp = 3
            else:
                r = S % 3
                if r == 0:
                    mp = pow(3, S // 3)
                elif r == 1:
                    mp = 4 * pow(3, (S - 4) // 3)
                else:
                    mp = 2 * pow(3, (S - 2) // 3)
            if mp >= n:
                brute = S if n > 1 else 1
                break
            S += 1
        assert solve_case(str(n)) == brute
    solve_all()

if __name__ == "__main__":
    main()
\end{minted}
\NotePages{3}

% ============ 8. Review & Pitfalls (own page) ============
\section{Review \& Pitfalls}
\WHAT{Minimize $\sum b_i$ such that $\prod b_i \ge n$; equivalently, find minimal $S$ with $M(S)\ge n$, where $M(S)$ is the product-maximizing partition of $S$.}
\WHY{This tests knowledge of integer partition maximizing products, continuous-relaxation intuition, and numerical comparisons for huge integers.}
\CHECKLIST{
- Recognize product-maximizing partition into $3$'s and $2$'s. 
- Derive $M(S)$ closed form by $S \bmod 3$. 
- Note monotonicity of $M(S)$ in $S$. 
- Compare via $\log_{10}$ using only a prefix of $n$. 
- Binary search minimal $S$.}
\EDGECASES{
- $n=1 \Rightarrow S=1$. 
- $S<4$ base cases for $M(S)$. 
- Equality at $n \in \{3^k, 2\cdot 3^k, 4\cdot 3^k\}$. 
- Very short inputs like $n\in\{2,3,4,5\}$. 
- Extremely long $n$ (millions of digits). 
- Prefix extraction when $L<K$ (use $K=L$).}
\PITFALLS{
- Using $1$'s in partitions (never optimal). 
- Forgetting the $(3,1)\to(2,2)$ replacement for $S\equiv 1\pmod{3}$. 
- Direct big-integer exponentiation for huge $n$ (too slow/memory-heavy). 
- Insufficient float precision: use a generous step gap and small $\varepsilon$. 
- Not handling $S\in\{0,1,2,3\}$ separately in $\log M$. 
- Off-by-one in binary search bounds.}
\FAILMODES{Baseline integer-based methods fail on huge $n$ due to memory/time; the log-based comparison survives by avoiding big integers entirely, relying on wide multiplicative gaps.}
\ELI{We want the cheapest way to reach at least $n$ by multiplying integers whose sum is our cost. The best way to multiply big numbers from a fixed sum is to use $3$'s and maybe a $2$, so we ask: what is the smallest sum $S$ so that this best-possible product reaches $n$? Compute and compare on the log scale, and search $S$ by binary search.}
\NotePages{3}

\end{document}