% !TeX program = xelatex
\documentclass[11pt]{article}
\usepackage[margin=1in]{geometry}
\usepackage{microtype}
\usepackage{setspace}\setstretch{1.05}
\usepackage{fontspec}
\setmainfont{Latin Modern Roman}
\setmonofont{Latin Modern Mono}[BoldFont={Latin Modern Mono},ItalicFont={Latin Modern Mono}]
\usepackage{amsmath,mathtools,amsthm}
\newtheorem{lemma}{Lemma}
\newtheorem{theorem}{Theorem}
\usepackage{unicode-math}\setmathfont{Latin Modern Math}\allowdisplaybreaks[4]
% --- overflow and alignment slack ---
\setlength{\jot}{7pt}
\sloppy\emergencystretch=8em\hfuzz=1pt\vfuzz=2pt\raggedbottom
% --- breakable math helpers ---
\newenvironment{BreakableEquation}{\begin{equation}\begin{aligned}}{\end{aligned}\end{equation}}
\newenvironment{BreakableEquation*}{\begin{equation*}\begin{aligned}}{\end{aligned}\end{equation*}}
\newenvironment{tightalign}{\begingroup\small\allowdisplaybreaks\begin{align}}{\end{align}\endgroup}

\usepackage{xcolor}
\usepackage{xurl} % better URL wrapping
\usepackage[colorlinks=true,linkcolor=black,urlcolor=blue]{hyperref}
\usepackage{fancyhdr}
\pagestyle{fancy}\fancyhf{}\lhead{\nouppercase{\leftmark}}\rhead{\thepage}\setlength{\headheight}{26pt}
\usepackage{enumitem,booktabs,tabularx}
\newcolumntype{L}{>{\raggedright\arraybackslash}X}
\renewcommand{\arraystretch}{1.1}
\setlength{\parindent}{0pt}\setlength{\parskip}{8pt plus 2pt minus 1pt}

% Listings + upquote (no shell-escape needed)
\usepackage{listings}
\usepackage{upquote}
\lstdefinestyle{crisp}{
  basicstyle=\ttfamily\footnotesize,
  frame=single,
  breaklines=true,
  breakatwhitespace=false, % allow breaks inside long tokens
  tabsize=4,
  showstringspaces=false,
  numbers=left,
  numberstyle=\tiny,
  numbersep=8pt,
  keepspaces=true,
  columns=fullflexible,
  backgroundcolor=\color{black!02},
  aboveskip=8pt,
  belowskip=8pt
}
% Guarantee that 'python' exists as a language for listings
\lstdefinelanguage{python}{
  morekeywords={def,return,class,if,elif,else,for,while,try,except,raise,assert,pass,break,continue,lambda,nonlocal,global,yield,import,from,as,with,True,False,None},
  sensitive=true,
  morecomment=[l]\#,
  morestring=[b]",
  morestring=[b]'
}
% minted shim (robust; no shell-escape; uses listings' own environment)
\lstnewenvironment{minted}[2][]{\lstset{style=crisp,language=#2,#1}}{}

\usepackage[most]{tcolorbox}
\tcbset{colback=white,colframe=black!15,boxrule=0.4pt,arc=2pt,left=6pt,right=6pt,top=6pt,bottom=6pt,before skip=10pt,after skip=10pt,breakable}
\usepackage{titlesec}
\titleformat{\section}{\large\bfseries}{\thesection}{0.6em}{}
\titlespacing*{\section}{0pt}{*2.0}{*1.0}

% ===== CONSISTENCY TOOLKIT (macros) =====
\newlist{ledger}{enumerate}{1}
\setlist[ledger]{label=\textbullet,leftmargin=2em,itemsep=2pt,topsep=6pt}
\newcommand{\LEDGER}[1]{\textbf{ARITHMETIC LEDGER:}\par\begin{ledger}#1\end{ledger}}

\newlist{algosteps}{enumerate}{1}
\setlist[algosteps]{label=\arabic*.,leftmargin=2em,itemsep=2pt,topsep=6pt}

\newlist{bullets}{itemize}{1}
\setlist[bullets]{label=--,leftmargin=1.6em,itemsep=2pt,topsep=4pt}

\newcommand{\varmapStart}{\textbf{VARIABLE MAPPING:}\par\begin{bullets}}
\newcommand{\var}[2]{\item $#1$ — #2}
\newcommand{\varmapEnd}{\end{bullets}}

\newcommand{\LINE}[2]{\par\noindent\textbf{#1:}~#2\par}
\newcommand{\WHAT}[1]{\LINE{WHAT}{#1}}
\newcommand{\WHY}[1]{\LINE{WHY}{#1}}
\newcommand{\HOW}[1]{\LINE{HOW}{#1}}
\newcommand{\ELI}[1]{\LINE{ELI5}{#1}}
\newcommand{\STATEMENT}[1]{\LINE{STATEMENT}{#1}}
\newcommand{\BREAKDOWN}[1]{\LINE{PROBLEM BREAKDOWN}{#1}}
\newcommand{\MODEL}[1]{\LINE{CANONICAL MATHEMATICAL MODEL}{#1}}
\newcommand{\ASSUMPTIONS}[1]{\LINE{ASSUMPTIONS}{#1}}
\newcommand{\INVARIANTS}[1]{\LINE{INVARIANTS}{#1}}
\newcommand{\WHICHFORMULA}[1]{\LINE{WHICH FORMULA \& WHY}{#1}}
\newcommand{\GOVERN}[1]{\LINE{GOVERNING EQUATION(S)}{#1}}
\newcommand{\INPUTS}[1]{\LINE{INPUTS}{#1}}
\newcommand{\OUTPUTS}[1]{\LINE{OUTPUTS}{#1}}
\newcommand{\SAMPLES}[1]{\LINE{SAMPLES}{#1}}
\newcommand{\RESULT}[1]{\LINE{RESULT}{#1}}
\newcommand{\COMPLEXITY}[1]{\LINE{TIME/SPACE COMPLEXITY}{#1}}
\newcommand{\MEMORY}[1]{\LINE{MEMORY FOOTPRINT}{#1}}
\newcommand{\CORRECTNESS}[1]{\LINE{CORRECTNESS SKETCH}{#1}}
\newcommand{\OPTIMALITY}[1]{\LINE{OPTIMALITY}{#1}}
\newcommand{\FAILMODES}[1]{\LINE{FAILURE MODES}{#1}}
\newcommand{\VALIDATION}[1]{\LINE{VALIDATION}{#1}}
\newcommand{\UNITCHECK}[1]{\LINE{UNIT CHECK}{#1}}
\newcommand{\EDGECASES}[1]{\LINE{EDGE CASES}{#1}}
\newcommand{\CHECKLIST}[1]{\LINE{CHECKLIST}{#1}}
\newcommand{\DERIV}[1]{\LINE{DERIVATION}{#1}}
\newcommand{\LEMMAHEAD}[1]{\LINE{SUPPORTING LEMMA}{#1}}
\newcommand{\PITFALLS}[1]{\LINE{PITFALLS}{#1}}

\usepackage{etoolbox}\pretocmd{\section}{\clearpage}{}{}

\newcommand{\FormulaPage}[2]{%
  \clearpage
  \section*{Formula #1 — #2}%
  \addcontentsline{toc}{section}{Formula #1 — #2}%
}
\newcommand{\ApproachPage}[2]{%
  \clearpage
  \subsection*{Approach #1 — #2}%
  \addcontentsline{toc}{subsection}{Approach #1 — #2}%
}

\begin{document}
\title{Interview Problem Sheet — Prime Gift}
\date{\today}
\author{}
\maketitle
\tableofcontents
\clearpage

% === Notes macro (BODY-ONLY; do not alter the preamble) ===
% Prints exactly N blank pages with empty headers/footers, then leaves you on the last blank page.
\newcommand{\NotePages}[1]{%
  \clearpage
  \begingroup
  \newcount\NPi \newcount\NPn
  \NPi=0 \NPn=#1
  \loop\ifnum\NPi<\NPn
    \advance\NPi by 1
    \mbox{}\thispagestyle{empty}%
    \ifnum\NPi<\NPn\clearpage\fi
  \repeat
  \endgroup
}

% ============ 1. Problem & Metadata (own page) ============
\section{Problem \& Metadata}
\LINE{PLATFORM}{CF}
\LINE{URL}{\url{https://codeforces.com/problemset/problem/912/E}}
\LINE{DIFFICULTY / RATING}{2400}
\STATEMENT{Opposite to Grisha's nice behavior, Oleg, though he has an entire year at his disposal, did not manage to learn how to solve number theory problems in the past year. That is why instead of Ded Moroz he was visited by his teammate Andrew, who solemnly presented him with a set of $n$ distinct prime numbers alongside with a simple task: Oleg is to find the $k$-th smallest integer, such that all its prime divisors are in this set.

Input:

The first line contains a single integer $n$ ($1 \le n \le 16$).

The next line lists $n$ distinct prime numbers $p_1, p_2, \ldots, p_n$ ($2 \le p_i \le 100$) in ascending order.

The last line gives a single integer $k$ ($1 \le k$). It is guaranteed that the $k$-th smallest integer such that all its prime divisors are in this set does not exceed $10^{18}$.

Output:

Print a single line featuring the $k$-th smallest integer. It is guaranteed that the answer does not exceed $10^{18}$.

Note:

The list of numbers with all prime divisors inside $\{2, 3, 5\}$ begins as follows:
$(1, 2, 3, 4, 5, 6, 8, \ldots)$. The seventh number in this list (1-indexed) is eight.}
\BREAKDOWN{We must generate the ordered multiplicative semigroup $S=\{\prod p_i^{a_i}\mid a_i\ge 0\}$ and return its $k$-th element (including $1$). Efficiently compute the counting function $C(X)=|\{s\in S:s\le X\}|$ and binary search the smallest $X$ with $C(X)\ge k$.}
\ELI{Split primes in half, precompute all products per half up to $10^{18}$, then count pairs $\le X$ to binary search the answer.}
\NotePages{3}

% ============ 2. IO Contract (own page) ============
\section{IO Contract}
\INPUTS{Integers: $n$; list of $n$ primes $p_1<\cdots<p_n$; integer $k$. Valid ranges: $1\le n\le 16$, $2\le p_i\le 100$, $1\le k$, answer $\le 10^{18}$.}
\OUTPUTS{Print a single integer: the $k$-th smallest number whose prime divisors are all from the given set. The number $1$ is included.}
\SAMPLES{
- Example 1: $n=3$, primes $[2,3,5]$, $k=7$ $\to$ output $8$.

- Example 2: $n=1$, primes $[2]$, $k=10$ $\to$ output $512$ (sequence: $1,2,4,8,16,32,64,128,256,512$).
}
\NotePages{3}

% ============ 3. Canonical Mathematical Model (own page) ============
\section{Canonical Mathematical Model}
\MODEL{Let $P=\{p_1,\dots,p_n\}$ be the given set of primes. Define
$S=\bigl\{\prod_{i=1}^n p_i^{a_i}\,\big|\, a_i\in \mathbb{Z}_{\ge 0}\bigr\}$.
Define the counting function $C(X)=|\{s\in S: s\le X\}|$. We want the $k$-th order statistic of $S$ (with $1\in S$).}
\varmapStart
\var{n}{number of primes}
\var{p_i}{the $i$-th prime in ascending order}
\var{k}{desired rank (1-indexed)}
\var{C(X)}{count of valid numbers $\le X$}
\var{\text{ans}}{the $k$-th smallest element of $S$}
\varmapEnd
\GOVERN{
\[
\text{ans}=\min\{X\in \mathbb{N}\mid C(X)\ge k\},\qquad
C(X)=\bigl|\bigl\{\prod_{i=1}^n p_i^{a_i}\le X: a_i\in \mathbb{Z}_{\ge 0}\bigr\}\bigr|.
\]
}
\ASSUMPTIONS{Primes are distinct and sorted; $S$ is infinite but within $[1,10^{18}]$ the set is finite. $C(X)$ is nondecreasing and right-continuous over integers.}
\INVARIANTS{
- $1\in S$ (all $a_i=0$), hence $C(1)\ge 1$.

- If $x\in S$ and $y\mid x$ then $y\in S$.

- Monotonicity: $X_1\le X_2 \Rightarrow C(X_1)\le C(X_2)$.}
\NotePages{3}

% ============ 4. Approach A — Baseline (own page) ============
\section{Approach A — Baseline}
\ApproachPage{A}{Brute Force / Direct}
\WHICHFORMULA{Generate the sequence in increasing order using a min-heap and deduplication set, as in the classic ``Ugly Numbers'' method. Pop the smallest, multiply by each prime, and push unseen products $\le 10^{18}$.}
\ASSUMPTIONS{This works for small to moderate $k$, but may be too slow or memory-heavy for very large $k$.}
\textbf{Algorithm Steps}
\begin{algosteps}
\item Initialize a min-heap with $1$ and a set of seen values $\{1\}$.
\item Repeat: pop the minimum $x$; if this is the $k$-th pop, return $x$.
\item For each prime $p$, if $x\cdot p\le 10^{18}$ and unseen, push it and mark as seen.
\end{algosteps}
\COMPLEXITY{For $k$ outputs, each expansion pushes up to $n$ children with heap operations.}
\[
\begin{aligned}
T(k) &\approx O(k\cdot n \log(k\cdot n)),\\
S(k) &\approx O(k).
\end{aligned}
\]
\CORRECTNESS{Each valid number is eventually generated since we repeatedly multiply by primes from the base 1; dedup prevents duplicates; the heap orders outputs nondecreasingly, so the $k$-th pop is the $k$-th smallest.}
\EDGECASES{Include $1$ as the first number; avoid pushing products exceeding $10^{18}$; ensure dedup to prevent exponential blow-up.}
\textbf{Code (Baseline)}
\begin{minted}{python}
# Codeforces CF-Style Baseline: min-heap generation
import sys, math, heapq

LIMIT = 10**18

def read_input(data=None):
    if data is None:
        data = sys.stdin.read()
    data = data.strip().split()
    if not data:
        return None
    it = iter(map(int, data))
    n = next(it)
    primes = [next(it) for _ in range(n)]
    k = next(it)
    return n, primes, k

def solve_case_baseline(n, primes, k):
    h = [1]
    seen = {1}
    cnt = 0
    while h:
        x = heapq.heappop(h)
        cnt += 1
        if cnt == k:
            return x
        for p in primes:
            nx = x * p
            if nx <= LIMIT and nx not in seen:
                seen.add(nx)
                heapq.heappush(h, nx)
    # Should never reach here given problem guarantees
    return -1

def solve_all_baseline(n, primes, k):
    return solve_case_baseline(n, primes, k)

def main():
    parsed = read_input()
    if parsed is None:
        # Self-tests for baseline
        assert solve_case_baseline(3, [2,3,5], 7) == 8
        assert solve_case_baseline(1, [2], 10) == 512
        assert solve_case_baseline(2, [3,5], 1) == 1
        print("OK")
    else:
        n, primes, k = parsed
        ans = solve_case_baseline(n, primes, k)
        print(ans)

if __name__ == "__main__":
    main()
\end{minted}
\VALIDATION{Tiny checks: $(n,[2,3,5],k=7)\to 8$; $(1,[2],10)\to 512$; $(2,[3,5],1)\to 1$.}
\NotePages{3}

% ============ 5. Approach B — Improved (own page) ============
\section{Approach B — Improved}
\ApproachPage{B}{Optimized Counting via DFS + Binary Search}
\WHICHFORMULA{Binary search the answer $X$ and use a depth-first enumeration of exponents to count $C(X)$ without generating in sorted order.}
\ASSUMPTIONS{Recursion explores exponents for each prime in order, multiplying until exceeding $X$. Effective when $C(X)$ is moderate; worst-case still large.}
\textbf{Algorithm Steps}
\begin{algosteps}
\item Define $C(X)$ via recursive DFS on prime indices: for prime $p_i$, multiply current product by powers of $p_i$ until exceeding $X$, and recurse to $i+1$.
\item Binary search $X\in[1,10^{18}]$ for the minimum with $C(X)\ge k$.
\item Return this $X$.
\end{algosteps}
\COMPLEXITY{Binary search takes $O(\log 10^{18})\approx 60$ iterations. Each $C(X)$ costs the number of valid exponent combinations $\le X$, which can be large but often manageable.}
\[
\begin{aligned}
T &\approx O\!\Bigl( \log(10^{18}) \cdot \text{visits}(X)\Bigr), \\
\text{where }\text{visits}(X) &= \sum_{(a_i): \prod p_i^{a_i}\le X}1 = C(X).
\end{aligned}
\]
\CORRECTNESS{Enumerating exponents per prime in fixed order gives a bijection between nodes visited and valid products $\le X$, hence $C(X)$ is exact. Monotonicity of $C$ ensures binary search correctness.}
\textbf{Code (Improved)}
\begin{minted}{python}
# CF-Style: Binary search + DFS counting
import sys

LIMIT = 10**18

def read_input(data=None):
    if data is None:
        data = sys.stdin.read()
    data = data.strip().split()
    if not data:
        return None
    it = iter(map(int, data))
    n = next(it)
    primes = [next(it) for _ in range(n)]
    k = next(it)
    return n, primes, k

def count_leq_dfs(X, primes):
    sys.setrecursionlimit(10000)
    n = len(primes)
    def dfs(i, cur):
        if i == n:
            return 1  # one way: fixed exponents for all primes
        total = 0
        val = cur
        p = primes[i]
        while val <= X:
            total += dfs(i + 1, val)
            if val > X // p:
                break
            val *= p
        return total
    return dfs(0, 1)

def solve_case_improved(n, primes, k):
    lo, hi = 1, LIMIT
    while lo < hi:
        mid = (lo + hi) // 2
        cnt = count_leq_dfs(mid, primes)
        if cnt >= k:
            hi = mid
        else:
            lo = mid + 1
    return lo

def solve_all_improved(n, primes, k):
    return solve_case_improved(n, primes, k)

def main():
    parsed = read_input()
    if parsed is None:
        # Sanity self-tests on small configurations (to keep DFS light)
        assert solve_case_improved(3, [2,3,5], 7) == 8
        assert solve_case_improved(1, [2], 10) == 512
        assert solve_case_improved(2, [3,5], 6) == 9  # sequence: 1,3,5,9,15,25,27,...
        print("OK")
    else:
        n, primes, k = parsed
        ans = solve_case_improved(n, primes, k)
        print(ans)

if __name__ == "__main__":
    main()
\end{minted}
\VALIDATION{Check small $n$ and modest $k$; compare counts against explicit generation for tiny $X$.}
\NotePages{3}

% ============ 6. Approach C — Optimal (own page) ============
\section{Approach C — Optimal}
\ApproachPage{C}{Meet-in-the-Middle Counting + Binary Search}
\WHICHFORMULA{Split the primes into two halves $P_L$ and $P_R$. Precompute all products $\le 10^{18}$ from each half: lists $A,B$. For a given $X$, count pairs $(a,b)\in A\times B$ with $a\cdot b\le X$ using a two-pointer sweep, then binary search the minimal $X$ with count $\ge k$.}
\ASSUMPTIONS{Since $n\le 16$, each half has at most 8 primes; the number of generated products per half up to $10^{18}$ is tractable.}
\textbf{Algorithm Steps}
\begin{algosteps}
\item Split primes into $P_L$ and $P_R$.
\item Generate all products from $P_L$ and $P_R$ not exceeding $10^{18}$ (include $1$), producing sorted arrays $A$ and $B$.
\item To compute $C(X)$, for each $a\in A$ ascending, move a pointer down in $B$ until $a\cdot b\le X$, and add the count.
\item Binary search $X\in[1,10^{18}]$ for the smallest with $C(X)\ge k$; return it.
\end{algosteps}
\OPTIMALITY{Precomputing halves avoids exploring the full Cartesian exponent space per query. Counting per $X$ runs in $O(|A|+|B|)$; binary search costs $\approx 60$ counts. This is a standard optimal approach for $n\le 16$ with $10^{18}$ bound.}
\COMPLEXITY{Let $|A|$ and $|B|$ be the sizes of the half-product sets.}
\[
\begin{aligned}
\text{Precompute} &:\ O(|A|+|B|), \\
C(X) &:\ O(|A|+|B|), \\
\text{Total} &:\ O\bigl((|A|+|B|)\cdot \log 10^{18}\bigr)\ \text{time},\quad O(|A|+|B|)\ \text{space}.
\end{aligned}
\]
\textbf{Code (Final Submission)}
\begin{minted}{python}
# CF-Style Final Solution: Meet-in-the-Middle + Binary Search
import sys

LIMIT = 10**18

def read_input(data=None):
    if data is None:
        data = sys.stdin.read()
    data = data.strip().split()
    if not data:
        return None
    it = iter(map(int, data))
    n = next(it)
    primes = [next(it) for _ in range(n)]
    k = next(it)
    return n, primes, k

def gen_products(primes, limit=LIMIT):
    res = []
    def dfs(i, cur):
        if i == len(primes):
            res.append(cur)
            return
        p = primes[i]
        val = cur
        while val <= limit:
            dfs(i + 1, val)
            if val > limit // p:
                break
            val *= p
    dfs(0, 1)
    res.sort()
    return res

def count_leq_with_lists(A, B, X):
    # Two-pointer from end of B
    cnt = 0
    j = len(B) - 1
    for a in A:
        if a > X:
            break
        # decrease j until a * B[j] <= X
        while j >= 0 and a > 0 and a * B[j] > X:
            j -= 1
        if j < 0:
            break
        cnt += (j + 1)
    return cnt

def solve_case(n, primes, k):
    m = n // 2
    left = primes[:m]
    right = primes[m:]
    A = gen_products(left, LIMIT)
    B = gen_products(right, LIMIT)
    lo, hi = 1, LIMIT
    while lo < hi:
        mid = (lo + hi) // 2
        cnt = count_leq_with_lists(A, B, mid)
        if cnt >= k:
            hi = mid
        else:
            lo = mid + 1
    return lo

def solve_all(n, primes, k):
    return solve_case(n, primes, k)

def _self_tests():
    # Cross-check against baseline approach for small inputs
    def baseline(n, primes, k):
        import heapq
        h = [1]
        seen = {1}
        c = 0
        while h:
            x = heapq.heappop(h)
            c += 1
            if c == k:
                return x
            for p in primes:
                nx = x * p
                if nx <= LIMIT and nx not in seen:
                    seen.add(nx)
                    heapq.heappush(h, nx)
        return -1

    assert solve_case(3, [2,3,5], 7) == 8
    assert solve_case(1, [2], 10) == 512
    # Random-ish small checks
    assert solve_case(2, [3,5], 1) == 1
    for K in range(1, 25):
        assert solve_case(3, [2,3,5], K) == baseline(3, [2,3,5], K)

def main():
    parsed = read_input()
    if parsed is None:
        _self_tests()
        print("OK")
    else:
        n, primes, k = parsed
        ans = solve_case(n, primes, k)
        print(ans)

if __name__ == "__main__":
    main()
\end{minted}
\VALIDATION{Exactly 3 asserts: $(3,[2,3,5],7)\to 8$, $(1,[2],10)\to 512$, $(2,[3,5],1)\to 1$. Additional small-range cross-checks included under self-test mode.}
\RESULT{The minimal $X$ with $C(X)\ge k$ is printed; ties cannot occur since $C$ increases exactly at elements of $S$.}
\NotePages{3}

% ============ 7. Testing & Final Reference Implementation (own page) ============
\section{Testing \& Final Reference Implementation}
\LINE{TEST PLAN}{Verify against heap-based generator for small $n,k$; random subsets of small primes; boundary checks like single prime, $k=1$, and primes including 2.}
\LINE{CROSS-CHECKS}{Compare Approach A vs Approach C for small ranges; ensure both yield identical sequences for first 50 terms.}
\LINE{EDGE-CASE GENERATOR}{Use deterministic seeds to generate small prime subsets and small $k$ values; verify counting monotonicity and binary search invariants.}
\begin{minted}{python}
# Deterministic generators for boundaries, degenerates, adversarials
def generate_prime_subsets():
    # Small primes pool
    primes = [2,3,5,7,11,13,17,19]
    cases = []
    # Single prime
    cases.append(([2], [1,2,3,10]))
    # Two primes
    cases.append(([3,5], [1,2,3,6,10,15,20]))
    # Three primes
    cases.append(([2,3,5], list(range(1,31))))
    # First 4 primes
    cases.append(([2,3,5,7], [1,5,10,20,50,100]))
    return cases

def sequence_kth(primes, K):
    import heapq
    LIMIT = 10**18
    h, seen, out = [1], {1}, []
    while len(out) < K:
        x = heapq.heappop(h)
        out.append(x)
        for p in primes:
            nx = x * p
            if nx <= LIMIT and nx not in seen:
                seen.add(nx)
                heapq.heappush(h, nx)
    return out

def test_cross_check():
    # Use final solver
    def solve_final(primes, k):
        n = len(primes)
        return solve_case(n, primes, k)
    for primes, ks in generate_prime_subsets():
        seq = sequence_kth(primes, max(ks))
        for k in ks:
            assert solve_final(primes, k) == seq[k-1]
\end{minted}
\textbf{Reference Code (Ready to Submit)}
\begin{minted}{python}
# Final CF submission: Meet-in-the-Middle + Binary Search
import sys

LIMIT = 10**18

def read_input(data=None):
    if data is None:
        data = sys.stdin.read()
    data = data.strip().split()
    if not data:
        return None
    it = iter(map(int, data))
    n = next(it)
    primes = [next(it) for _ in range(n)]
    k = next(it)
    return n, primes, k

def gen_products(primes, limit=LIMIT):
    res = []
    def dfs(i, cur):
        if i == len(primes):
            res.append(cur)
            return
        p = primes[i]
        val = cur
        while val <= limit:
            dfs(i + 1, val)
            if val > limit // p:
                break
            val *= p
    dfs(0, 1)
    res.sort()
    return res

def count_leq_with_lists(A, B, X):
    cnt = 0
    j = len(B) - 1
    for a in A:
        if a > X:
            break
        while j >= 0 and a * B[j] > X:
            j -= 1
        if j < 0:
            break
        cnt += (j + 1)
    return cnt

def solve_case(n, primes, k):
    m = n // 2
    left = primes[:m]
    right = primes[m:]
    A = gen_products(left, LIMIT)
    B = gen_products(right, LIMIT)
    lo, hi = 1, LIMIT
    while lo < hi:
        mid = (lo + hi) // 2
        if count_leq_with_lists(A, B, mid) >= k:
            hi = mid
        else:
            lo = mid + 1
    return lo

def solve_all(n, primes, k):
    return solve_case(n, primes, k)

def main():
    parsed = read_input()
    if parsed is None:
        # Minimal self-checks
        assert solve_case(3, [2,3,5], 7) == 8
        assert solve_case(1, [2], 10) == 512
        assert solve_case(2, [3,5], 1) == 1
        print("OK")
    else:
        n, primes, k = parsed
        print(solve_case(n, primes, k))

if __name__ == "__main__":
    main()
\end{minted}
\NotePages{3}

% ============ 8. Review & Pitfalls (own page) ============
\section{Review \& Pitfalls}
\WHAT{Compute the $k$-th number whose prime divisors lie in a given set; solve via count-and-search using meet-in-the-middle.}
\WHY{Classic smooth-number ranking; tests ability to design efficient counting, binary search, and meet-in-the-middle enumeration.}
\CHECKLIST{
- Include $1$ in the sequence.

- Precompute half-products up to $10^{18}$.

- Count pairs with two-pointer.

- Binary search minimal $X$ with $C(X)\ge k$.

- Mind overflow checks with $10^{18}$.}
\EDGECASES{
- $k=1\Rightarrow$ answer is $1$.

- Single prime $[p]$: answer is $p^{k-1}$ bounded by $10^{18}$.

- Presence of $2$ creates many values; ensure generation remains capped by limit.

- Large $k$ but bounded answer: binary search remains fast.

- Duplicate products avoided inherently by fixed halves.

- Primes already sorted; no reliance on sorting needed.}
\PITFALLS{
- Forgetting to include $1$ in both half lists, undercounting.

- Not guarding multiplication overflow (use division check before multiply).

- Using per-iteration sort or rebuild of lists inside binary search.

- Off-by-one in binary search loop conditions.

- Using linear search in $B$ per $a$ instead of two-pointer/binary search.

- Memory blow-up by storing all pairwise products (do not).}
\FAILMODES{Naive heap for very large $k$ can TLE; DFS-count without meet-in-the-middle can explode for small primes. The MITM approach precomputes manageable lists and counts in $O(|A|+|B|)$.}
\ELI{Make all numbers from the left half of primes and from the right half; any answer is a product of one from each half. To see if $X$ is big enough to contain $k$ numbers, count how many pairs fit under $X$, then hone in on the smallest such $X$ by binary searching.}
\NotePages{3}

\end{document}