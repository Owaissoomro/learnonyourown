% !TeX program = xelatex
\documentclass[11pt]{article}
\usepackage[margin=1in]{geometry}
\usepackage{microtype}
\usepackage{setspace}\setstretch{1.05}
\usepackage{fontspec}
\setmainfont{Latin Modern Roman}
\setmonofont{Latin Modern Mono}[BoldFont={Latin Modern Mono},ItalicFont={Latin Modern Mono}]
\usepackage{amsmath,mathtools,amsthm}
\newtheorem{lemma}{Lemma}
\newtheorem{theorem}{Theorem}
\usepackage{unicode-math}\setmathfont{Latin Modern Math}\allowdisplaybreaks[4]
% --- overflow and alignment slack ---
\setlength{\jot}{7pt}
\sloppy\emergencystretch=8em\hfuzz=1pt\vfuzz=2pt\raggedbottom
% --- breakable math helpers ---
\newenvironment{BreakableEquation}{\begin{equation}\begin{aligned}}{\end{aligned}\end{equation}}
\newenvironment{BreakableEquation*}{\begin{equation*}\begin{aligned}}{\end{aligned}\end{equation*}}
\newenvironment{tightalign}{\begingroup\small\allowdisplaybreaks\begin{align}}{\end{align}\endgroup}

\usepackage{xcolor}
\usepackage{xurl} % better URL wrapping
\usepackage[colorlinks=true,linkcolor=black,urlcolor=blue]{hyperref}
\usepackage{fancyhdr}
\pagestyle{fancy}\fancyhf{}\lhead{\nouppercase{\leftmark}}\rhead{\thepage}\setlength{\headheight}{26pt}
\usepackage{enumitem,booktabs,tabularx}
\newcolumntype{L}{>{\raggedright\arraybackslash}X}
\renewcommand{\arraystretch}{1.1}
\setlength{\parindent}{0pt}\setlength{\parskip}{8pt plus 2pt minus 1pt}

% Listings + upquote (no shell-escape needed)
\usepackage{listings}
\usepackage{upquote}
\lstdefinestyle{crisp}{
  basicstyle=\ttfamily\footnotesize,
  frame=single,
  breaklines=true,
  breakatwhitespace=false, % allow breaks inside long tokens
  tabsize=4,
  showstringspaces=false,
  numbers=left,
  numberstyle=\tiny,
  numbersep=8pt,
  keepspaces=true,
  columns=fullflexible,
  backgroundcolor=\color{black!02},
  aboveskip=8pt,
  belowskip=8pt
}
% Guarantee that 'python' exists as a language for listings
\lstdefinelanguage{python}{
  morekeywords={def,return,class,if,elif,else,for,while,try,except,raise,assert,pass,break,continue,lambda,nonlocal,global,yield,import,from,as,with,True,False,None},
  sensitive=true,
  morecomment=[l]\#,
  morestring=[b]",
  morestring=[b]'
}
% minted shim (robust; no shell-escape; uses listings' own environment)
\lstnewenvironment{minted}[2][]{\lstset{style=crisp,language=#2,#1}}{}

\usepackage[most]{tcolorbox}
\tcbset{colback=white,colframe=black!15,boxrule=0.4pt,arc=2pt,left=6pt,right=6pt,top=6pt,bottom=6pt,before skip=10pt,after skip=10pt,breakable}
\usepackage{titlesec}
\titleformat{\section}{\large\bfseries}{\thesection}{0.6em}{}
\titlespacing*{\section}{0pt}{*2.0}{*1.0}

% ===== CONSISTENCY TOOLKIT (macros) =====
\newlist{ledger}{enumerate}{1}
\setlist[ledger]{label=\textbullet,leftmargin=2em,itemsep=2pt,topsep=6pt}
\newcommand{\LEDGER}[1]{\textbf{ARITHMETIC LEDGER:}\par\begin{ledger}#1\end{ledger}}

\newlist{algosteps}{enumerate}{1}
\setlist[algosteps]{label=\arabic*.,leftmargin=2em,itemsep=2pt,topsep=6pt}

\newlist{bullets}{itemize}{1}
\setlist[bullets]{label=--,leftmargin=1.6em,itemsep=2pt,topsep=4pt}

\newcommand{\varmapStart}{\textbf{VARIABLE MAPPING:}\par\begin{bullets}}
\newcommand{\var}[2]{\item $#1$ — #2}
\newcommand{\varmapEnd}{\end{bullets}}

\newcommand{\LINE}[2]{\par\noindent\textbf{#1:}~#2\par}
\newcommand{\WHAT}[1]{\LINE{WHAT}{#1}}
\newcommand{\WHY}[1]{\LINE{WHY}{#1}}
\newcommand{\HOW}[1]{\LINE{HOW}{#1}}
\newcommand{\ELI}[1]{\LINE{ELI5}{#1}}
\newcommand{\STATEMENT}[1]{\LINE{STATEMENT}{#1}}
\newcommand{\BREAKDOWN}[1]{\LINE{PROBLEM BREAKDOWN}{#1}}
\newcommand{\MODEL}[1]{\LINE{CANONICAL MATHEMATICAL MODEL}{#1}}
\newcommand{\ASSUMPTIONS}[1]{\LINE{ASSUMPTIONS}{#1}}
\newcommand{\INVARIANTS}[1]{\LINE{INVARIANTS}{#1}}
\newcommand{\WHICHFORMULA}[1]{\LINE{WHICH FORMULA \& WHY}{#1}}
\newcommand{\GOVERN}[1]{\LINE{GOVERNING EQUATION(S)}{#1}}
\newcommand{\INPUTS}[1]{\LINE{INPUTS}{#1}}
\newcommand{\OUTPUTS}[1]{\LINE{OUTPUTS}{#1}}
\newcommand{\SAMPLES}[1]{\LINE{SAMPLES}{#1}}
\newcommand{\RESULT}[1]{\LINE{RESULT}{#1}}
\newcommand{\COMPLEXITY}[1]{\LINE{TIME/SPACE COMPLEXITY}{#1}}
\newcommand{\MEMORY}[1]{\LINE{MEMORY FOOTPRINT}{#1}}
\newcommand{\CORRECTNESS}[1]{\LINE{CORRECTNESS}{#1}}
\newcommand{\OPTIMALITY}[1]{\LINE{OPTIMALITY}{#1}}
\newcommand{\FAILMODES}[1]{\LINE{FAILURE MODES}{#1}}
\newcommand{\VALIDATION}[1]{\LINE{VALIDATION}{#1}}
\newcommand{\UNITCHECK}[1]{\LINE{UNIT CHECK}{#1}}
\newcommand{\EDGECASES}[1]{\LINE{EDGE CASES}{#1}}
\newcommand{\CHECKLIST}[1]{\LINE{CHECKLIST}{#1}}
\newcommand{\DERIV}[1]{\LINE{DERIVATION}{#1}}
\newcommand{\LEMMAHEAD}[1]{\LINE{SUPPORTING LEMMA}{#1}}
\newcommand{\PITFALLS}[1]{\LINE{PITFALLS}{#1}}

\usepackage{etoolbox}\pretocmd{\section}{\clearpage}{}{}

\newcommand{\FormulaPage}[2]{%
  \clearpage
  \section*{Formula #1 — #2}%
  \addcontentsline{toc}{section}{Formula #1 — #2}%
}
\newcommand{\ApproachPage}[2]{%
  \clearpage
  \subsection*{Approach #1 — #2}%
  \addcontentsline{toc}{subsection}{Approach #1 — #2}%
}

\begin{document}
\title{Interview Problem Sheet — Heap Operations}
\date{\today}
\author{}
\maketitle
\tableofcontents
\clearpage

% === Notes macro (BODY-ONLY; do not alter the preamble) ===
% Prints exactly N blank pages with empty headers/footers, then leaves you on the last blank page.
\newcommand{\NotePages}[1]{%
  \clearpage
  \begingroup
  \newcount\NPi \newcount\NPn
  \NPi=0 \NPn=#1
  \loop\ifnum\NPi<\NPn
    \advance\NPi by 1
    \mbox{}\thispagestyle{empty}%
    \ifnum\NPi<\NPn\clearpage\fi
  \repeat
  \endgroup
}

% ============ 1. Problem & Metadata (own page) ============
\section{Problem \& Metadata}
\LINE{PLATFORM}{CF}
\LINE{URL}{\url{https://codeforces.com/problemset/problem/681/C}}
\LINE{DIFFICULTY / RATING}{1600}
\STATEMENT{Petya has recently learned data structure named ``Binary heap''.

The heap he is now operating with allows the following operations:
\begin{bullets}
\item put the given number into the heap;
\item get the value of the minimum element in the heap;
\item extract the minimum element from the heap;
\end{bullets}

Thus, at any moment of time the heap contains several integers (possibly none), some of them might be equal.

In order to better learn this data structure Petya took an empty heap and applied some operations above to it. Also, he carefully wrote down all the operations and their results to his event log, following the format:
\begin{bullets}
\item insert x — put the element with value x in the heap;
\item getMin x — the value of the minimum element contained in the heap was equal to x;
\item removeMin — the minimum element was extracted from the heap (only one instance, if there were many).
\end{bullets}

All the operations were correct, i.e. there was at least one element in the heap each time getMin or removeMin operations were applied.

While Petya was away for a lunch, his little brother Vova came to the room, took away some of the pages from Petya's log and used them to make paper boats.

Now Vova is worried, if he made Petya's sequence of operations inconsistent. For example, if one applies operations one-by-one in the order they are written in the event log, results of getMin operations might differ from the results recorded by Petya, and some of getMin or removeMin operations may be incorrect, as the heap is empty at the moment they are applied.

Now Vova wants to add some new operation records to the event log in order to make the resulting sequence of operations correct. That is, the result of each getMin operation is equal to the result in the record, and the heap is non-empty when getMin and removeMin are applied. Vova wants to complete this as fast as possible, as Petya may get back at any moment. He asks you to add the least possible number of operation records to the current log. Note that an arbitrary number of operations may be added at the beginning, between any two other operations, or at the end of the log.

Input:
The first line of the input contains the only integer $n$ ($1 \le n \le 100{,}000$) — the number of the records left in Petya's journal.

Each of the following $n$ lines describes the records in the current log in the order they are applied. Format described in the statement is used. All numbers in the input are integers not exceeding $10^9$ by their absolute value.

Output:
The first line of the output should contain a single integer $m$ — the minimum possible number of records in the modified sequence of operations.

Next $m$ lines should contain the corrected sequence of records following the format of the input (described in the statement), one per line and in the order they are applied. All the numbers in the output should be integers not exceeding $10^9$ by their absolute value.

Note that the input sequence of operations must be the subsequence of the output sequence.

It is guaranteed that there exists the correct answer consisting of no more than $1{,}000{,}000$ operations.

Note:
In the first sample, after number $3$ is inserted into the heap, the minimum number is $3$. To make the result of the first getMin equal to $4$ one should first remove number $3$ from the heap and then add number $4$ into the heap.

In the second sample case number $1$ is inserted two times, so it should be similarly removed twice.}
\BREAKDOWN{We must minimally augment a partially observed sequence of heap operations so that all getMin and removeMin steps are valid and every getMin returns the recorded value. We can simulate a min-heap and insert/remove as needed before each recorded operation.}
\ELI{Before each logged command, fix the heap by removing too-small elements or inserting the needed value, and if a removal is requested on an empty heap, insert a dummy first.}
\NotePages{3}

% ============ 2. IO Contract (own page) ============
\section{IO Contract}
\INPUTS{Single test case. First line: integer $n$ ($1 \le n \le 100{,}000$). Next $n$ lines each is one of:
\begin{bullets}
\item ``insert x'' with integer $x$ ($|x| \le 10^9$),
\item ``getMin x'' with integer $x$ ($|x| \le 10^9$),
\item ``removeMin''.
\end{bullets}}
\OUTPUTS{First line: integer $m$ — minimal total number of operations in corrected log. Next $m$ lines: a valid operation sequence containing the original $n$ operations as a subsequence and satisfying all semantics.}
\SAMPLES{Example 1:
\begin{tcolorbox}
Input
\begin{verbatim}
5
insert 3
getMin 4
removeMin
getMin 4
removeMin
\end{verbatim}
Output
\begin{verbatim}
8
insert 3
removeMin
insert 4
getMin 4
removeMin
insert 4
getMin 4
removeMin
\end{verbatim}
\end{tcolorbox}
Example 2:
\begin{tcolorbox}
Input
\begin{verbatim}
3
getMin 1
getMin 1
removeMin
\end{verbatim}
Output
\begin{verbatim}
5
insert 1
getMin 1
getMin 1
removeMin
removeMin
\end{verbatim}
\end{tcolorbox}}
\NotePages{3}

% ============ 3. Canonical Mathematical Model (own page) ============
\section{Canonical Mathematical Model}
\MODEL{Let $H$ be a multiset representing the heap. We are given a sequence $S$ of $n$ operations over $\{\text{insert}(x), \text{getMin}(x), \text{removeMin}\}$. We must produce a supersequence $T$ of minimal length such that applying $T$ to $H=\varnothing$ yields that $S$ is a subsequence of $T$, every $\text{getMin}(x)$ observes $\min(H)=x$, and $\text{removeMin}$ occurs only when $H \ne \varnothing$.}
\varmapStart
\var{H}{current multiset (heap) state}
\var{S}{given sequence of length $n$}
\var{T}{output sequence, to be minimized in length}
\var{x}{integer argument for operations where applicable}
\varmapEnd
\GOVERN{
\[
\begin{aligned}
&\text{At each step before a recorded op:} \\
&\quad \text{if op}=\text{getMin}(x):~
\text{while }H\ne\varnothing\text{ and }\min(H)<x:~H\leftarrow H\setminus\{\min(H)\};\\
&\qquad \text{if }H=\varnothing\text{ or }\min(H)>x:~H\leftarrow H\cup\{x\};\\
&\quad \text{if op}=\text{removeMin}:~
\text{if }H=\varnothing:~H\leftarrow H\cup\{0\};~H\leftarrow H\setminus\{\min(H)\}.\\
&\text{Record each fix-up as operations, then the original op.}
\end{aligned}
\]
}
\ASSUMPTIONS{Heap is a min-heap over integers; duplicates allowed. Insert and extract operate on single elements. Any fixed dummy value (e.g., $0$) is valid on emergency insert because only relative order matters.}
\INVARIANTS{
\begin{bullets}
\item Before recording any $\text{getMin}(x)$, the heap satisfies $\min(H)=x$.
\item Before recording any $\text{removeMin}$, $H\ne\varnothing$.
\item The original sequence $S$ appears as a subsequence of $T$ in order.
\end{bullets}}
\NotePages{3}

% ============ 4. Approach A — Baseline (own page) ============
\section{Approach A — Baseline}
\ApproachPage{A}{Brute Force / Direct}
\WHICHFORMULA{Simulate the heap as a sorted list (multiset). Fix inconsistencies just-in-time before each logged operation.}
\ASSUMPTIONS{Maintain $H$ as a sorted Python list; use linear removals and binary insertion. Simplicity over performance.}
\textbf{Algorithm Steps}
\begin{algosteps}
\item Initialize empty multiset $H$ (sorted list) and empty output list $T$.
\item For each recorded op:
  \begin{bullets}
  \item If getMin $x$: remove all elements $<x$; if empty or current min $>x$, insert $x$; then append getMin $x$.
  \item If removeMin: if empty, insert dummy $0$; then remove the minimum; append removeMin.
  \item If insert $x$: insert $x$; append insert $x$.
  \end{bullets}
\item Print $m=|T|$ and the lines of $T$.
\end{algosteps}
\COMPLEXITY{Let $n$ be the number of input operations and let $k$ be the number of added operations. Using a sorted list:
\[
\begin{aligned}
T(n) &\le O\big((n+k)\cdot n\big)\quad\text{(worst case due to $O(n)$ deletions from front and insertions)}\\
S(n) &= O(n+k).
\end{aligned}
\]
}
\CORRECTNESS{The fix-up loop ensures no element smaller than $x$ remains before getMin $x$, so the minimum is at least $x$. If it exceeds $x$ or $H$ is empty, inserting $x$ makes the minimum exactly $x$. For removeMin, pre-insert ensures non-emptiness. We append the original op after fixes, so the input is a subsequence.}
\EDGECASES{Empty heap on removeMin; empty heap on getMin; multiple equal values; negative values; long runs of decreasing getMin values causing many removals.}
\textbf{Code (Baseline)}
\begin{minted}{python}
# CF TEMPLATE (must include read_input / solve_case / main + guard + asserts)
from bisect import bisect_left, insort

def read_input(lines):
    it = iter(lines)
    n = int(next(it).strip())
    ops = []
    for _ in range(n):
        s = next(it).strip()
        parts = s.split()
        if parts[0] == "insert":
            ops.append(("insert", int(parts[1])))
        elif parts[0] == "getMin":
            ops.append(("getMin", int(parts[1])))
        else:
            ops.append(("removeMin", None))
    return ops

def solve_all(ops):
    arr = []  # sorted list as multiset
    out = []
    def remove_min():
        # remove smallest element if exists
        if arr:
            arr.pop(0)
    def insert_x(x):
        insort(arr, x)
        out.append(f"insert {x}")
    for typ, val in ops:
        if typ == "insert":
            insert_x(val)
        elif typ == "getMin":
            x = val
            # remove all < x
            while arr and arr[0] < x:
                remove_min()
                out.append("removeMin")
            # ensure min == x
            if not arr or arr[0] > x:
                insert_x(x)
            out.append(f"getMin {x}")
        else:  # removeMin
            if not arr:
                insert_x(0)
            remove_min()
            out.append("removeMin")
    return out

def main():
    import sys
    data = sys.stdin.read().splitlines()
    if not data:
        # basic tests when no stdin provided
        def run_io(inp):
            lines = inp.strip().splitlines()
            ops = read_input(lines)
            out = solve_all(ops)
            return str(len(out)) + "\n" + "\n".join(out) + "\n"
        # Test 1
        inp1 = """
5
insert 3
getMin 4
removeMin
getMin 4
removeMin
""".strip()
        out1 = run_io(inp1)
        exp1 = """
8
insert 3
removeMin
insert 4
getMin 4
removeMin
insert 4
getMin 4
removeMin
""".lstrip()
        assert out1 == exp1
        # Test 2
        inp2 = """
3
getMin 1
getMin 1
removeMin
""".strip()
        out2 = run_io(inp2)
        exp2 = """
5
insert 1
getMin 1
getMin 1
removeMin
removeMin
""".lstrip()
        assert out2 == exp2
        # Test 3: consecutive removeMin on empty
        inp3 = """
2
removeMin
removeMin
""".strip()
        out3 = run_io(inp3)
        # Should be 4 operations: insert 0, remove; insert 0, remove
        assert out3.splitlines()[0] == "4"
        print("OK")
    else:
        ops = read_input(data)
        out = solve_all(ops)
        sys.stdout.write(str(len(out)) + "\n")
        sys.stdout.write("\n".join(out) + "\n")

if __name__ == "__main__":
    main()
\end{minted}
\VALIDATION{Checked that:
\begin{bullets}
\item getMin on larger value removes smaller ones, then inserts target.
\item removeMin on empty first inserts dummy $0$.
\item Input appears as a subsequence in output.
\end{bullets}}
\NotePages{3}

% ============ 5. Approach B — Improved (own page) ============
\section{Approach B — Improved}
\ApproachPage{B}{Optimized Data Structure / Pruning / DP}
\WHICHFORMULA{Use a binary min-heap for $O(\log n)$ insertions/removals. The same fix-up logic applies, but operations are efficient.}
\ASSUMPTIONS{Python \texttt{heapq} as the min-heap. We use a fixed dummy value, e.g., $0$, for emergency inserts.}
\textbf{Algorithm Steps}
\begin{algosteps}
\item Maintain a min-heap $H$.
\item For getMin $x$: while $H$ non-empty and $\min(H)<x$, pop and record removeMin. If empty or $\min(H)>x$, push $x$ and record insert. Then record getMin $x$.
\item For removeMin: if empty, push $0$ and record insert. Then pop and record removeMin.
\item For insert $x$: push $x$ and record insert $x$.
\end{algosteps}
\COMPLEXITY{Each element is inserted and removed at most once during fix-ups in addition to logged inserts/removes.
\[
\begin{aligned}
T(n) &= O\big((n+k)\log(n+k)\big),\\
S(n) &= O(n+k).
\end{aligned}
\]
Here $k$ is the number of added operations in the minimal fix.}
\CORRECTNESS{Identical reasoning as baseline, with the heap property guaranteeing correctness of comparisons and removals. The while-loop ensures no smaller value than $x$ remains before a getMin $x$.}
\textbf{Code (Improved)}
\begin{minted}{python}
# Same API/signature as baseline; deterministic; with asserts
import heapq

def read_input(lines):
    it = iter(lines)
    n = int(next(it).strip())
    ops = []
    for _ in range(n):
        s = next(it).strip()
        parts = s.split()
        if parts[0] == "insert":
            ops.append(("insert", int(parts[1])))
        elif parts[0] == "getMin":
            ops.append(("getMin", int(parts[1])))
        else:
            ops.append(("removeMin", None))
    return ops

def solve_all(ops):
    h = []
    out = []
    def insert_x(x):
        heapq.heappush(h, x)
        out.append(f"insert {x}")
    for typ, val in ops:
        if typ == "insert":
            insert_x(val)
        elif typ == "getMin":
            x = val
            while h and h[0] < x:
                heapq.heappop(h)
                out.append("removeMin")
            if not h or h[0] > x:
                insert_x(x)
            out.append(f"getMin {x}")
        else:  # removeMin
            if not h:
                insert_x(0)
            heapq.heappop(h)
            out.append("removeMin")
    return out

def main():
    import sys
    data = sys.stdin.read().splitlines()
    if not data:
        def run_io(inp):
            lines = inp.strip().splitlines()
            ops = read_input(lines)
            out = solve_all(ops)
            return str(len(out)) + "\n" + "\n".join(out) + "\n"
        # Tests
        out1 = run_io("1\nremoveMin\n")
        assert out1.splitlines() == ["2", "insert 0", "removeMin"]
        out2 = run_io("2\ninsert 5\ngetMin 5\n")
        assert out2.splitlines() == ["2", "insert 5", "getMin 5"]
        out3 = run_io("3\ninsert 3\ngetMin 4\nremoveMin\n")
        lines = out3.splitlines()
        assert lines[0] == "5" and lines[1] == "insert 3" and lines[2] == "removeMin"
        print("OK")
    else:
        ops = read_input(data)
        out = solve_all(ops)
        sys.stdout.write(str(len(out)) + "\n")
        sys.stdout.write("\n".join(out) + "\n")

if __name__ == "__main__":
    main()
\end{minted}
\VALIDATION{Edge checks:
\begin{bullets}
\item Multiple getMin with same $x$ do not cause duplicate inserts if $x$ is already at top.
\item Cascading removals when current min $<x$.
\item removeMin on empty first inserts dummy.
\end{bullets}}
\NotePages{3}

% ============ 6. Approach C — Optimal (own page) ============
\section{Approach C — Optimal}
\ApproachPage{C}{Provably Minimal Edit via Greedy Heap Simulation}
\WHICHFORMULA{Greedy simulation with a min-heap. Before each logged operation, do the least number of edits to satisfy its preconditions. This strategy is optimal because any smaller edits would violate the operation semantics.}
\ASSUMPTIONS{Standard min-heap; integers within $|x|\le 10^9$. Dummy value for emergency insert can be any integer (choose $0$).}
\textbf{Algorithm Steps}
\begin{algosteps}
\item Initialize empty heap $H$ and output list $T$.
\item For each logged operation:
\begin{bullets}
\item insert $x$: push $x$; append ``insert $x$''.
\item getMin $x$: while $H$ non-empty and $\min(H)<x$, pop/removeMin; if empty or $\min(H)>x$, push $x$/insert $x$; append ``getMin $x$''.
\item removeMin: if $H$ empty, insert $0$; then pop; append ``removeMin''.
\end{bullets}
\item Print $|T|$ and $T$.
\end{algosteps}
\OPTIMALITY{For getMin $x$, any valid state must have no elements $<x$ and at least one $x$. Removing all $<x$ and inserting $x$ if absent is the unique minimal fix. For removeMin on empty, inserting exactly one element is the unique minimal fix. Hence the greedy edits are minimal per step and globally minimal.}
\COMPLEXITY{
\[
\begin{aligned}
T(n) &= O\!\left((n+k)\log(n+k)\right),\\
S(n) &= O(n+k),
\end{aligned}
\]
where $k$ is the number of added operations (bounded by $1{,}000{,}000 - n$ by problem guarantee).}
\textbf{Code (Final Submission)}
\begin{minted}{python}
# CF: read_input(), solve_case()/solve_all(), main()+guard + asserts
import sys, heapq

def read_input(lines):
    it = iter(lines)
    n = int(next(it).strip())
    ops = []
    for _ in range(n):
        s = next(it).strip()
        parts = s.split()
        if parts[0] == "insert":
            ops.append(("insert", int(parts[1])))
        elif parts[0] == "getMin":
            ops.append(("getMin", int(parts[1])))
        else:
            ops.append(("removeMin", None))
    return ops

def solve_all(ops):
    h = []
    out = []
    def insert_x(x):
        heapq.heappush(h, x)
        out.append(f"insert {x}")
    for typ, val in ops:
        if typ == "insert":
            insert_x(val)
        elif typ == "getMin":
            x = val
            while h and h[0] < x:
                heapq.heappop(h)
                out.append("removeMin")
            if not h or h[0] > x:
                insert_x(x)
            out.append(f"getMin {x}")
        else:  # removeMin
            if not h:
                insert_x(0)
            heapq.heappop(h)
            out.append("removeMin")
    return out

def main():
    data = sys.stdin.read().splitlines()
    if not data:
        # Self-checks when no stdin provided
        def run_io(inp):
            lines = inp.strip().splitlines()
            ops = read_input(lines)
            out = solve_all(ops)
            return str(len(out)) + "\n" + "\n".join(out) + "\n"
        # Exactly 3 asserts
        # 1) Minimal fix for getMin on empty
        assert run_io("1\ngetMin 7\n") == "2\ninsert 7\ngetMin 7\n"
        # 2) Remove too-small then enforce target
        assert run_io("2\ninsert 3\ngetMin 4\n").splitlines()[:3] == ["3", "insert 3", "removeMin"]
        # 3) Two removeMin on empty => two inserts
        assert run_io("2\nremoveMin\nremoveMin\n").splitlines()[0] == "4"
        # Print OK so local runs show success
        print("OK")
    else:
        ops = read_input(data)
        out = solve_all(ops)
        sys.stdout.write(str(len(out)) + "\n")
        sys.stdout.write("\n".join(out) + "\n")

if __name__ == "__main__":
    main()
\end{minted}
\VALIDATION{Exactly 3 asserts included focusing on:
\begin{bullets}
\item getMin on empty inserts target then gets it.
\item Removal of smaller elements before getMin.
\item Consecutive removeMin on empty creates required inserts.
\end{bullets}}
\RESULT{Produces a minimal-length corrected operation sequence that preserves the input as a subsequence and satisfies heap semantics. Ties do not arise; the sequence is uniquely determined by the greedy fix-ups.}
\NotePages{3}

% ============ 7. Testing & Final Reference Implementation (own page) ============
\section{Testing \& Final Reference Implementation}
\LINE{TEST PLAN}{Unit tests for emptiness handling, repeated targets, and chains of increasing/decreasing getMin values. Property check: the simulated heap under the output sequence must never violate preconditions before any recorded operation.}
\LINE{CROSS-CHECKS}{Compare outputs of Baseline vs Improved vs Final on small random cases; ensure both produce identical sequences and lengths.}
\LINE{EDGE-CASE GENERATOR}{Generate sequences with:
\begin{bullets}
\item Starts with removeMin or getMin.
\item Long runs of getMin with descending values.
\item Interleaved inserts and removes.
\item Large absolute values and negatives.
\end{bullets}}
\begin{minted}{python}
# Deterministic generators for boundaries, degenerates, adversarials
import random

def gen_case(seed=0, n=50):
    random.seed(seed)
    ops = []
    for _ in range(n):
        t = random.randrange(3)
        if t == 0:
            x = random.randint(-10, 10)
            ops.append(("insert", x))
        elif t == 1:
            x = random.randint(-10, 10)
            ops.append(("getMin", x))
        else:
            ops.append(("removeMin", None))
    return ops

def ops_to_text(ops):
    out = [str(len(ops))]
    for typ, val in ops:
        if typ == "insert":
            out.append(f"insert {val}")
        elif typ == "getMin":
            out.append(f"getMin {val}")
        else:
            out.append("removeMin")
    return "\n".join(out) + "\n"

# Cross-check driver (baseline vs final) for local use
if __name__ == "__main__":
    from copy import deepcopy
    import heapq
    def solve_heap(ops):
        # final method
        h = []
        out = []
        def ins(x):
            heapq.heappush(h, x)
            out.append(f"insert {x}")
        for typ, val in ops:
            if typ == "insert":
                ins(val)
            elif typ == "getMin":
                x = val
                while h and h[0] < x:
                    heapq.heappop(h)
                    out.append("removeMin")
                if not h or h[0] > x:
                    ins(x)
                out.append(f"getMin {x}")
            else:
                if not h:
                    ins(0)
                heapq.heappop(h)
                out.append("removeMin")
        return out
    for s in range(5):
        ops = gen_case(s, 40)
        out = solve_heap(ops)
        # consistency checks
        assert len(out) >= len(ops)
        # subsequence check
        it = iter(out)
        for typ, val in ops:
            target = f"{typ} {val}" if val is not None else typ
            for line in it:
                if line == target:
                    break
            else:
                raise AssertionError("Subsequence not preserved")
    print("Cross-checks OK")
\end{minted}
\textbf{Reference Code (Ready to Submit)}
\begin{minted}{python}
# Single, final reference solution with the required API (CF/LC) + asserts
import sys, heapq

def read_input(lines):
    it = iter(lines)
    n = int(next(it).strip())
    ops = []
    for _ in range(n):
        s = next(it).strip()
        parts = s.split()
        if parts[0] == "insert":
            ops.append(("insert", int(parts[1])))
        elif parts[0] == "getMin":
            ops.append(("getMin", int(parts[1])))
        else:
            ops.append(("removeMin", None))
    return ops

def solve_all(ops):
    h = []
    out = []
    def insert_x(x):
        heapq.heappush(h, x)
        out.append(f"insert {x}")
    for typ, val in ops:
        if typ == "insert":
            insert_x(val)
        elif typ == "getMin":
            x = val
            while h and h[0] < x:
                heapq.heappop(h)
                out.append("removeMin")
            if not h or h[0] > x:
                insert_x(x)
            out.append(f"getMin {x}")
        else:
            if not h:
                insert_x(0)
            heapq.heappop(h)
            out.append("removeMin")
    return out

def main():
    data = sys.stdin.read().splitlines()
    if not data:
        # Local asserts when no stdin
        def run_io(inp):
            ops = read_input(inp.strip().splitlines())
            out = solve_all(ops)
            return str(len(out)) + "\n" + "\n".join(out) + "\n"
        a1 = run_io("1\ngetMin 5\n")
        assert a1 == "2\ninsert 5\ngetMin 5\n"
        a2 = run_io("1\nremoveMin\n").splitlines()
        assert a2 == ["2", "insert 0", "removeMin"]
        a3 = run_io("2\ninsert -1\ngetMin 0\n").splitlines()
        assert a3[0] == "3" and a3[1] == "insert -1" and a3[2] == "removeMin"
        print("OK")
    else:
        ops = read_input(data)
        out = solve_all(ops)
        sys.stdout.write(str(len(out)) + "\n")
        sys.stdout.write("\n".join(out) + "\n")

if __name__ == "__main__":
    main()
\end{minted}
\NotePages{3}

% ============ 8. Review & Pitfalls (own page) ============
\section{Review \& Pitfalls}
\WHAT{Greedy heap simulation: before each logged op, minimally fix the heap to satisfy its preconditions and exact getMin value.}
\WHY{A classic edit-to-consistency task tests understanding of heaps, simulation, and greedy minimality under constraints. Common in CF and interviews.}
\CHECKLIST{
\begin{bullets}
\item Is the heap empty before removeMin? Insert dummy.
\item Is current min less than target $x$ before getMin? Remove mins until $\ge x$.
\item Is heap empty or min greater than $x$ before getMin? Insert $x$.
\item Append the original op after fixes to preserve subsequence.
\item Use a min-heap; avoid extra operations.
\end{bullets}}
\EDGECASES{
\begin{bullets}
\item First operation is removeMin.
\item First operation is getMin $x$.
\item Many consecutive getMin with the same $x$.
\item Decreasing sequence of getMin values causing many removals.
\item Negative values and zeros.
\item Large absolute values near $10^9$.
\item Long runs of removeMin exceeding inserts.
\item Duplicate inserts then removes.
\item Interleaved inserts of different values around getMin.
\item Empty input lines are not present by spec, but robust parsers should be strict.
\end{bullets}}
\PITFALLS{
\begin{bullets}
\item Forgetting to output the original operation after fix-ups.
\item Inserting $x$ even when min already equals $x$ (adds unnecessary ops).
\item Popping only once when multiple smaller elements exist.
\item Using a max-heap by mistake.
\item Mishandling signs or parsing integers.
\item Emitting outputs exceeding the guaranteed bound due to a bug.
\item Using unstable checks that reorder or drop original operations.
\item Off-by-one in counting output operations.
\end{bullets}}
\FAILMODES{Baseline sorted-list implementation may time out for maximal $n$, but the heap-based method runs in $O((n+k)\log(n+k))$ and respects the $1{,}000{,}000$ bound. Any approach that tries to defer removals or batch them without checking getMin values can violate semantics.}
\ELI{Keep the heap honest before every logged action. If a getMin asks for $x$, throw away all smaller numbers and add $x$ if needed. If asked to remove on empty, add something first. This way, every log line is correct and we add as few fixes as possible.}
\NotePages{3}

\end{document}