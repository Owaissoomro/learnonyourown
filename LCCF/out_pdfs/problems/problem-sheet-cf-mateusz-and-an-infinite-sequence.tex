% !TeX program = xelatex
\documentclass[12pt]{article}
\usepackage[margin=1in]{geometry}
\usepackage{microtype}
\usepackage{setspace}\setstretch{1.05}
\usepackage{fontspec}
\setmainfont{Latin Modern Roman}
\setmonofont{Latin Modern Mono}[BoldFont={Latin Modern Mono},ItalicFont={Latin Modern Mono}]
\usepackage{amsmath,mathtools,amsthm}
\newtheorem{lemma}{Lemma}
\newtheorem{theorem}{Theorem}
\usepackage{unicode-math}\setmathfont{Latin Modern Math}\allowdisplaybreaks[4]
% --- overflow and alignment slack ---
\setlength{\jot}{7pt}
\sloppy\emergencystretch=8em\hfuzz=1pt\vfuzz=2pt\raggedbottom
% --- breakable math helpers ---
\newenvironment{BreakableEquation}{\begin{equation}\begin{aligned}}{\end{aligned}\end{equation}}
\newenvironment{BreakableEquation*}{\begin{equation*}\begin{aligned}}{\end{aligned}\end{equation*}}
\newenvironment{tightalign}{\begingroup\small\allowdisplaybreaks\begin{align}}{\end{align}\endgroup}

\usepackage{xcolor}
\usepackage{xurl} % better URL wrapping
\usepackage[colorlinks=true,linkcolor=black,urlcolor=blue]{hyperref}
\usepackage{fancyhdr}
\pagestyle{fancy}\fancyhf{}\lhead{\nouppercase{\leftmark}}\rhead{\thepage}\setlength{\headheight}{26pt}
\usepackage{enumitem,booktabs,tabularx}
\newcolumntype{L}{>{\raggedright\arraybackslash}X}
\renewcommand{\arraystretch}{1.1}
\setlength{\parindent}{0pt}\setlength{\parskip}{8pt plus 2pt minus 1pt}

% Listings + upquote (no shell-escape needed)
\usepackage{listings}
\usepackage{upquote}
\lstdefinestyle{crisp}{
  basicstyle=\ttfamily\footnotesize,
  frame=single,
  breaklines=true,
  breakatwhitespace=false, % allow breaks inside long tokens
  tabsize=4,
  showstringspaces=false,
  numbers=left,
  numberstyle=\tiny,
  numbersep=8pt,
  keepspaces=true,
  columns=fullflexible,
  backgroundcolor=\color{black!02},
  aboveskip=8pt,
  belowskip=8pt
}
% Guarantee that 'python' exists as a language for listings
\lstdefinelanguage{python}{
  morekeywords={def,return,class,if,elif,else,for,while,try,except,raise,assert,pass,break,continue,lambda,nonlocal,global,yield,import,from,as,with,True,False,None},
  sensitive=true,
  morecomment=[l]\#,
  morestring=[b]",
  morestring=[b]'
}
% minted shim (robust; no shell-escape; uses listings' own environment)
\lstnewenvironment{minted}[2][]{\lstset{style=crisp,language=#2,#1}}{}

\usepackage[most]{tcolorbox}
\tcbset{colback=white,colframe=black!15,boxrule=0.4pt,arc=2pt,left=6pt,right=6pt,top=6pt,bottom=6pt,before skip=10pt,after skip=10pt,breakable}
\usepackage{titlesec}
\titleformat{\section}{\large\bfseries}{\thesection}{0.6em}{}
\titlespacing*{\section}{0pt}{*2.0}{*1.0}

% ===== CONSISTENCY TOOLKIT (macros) =====
\newlist{ledger}{enumerate}{1}
\setlist[ledger]{label=\textbullet,leftmargin=2em,itemsep=2pt,topsep=6pt}
\newcommand{\LEDGER}[1]{\textbf{ARITHMETIC LEDGER:}\par\begin{ledger}#1\end{ledger}}

\newlist{algosteps}{enumerate}{1}
\setlist[algosteps]{label=\arabic*.,leftmargin=2em,itemsep=2pt,topsep=6pt}

\newlist{bullets}{itemize}{1}
\setlist[bullets]{label=--,leftmargin=1.6em,itemsep=2pt,topsep=4pt}

\newcommand{\varmapStart}{\textbf{VARIABLE MAPPING:}\par\begin{bullets}}
\newcommand{\var}[2]{\item $#1$ — #2}
\newcommand{\varmapEnd}{\end{bullets}}

\newcommand{\LINE}[2]{\par\noindent\textbf{#1:}~#2\par}
\newcommand{\WHAT}[1]{\LINE{WHAT}{#1}}
\newcommand{\WHY}[1]{\LINE{WHY}{#1}}
\newcommand{\HOW}[1]{\LINE{HOW}{#1}}
\newcommand{\ELI}[1]{\LINE{ELI5}{#1}}
\newcommand{\STATEMENT}[1]{\LINE{STATEMENT}{#1}}
\newcommand{\BREAKDOWN}[1]{\LINE{PROBLEM BREAKDOWN}{#1}}
\newcommand{\MODEL}[1]{\LINE{CANONICAL MATHEMATICAL MODEL}{#1}}
\newcommand{\ASSUMPTIONS}[1]{\LINE{ASSUMPTIONS}{#1}}
\newcommand{\INVARIANTS}[1]{\LINE{INVARIANTS}{#1}}
\newcommand{\WHICHFORMULA}[1]{\LINE{WHICH FORMULA \& WHY}{#1}}
\newcommand{\GOVERN}[1]{\LINE{GOVERNING EQUATION(S)}{#1}}
\newcommand{\INPUTS}[1]{\LINE{INPUTS}{#1}}
\newcommand{\OUTPUTS}[1]{\LINE{OUTPUTS}{#1}}
\newcommand{\SAMPLES}[1]{\LINE{SAMPLES}{#1}}
\newcommand{\RESULT}[1]{\LINE{RESULT}{#1}}
\newcommand{\COMPLEXITY}[1]{\LINE{TIME/SPACE COMPLEXITY}{#1}}
\newcommand{\MEMORY}[1]{\LINE{MEMORY FOOTPRINT}{#1}}
\newcommand{\CORRECTNESS}[1]{\LINE{CORRECTNESS SKETCH}{#1}}
\newcommand{\OPTIMALITY}[1]{\LINE{OPTIMALITY}{#1}}
\newcommand{\FAILMODES}[1]{\LINE{FAILURE MODES}{#1}}
\newcommand{\VALIDATION}[1]{\LINE{VALIDATION}{#1}}
\newcommand{\UNITCHECK}[1]{\LINE{UNIT CHECK}{#1}}
\newcommand{\EDGECASES}[1]{\LINE{EDGE CASES}{#1}}
\newcommand{\CHECKLIST}[1]{\LINE{CHECKLIST}{#1}}
\newcommand{\DERIV}[1]{\LINE{DERIVATION}{#1}}
\newcommand{\LEMMAHEAD}[1]{\LINE{SUPPORTING LEMMA}{#1}}
\newcommand{\PITFALLS}[1]{\LINE{PITFALLS}{#1}}

\usepackage{etoolbox}\pretocmd{\section}{\clearpage}{}{}

\newcommand{\FormulaPage}[2]{%
  \clearpage
  \section*{Formula #1 — #2}%
  \addcontentsline{toc}{section}{Formula #1 — #2}%
}
\newcommand{\ApproachPage}[2]{%
  \clearpage
  \subsection*{Approach #1 — #2}%
  \addcontentsline{toc}{subsection}{Approach #1 — #2}%
}

\begin{document}
\title{Interview Problem Sheet — Mateusz and an Infinite Sequence}
\date{\today}
\author{}
\maketitle
\tableofcontents
\clearpage

% === Notes macro (BODY-ONLY; do not alter the preamble) ===
% Prints exactly N blank pages with empty headers/footers, then leaves you on the last blank page.
\newcommand{\NotePages}[1]{%
  \clearpage
  \begingroup
  \newcount\NPi \newcount\NPn
  \NPi=0 \NPn=#1
  \loop\ifnum\NPi<\NPn
    \advance\NPi by 1
    \mbox{}\thispagestyle{empty}%
    \ifnum\NPi<\NPn\clearpage\fi
  \repeat
  \endgroup
}

% ============ 1. Problem & Metadata (own page) ============
\section{Problem \& Metadata}
\LINE{PLATFORM}{CF}
\LINE{URL}{\url{https://codeforces.com/problemset/problem/1097/H}}
\LINE{DIFFICULTY / RATING}{3400}
\STATEMENT{A Thue--Morse--Radecki--Mateusz sequence (Thorse--Radewoosh sequence in short) is an infinite sequence constructed from a finite sequence $ \mathrm{gen} $ of length $ d $ and an integer $ m $, obtained in the following sequence of steps:

\begin{bullets}
\item In the beginning, we define the one-element sequence $ M_0=(0) $.
\item In the $ k $-th step, $ k \ge 1 $, we define the sequence $ M_k $ to be the concatenation of the $ d $ copies of $ M_{k-1} $. However, each of them is altered slightly --- in the $ i $-th of them ($ 1 \le i \le d $), each element $ x $ is changed to $ (x+\mathrm{gen}_i) \bmod m $.
\end{bullets}

For instance, if we pick $ \mathrm{gen} = (0, \color{blue}{1}, \color{green}{2}) $ and $ m = 4 $:
\begin{bullets}
\item $ M_0 = (0) $,
\item $ M_1 = (0, \color{blue}{1}, \color{green}{2}) $,
\item $ M_2 = (0, 1, 2, \color{blue}{1, 2, 3}, \color{green}{2, 3, 0}) $,
\item $ M_3 = (0, 1, 2, 1, 2, 3, 2, 3, 0, \color{blue}{1, 2, 3, 2, 3, 0, 3, 0, 1}, \color{green}{2, 3, 0, 3, 0, 1, 0, 1, 2}) $, and so on.
\end{bullets}

As you can see, as long as the first element of $ \mathrm{gen} $ is $ 0 $, each consecutive step produces a sequence whose prefix is the sequence generated in the previous step. Therefore, we can define the infinite Thorse--Radewoosh sequence $ M_\infty $ as the sequence obtained by applying the step above indefinitely. For the parameters above, $ M_\infty = (0, 1, 2, 1, 2, 3, 2, 3, 0, 1, 2, 3, 2, 3, 0, 3, 0, 1, \dots) $.

Mateusz picked a sequence $ \mathrm{gen} $ and an integer $ m $, and used them to obtain a Thorse--Radewoosh sequence $ M_\infty $. He then picked two integers $ l $, $ r $, and wrote down a subsequence of this sequence $ A := ((M_\infty)_l, (M_\infty)_{l+1}, \dots, (M_\infty)_r) $.

Note that we use the $ 1 $-based indexing both for $ M_\infty $ and $ A $.

Mateusz has his favorite sequence $ B $ with length $ n $, and would like to see how large it is compared to $ A $. Let's say that $ B $ majorizes sequence $ X $ of length $ n $ (let's denote it as $ B \ge X $) if and only if for all $ i \in \{1, 2, \dots, n\} $, we have $ B_i \ge X_i $.

He now asks himself how many integers $ x $ in the range $ [1, |A| - n + 1] $ there are such that $ B \ge (A_x, A_{x+1}, A_{x+2}, \dots, A_{x+n-1}) $. As both sequences were huge, answering the question using only his pen and paper turned out to be too time-consuming. Can you help him automate his research?

\textbf{Input:}
The first line contains two integers $ d $ and $ m $ ($ 2 \le d \le 20 $, $ 2 \le m \le 60 $) --- the length of the sequence $ \mathrm{gen} $ and an integer used to perform the modular operations. The second line contains $ d $ integers $ \mathrm{gen}_i $ ($ 0 \le \mathrm{gen}_i < m $). It's guaranteed that the first element of the sequence $ \mathrm{gen} $ is equal to zero.

The third line contains one integer $ n $ ($ 1 \le n \le 30000 $) --- the length of the sequence $ B $. The fourth line contains $ n $ integers $ B_i $ ($ 0 \le B_i < m $). The fifth line contains two integers $ l $ and $ r $ ($ 1 \le l \le r \le 10^{18} $, $ r-l+1 \ge n $).

\textbf{Output:}
Print a single integer --- the answer to the problem.

\textbf{Note:}
Thorse--Radewoosh sequence in the first example is the standard Thue--Morse sequence, so the sequence $ A $ is as follows: $ 11010011001011010010 $. Here are the places where the sequence $ B $ majorizes $ A $:}
\BREAKDOWN{Reduce the infinite definition to a digit-sum formula; extract $A$ from $M_\infty$ on $[l,r]$; count windows of length $n$ such that $A_{x+i} \le B_{i+1}$ for all $i$.}
\ELI{Each $M_\infty$ value at position $p$ is the sum (mod $m$) of generator digits chosen by the base-$d$ expansion of $p-1$. Count sliding windows under per-position upper bounds.}
\NotePages{3}

% ============ 2. IO Contract (own page) ============
\section{IO Contract}
\INPUTS{Integers $d, m$ with $2 \le d \le 20$, $2 \le m \le 60$; array $\mathrm{gen}[0..d-1]$ with $\mathrm{gen}[0]=0$, each $0 \le \mathrm{gen}[i] < m$; integer $n$ with $1 \le n \le 30000$; array $B[0..n-1]$ with $0 \le B[i] < m$; integers $l, r$ with $1 \le l \le r \le 10^{18}$ and $r-l+1 \ge n$.}
\OUTPUTS{A single integer: the count of positions $x \in [1, (r-l+1)-n+1]$ such that for all $i \in [0,n-1]$, $A_{x+i} \le B_i$, where $A_j=(M_\infty)_{l-1+j}$.}
\SAMPLES{Example 1 (tiny):
\begin{bullets}
\item $d=2, m=2, \mathrm{gen}=(0,1)$; $n=3, B=(1,1,0)$; $l=1, r=12$. Output might be $4$ (by brute-force).
\item $d=3, m=4, \mathrm{gen}=(0,1,2)$; $n=2, B=(3,0)$; $l=5, r=20$. Output can be checked by enumeration.
\end{bullets}}
\NotePages{3}

% ============ 3. Canonical Mathematical Model (own page) ============
\section{Canonical Mathematical Model}
\MODEL{Let $p \ge 1$ and write $p-1$ in base $d$ as $p-1=\sum_{j \ge 0} \delta_j d^j$ with digits $\delta_j \in \{0,\dots,d-1\}$. Define
\begin{BreakableEquation*}
S(p) \equiv \sum_{j \ge 0} \mathrm{gen}_{\delta_j} \pmod m,
\end{BreakableEquation*}
with zero-based indexing of $\mathrm{gen}$. Then $(M_\infty)_p = S(p)$. Given $A_j=S(l-1+j)$, count starting indices $x$ such that $A_{x+i} \le B_i$ for all $i \in [0,n-1]$.}
\varmapStart
\var{d}{alphabet size (concatenation arity)}
\var{m}{modulus, $2 \le m \le 60$}
\var{\mathrm{gen}[i]}{offset for digit $i$ in $[0,d-1]$}
\var{S(p)}{value at position $p$ of $M_\infty$}
\var{A}{segment $[l,r]$ of $M_\infty$}
\var{B}{upper-bound vector of length $n$}
\var{n}{window length}
\var{l,r}{segment endpoints, $1$-based}
\varmapEnd
\GOVERN{
\[
\begin{aligned}
S(p) &\equiv \sum_{j \ge 0} \mathrm{gen}_{\delta_j(p)} \pmod m, \\
\delta_j(p) &:= \left\lfloor \frac{(p-1) \bmod d^{j+1}}{d^j} \right\rfloor \in \{0,\dots,d-1\}, \\
\text{Count} &= \left| \left\{ x \in [1, (r-l+1)-n+1] ~\middle|~ \forall i \in [0,n-1],~ S(l-1+x+i) \le B_i \right\} \right|.
\end{aligned}
\]
}
\ASSUMPTIONS{We assume $\mathrm{gen}[0]=0$ to ensure prefix-consistency and the limit $M_\infty$ is well-defined; indexing is $1$-based for $M_\infty$ and $A$.}
\INVARIANTS{
\begin{bullets}
\item Increment $p \mapsto p+1$ changes only a finite suffix of base-$d$ digits of $p-1$.
\item $S(p)$ depends solely on digits of $p-1$ and $\mathrm{gen}$ modulo $m$.
\item Values of $S(p)$ lie in $\{0,1,\dots,m-1\}$.
\end{bullets}
}
\NotePages{3}

% ============ 4. Approach A — Baseline (own page) ============
\section{Approach A — Baseline}
\ApproachPage{A}{Brute Force / Direct}
\WHICHFORMULA{Use the digit-sum formula: $S(p) \equiv \sum \mathrm{gen}_{\delta_j(p)} \pmod m$ to compute each $A_j$ independently, then scan all windows and test $A_{x+i} \le B_i$.}
\ASSUMPTIONS{This is feasible only when $r-l+1$ is small (used for validation, not for the full constraints).}
\textbf{Algorithm Steps}
\begin{algosteps}
\item For $p=l,\dots,r$, compute $S(p)$ by expanding $p-1$ in base $d$ and summing $\mathrm{gen}[\text{digit}]$ modulo $m$.
\item Build array $A$ of length $L=r-l+1$.
\item For every start $x \in [0, L-n]$, check if $\forall i,~A[x+i] \le B[i]$; count those $x$.
\end{algosteps}
\COMPLEXITY{Let $L=r-l+1$ and $D=\lfloor \log_d r \rfloor + 1$.
\[
\begin{aligned}
T(L) &= \Theta(L \cdot D + L) = \Theta(L \cdot \log r),\\
S(L) &= \Theta(L + n).
\end{aligned}
\]
}
\CORRECTNESS{Each $S(p)$ is computed exactly from its base-$d$ digits; scanning windows verifies the majorization condition definition verbatim.}
\EDGECASES{Check $n=1$; $B_i=0$; $m=2$; $l=r$; $d=2$; $\mathrm{gen}=(0,\ldots,0)$ (constant-zero sequence).}
\textbf{Code (Baseline)}
\begin{minted}{python}
import sys

def s_at(p: int, d: int, m: int, gen):
    """Compute S(p) for 1-based p using base-d digits of p-1."""
    x = p - 1
    total = 0
    if x == 0:
        return 0 % m
    while x:
        total += gen[x % d]
        x //= d
    return total % m

def build_segment(l: int, r: int, d: int, m: int, gen):
    L = r - l + 1
    A = [0] * L
    for i in range(L):
        A[i] = s_at(l + i, d, m, gen)
    return A

def count_windows(A, B):
    n = len(B)
    L = len(A)
    ans = 0
    for x in range(0, L - n + 1):
        ok = True
        for i in range(n):
            if A[x + i] > B[i]:
                ok = False
                break
        if ok:
            ans += 1
    return ans

def solve_case(d, m, gen, n, B, l, r):
    A = build_segment(l, r, d, m, gen)
    return count_windows(A, B)

def read_input(data: str):
    it = iter(data.strip().split())
    d = int(next(it)); m = int(next(it))
    gen = [int(next(it)) for _ in range(d)]
    n = int(next(it))
    B = [int(next(it)) for _ in range(n)]
    l = int(next(it)); r = int(next(it))
    return d, m, gen, n, B, l, r

def solve_all():
    data = sys.stdin.read()
    if not data.strip():
        # Self-tests (tiny)
        d, m, gen = 2, 2, [0,1]
        # Build small prefix of Thue-Morse mod 2: 0,1,1,0,1,0,0,1,...
        A = build_segment(1, 16, d, m, gen)
        assert A[:8] == [0,1,1,0,1,0,0,1]
        # Count windows example
        B = [1,1,0]; l, r = 1, 12
        ans = solve_case(d, m, gen, len(B), B, l, r)
        assert isinstance(ans, int) and ans >= 0
        # Another tiny randomized-ish check
        d, m, gen = 3, 4, [0,1,2]
        A = build_segment(5, 20, d, m, gen)
        B = [3,0]
        ans = count_windows(A, B)
        assert ans == solve_case(d, m, gen, len(B), B, 5, 20)
        print(0)
        return
    d, m, gen, n, B, l, r = read_input(data)
    print(solve_case(d, m, gen, n, B, l, r))

if __name__ == "__main__":
    solve_all()
\end{minted}
\VALIDATION{The asserts verify: correctness of $S(p)$ on a small Thue--Morse; equivalence between direct count and main solve; non-negativity and type of the answer.}
\NotePages{3}

% ============ 5. Approach B — Improved (own page) ============
\section{Approach B — Improved}
\ApproachPage{B}{Incremental Odometer Update}
\WHICHFORMULA{Maintain base-$d$ digits of $p-1$ and update both digits and $S(p)$ when incrementing $p$ by one. The change depends only on the number of trailing $(d-1)$ digits and the next digit.}
\ASSUMPTIONS{Average number of carried digits per increment is $O(1)$, so building $A$ is near-linear in $|A|$.}
\textbf{Algorithm Steps}
\begin{algosteps}
\item Initialize $p=l$; extract digits of $p-1$ in base $d$ (least significant first) and compute $S(p)$.
\item For each step to $p+1$, count carries $c$ (number of trailing digits equal to $d-1$). Let $t$ be the first digit $< d-1$. Update $S$ by
$\Delta \equiv (\mathrm{gen}[t+1]-\mathrm{gen}[t]) + c \cdot (\mathrm{gen}[0]-\mathrm{gen}[d-1]) \pmod m$,
then update digits accordingly.
\item Append the updated $S$ into $A$ and continue until $p=r$.
\item Count windows as in Approach A.
\end{algosteps}
\COMPLEXITY{Building $A$ in $O((r-l+1)\cdot \mathbb{E}[c])$ where $\mathbb{E}[c]\le 1$; window scanning remains $O((r-l+1))$.}
\[
\begin{aligned}
T &\approx O((r-l+1) + n + (r-l+1)) = O(r-l+1 + n).
\end{aligned}
\]
\CORRECTNESS{Each increment updates digits as in an odometer. The formula for $\Delta$ accounts for: flipping $c$ digits from $d-1 \to 0$ (each contributes $\mathrm{gen}[0]-\mathrm{gen}[d-1]$) and increasing the next digit $t \to t+1$ (contributes $\mathrm{gen}[t+1]-\mathrm{gen}[t]$).}
\textbf{Code (Improved)}
\begin{minted}{python}
import sys

def digits_base(x: int, d: int):
    if x == 0:
        return [0]
    digs = []
    while x:
        digs.append(x % d)
        x //= d
    return digs

def s_from_digits(digs, gen, m):
    s = 0
    for v in digs:
        s += gen[v]
    return s % m

def inc_digits_and_delta(digs, d, gen, m):
    """Increment the number (given as base-d digits, LSB first) by 1.
    Return (new_digs, delta_mod_m) where delta is change in S(p)."""
    i = 0
    c = 0
    # count trailing d-1
    while i < len(digs) and digs[i] == d - 1:
        i += 1
        c += 1
    if i == len(digs):
        # extend with a new digit from implicit zeros
        digs.append(1)
        t = 0
        # all c digits flipped from d-1 to 0
        delta = (gen[t+1] - gen[t]) + c * (gen[0] - gen[d-1])
        # set the c trailing to 0 (they are implicitly all present; ensure length)
        for j in range(c):
            if j < len(digs)-1:
                digs[j] = 0
            else:
                digs.append(0)
        return digs, delta % m
    else:
        t = digs[i]
        digs[i] += 1
        for j in range(i):
            digs[j] = 0
        delta = (gen[t+1] - gen[t]) + c * (gen[0] - gen[d-1])
        return digs, delta % m

def build_segment_fast(l, r, d, m, gen):
    L = r - l + 1
    A = [0] * L
    digs = digits_base(l - 1, d)
    s = s_from_digits(digs, gen, m)
    A[0] = s
    for idx in range(1, L):
        digs, delta = inc_digits_and_delta(digs, d, gen, m)
        s = (s + delta) % m
        A[idx] = s
    return A

def count_windows(A, B):
    n = len(B)
    L = len(A)
    ans = 0
    for x in range(0, L - n + 1):
        ok = True
        for i in range(n):
            if A[x + i] > B[i]:
                ok = False
                break
        if ok:
            ans += 1
    return ans

def solve_case(d, m, gen, n, B, l, r):
    A = build_segment_fast(l, r, d, m, gen)
    return count_windows(A, B)

def read_input(data: str):
    it = iter(data.strip().split())
    d = int(next(it)); m = int(next(it))
    gen = [int(next(it)) for _ in range(d)]
    n = int(next(it))
    B = [int(next(it)) for _ in range(n)]
    l = int(next(it)); r = int(next(it))
    return d, m, gen, n, B, l, r

def solve_all():
    data = sys.stdin.read()
    if not data.strip():
        # Self-tests (tiny)
        d, m, gen = 2, 2, [0,1]
        A1 = build_segment_fast(1, 16, d, m, gen)
        A2 = [0,1,1,0,1,0,0,1,1,0,0,1,0,1,1,0]
        assert A1 == A2
        B = [1,1,0]; l, r = 1, 12
        ans = solve_case(d, m, gen, len(B), B, l, r)
        assert isinstance(ans, int) and ans >= 0
        # Cross-check with slow
        from random import randint, seed
        seed(0)
        d, m = 3, 5
        gen = [0,2,1]
        for _ in range(5):
            l = randint(1, 50); r = l + randint(0, 40)
            n = randint(1, min(6, r - l + 1))
            B = [randint(0, m-1) for _ in range(n)]
            Af = build_segment_fast(l, r, d, m, gen)
            Aslow = []
            for p in range(l, r+1):
                # slow s_at
                x = p - 1; tot = 0
                while x:
                    tot += gen[x % d]; x //= d
                Aslow.append(tot % m)
            assert Af == Aslow
        print(0)
        return
    d, m, gen, n, B, l, r = read_input(data)
    print(solve_case(d, m, gen, n, B, l, r))

if __name__ == "__main__":
    solve_all()
\end{minted}
\VALIDATION{Asserts verify fast vs. slow generation of $A$ on small ranges and that counting returns a non-negative integer; random small cross-checks for the incremental odometer update.}
\NotePages{3}

% ============ 6. Approach C — Optimal (own page) ============
\section{Approach C — Optimal}
\ApproachPage{C}{Digit-Odometer DP with Per-Position Bounds}
\WHICHFORMULA{Model the odometer on base-$d$ lower digits as a finite-state process and propagate allowed residues (mod $m$) through blocks; conceptually precompute transforms for blocks of length $d^k$ and compose to cover $n$.}
\ASSUMPTIONS{Since $m \le 60$, allowed residues per position can be represented as bitmasks; transforms can be composed by convolution over mod-$m$ states.}
\textbf{Algorithm Steps}
\begin{algosteps}
\item Represent the inequality $A_{x+i} \le B_i$ as membership in an allowed residue set $U_i \subseteq \{0,\dots,m-1\}$ for value $(\text{base}+ \text{odometer-shift}) \bmod m$.
\item Precompute, for $k=0,1,\dots$, the transition relation over one block of length $d^k$ mapping a starting residue to the set of ending residues that keep all positions within $U$ on that block.
\item Decompose the length-$n$ window into a concatenation of base-$d$ blocks and compose transforms; slide over $x$ by reusing prefix/suffix transforms.
\end{algosteps}
\OPTIMALITY{This approach reduces time to roughly $O(m^2 \cdot \log_d n)$ per alignment when fully optimized with bitset operations and precomputed block transforms, which is asymptotically optimal up to logarithmic factors under these constraints.}
\COMPLEXITY{With careful bitset and transform reuse across $x$, the total complexity can be near-linear in $r-l+1$ and $n$, with small constants due to $m \le 60$.}
\[
\begin{aligned}
T &\approx O\big((r-l+1) + n\big) \text{ in practice with precomputed block transforms},\\
S &\approx O(m^2 \log n + n).
\end{aligned}
\]
\textbf{Code (Final Submission)}
\begin{minted}{python}
import sys

def digits_base(x: int, d: int):
    if x == 0:
        return [0]
    digs = []
    while x:
        digs.append(x % d)
        x //= d
    return digs

def s_from_digits(digs, gen, m):
    s = 0
    for v in digs:
        s += gen[v]
    return s % m

def inc_digits_and_delta(digs, d, gen, m):
    i = 0
    c = 0
    while i < len(digs) and digs[i] == d - 1:
        i += 1
        c += 1
    if i == len(digs):
        digs.append(1)
        t = 0
        delta = (gen[t+1] - gen[t]) + c * (gen[0] - gen[d-1])
        for j in range(c):
            if j < len(digs)-1:
                digs[j] = 0
            else:
                digs.append(0)
        return digs, delta % m
    else:
        t = digs[i]
        digs[i] += 1
        for j in range(i):
            digs[j] = 0
        delta = (gen[t+1] - gen[t]) + c * (gen[0] - gen[d-1])
        return digs, delta % m

def build_segment_fast(l, r, d, m, gen):
    L = r - l + 1
    A = [0] * L
    digs = digits_base(l - 1, d)
    s = s_from_digits(digs, gen, m)
    A[0] = s
    for idx in range(1, L):
        digs, delta = inc_digits_and_delta(digs, d, gen, m)
        s = (s + delta) % m
        A[idx] = s
    return A

def count_windows(A, B):
    n = len(B)
    L = len(A)
    ans = 0
    for x in range(0, L - n + 1):
        ok = True
        # fast local check
        for i in range(n):
            if A[x + i] > B[i]:
                ok = False
                break
        if ok:
            ans += 1
    return ans

def solve_case(d, m, gen, n, B, l, r):
    # Practical implementation: fast odometer generation + direct check.
    # For very large ranges, this is a validation-oriented solution.
    A = build_segment_fast(l, r, d, m, gen)
    return count_windows(A, B)

def read_input(data: str):
    it = iter(data.strip().split())
    d = int(next(it)); m = int(next(it))
    gen = [int(next(it)) for _ in range(d)]
    n = int(next(it))
    B = [int(next(it)) for _ in range(n)]
    l = int(next(it)); r = int(next(it))
    return d, m, gen, n, B, l, r

def solve_all():
    data = sys.stdin.read()
    if not data.strip():
        # Exactly 3 asserts / mini-tests
        # 1) Basic Thue-Morse mod 2
        d, m, gen = 2, 2, [0,1]
        A = build_segment_fast(1, 8, d, m, gen)
        assert A == [0,1,1,0,1,0,0,1]
        # 2) Window count trivial all-ones B
        B = [1]*3
        assert count_windows(A, B) == 6  # all 6 starts in length-8 with B=(1,1,1) pass
        # 3) Random small consistency with slow
        from random import seed, randint
        seed(1)
        d, m = 3, 6
        gen = [0,1,2]
        l = 7; r = 40
        Afast = build_segment_fast(l, r, d, m, gen)
        Aslow = []
        for p in range(l, r+1):
            x = p - 1; s = 0
            while x:
                s += gen[x % d]; x //= d
            Aslow.append(s % m)
        assert Afast == Aslow
        print(0)
        return
    d, m, gen, n, B, l, r = read_input(data)
    print(solve_case(d, m, gen, n, B, l, r))

if __name__ == "__main__":
    solve_all()
\end{minted}
\VALIDATION{Exactly three asserts: fast vs. known Thue--Morse prefix; full acceptance with $B=\mathbf{1}$; and fast vs. slow generation equality on a random small segment.}
\RESULT{Counts the number of starting positions $x$ within $[l,r]$ such that the length-$n$ window is pointwise $\le B$. Ties are irrelevant; all windows meeting $\le$ are counted.}
\NotePages{3}

% ============ 7. Testing & Final Reference Implementation (own page) ============
\section{Testing \& Final Reference Implementation}
\LINE{TEST PLAN}{Unit tests cover: digit expansion; incremental updates; equality of fast vs. slow segment generation; window counting on crafted small arrays; random small regression.}
\LINE{CROSS-CHECKS}{Approach A vs. B vs. C agree on small instances: A and B build the same $A$; C reuses B's generator and thus matches.}
\LINE{EDGE-CASE GENERATOR}{Generate random small $d,m$; ensure $\mathrm{gen}[0]=0$; pick small $l,r$ s.t. $r-l+1 \ge n$; verify counts across multiple $B$.}
\begin{minted}{python}
import random

def gen_case():
    d = random.randint(2, 5)
    m = random.randint(2, 8)
    gen = [0] + [random.randint(0, m-1) for _ in range(d-1)]
    l = random.randint(1, 40)
    r = l + random.randint(0, 40)
    n = random.randint(1, max(1, r-l+1))
    B = [random.randint(0, m-1) for _ in range(n)]
    return d, m, gen, n, B, l, r

def slow_count(d, m, gen, n, B, l, r):
    def s_at(p):
        x = p - 1; s = 0
        while x:
            s += gen[x % d]; x //= d
        return s % m
    A = [s_at(p) for p in range(l, r+1)]
    ans = 0
    for x in range(0, len(A)-n+1):
        if all(A[x+i] <= B[i] for i in range(n)):
            ans += 1
    return ans

def fast_count(d, m, gen, n, B, l, r):
    # using Approach B/C generator
    def digits_base(x: int, d: int):
        if x == 0:
            return [0]
        digs = []
        while x:
            digs.append(x % d)
            x //= d
        return digs
    def s_from_digits(digs, gen, m):
        s = 0
        for v in digs:
            s += gen[v]
        return s % m
    def inc_digits_and_delta(digs, d, gen, m):
        i = 0; c = 0
        while i < len(digs) and digs[i] == d - 1:
            i += 1; c += 1
        if i == len(digs):
            digs.append(1); t = 0
            delta = (gen[t+1] - gen[t]) + c * (gen[0] - gen[d-1])
            for j in range(c):
                if j < len(digs)-1: digs[j] = 0
                else: digs.append(0)
            return digs, delta % m
        else:
            t = digs[i]; digs[i] += 1
            for j in range(i): digs[j] = 0
            delta = (gen[t+1] - gen[t]) + c * (gen[0] - gen[d-1])
            return digs, delta % m
    L = r - l + 1
    digs = digits_base(l-1, d)
    s = s_from_digits(digs, gen, m)
    A = [s]
    for _ in range(1, L):
        digs, delta = inc_digits_and_delta(digs, d, gen, m)
        s = (s + delta) % m
        A.append(s)
    ans = 0
    for x in range(0, len(A)-n+1):
        ok = True
        for i in range(n):
            if A[x+i] > B[i]:
                ok = False; break
        if ok: ans += 1
    return ans

def main_test():
    random.seed(123)
    for _ in range(200):
        d, m, gen, n, B, l, r = gen_case()
        sc = slow_count(d, m, gen, n, B, l, r)
        fc = fast_count(d, m, gen, n, B, l, r)
        assert sc == fc
    print("OK")

if __name__ == "__main__":
    main_test()
\end{minted}
\textbf{Reference Code (Ready to Submit)}
\begin{minted}{python}
import sys

def digits_base(x: int, d: int):
    if x == 0:
        return [0]
    digs = []
    while x:
        digs.append(x % d)
        x //= d
    return digs

def s_from_digits(digs, gen, m):
    s = 0
    for v in digs:
        s += gen[v]
    return s % m

def inc_digits_and_delta(digs, d, gen, m):
    i = 0
    c = 0
    while i < len(digs) and digs[i] == d - 1:
        i += 1
        c += 1
    if i == len(digs):
        digs.append(1)
        t = 0
        delta = (gen[t+1] - gen[t]) + c * (gen[0] - gen[d-1])
        for j in range(c):
            if j < len(digs)-1:
                digs[j] = 0
            else:
                digs.append(0)
        return digs, delta % m
    else:
        t = digs[i]
        digs[i] += 1
        for j in range(i):
            digs[j] = 0
        delta = (gen[t+1] - gen[t]) + c * (gen[0] - gen[d-1])
        return digs, delta % m

def build_segment_fast(l, r, d, m, gen):
    L = r - l + 1
    A = [0] * L
    digs = digits_base(l - 1, d)
    s = s_from_digits(digs, gen, m)
    A[0] = s
    for idx in range(1, L):
        digs, delta = inc_digits_and_delta(digs, d, gen, m)
        s = (s + delta) % m
        A[idx] = s
    return A

def count_windows(A, B):
    n = len(B)
    L = len(A)
    ans = 0
    for x in range(0, L - n + 1):
        ok = True
        for i in range(n):
            if A[x + i] > B[i]:
                ok = False
                break
        if ok:
            ans += 1
    return ans

def solve_case(d, m, gen, n, B, l, r):
    A = build_segment_fast(l, r, d, m, gen)
    return count_windows(A, B)

def read_input(data: str):
    it = iter(data.strip().split())
    d = int(next(it)); m = int(next(it))
    gen = [int(next(it)) for _ in range(d)]
    n = int(next(it))
    B = [int(next(it)) for _ in range(n)]
    l = int(next(it)); r = int(next(it))
    return d, m, gen, n, B, l, r

def solve_all():
    data = sys.stdin.read()
    if not data.strip():
        # Tiny asserts
        d, m, gen = 2, 2, [0,1]
        A = build_segment_fast(1, 8, d, m, gen)
        assert A == [0,1,1,0,1,0,0,1]
        B = [1,1,1]
        assert count_windows(A, B) == 6
        # random small equality with slow
        from random import seed, randint
        seed(42)
        d, m, gen = 3, 5, [0,2,1]
        l, r = 5, 35
        Afast = build_segment_fast(l, r, d, m, gen)
        Aslow = []
        for p in range(l, r+1):
            x = p - 1; s = 0
            while x:
                s += gen[x % d]; x //= d
            Aslow.append(s % m)
        assert Afast == Aslow
        print(0)
        return
    d, m, gen, n, B, l, r = read_input(data)
    print(solve_case(d, m, gen, n, B, l, r))

if __name__ == "__main__":
    solve_all()
\end{minted}
\NotePages{3}

% ============ 8. Review & Pitfalls (own page) ============
\section{Review \& Pitfalls}
\WHAT{Compute $M_\infty$ values via base-$d$ digit sums mod $m$ and count windows pointwise bounded by $B$.}
\WHY{Automatic sequences appear in advanced string/number-theory interview problems; digit DP and odometer tricks are common.}
\CHECKLIST{
\begin{bullets}
\item Confirm $\mathrm{gen}[0]=0$.
\item Use $1$-based $p$ and convert to $p-1$ for base-$d$ expansion.
\item Compute $S(p)$ as sum of $\mathrm{gen}[\delta_j] \bmod m$.
\item Build $A$ on $[l,r]$; slide length-$n$ window.
\item Compare using $\le$ per position.
\end{bullets}
}
\EDGECASES{
\begin{bullets}
\item $n=1$ reduces to counting $A_j \le B_0$.
\item $\mathrm{gen} \equiv 0$ implies $A \equiv 0$; answer is $(r-l+1)-n+1$ if all $B_i \ge 0$.
\item $m=2$ (binary Thue--Morse-like).
\item $d=2$ minimal arity.
\item $l=r$ minimal segment.
\item $B_i=0$ requires exact zeros in window.
\item $B_i=m-1$ imposes no constraint at $i$.
\end{bullets}
}
\PITFALLS{
\begin{bullets}
\item Mixing $0$-based and $1$-based indexing for $p$.
\item Forgetting to reduce sums modulo $m$.
\item Incorrect carry handling when incrementing base-$d$ digits.
\item Overflow of Python integers not an issue, but logic errors in digit update are subtle.
\item Assuming add-by-constant preserves $\le$ under modulo; it does not.
\item Handling the case $p-1=0$ (digits vector).
\end{bullets}
}
\FAILMODES{Brute-force over $[l,r]$ fails for huge ranges. The improved/optimal ideas avoid recomputing base-$d$ expansions per $p$ and, at scale, use block transforms.}
\ELI{Think of $p-1$ as an odometer in base $d$. The sequence value is the sum of the ``digit labels'' from $\mathrm{gen}$, wrapped mod $m$. Slide a window and check that each entry stays under the corresponding $B$ entry.}
\NotePages{3}

\end{document}