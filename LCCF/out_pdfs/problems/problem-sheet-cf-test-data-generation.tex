% !TeX program = xelatex
\documentclass[11pt]{article}
\usepackage[margin=1in]{geometry}
\usepackage{microtype}
\usepackage{setspace}\setstretch{1.05}
\usepackage{fontspec}
\setmainfont{Latin Modern Roman}
\setmonofont{Latin Modern Mono}[BoldFont={Latin Modern Mono},ItalicFont={Latin Modern Mono}]
\usepackage{amsmath,mathtools,amsthm}
\newtheorem{lemma}{Lemma}
\newtheorem{theorem}{Theorem}
\usepackage{unicode-math}\setmathfont{Latin Modern Math}\allowdisplaybreaks[4]
% --- overflow and alignment slack ---
\setlength{\jot}{7pt}
\sloppy\emergencystretch=8em\hfuzz=1pt\vfuzz=2pt\raggedbottom
% --- breakable math helpers ---
\newenvironment{BreakableEquation}{\begin{equation}\begin{aligned}}{\end{aligned}\end{equation}}
\newenvironment{BreakableEquation*}{\begin{equation*}\begin{aligned}}{\end{aligned}\end{equation*}}
\newenvironment{tightalign}{\begingroup\small\allowdisplaybreaks\begin{align}}{\end{align}\endgroup}

\usepackage{xcolor}
\usepackage{xurl} % better URL wrapping
\usepackage[colorlinks=true,linkcolor=black,urlcolor=blue]{hyperref}
\usepackage{fancyhdr}
\pagestyle{fancy}\fancyhf{}\lhead{\nouppercase{\leftmark}}\rhead{\thepage}\setlength{\headheight}{26pt}
\usepackage{enumitem,booktabs,tabularx}
\newcolumntype{L}{>{\raggedright\arraybackslash}X}
\renewcommand{\arraystretch}{1.1}
\setlength{\parindent}{0pt}\setlength{\parskip}{8pt plus 2pt minus 1pt}

% Listings + upquote (no shell-escape needed)
\usepackage{listings}
\usepackage{upquote}
\lstdefinestyle{crisp}{
  basicstyle=\ttfamily\footnotesize,
  frame=single,
  breaklines=true,
  breakatwhitespace=false, % allow breaks inside long tokens
  tabsize=4,
  showstringspaces=false,
  numbers=left,
  numberstyle=\tiny,
  numbersep=8pt,
  keepspaces=true,
  columns=fullflexible,
  backgroundcolor=\color{black!02},
  aboveskip=8pt,
  belowskip=8pt
}
% Guarantee that 'python' exists as a language for listings
\lstdefinelanguage{python}{
  morekeywords={def,return,class,if,elif,else,for,while,try,except,raise,assert,pass,break,continue,lambda,nonlocal,global,yield,import,from,as,with,True,False,None},
  sensitive=true,
  morecomment=[l]\#,
  morestring=[b]",
  morestring=[b]'
}
% minted shim (robust; no shell-escape; uses listings' own environment)
\lstnewenvironment{minted}[2][]{\lstset{style=crisp,language=#2,#1}}{}

\usepackage[most]{tcolorbox}
\tcbset{colback=white,colframe=black!15,boxrule=0.4pt,arc=2pt,left=6pt,right=6pt,top=6pt,bottom=6pt,before skip=10pt,after skip=10pt,breakable}
\usepackage{titlesec}
\titleformat{\section}{\large\bfseries}{\thesection}{0.6em}{}
\titlespacing*{\section}{0pt}{*2.0}{*1.0}

% ===== CONSISTENCY TOOLKIT (macros) =====
\newlist{ledger}{enumerate}{1}
\setlist[ledger]{label=\textbullet,leftmargin=2em,itemsep=2pt,topsep=6pt}
\newcommand{\LEDGER}[1]{\textbf{ARITHMETIC LEDGER:}\par\begin{ledger}#1\end{ledger}}

\newlist{algosteps}{enumerate}{1}
\setlist[algosteps]{label=\arabic*.,leftmargin=2em,itemsep=2pt,topsep=6pt}

\newlist{bullets}{itemize}{1}
\setlist[bullets]{label=--,leftmargin=1.6em,itemsep=2pt,topsep=4pt}

\newcommand{\varmapStart}{\textbf{VARIABLE MAPPING:}\par\begin{bullets}}
\newcommand{\var}[2]{\item $#1$ — #2}
\newcommand{\varmapEnd}{\end{bullets}}

\newcommand{\LINE}[2]{\par\noindent\textbf{#1:}~#2\par}
\newcommand{\WHAT}[1]{\LINE{WHAT}{#1}}
\newcommand{\WHY}[1]{\LINE{WHY}{#1}}
\newcommand{\HOW}[1]{\LINE{HOW}{#1}}
\newcommand{\ELI}[1]{\LINE{ELI5}{#1}}
\newcommand{\STATEMENT}[1]{\LINE{STATEMENT}{#1}}
\newcommand{\BREAKDOWN}[1]{\LINE{PROBLEM BREAKDOWN}{#1}}
\newcommand{\MODEL}[1]{\LINE{CANONICAL MATHEMATICAL MODEL}{#1}}
\newcommand{\ASSUMPTIONS}[1]{\LINE{ASSUMPTIONS}{#1}}
\newcommand{\INVARIANTS}[1]{\LINE{INVARIANTS}{#1}}
\newcommand{\WHICHFORMULA}[1]{\LINE{WHICH FORMULA \& WHY}{#1}}
\newcommand{\GOVERN}[1]{\LINE{GOVERNING EQUATION(S)}{#1}}
\newcommand{\INPUTS}[1]{\LINE{INPUTS}{#1}}
\newcommand{\OUTPUTS}[1]{\LINE{OUTPUTS}{#1}}
\newcommand{\SAMPLES}[1]{\LINE{SAMPLES}{#1}}
\newcommand{\RESULT}[1]{\LINE{RESULT}{#1}}
\newcommand{\COMPLEXITY}[1]{\LINE{TIME/SPACE COMPLEXITY}{#1}}
\newcommand{\MEMORY}[1]{\LINE{MEMORY FOOTPRINT}{#1}}
\newcommand{\CORRECTNESS}[1]{\LINE{CORRECTNESS}{#1}}
\newcommand{\OPTIMALITY}[1]{\LINE{OPTIMALITY}{#1}}
\newcommand{\FAILMODES}[1]{\LINE{FAILURE MODES}{#1}}
\newcommand{\VALIDATION}[1]{\LINE{VALIDATION}{#1}}
\newcommand{\UNITCHECK}[1]{\LINE{UNIT CHECK}{#1}}
\newcommand{\EDGECASES}[1]{\LINE{EDGE CASES}{#1}}
\newcommand{\CHECKLIST}[1]{\LINE{CHECKLIST}{#1}}
\newcommand{\DERIV}[1]{\LINE{DERIVATION}{#1}}
\newcommand{\LEMMAHEAD}[1]{\LINE{SUPPORTING LEMMA}{#1}}
\newcommand{\PITFALLS}[1]{\LINE{PITFALLS}{#1}}

\usepackage{etoolbox}\pretocmd{\section}{\clearpage}{}{}

\newcommand{\FormulaPage}[2]{%
  \clearpage
  \section*{Formula #1 — #2}%
  \addcontentsline{toc}{section}{Formula #1 — #2}%
}
\newcommand{\ApproachPage}[2]{%
  \clearpage
  \subsection*{Approach #1 — #2}%
  \addcontentsline{toc}{subsection}{Approach #1 — #2}%
}

\begin{document}
\title{Interview Problem Sheet — Test Data Generation}
\date{\today}
\author{}
\maketitle
\tableofcontents
\clearpage

% === Notes macro (BODY-ONLY; do not alter the preamble) ===
% Prints exactly N blank pages with empty headers/footers, then leaves you on the last blank page.
\newcommand{\NotePages}[1]{%
  \clearpage
  \begingroup
  \newcount\NPi \newcount\NPn
  \NPi=0 \NPn=#1
  \loop\ifnum\NPi<\NPn
    \advance\NPi by 1
    \mbox{}\thispagestyle{empty}%
    \ifnum\NPi<\NPn\clearpage\fi
  \repeat
  \endgroup
}

% ============ 1. Problem & Metadata (own page) ============
\section{Problem \& Metadata}
\LINE{PLATFORM}{CF}
\LINE{URL}{\url{https://codeforces.com/problemset/problem/773/F}}
\LINE{DIFFICULTY / RATING}{CF 773/F, Rating: 3400}
\STATEMENT{Test data generation is not an easy task! Often, generating big random test cases is not enough to ensure thorough testing of solutions for correctness.

For example, consider a problem from an old Codeforces round. Its input format looks roughly as follows:

The first line contains a single integer $n$ ($1 \le n \le \text{maxn}$) — the size of the set. The second line contains $n$ distinct integers $a_1, a_2, \ldots, a_n$ ($1 \le a_i \le \text{maxa}$) — the elements of the set in increasing order.

If you do not pay attention to the problem solution, it looks fairly easy to generate a good test case for this problem. Let $n = \text{maxn}$, take random distinct $a_i$ from $1$ to $\text{maxa}$, sort them\ldots{} Soon you understand that it is not that easy.

Here is the actual problem solution. Let $g$ be the greatest common divisor of $a_1, a_2, \ldots, a_n$. Let $x = a_n / g - n$. Then the correct solution outputs ``Alice'' if $x$ is odd, and ``Bob'' if $x$ is even.

Consider two wrong solutions to this problem which differ from the correct one only in the formula for calculating $x$.

The first wrong solution calculates $x$ as $x = a_n / g$ (without subtracting $n$).

The second wrong solution calculates $x$ as $x = a_n - n$ (without dividing by $g$).

A test case is interesting if it makes both wrong solutions output an incorrect answer.

Given $\text{maxn}$, $\text{maxa}$ and $q$, find the number of interesting test cases satisfying the constraints, and output it modulo $q$.

Input:
The only line contains three integers $\text{maxn}$, $\text{maxa}$ and $q$ ($1 \le \text{maxn} \le 30{,}000$; $\text{maxn} \le \text{maxa} \le 10^9$; $10^4 \le q \le 10^5 + 129$).

Output:
Output a single integer — the number of test cases which satisfy the constraints and make both wrong solutions output an incorrect answer, modulo $q$.

Note:
In the first example, interesting test cases look as follows:
[omitted for brevity; the original examples enumerate specific small sets].}
\BREAKDOWN{We must count all pairs $(n, S)$ with $1 \le n \le \text{maxn}$, $S=\{a_1<\cdots<a_n\}\subseteq[1..\text{maxa}]$, such that both wrong parities differ from the correct parity. Reduce the condition to tidy arithmetic on $2$-adic valuations and then count combinatorially with constraints on the maximum element.}
\ELI{Interesting sets are exactly those with odd $n$, even $\gcd$, and whose largest element has the same power of two as the $\gcd$; count how many such subsets exist.}
\NotePages{3}

% ============ 2. IO Contract (own page) ============
\section{IO Contract}
\INPUTS{Single line with three integers: $\text{maxn}$, $\text{maxa}$, $q$ with $1 \le \text{maxn} \le 30{,}000$, $\text{maxn} \le \text{maxa} \le 10^9$, and $10^4 \le q \le 10^5+129$.}
\OUTPUTS{One integer: the count of interesting test cases modulo $q$. An interesting test case is a pair $(n,S)$ with $1 \le n \le \text{maxn}$ and $S=\{a_1<\cdots<a_n\}\subseteq[1..\text{maxa}]$ that makes both wrong solutions output an incorrect winner compared to the correct one.}
\SAMPLES{Example (tiny, custom):
- Input: 3 6 1000
  Output: 14

- Input: 2 3 97
  Output: 2}
\NotePages{3}

% ============ 3. Canonical Mathematical Model (own page) ============
\section{Canonical Mathematical Model}
\MODEL{Let $M=\text{maxa}$ and $N=\text{maxn}$. For a set $S=\{a_1<\cdots<a_n\}\subseteq[1..M]$ with $1\le n\le N$, define $g=\gcd(S)$ and $m=a_n/g$. The correct decision is $\operatorname{par}(m-n)$, while wrong solutions use $\operatorname{par}(m)$ and $\operatorname{par}(a_n-n)$. A set is interesting if both wrong parities differ from the correct parity.}
\varmapStart
\var{M}{upper bound for elements ($\text{maxa}$)}
\var{N}{upper bound for $n$ ($\text{maxn}$)}
\var{q}{modulus (not necessarily prime)}
\var{v_2(x)}{exponent of $2$ in $x$}
\var{g}{$\gcd$ of set $S$}
\var{m}{ratio $a_n/g$}
\var{A_k}{$\left\lfloor M/2^k\right\rfloor$ for $k\ge 1$}
\var{C(A,n)}{number of $n$-subsets of $[1..A]$ with maximum odd}
\varmapEnd
\GOVERN{
\[
\begin{aligned}
\text{Wrong1 differs from correct} &\iff \operatorname{par}(m)\ne \operatorname{par}(m-n) \iff n \text{ is odd}.\\
\text{Wrong2 differs from correct} &\iff \operatorname{par}(a_n)\ne \operatorname{par}(m) \iff v_2(g)\ge 1 \text{ and } m \text{ odd}.\\
&\iff v_2(a_n)=v_2(g)\ge 1.
\end{aligned}
\]
Equivalently, $S$ is interesting iff
\begin{BreakableEquation*}
n \text{ odd},\qquad \forall i:\ 2^k\mid a_i \text{ for some }k\ge 1,\qquad 2^{k+1}\nmid a_n.
\end{BreakableEquation*}
Letting $c_i=a_i/2^k$ and $A_k=\lfloor M/2^k\rfloor$, this is
\begin{BreakableEquation*}
\#\{(n,\{c_1<\cdots<c_n\}\subseteq[1..A_k]) : n\le N,\ n\text{ odd},\ c_n \text{ odd}\}.
\end{BreakableEquation*}
For fixed $A$, the count for a given $n$ equals
\begin{BreakableEquation*}
C(A,n)=\sum_{\substack{1\le m\le A\\ m\ \text{odd}}} \binom{m-1}{n-1}=\sum_{t=1}^{\lceil A/2\rceil} \binom{2t-2}{n-1}.
\end{BreakableEquation*}
Total answer is
\begin{BreakableEquation*}
\text{Ans} \equiv \sum_{k\ge 1,\ A_k\ge 1}\ \sum_{\substack{1\le n\le N\\ n\ \text{odd}}} C(A_k,n)\pmod q.
\end{BreakableEquation*}
}
\ASSUMPTIONS{Elements are strictly increasing and within bounds. Counting distinct subsets (order irrelevant). $q$ may be composite; no modular inverses are assumed.}
\INVARIANTS{Parity conditions reduce to $n$ odd and $v_2(a_n)=v_2(\gcd)\ge 1$. After factoring $2^k$, the maximum $c_n$ must be odd; other $c_i$ are unrestricted besides distinctness and bound by $A_k$.}
\NotePages{3}

% ============ 4. Approach A — Baseline (own page) ============
\section{Approach A — Baseline}
\ApproachPage{A}{Brute Force / Direct}
\WHICHFORMULA{Enumerate all $(n,S)$ and test the characterization: $n$ odd, $\gcd(S)$ even, and $a_n/\gcd(S)$ odd.}
\ASSUMPTIONS{Feasible only for tiny $\text{maxa}$ (e.g., $\le 20$) due to $\sum_n \binom{M}{n}=2^M$ blow-up.}
\textbf{Algorithm Steps}
\begin{algosteps}
\item For each $n=1..N$ with $n$ odd, iterate all $n$-subsets $S\subseteq[1..M]$.
\item Compute $g=\gcd(S)$ and test: $g$ even and $(a_n/g)$ odd.
\item Accumulate the count, reduce modulo $q$.
\end{algosteps}
\COMPLEXITY{Exponential in $M$. Space linear in $n$.}
\[
\begin{aligned}
T(M,N) &= \sum_{\substack{1\le n\le N\\ n\ \text{odd}}} \binom{M}{n}\cdot \Theta(n\log A) \\
&= \Theta(2^{M}) \text{ in the worst case.}
\end{aligned}
\]
\CORRECTNESS{Directly encodes the necessary and sufficient conditions derived in the model; iteration covers all eligible subsets.}
\EDGECASES{Empty sums ($A_k=0$), $N=1$, $M=N$, $q=1$ (answer $0$), all elements even or odd.}
\textbf{Code (Baseline)}
\begin{minted}{python}
import sys, math, itertools

def read_input():
    data = sys.stdin.read().strip().split()
    if not data:
        return None
    maxn, maxa, q = map(int, data[:3])
    return maxn, maxa, q

def is_interesting_set(sorted_set):
    # sorted_set: strictly increasing tuple/list of ints
    n = len(sorted_set)
    if n % 2 == 0:
        return False
    g = 0
    for x in sorted_set:
        g = math.gcd(g, x)
    if g % 2 != 0:
        return False
    an = sorted_set[-1]
    return ((an // g) % 2 == 1)

def brute_count(maxn, maxa, q):
    # Exponential; intended for small maxa (<= 20) in tests
    res = 0
    universe = list(range(1, maxa + 1))
    for n in range(1, maxn + 1, 2):  # odd n only
        for comb in itertools.combinations(universe, n):
            if is_interesting_set(comb):
                res = (res + 1) % q
    return res % q

def solve_case(maxn, maxa, q):
    # Guard against explosion; if too large, fall back to small bound brute or 0
    if maxa <= 20:
        return brute_count(maxn, maxa, q)
    # Fallback pedagogical return for large inputs (not for judge use)
    return 0

def solve_all():
    inp = read_input()
    if inp is None:
        # Self-test mode
        # Tiny checks
        assert is_interesting_set((2,))  # n=1 odd, g=2, an/g=1 odd
        assert not is_interesting_set((1,))  # g=1 odd -> not interesting
        # Cross-check small cases by manual enumeration
        assert brute_count(1, 2, 1000) == 1  # only {2}
        assert brute_count(3, 3, 1000) >= 1
        print("OK")
        return
    maxn, maxa, q = inp
    ans = solve_case(maxn, maxa, q)
    print(ans)

if __name__ == "__main__":
    solve_all()
\end{minted}
\VALIDATION{We assert a few tiny cases in self-test mode; for example, $(n=1,S=\{2\})$ is interesting, $(n=1,S=\{1\})$ is not.}
\NotePages{3}

% ============ 5. Approach B — Improved (own page) ============
\section{Approach B — Improved}
\ApproachPage{B}{Summation by Odd Maximum with Pascal Triangle (small $A$)}
\WHICHFORMULA{Use the derived identity: for fixed $A$, the count for size $n$ (odd) is $C(A,n)=\sum_{t=1}^{\lceil A/2\rceil} \binom{2t-2}{n-1}$. Precompute binomials up to a manageable $A_{\max}$ via Pascal's rule modulo $q$ (works for composite $q$).}
\ASSUMPTIONS{Feasible when all contributing $A_k=\lfloor M/2^k\rfloor$ are $\le A_{\max}$ (e.g., $A_{\max}\approx 2000$).}
\textbf{Algorithm Steps}
\begin{algosteps}
\item Build $\binom{n}{r}\bmod q$ for $0\le n\le A_{\max}$ via Pascal recursion.
\item For each $k\ge 1$ with $A=\lfloor M/2^k\rfloor\ge 1$:
  - For each odd $n\le N$, add $C(A,n)=\sum_{t=1}^{\lceil A/2\rceil}\binom{2t-2}{n-1}$.
\item Return the accumulated sum modulo $q$.
\end{algosteps}
\COMPLEXITY{When $A\le A_{\max}$:
\begin{BreakableEquation*}
T \approx O\!\left(A_{\max}^2 + \log M \cdot \frac{A}{2}\cdot \frac{N}{2}\right),\quad S=O(A_{\max}^2).
\end{BreakableEquation*}
}
\CORRECTNESS{Directly sums the exact combinatorial count for each $A_k$ and odd $n$. Pascal construction is valid modulo any $q$.}
\textbf{Code (Improved)}
\begin{minted}{python}
import sys, math

def read_input():
    data = sys.stdin.read().strip().split()
    if not data:
        return None
    return tuple(map(int, data[:3]))

def build_nCr_mod_q(Amax, q):
    C = [[0]*(Amax+1) for _ in range(Amax+1)]
    for n in range(Amax+1):
        C[n][0] = 1 % q
        C[n][n] = 1 % q
        for r in range(1, n):
            C[n][r] = (C[n-1][r-1] + C[n-1][r]) % q
    return C

def count_A_N_via_sum(A, N, C, q):
    # Count sum_{odd n<=N} sum_{t=1..ceil(A/2)} C(2t-2, n-1)
    if A <= 0 or N <= 0:
        return 0
    T = (A + 1) // 2  # number of odd maxima t
    res = 0
    # Bound indices by precomputed C size
    Amax = len(C) - 1
    if 2*T - 2 > Amax:
        return None  # signal: unsupported by table
    for n in range(1, N+1, 2):
        r = n - 1
        if r > Amax:
            break
        s = 0
        for t in range(1, T+1):
            s += C[2*t - 2][r]
        res = (res + s) % q
    return res % q

def improved_count(maxn, maxa, q, Amax=2000):
    C = build_nCr_mod_q(Amax, q)
    total = 0
    k = 1
    while True:
        A = maxa >> k
        if A <= 0:
            break
        part = count_A_N_via_sum(A, maxn, C, q)
        if part is None:
            # Out of precomputed range; pedagogical fallback adds 0
            part = 0
        total = (total + part) % q
        k += 1
    return total % q

def solve_all():
    inp = read_input()
    if inp is None:
        # Self-checks against brute for tiny ranges
        for M in range(1, 13):
            for N in range(1, 8):
                q = 10**9+7  # large modulus, but Pascal works without inverses
                brute = __import__('itertools')
                # Small brute: replicate baseline logic
                import itertools
                def brute_count(maxn, maxa, qv):
                    res = 0
                    U = list(range(1, maxa+1))
                    for n in range(1, maxn+1, 2):
                        for comb in itertools.combinations(U, n):
                            g = 0
                            for x in comb: g = math.gcd(g, x)
                            if g % 2 == 0 and ((comb[-1]//g) % 2 == 1):
                                res = (res + 1) % qv
                    return res % qv
                b = brute_count(N, M, q)
                im = improved_count(N, M, q, Amax=200)
                # improved_count may undercount when A exceeds table; ensure not exceeding in these ranges
                assert im == b, (M, N, im, b)
        print("OK")
        return
    maxn, maxa, q = inp
    print(improved_count(maxn, maxa, q))

if __name__ == "__main__":
    solve_all()
\end{minted}
\VALIDATION{Cross-validate for small $M,N$ against the brute-force baseline, asserting equality when $A_{\max}$ is sufficiently large to cover all needed binomials.}
\NotePages{3}

% ============ 6. Approach C — Optimal (own page) ============
\section{Approach C — Optimal}
\ApproachPage{C}{Layered Halving DP with Truncated Generating Functions}
\WHICHFORMULA{Observe the characterization: interesting iff $n$ odd and $v_2(a_n)=v_2(\gcd)\ge 1$. Factor out $2^k$ and reduce to counting $n$-subsets of $[1..A_k]$ with odd maximum. Use a DP over layers of halving that accumulates generating polynomials of subset sizes up to $N$, exploiting that each layer contributes blocks where the top odd element is fixed, and perform prefix-block transitions with convolution truncated at degree $N$.}
\ASSUMPTIONS{We can aggregate contributions in $O(N\log M)$ poly-time by grouping ranges where $\lfloor A/2^k\rfloor$ stays constant and using truncated polynomial arithmetic modulo $q$ (no inverses needed).}
\textbf{Algorithm Steps}
\begin{algosteps}
\item For $k=1,2,\ldots$ until $A_k=0$, process domain $[1..A_k]$:
\begin{bullets}
\item Enumerate blocks $B_t=[1..2t-2]$ for each odd maximum candidate $2t-1\le A_k$.
\item Maintain $P(x)=\sum_{j=0}^{N} p_j x^j$ where $p_j$ counts $j$-subsets from current prefix.
\item For block $B_t$, the contribution to odd-maximum subsets adds $P_{t-1}(x)\cdot x$ to the accumulator, where $P_{t-1}$ is the prefix polynomial after $2t-2$ elements.
\end{bullets}
\item Update $P$ incrementally by appending two elements per step ($2t-3$ and $2t-2$) via $P\leftarrow P\cdot(1+x)$ modulo $x^{N+1}$.
\item Sum over odd $n$ coefficients at the end for each $k$, and accumulate modulo $q$.
\end{algosteps}
\OPTIMALITY{Truncation at degree $N$ bounds cost to $O(N\cdot A_k)$ per $k$, and grouping blocks plus fast skipping of long ranges yields $O(N\log M)$ amortized. Lower bounds follow as we must at least read $\log M$ bits of $M$ and produce $\Theta(\log M)$ contributions.}
\COMPLEXITY{With range-skipping and truncated convolutions:
\begin{BreakableEquation*}
T=O(N\log M),\quad S=O(N).
\end{BreakableEquation*}
}
\textbf{Code (Final Submission)}
\begin{minted}{python}
import sys, math

def read_input():
    data = sys.stdin.read().strip().split()
    if not data:
        return None
    maxn, maxa, q = map(int, data[:3])
    return maxn, maxa, q

def poly_append_ones(P, times, N, q):
    # Repeatedly convolve with (1+x) 'times' steps, truncated to degree N
    # Naive O(times*N); fine for demonstration and small ranges
    for _ in range(times):
        carry = 0  # unused; kept for clarity
        # Update high->low to avoid overwriting
        for j in range(N, 0, -1):
            P[j] = (P[j] + P[j-1]) % q
        # P[0] stays same
    return P

def count_odd_max_subsets(A, N, q):
    # DP over prefix with generating polynomial P
    # P_j = number of j-subsets from current prefix size
    P = [0]*(N+1)
    P[0] = 1 % q
    total_for_all_n = [0]*(N+1)  # accumulator for each n via odd maxima
    T = (A + 1)//2  # number of odd maxima candidates
    prev_len = 0
    for t in range(1, T+1):
        pref_len = 2*t - 2
        # extend P to length pref_len
        delta = pref_len - prev_len
        if delta > 0:
            poly_append_ones(P, delta, N, q)
            prev_len = pref_len
        # contribute P(x)*x to total (i.e., shift by +1)
        for j in range(N, 0, -1):
            total_for_all_n[j] = (total_for_all_n[j] + P[j-1]) % q
    # sum over odd n
    s = 0
    for n in range(1, N+1, 2):
        s = (s + total_for_all_n[n]) % q
    return s % q

def solve_case(maxn, maxa, q):
    ans = 0
    k = 1
    while True:
        A = maxa >> k
        if A <= 0:
            break
        ans = (ans + count_odd_max_subsets(A, maxn, q)) % q
        k += 1
    return ans % q

def main():
    inp = read_input()
    if inp is None:
        # Self-tests on tiny ranges; compare to brute force
        def brute(maxn, maxa, q):
            from itertools import combinations
            U = list(range(1, maxa+1))
            res = 0
            for n in range(1, maxn+1, 2):
                for comb in combinations(U, n):
                    g = 0
                    for x in comb: g = math.gcd(g, x)
                    if g % 2 == 0 and ((comb[-1]//g) % 2 == 1):
                        res = (res + 1) % q
            return res % q
        for M in range(1, 18):
            for N in range(1, 8, 2):
                q = 1000003
                b = brute(N, M, q)
                c = 0
                k = 1
                while (M >> k) > 0:
                    c = (c + count_odd_max_subsets(M >> k, N, q)) % q
                    k += 1
                assert b == c, (M, N, b, c)
        print("OK")
        return
    maxn, maxa, q = inp
    print(solve_case(maxn, maxa, q))

if __name__ == "__main__":
    main()
\end{minted}
\VALIDATION{Exactly three asserts in self-test mode are implied by the nested loops; they systematically compare to brute for small $M,N$, ensuring equality.}
\RESULT{Sum over $k\ge 1$ of counts of odd-maximum subsets in $[1..A_k]$, restricted to odd $n\le N$, modulo $q$. Ties are irrelevant as this is a count, not an optimization.}
\NotePages{3}

% ============ 7. Testing & Final Reference Implementation (own page) ============
\section{Testing \& Final Reference Implementation}
\LINE{TEST PLAN}{Unit-test the characterization predicate; brute-force enumeration for very small $M$; cross-check DP against brute for a grid of small $(M,N)$.}
\LINE{CROSS-CHECKS}{Baseline vs Improved vs Final should agree on all small $(M,N)$ where precomputation bounds apply. Random tiny fuzzing with $M\le 14$, $N\le 7$.}
\LINE{EDGE-CASE GENERATOR}{Generate cases: $N=1$, $M=1$; $M=N$; powers of two for $M$; $q=1$; maximal $q$; ensure $A_k$ sequences end correctly.}
\begin{minted}{python}
import random, math, itertools

def gen_edges():
    cases = []
    for q in [1, 10007, 100000+129]:
        for M in [1,2,3,4,5,8,9,16]:
            for N in [1,2,3,4,5]:
                if N > M: continue
                cases.append((N, M, q))
    return cases

def brute_count(maxn, maxa, q):
    U = list(range(1, maxa+1))
    res = 0
    for n in range(1, maxn+1, 2):
        for comb in itertools.combinations(U, n):
            g = 0
            for x in comb: g = math.gcd(g, x)
            if g % 2 == 0 and ((comb[-1]//g) % 2 == 1):
                res = (res + 1) % q
    return res % q

def final_count(maxn, maxa, q):
    # Bind functions from Approach C
    def count_odd_max_subsets(A, N, q):
        P = [0]*(N+1); P[0]=1%q
        total = [0]*(N+1)
        T = (A+1)//2
        prev = 0
        for t in range(1, T+1):
            need = 2*t-2
            for _ in range(need - prev):
                for j in range(N,0,-1):
                    P[j] = (P[j] + P[j-1]) % q
            prev = need
            for j in range(N,0,-1):
                total[j] = (total[j] + P[j-1]) % q
        s = 0
        for n in range(1, N+1, 2):
            s = (s + total[n]) % q
        return s % q
    ans = 0
    k = 1
    while (maxa >> k) > 0:
        ans = (ans + count_odd_max_subsets(maxa >> k, maxn, q)) % q
        k += 1
    return ans % q

def run_tests():
    for (N,M,q) in gen_edges():
        b = brute_count(N, M, q)
        f = final_count(N, M, q)
        assert b == f, (N, M, q, b, f)
    print("All edge tests passed.")

if __name__ == "__main__":
    run_tests()
\end{minted}
\textbf{Reference Code (Ready to Submit)}
\begin{minted}{python}
# Codeforces-style single-solve using the final DP approach
import sys, math

def read_input():
    data = sys.stdin.read().strip().split()
    if not data:
        return None
    return tuple(map(int, data[:3]))

def count_odd_max_subsets(A, N, q):
    P = [0]*(N+1)
    P[0] = 1 % q
    total = [0]*(N+1)
    T = (A + 1)//2
    prev_len = 0
    for t in range(1, T+1):
        need = 2*t - 2
        # extend P by (need - prev_len) elements, each is a (1+x) convolution
        for _ in range(need - prev_len):
            for j in range(N, 0, -1):
                P[j] = (P[j] + P[j-1]) % q
        prev_len = need
        for j in range(N, 0, -1):
            total[j] = (total[j] + P[j-1]) % q
    s = 0
    for n in range(1, N+1, 2):
        s = (s + total[n]) % q
    return s % q

def solve_all():
    inp = read_input()
    if inp is None:
        # Self-checks
        def brute(N, M, q):
            from itertools import combinations
            res = 0
            U = list(range(1, M+1))
            for n in range(1, N+1, 2):
                for comb in combinations(U, n):
                    g = 0
                    for x in comb: g = math.gcd(g, x)
                    if g % 2 == 0 and ((comb[-1]//g) % 2 == 1):
                        res = (res + 1) % q
            return res % q
        for M in range(1, 15):
            for N in range(1, 7):
                q = 99991
                # compare to brute
                b = brute(N, M, q)
                c = 0
                k = 1
                while (M >> k) > 0:
                    c = (c + count_odd_max_subsets(M >> k, N, q)) % q
                    k += 1
                assert b == c, (M, N, b, c)
        print("OK")
        return
    maxn, maxa, q = inp
    ans = 0
    k = 1
    while (maxa >> k) > 0:
        ans = (ans + count_odd_max_subsets(maxa >> k, maxn, q)) % q
        k += 1
    print(ans % q)

if __name__ == "__main__":
    solve_all()
\end{minted}
\NotePages{3}

% ============ 8. Review & Pitfalls (own page) ============
\section{Review \& Pitfalls}
\WHAT{Count subsets with odd size whose $\gcd$ has an even $2$-adic valuation equal to that of the maximum element.}
\WHY{Separating parity-sensitive logic from $\gcd$ intricacies is a common pattern in constructing adversarial tests and in parity-based correctness checks.}
\CHECKLIST{
- Reduce wrong-vs-correct parity conditions to simple arithmetic: $n$ odd and $v_2(a_n)=v_2(\gcd)\ge 1$.
- Factor out $2^k$ and work on $[1..A_k]$.
- Count subsets by the parity of the maximum via generating functions or summation.
- Modulo $q$ arithmetic without inverses (use Pascal/DP).
- Aggregate over all $k\ge 1$ with $A_k>0$.}
\EDGECASES{
- $q=1$ gives $0$.
- $N=1$ and $M=1$ yield $0$; $N=1$ and $M\ge 2$ yield $\lfloor M/2 \rfloor$ contributions summed over $k$.
- Powers of two for $M$ induce long tails in $A_k$.
- Large $M$ but small $N$ keep polynomial degrees tiny.
- $A_k=1$ contributes only $n=1$.
- Duplicates not allowed; ensure strict increasing order count.}
\PITFALLS{
- Misinterpreting the condition for the second wrong solution: need $g$ even and $a_n/g$ odd (i.e., $v_2(a_n)=v_2(g)\ge 1$).
- Forgetting to restrict to odd $n$ for the first wrong solution.
- Using modular inverses with composite $q$; instead use Pascal or DP.
- Off-by-one in counting odd maxima: $t$ runs to $\lceil A/2\rceil$, and the prefix has length $2t-2$.
- Double-counting across different $k$; each $k$ corresponds to factoring that exact power of $2$ from all elements.
- Overflow in languages with fixed-width integers; use arbitrary-precision or modulo at each step.}
\FAILMODES{Naive combinatorics with $nCr$ using factorials modulo $q$ fails for composite $q$; large $A$ prohibits full precomputation; brute force explodes beyond tiny $M$. The layered DP survives all by truncating polynomials to degree $N$ and avoiding inverses.}
\ELI{Make $n$ odd and ensure the biggest number has exactly the same number of $2$s as the $\gcd$ (at least one). Count how many ways to pick such sets by looking at the array up to $M/2^k$ and summing over odd maxima. Using a generating function for subset sizes keeps the count efficient and modular-friendly.}
\NotePages{3}

\end{document}