% !TeX program = xelatex
\documentclass{article}
\usepackage[margin=1in]{geometry}
\usepackage{microtype}
\usepackage{setspace}\setstretch{1.05}
\usepackage{fontspec}
\setmainfont{Latin Modern Roman}
\setmonofont{Latin Modern Mono}[BoldFont={Latin Modern Mono},ItalicFont={Latin Modern Mono}]
\usepackage{amsmath,mathtools,amsthm}
\newtheorem{lemma}{Lemma}
\newtheorem{theorem}{Theorem}
\usepackage{unicode-math}\setmathfont{Latin Modern Math}\allowdisplaybreaks[4]
% --- overflow and alignment slack ---
\setlength{\jot}{7pt}
\sloppy\emergencystretch=8em\hfuzz=1pt\vfuzz=2pt\raggedbottom
% --- breakable math helpers ---
\newenvironment{BreakableEquation}{\begin{equation}\begin{aligned}}{\end{aligned}\end{equation}}
\newenvironment{BreakableEquation*}{\begin{equation*}\begin{aligned}}{\end{aligned}\end{equation*}}
\newenvironment{tightalign}{\begingroup\small\allowdisplaybreaks\begin{align}}{\end{align}\endgroup}

\usepackage{xcolor}
\usepackage{xurl} % better URL wrapping
\usepackage[colorlinks=true,linkcolor=black,urlcolor=blue]{hyperref}
\usepackage{fancyhdr}
\pagestyle{fancy}\fancyhf{}\lhead{\nouppercase{\leftmark}}\rhead{\thepage}\setlength{\headheight}{26pt}
\usepackage{enumitem,booktabs,tabularx}
\newcolumntype{L}{>{\raggedright\arraybackslash}X}
\renewcommand{\arraystretch}{1.1}
\setlength{\parindent}{0pt}\setlength{\parskip}{8pt plus 2pt minus 1pt}

% Listings + upquote (no shell-escape needed)
\usepackage{listings}
\usepackage{upquote}
\lstdefinestyle{crisp}{
  basicstyle=\ttfamily\footnotesize,
  frame=single,
  breaklines=true,
  breakatwhitespace=false, % allow breaks inside long tokens
  tabsize=4,
  showstringspaces=false,
  numbers=left,
  numberstyle=\tiny,
  numbersep=8pt,
  keepspaces=true,
  columns=fullflexible,
  backgroundcolor=\color{black!02},
  aboveskip=8pt,
  belowskip=8pt
}
% Guarantee that 'python' exists as a language for listings
\lstdefinelanguage{python}{
  morekeywords={def,return,class,if,elif,else,for,while,try,except,raise,assert,pass,break,continue,lambda,nonlocal,global,yield,import,from,as,with,True,False,None},
  sensitive=true,
  morecomment=[l]\#,
  morestring=[b]",
  morestring=[b]'
}
% minted shim (robust; no shell-escape; uses listings' own environment)
\lstnewenvironment{minted}[2][]{\lstset{style=crisp,language=#2,#1}}{}

\usepackage[most]{tcolorbox}
\tcbset{colback=white,colframe=black!15,boxrule=0.4pt,arc=2pt,left=6pt,right=6pt,top=6pt,bottom=6pt,before skip=10pt,after skip=10pt,breakable}
\usepackage{titlesec}
\titleformat{\section}{\large\bfseries}{\thesection}{0.6em}{}
\titlespacing*{\section}{0pt}{*2.0}{*1.0}

% ===== CONSISTENCY TOOLKIT (macros) =====
\newlist{ledger}{enumerate}{1}
\setlist[ledger]{label=\textbullet,leftmargin=2em,itemsep=2pt,topsep=6pt}
\newcommand{\LEDGER}[1]{\textbf{ARITHMETIC LEDGER:}\par\begin{ledger}#1\end{ledger}}

\newlist{algosteps}{enumerate}{1}
\setlist[algosteps]{label=\arabic*.,leftmargin=2em,itemsep=2pt,topsep=6pt}

\newlist{bullets}{itemize}{1}
\setlist[bullets]{label=--,leftmargin=1.6em,itemsep=2pt,topsep=4pt}

\newcommand{\varmapStart}{\textbf{VARIABLE MAPPING:}\par\begin{bullets}}
\newcommand{\var}[2]{\item $#1$ — #2}
\newcommand{\varmapEnd}{\end{bullets}}

\newcommand{\LINE}[2]{\par\noindent\textbf{#1:}~#2\par}
\newcommand{\WHAT}[1]{\LINE{WHAT}{#1}}
\newcommand{\WHY}[1]{\LINE{WHY}{#1}}
\newcommand{\HOW}[1]{\LINE{HOW}{#1}}
\newcommand{\ELI}[1]{\LINE{ELI5}{#1}}
\newcommand{\STATEMENT}[1]{\LINE{STATEMENT}{#1}}
\newcommand{\BREAKDOWN}[1]{\LINE{PROBLEM BREAKDOWN}{#1}}
\newcommand{\MODEL}[1]{\LINE{CANONICAL MATHEMATICAL MODEL}{#1}}
\newcommand{\ASSUMPTIONS}[1]{\LINE{ASSUMPTIONS}{#1}}
\newcommand{\INVARIANTS}[1]{\LINE{INVARIANTS}{#1}}
\newcommand{\WHICHFORMULA}[1]{\LINE{WHICH FORMULA \& WHY}{#1}}
\newcommand{\GOVERN}[1]{\LINE{GOVERNING EQUATION(S)}{#1}}
\newcommand{\INPUTS}[1]{\LINE{INPUTS}{#1}}
\newcommand{\OUTPUTS}[1]{\LINE{OUTPUTS}{#1}}
\newcommand{\SAMPLES}[1]{\LINE{SAMPLES}{#1}}
\newcommand{\RESULT}[1]{\LINE{RESULT}{#1}}
\newcommand{\COMPLEXITY}[1]{\LINE{TIME/SPACE COMPLEXITY}{#1}}
\newcommand{\MEMORY}[1]{\LINE{MEMORY FOOTPRINT}{#1}}
\newcommand{\CORRECTNESS}[1]{\LINE{CORRECTNESS SKETCH}{#1}}
\newcommand{\OPTIMALITY}[1]{\LINE{OPTIMALITY}{#1}}
\newcommand{\FAILMODES}[1]{\LINE{FAILURE MODES}{#1}}
\newcommand{\VALIDATION}[1]{\LINE{VALIDATION}{#1}}
\newcommand{\UNITCHECK}[1]{\LINE{UNIT CHECK}{#1}}
\newcommand{\EDGECASES}[1]{\LINE{EDGE CASES}{#1}}
\newcommand{\CHECKLIST}[1]{\LINE{CHECKLIST}{#1}}
\newcommand{\DERIV}[1]{\LINE{DERIVATION}{#1}}
\newcommand{\LEMMAHEAD}[1]{\LINE{SUPPORTING LEMMA}{#1}}
\newcommand{\PITFALLS}[1]{\LINE{PITFALLS}{#1}}

\usepackage{etoolbox}\pretocmd{\section}{\clearpage}{}{}

\newcommand{\FormulaPage}[2]{%
  \clearpage
  \section*{Formula #1 — #2}%
  \addcontentsline{toc}{section}{Formula #1 — #2}%
}
\newcommand{\ApproachPage}[2]{%
  \clearpage
  \subsection*{Approach #1 — #2}%
  \addcontentsline{toc}{subsection}{Approach #1 — #2}%
}

\begin{document}
\title{Interview Problem Sheet — String Distance}
\date{\today}
\author{}
\maketitle
\tableofcontents
\clearpage

% === Notes macro (BODY-ONLY; do not alter the preamble) ===
% Prints exactly N blank pages with empty headers/footers, then leaves you on the last blank page.
\newcommand{\NotePages}[1]{%
  \clearpage
  \begingroup
  \newcount\NPi \newcount\NPn
  \NPi=0 \NPn=#1
  \loop\ifnum\NPi<\NPn
    \advance\NPi by 1
    \mbox{}\thispagestyle{empty}%
    \ifnum\NPi<\NPn\clearpage\fi
  \repeat
  \endgroup
}

% ============ 1. Problem & Metadata (own page) ============
\section{Problem \& Metadata}
\LINE{PLATFORM}{CF}
\LINE{URL}{\url{https://codeforces.com/problemset/problem/1535/F}}
\LINE{DIFFICULTY / RATING}{3000}
\STATEMENT{Suppose you are given two strings $a$ and $b$. You can apply the following operation any number of times: choose any contiguous substring of $a$ or $b$, and sort the characters in it in non-descending order. Let $f(a, b)$ be the minimum number of operations you have to apply in order to make them equal (or $f(a, b) = 1337$ if it is impossible to make $a$ and $b$ equal using these operations).

For example:
\begin{bullets}
\item $f(\text{ab}, \text{ab}) = 0$;
\item $f(\text{ba}, \text{ab}) = 1$ (in one operation, we can sort the whole first string);
\item $f(\text{ebcda}, \text{ecdba}) = 1$ (in one operation, we can sort the substring of the second string starting from the $2$-nd character and ending with the $4$-th character);
\item $f(\text{a}, \text{b}) = 1337$.
\end{bullets}

You are given $n$ strings $s_1, s_2, \ldots, s_n$ having equal length. Calculate $\sum \limits_{i = 1}^{n} \sum\limits_{j = i + 1}^{n} f(s_i, s_j)$.

Input: The first line contains one integer $n$ ($1 \le n \le 2 \cdot 10^5$) — the number of strings.

Then $n$ lines follow, each line contains one of the strings $s_i$, consisting of lowercase Latin letters. $|s_1| = |s_2| = \ldots = |s_n|$, and $n \cdot |s_1| \le 2 \cdot 10^5$. All these strings are pairwise distinct.

Output: Print one integer: $\sum \limits_{i = 1}^{n} \sum\limits_{j = i + 1}^{n} f(s_i, s_j)$.}
\BREAKDOWN{Reduce $f(a,b)$ to a small set of values by characterizing what one substring sort can achieve. Group strings by letter multiset to isolate impossible pairs ($1337$). Within each group, count pairs reachable by one sort; the rest require two sorts.}
\ELI{Either the letters differ and we add $1337$, or they match and we add $1$ if one string arises from the other by sorting one block, else $2$.}
\NotePages{3}

% ============ 2. IO Contract (own page) ============
\section{IO Contract}
\INPUTS{One integer $n$; then $n$ distinct strings of equal length over $\{a,\ldots,z\}$. Constraints: $1 \le n \le 2\cdot 10^5$, total input size $n\cdot |s_1| \le 2\cdot 10^5$.}
\OUTPUTS{A single integer: $\sum_{1 \le i < j \le n} f(s_i, s_j)$.}
\SAMPLES{Examples of $f$ on pairs: $f(\text{ab},\text{ab})=0$, $f(\text{ba},\text{ab})=1$, $f(\text{a},\text{b})=1337$. Since input strings are distinct, $0$ does not occur in the final sum.}
\NotePages{3}

% ============ 3. Canonical Mathematical Model (own page) ============
\section{Canonical Mathematical Model}
\MODEL{Let $\Sigma=\{a,\ldots,z\}$. For a string $s\in\Sigma^L$, let $\mathrm{sort}(s[l..r])$ denote $s$ with the substring $[l,r]$ replaced by its non-decreasing sort. For two strings $a,b$ of equal length, define
\begin{BreakableEquation*}
f(a,b)=\min\{t\in \mathbb{N}_{\ge 0}:\ \exists \text{ a sequence of }t\text{ operations on }a\text{ or }b\text{ s.t. the results are equal}\},
\end{BreakableEquation*}
and $f(a,b)=1337$ if no such sequence exists.}
\varmapStart
\var{n}{number of strings}
\var{L}{common length of the strings}
\var{c(s)}{Parikh vector (letter counts) of $s$ in $\mathbb{N}^{26}$}
\var{D(s)}{set of descent indices $\{i: s[i] > s[i+1]\}$}
\var{l,r}{first and last mismatch positions between two strings}
\varmapEnd
\GOVERN{
\[
\begin{aligned}
f(a,b) &=
\begin{cases}
1337,& c(a)\ne c(b),\\
0,& a=b,\\
1,& c(a)=c(b),\ a\ne b,\ \text{and $a[l..r]$ or $b[l..r]$ is non-decreasing},\\
2,& \text{otherwise.}
\end{cases}
\end{aligned}
\]
}
\ASSUMPTIONS{Strings are pairwise distinct, equal length, lowercase letters only.}
\INVARIANTS{
\begin{bullets}
\item Sorting any substring preserves the Parikh vector.
\item If $c(a)=c(b)$, then sorting the whole strings makes both equal in at most $2$ operations.
\item If $a\ne b$ and $l,r$ are first/last mismatches, then $c\big(a[l..r]\big)=c\big(b[l..r]\big)$.
\end{bullets}
}
\NotePages{3}

% ============ 4. Approach A — Baseline (own page) ============
\section{Approach A — Baseline}
\ApproachPage{A}{Brute Force / Direct}
\WHICHFORMULA{Compute $f(a,b)$ pairwise using the piecewise definition from the model.}
\ASSUMPTIONS{Use $O(L)$ Parikh comparison and $O(L)$ non-decreasing checks per pair.}
\textbf{Algorithm Steps}
\begin{algosteps}
\item For each pair $(i,j)$, compare Parikh vectors $c(s_i)$ and $c(s_j)$; if different, add $1337$.
\item Else, find first and last mismatch indices $l,r$.
\item If the strings are identical, add $0$; else if $s_i[l..r]$ or $s_j[l..r]$ is non-decreasing, add $1$; otherwise add $2$.
\end{algosteps}
\COMPLEXITY{For $n$ strings of length $L$, the baseline costs $O(n^2\cdot L)$ time and $O(1)$ extra space.}
\[
\begin{aligned}
T(n) &= \binom{n}{2}\cdot O(L) \\
     &= O(n^2 L) \\
\end{aligned}
\]
\CORRECTNESS{Directly matches the characterization: if Parikh vectors differ, impossibility $1337$; if equal and strings differ only within $[l,r]$ where one side is already sorted, one operation suffices; otherwise two operations (sort whole of both).}
\EDGECASES{All strings distinct so $0$ does not occur across pairs; strings of length $1$; highly repetitive letters; entirely sorted strings.}
\textbf{Code (Baseline)}
\begin{minted}{python}
from typing import List, Tuple
import sys

def is_nondec(seg: str) -> bool:
    # check non-decreasing
    for i in range(len(seg) - 1):
        if seg[i] > seg[i+1]:
            return False
    return True

def f_pair(a: str, b: str) -> int:
    if len(a) != len(b):
        return 1337
    L = len(a)
    if a == b:
        return 0
    # check Parikh vectors
    cnt = [0]*26
    for ch in a:
        cnt[ord(ch)-97] += 1
    for ch in b:
        cnt[ord(ch)-97] -= 1
    if any(x != 0 for x in cnt):
        return 1337
    # find first and last mismatch
    l = 0
    while l < L and a[l] == b[l]:
        l += 1
    r = L - 1
    while r >= 0 and a[r] == b[r]:
        r -= 1
    if l > r:
        return 0
    # one operation if either substring is already sorted
    if is_nondec(a[l:r+1]) or is_nondec(b[l:r+1]):
        return 1
    return 2

def read_input() -> List[str]:
    data = sys.stdin.read().strip().split()
    if not data:
        return []
    n = int(data[0])
    arr = data[1:1+n]
    return arr

def solve_all(strings: List[str]) -> int:
    n = len(strings)
    ans = 0
    for i in range(n):
        for j in range(i+1, n):
            ans += f_pair(strings[i], strings[j])
    return ans

def main():
    arr = read_input()
    if not arr:
        return
    print(solve_all(arr))

if __name__ == "__main__":
    # tiny self-checks
    assert f_pair("ab", "ab") == 0
    assert f_pair("ba", "ab") == 1
    assert f_pair("ebcda", "ecdba") == 1
    assert f_pair("a", "b") == 1337
    # check sum on small set
    test = ["ab", "ba", "ac"]
    # pairs: (ab,ba)=1, (ab,ac)=1337 (different Parikh), (ba,ac)=1337
    assert solve_all(test) == 1 + 1337 + 1337
    main()
\end{minted}
\VALIDATION{Validated on the illustrative $f$ examples and a tiny multi-string sum.}
\NotePages{3}

% ============ 5. Approach B — Improved (own page) ============
\section{Approach B — Improved}
\ApproachPage{B}{Group by Parikh Vector}
\WHICHFORMULA{Separate the $1337$-pairs by grouping strings with identical letter counts. Within each group, only $1$ or $2$ is possible (strings are distinct).}
\ASSUMPTIONS{Strings are lowercase over fixed alphabet; Parikh vectors fit in 26-integer tuples.}
\textbf{Algorithm Steps}
\begin{algosteps}
\item Compute Parikh vector key for each string; group indices by this key.
\item Total pairs $P=\binom{n}{2}$. For each group size $m$, subtract $\binom{m}{2}$ from $P$ to get cross-group pairs; each contributes $1337$.
\item For each group, enumerate unordered pairs and add $1$ if the differing block is non-decreasing in either string; otherwise add $2$.
\end{algosteps}
\COMPLEXITY{Grouping is $O(nL)$. Counting within groups is $\sum O(m_g^2 \cdot L)$ across groups, which improves over naive if groups are small, and always isolates the $1337$ mass in $O(n)$.}
\[
\begin{aligned}
T(n) &= O(nL) + \sum_g O(m_g^2 \cdot L),\qquad \sum_g m_g = n.\\
\end{aligned}
\]
\CORRECTNESS{By invariants: cross-group pairs are impossible ($1337$). Within the same Parikh group, minimal operations are $1$ iff some single differing interval is already non-decreasing in one of the pair; otherwise $2$.}
\textbf{Code (Improved)}
\begin{minted}{python}
from typing import List, Dict, Tuple
import sys
from collections import defaultdict

def is_nondec_slice(s: str, l: int, r: int) -> bool:
    for i in range(l, r):
        if s[i] > s[i+1]:
            return False
    return True

def f_pair_same_parikh(a: str, b: str) -> int:
    # Precondition: Parikh vectors equal; a != b allowed
    if a == b:
        return 0
    L = len(a)
    l = 0
    while l < L and a[l] == b[l]:
        l += 1
    r = L - 1
    while r >= 0 and a[r] == b[r]:
        r -= 1
    if l > r:
        return 0
    if is_nondec_slice(a, l, r) or is_nondec_slice(b, l, r):
        return 1
    return 2

def parikh_key(s: str) -> Tuple[int, ...]:
    cnt = [0]*26
    for ch in s:
        cnt[ord(ch)-97] += 1
    return tuple(cnt)

def read_input() -> List[str]:
    data = sys.stdin.read().strip().split()
    if not data:
        return []
    n = int(data[0])
    arr = data[1:1+n]
    return arr

def solve_grouping(strings: List[str]) -> int:
    n = len(strings)
    groups: Dict[Tuple[int, ...], List[int]] = defaultdict(list)
    for idx, s in enumerate(strings):
        groups[parikh_key(s)].append(idx)
    total_pairs = n*(n-1)//2
    within_pairs_sum = 0
    ans_within = 0
    for key, idxs in groups.items():
        m = len(idxs)
        within_pairs_sum += m*(m-1)//2
        # sum contributions inside group
        for ii in range(m):
            i = idxs[ii]
            for jj in range(ii+1, m):
                j = idxs[jj]
                val = f_pair_same_parikh(strings[i], strings[j])
                # strings are distinct, so 0 shouldn't occur, but keep general
                if val == 0:
                    ans_within += 0
                elif val == 1:
                    ans_within += 1
                else:
                    ans_within += 2
    cross_pairs = total_pairs - within_pairs_sum
    ans = 1337 * cross_pairs + ans_within
    return ans

def main():
    arr = read_input()
    if not arr:
        return
    print(solve_grouping(arr))

if __name__ == "__main__":
    # basic checks
    assert f_pair_same_parikh("ab", "ba") == 1
    assert f_pair_same_parikh("abc", "abc") == 0
    assert solve_grouping(["ab","ba","ac"]) == (1337*2 + 1)
    main()
\end{minted}
\VALIDATION{Checked on small handcrafted cases; cross-group contribution isolated as $1337 \times \text{(cross pairs)}$.}
\NotePages{3}

% ============ 6. Approach C — Optimal (own page) ============
\section{Approach C — Optimal}
\ApproachPage{C}{Characterization and Counting}
\WHICHFORMULA{Use the characterization $f\in\{1337,1,2\}$ and count pairs via Parikh grouping plus single-interval monotonicity within differing blocks.}
\ASSUMPTIONS{Distinct strings; equality of Parikh vectors decides feasibility; in-group pairs differ on a contiguous block between first and last mismatches.}
\textbf{Algorithm Steps}
\begin{algosteps}
\item Group by Parikh vectors; compute $1337$ contributions in $O(n)$ after grouping.
\item For each group, for every pair compute the first and last mismatch and test if either differing block is non-decreasing.
\item Sum $1$ if yes, else $2$.
\end{algosteps}
\OPTIMALITY{The characterization of $f$ is tight: if letter multisets differ, $1337$ is necessary; if equal and strings are not identical, one operation suffices exactly when a single differing interval is already sorted in one of them; otherwise two operations (sort-whole on both) suffice and are necessary.}
\COMPLEXITY{Grouping is linear in input size; in-group pair scans are linear in $L$ per pair.}
\[
\begin{aligned}
T(n) &= O(nL) + \sum_g O(m_g^2 \cdot L),\\
S(n) &= O(nL)\ \text{to store input and group indices.}
\end{aligned}
\]
\textbf{Code (Final Submission)}
\begin{minted}{python}
# CF: read_input(), solve_case()/solve_all(), main()+guard + asserts
from typing import List, Dict, Tuple
import sys
from collections import defaultdict

def is_nondec_block(s: str, l: int, r: int) -> bool:
    # inclusive [l, r]
    for i in range(l, r):
        if s[i] > s[i+1]:
            return False
    return True

def f_in_group(a: str, b: str) -> int:
    if a == b:
        return 0
    L = len(a)
    l = 0
    while l < L and a[l] == b[l]:
        l += 1
    r = L - 1
    while r >= 0 and a[r] == b[r]:
        r -= 1
    if l > r:
        return 0
    if is_nondec_block(a, l, r) or is_nondec_block(b, l, r):
        return 1
    return 2

def key_parikh(s: str) -> Tuple[int, ...]:
    cnt = [0]*26
    for ch in s:
        cnt[ord(ch)-97] += 1
    return tuple(cnt)

def read_input() -> List[str]:
    data = sys.stdin.read().strip().split()
    if not data:
        return []
    n = int(data[0])
    arr = data[1:1+n]
    return arr

def solve_all(strings: List[str]) -> int:
    n = len(strings)
    groups: Dict[Tuple[int, ...], List[int]] = defaultdict(list)
    for i, s in enumerate(strings):
        groups[key_parikh(s)].append(i)
    total_pairs = n*(n-1)//2
    within_pairs = 0
    sum_within = 0
    for idxs in groups.values():
        m = len(idxs)
        within_pairs += m*(m-1)//2
        for ii in range(m):
            i = idxs[ii]
            for jj in range(ii+1, m):
                j = idxs[jj]
                val = f_in_group(strings[i], strings[j])
                if val == 1:
                    sum_within += 1
                elif val == 2:
                    sum_within += 2
                else:
                    # distinct strings in input, but keep for completeness
                    sum_within += 0
    cross_pairs = total_pairs - within_pairs
    return 1337 * cross_pairs + sum_within

def main():
    arr = read_input()
    if not arr:
        return
    print(solve_all(arr))

if __name__ == "__main__":
    # Exactly 3 asserts
    assert f_in_group("ba","ab") == 1
    assert f_in_group("cba","abc") == 1  # sort whole block once
    assert solve_all(["ab","ba","ac"]) == 1 + 1337 + 1337
    main()
\end{minted}
\VALIDATION{Three asserts cover single-operation reachability, whole-block sort, and mixed-group sum.}
\RESULT{Sum of $f$ over all unordered pairs: add $1337$ for cross-Parikh pairs; add $1$ or $2$ within Parikh groups according to whether a single differing block is non-decreasing in one of the pair.}
\NotePages{3}

% ============ 7. Testing & Final Reference Implementation (own page) ============
\section{Testing \& Final Reference Implementation}
\LINE{TEST PLAN}{Unit-test the pairwise $f$ characterization; verify grouping math; craft small sets with known contributions.}
\LINE{CROSS-CHECKS}{Compare baseline O($n^2 L$) versus improved grouping on small random cases for equality of outputs.}
\LINE{EDGE-CASE GENERATOR}{Generate length-$1$ strings, already-sorted strings, and strings differing in one interval.}
\begin{minted}{python}
# Deterministic generators for boundaries, degenerates, adversarials
from typing import List
import random

def gen_len1(k: int) -> List[str]:
    # k distinct single letters from 'a'..'z'
    return [chr(97+i) for i in range(k)]

def gen_sorted_and_variants() -> List[str]:
    base = "aabbcc"
    return [base, "ababcc", "aabcbc", "ccbbaa"]

def check_equal_outputs():
    random.seed(0)
    def f_pair(a: str, b: str) -> int:
        if a == b:
            return 0
        cnt = [0]*26
        for ch in a:
            cnt[ord(ch)-97]+=1
        for ch in b:
            cnt[ord(ch)-97]-=1
        if any(cnt):
            return 1337
        L = len(a)
        l = 0
        while l<L and a[l]==b[l]: l+=1
        r = L-1
        while r>=0 and a[r]==b[r]: r-=1
        if l>r: return 0
        def is_nd(s,l,r):
            for i in range(l,r):
                if s[i]>s[i+1]: return False
            return True
        return 1 if (is_nd(a,l,r) or is_nd(b,l,r)) else 2

    def brute(arr: List[str]) -> int:
        n=len(arr); ans=0
        for i in range(n):
            for j in range(i+1,n):
                ans += f_pair(arr[i],arr[j])
        return ans

    def grouped(arr: List[str]) -> int:
        from collections import defaultdict
        def key(s):
            cnt=[0]*26
            for ch in s: cnt[ord(ch)-97]+=1
            return tuple(cnt)
        groups=defaultdict(list)
        for i,s in enumerate(arr): groups[key(s)].append(i)
        n=len(arr); total_pairs = n*(n-1)//2
        within_pairs=0; sum_within=0
        def in_group(a,b):
            if a==b: return 0
            L=len(a); l=0
            while l<L and a[l]==b[l]: l+=1
            r=L-1
            while r>=0 and a[r]==b[r]: r-=1
            if l>r: return 0
            def is_nd(s,l,r):
                for i in range(l,r):
                    if s[i]>s[i+1]: return False
                return True
            return 1 if (is_nd(a,l,r) or is_nd(b,l,r)) else 2
        for idxs in groups.values():
            m=len(idxs); within_pairs += m*(m-1)//2
            for ii in range(m):
                for jj in range(ii+1,m):
                    val=in_group(arr[idxs[ii]],arr[idxs[jj]])
                    sum_within += 0 if val==0 else val
        cross = total_pairs - within_pairs
        return 1337*cross + sum_within

    # run small random tests
    for _ in range(50):
        n=6
        L=4
        seen=set()
        arr=[]
        # generate distinct
        while len(arr)<n:
            s="".join(random.choice("abc") for _ in range(L))
            if s in seen: continue
            seen.add(s); arr.append(s)
        assert brute(arr)==grouped(arr)

if __name__ == "__main__":
    # run local test
    check_equal_outputs()
\end{minted}
\textbf{Reference Code (Ready to Submit)}
\begin{minted}{python}
# Single, final reference solution with the required API (CF/LC) + asserts
from typing import List, Dict, Tuple
import sys
from collections import defaultdict

def is_nondec_block(s: str, l: int, r: int) -> bool:
    for i in range(l, r):
        if s[i] > s[i+1]:
            return False
    return True

def f_in_group(a: str, b: str) -> int:
    if a == b:
        return 0
    L = len(a)
    l = 0
    while l < L and a[l] == b[l]:
        l += 1
    r = L - 1
    while r >= 0 and a[r] == b[r]:
        r -= 1
    if l > r:
        return 0
    if is_nondec_block(a, l, r) or is_nondec_block(b, l, r):
        return 1
    return 2

def key_parikh(s: str) -> Tuple[int, ...]:
    cnt = [0]*26
    for ch in s:
        cnt[ord(ch)-97] += 1
    return tuple(cnt)

def read_input() -> List[str]:
    data = sys.stdin.read().strip().split()
    if not data:
        return []
    n = int(data[0])
    arr = data[1:1+n]
    return arr

def solve_all(strings: List[str]) -> int:
    n = len(strings)
    groups: Dict[Tuple[int, ...], List[int]] = defaultdict(list)
    for i, s in enumerate(strings):
        groups[key_parikh(s)].append(i)
    total_pairs = n*(n-1)//2
    within_pairs = 0
    sum_within = 0
    for idxs in groups.values():
        m = len(idxs)
        within_pairs += m*(m-1)//2
        for ii in range(m):
            i = idxs[ii]
            for jj in range(ii+1, m):
                j = idxs[jj]
                val = f_in_group(strings[i], strings[j])
                if val == 1:
                    sum_within += 1
                elif val == 2:
                    sum_within += 2
                else:
                    sum_within += 0
    cross_pairs = total_pairs - within_pairs
    return 1337 * cross_pairs + sum_within

def main():
    arr = read_input()
    if not arr:
        return
    print(solve_all(arr))

if __name__ == "__main__":
    # quick asserts
    assert f_in_group("ba","ab") == 1
    assert f_in_group("abc","cba") == 1
    assert solve_all(["ab","ba","ac"]) == (1 + 1337 + 1337)
    main()
\end{minted}
\NotePages{3}

% ============ 8. Review & Pitfalls (own page) ============
\section{Review \& Pitfalls}
\WHAT{Sum over all unordered pairs of the minimal count of substring-sorts to make two strings equal, or $1337$ if impossible.}
\WHY{Captures reasoning about what single local sorting can and cannot accomplish; typical of string transformation invariants.}
\CHECKLIST{
\begin{bullets}
\item Group by Parikh vectors (letter counts).
\item Cross-group pairs contribute $1337$.
\item Inside a group, find first/last mismatches $[l,r]$.
\item Check non-decreasing on $a[l..r]$ or $b[l..r]$.
\item Add $1$ if yes, else $2$.
\end{bullets}
}
\EDGECASES{
\begin{bullets}
\item Length $1$ strings.
\item Strings already identical (though input is distinct).
\item All characters same.
\item Only one mismatch position (then both substrings are length $1$, trivially non-decreasing $\Rightarrow 1$).
\item Entire string differs (sorting whole string once).
\item Repeated letters inside the differing block.
\end{bullets}
}
\PITFALLS{
\begin{bullets}
\item Forgetting impossibility when Parikh vectors differ.
\item Miscomputing $l,r$ when strings are equal.
\item Off-by-one in non-decreasing check on $[l,r]$.
\item Assuming stability of sort; it is irrelevant here.
\item Overflow when summing pairs: use Python big ints or 64-bit in C++.
\item Mixing up ordered vs unordered pairs.
\end{bullets}
}
\FAILMODES{Brute-force generating all single-sort results per string is quadratic or worse; grouping avoids wasted $1337$ checks but naive all-pairs in large groups can still be heavy without further structure.}
\ELI{Either the letters do not match (impossible $\Rightarrow 1337$) or they do. When they do, one substring sort is enough exactly when the part where they differ is already sorted in one of them. Otherwise you need two sorts (sort-whole on both).}
\NotePages{3}

\end{document}