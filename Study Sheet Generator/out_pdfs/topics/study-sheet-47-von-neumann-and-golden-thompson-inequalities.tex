% !TeX program = xelatex
\documentclass[11pt]{article}
\usepackage[margin=1in]{geometry}
\usepackage{microtype,setspace,amsmath,amssymb,mathtools,amsthm,unicode-math}
\setstretch{1.05}
\setmainfont{Latin Modern Roman}
\setmonofont{Latin Modern Mono}
\setmathfont{Latin Modern Math}

\allowdisplaybreaks[4]
\setlength{\jot}{7pt}
\setlength{\emergencystretch}{8em}
\sloppy

\usepackage{xcolor,fancyhdr,enumitem,inconsolata,listings}
\pagestyle{fancy}\fancyhf{}\lhead{\nouppercase{\leftmark}}\rhead{\thepage}
\setlength{\headheight}{26pt}
\setlength{\parindent}{0pt}
\setlength{\parskip}{8pt plus 2pt minus 1pt}
\raggedbottom

\newenvironment{BreakableEquation}{\begin{equation}\begin{aligned}}{\end{aligned}\end{equation}}
\newenvironment{BreakableEquation*}{\begin{equation*}\begin{aligned}}{\end{aligned}\end{equation*}}
\newenvironment{tightalign}{\begingroup\small\allowdisplaybreaks\begin{align}}{\end{align}\endgroup}

\providecommand{\enumlistm}{enumitem}
\newenvironment{minted}[2][]{%
  \lstset{style=code,language=#2,#1}\begin{lstlisting}%
}{\end{lstlisting}}

\newcommand{\inputminted}[3][]{\begin{lstlisting}\end{lstlisting}}

\newlist{bullets}{itemize}{1}
\setlist[bullets]{label=--,leftmargin=1.6em,itemsep=2pt,topsep=4pt}

\newcommand{\varmapStart}{\textbf{VARIABLE MAPPING:}\par\begin{bullets}}
\newcommand{\var}[2]{\item $#1$ — #2}
\newcommand{\varmapEnd}{\end{bullets}}

\newcommand{\glossx}[6]{%
  \textbf{#1}\par
  \begin{bullets}
    \item \textbf{What:} #2
    \item \textbf{Why:} #3
    \item \textbf{How:} #4
    \item \textbf{ELI5:} #5
    \item \textbf{Pitfall/Example:} #6
  \end{bullets}
}

\newtheorem{theorem}{Theorem}[section]
\newtheorem{lemma}[theorem]{Lemma}
\newtheorem{proposition}[theorem]{Proposition}
\newtheorem{corollary}[theorem]{Corollary}
\theoremstyle{definition}
\newtheorem{definition}[theorem]{Definition}
\newtheorem{claim}[theorem]{Claim}
\lstdefinestyle{code}{
  basicstyle=\ttfamily\footnotesize,
  backgroundcolor=\color{black!02},
  frame=single,
  numbers=left, numberstyle=\tiny, numbersep=8pt,
  breaklines=true, breakatwhitespace=true,
  tabsize=4, showstringspaces=false,
  upquote=true, keepspaces=true, columns=fullflexible,
  literate=
    {–}{{-}}1
    {—}{{-}}1
    {…}{{...}}1
    {≤}{{\ensuremath{\le}}}1
    {≥}{{\ensuremath{\ge}}}1
    {≠}{{\ensuremath{\ne}}}1
    {≈}{{\ensuremath{\approx}}}1
    {±}{{\ensuremath{\pm}}}1
    {→}{{\ensuremath{\to}}}1
    {←}{{\ensuremath{\leftarrow}}}1
    {∞}{{\ensuremath{\infty}}}1
    {√}{{\ensuremath{\sqrt{\ }}}}1
    {×}{{\ensuremath{\times}}}1
    {÷}{{\ensuremath{\div}}}1
}

\lstnewenvironment{codepy}[1][]%
  {\lstset{style=code,language=Python,#1}}%
  {}

\newcommand{\inlinecode}[1]{\lstinline[style=code]!#1!}

\newcommand{\LF}[2]{\par\noindent\textbf{#1:}~#2\par}
\newcommand{\WHAT}[1]{\LF{WHAT}{#1}}
\newcommand{\WHY}[1]{\LF{WHY}{#1}}
\newcommand{\HOW}[1]{\LF{HOW}{#1}}
\newcommand{\ELI}[1]{\LF{ELI5}{#1}}
\newcommand{\SCOPE}[1]{\LF{SCOPE}{#1}}
\newcommand{\CONFUSIONS}[1]{\LF{COMMON CONFUSIONS}{#1}}
\newcommand{\APPLICATIONS}[1]{\LF{APPLICATIONS}{#1}}
\newcommand{\FORMULA}[1]{\LF{FORMULA}{#1}}
\newcommand{\CANONICAL}[1]{\LF{CANONICAL FORM}{#1}}
\newcommand{\PRECONDS}[1]{\LF{PRECONDITIONS}{#1}}
\newcommand{\DERIVATION}[1]{\LF{DERIVATION}{#1}}
\newcommand{\EQUIV}[1]{\LF{EQUIVALENT FORMS}{#1}}
\newcommand{\LIMITS}[1]{\LF{LIMIT CASES}{#1}}
\newcommand{\INPUTS}[1]{\LF{INPUTS}{#1}}
\newcommand{\OUTPUTS}[1]{\LF{OUTPUTS}{#1}}
\newcommand{\RESULT}[1]{\LF{RESULT}{#1}}
\newcommand{\INTUITION}[1]{\LF{INTUITION}{#1}}
\newcommand{\PITFALLS}[1]{\LF{PITFALLS}{#1}}
\newcommand{\MODEL}[1]{\LF{CANONICAL MATH MODEL}{#1}}
\newcommand{\ASSUMPTIONS}[1]{\LF{ASSUMPTIONS}{#1}}
\newcommand{\WHICHFORMULA}[1]{\LF{WHICH FORMULA \& WHY}{#1}}
\newcommand{\GOVERN}[1]{\LF{GOVERNING EQUATION(S)}{#1}}
\newcommand{\UNITCHECK}[1]{\LF{UNIT CHECK}{#1}}
\newcommand{\EDGECASES}[1]{\LF{EDGE CASES}{#1}}
\newcommand{\ALTERNATE}[1]{\LF{ALTERNATE APPROACH (sketch)}{#1}}
\newcommand{\PROBLEM}[1]{\LF{PROBLEM}{#1}}
\newcommand{\API}[1]{\LF{API}{#1}}
\newcommand{\COMPLEXITY}[1]{\LF{COMPLEXITY}{#1}}
\newcommand{\FAILMODES}[1]{\LF{FAILURE MODES}{#1}}
\newcommand{\STABILITY}[1]{\LF{NUMERICAL STABILITY}{#1}}
\newcommand{\VALIDATION}[1]{\LF{VALIDATION}{#1}}
\newcommand{\EXPLANATION}[1]{\LF{EXPLANATION}{#1}}
\newcommand{\SCENARIO}[1]{\LF{SCENARIO}{#1}}
\newcommand{\PIPELINE}[1]{\LF{PIPELINE STEPS}{#1}}
\newcommand{\METRICS}[1]{\LF{METRICS}{#1}}
\newcommand{\INTERPRET}[1]{\LF{INTERPRETATION}{#1}}
\newcommand{\NEXTSTEPS}[1]{\LF{LIMITATIONS \& NEXT STEPS}{#1}}

\usepackage{titlesec}
\titleformat{\section}{\Large\bfseries}{\thesection}{0.6em}{}
\titlespacing*{\section}{0pt}{*2}{*1}
\usepackage{etoolbox}
\pretocmd{\section}{\clearpage}{}{}

\newcommand{\FormulaPage}[2]{%
  \clearpage
  \section*{Formula #1 — #2}%
  \addcontentsline{toc}{section}{Formula #1 — #2}%
}
\newcommand{\ProblemPage}[2]{%
  \clearpage
  \subsection*{Problem #1: #2}%
  \addcontentsline{toc}{subsection}{Problem #1: #2}%
}
\newcommand{\CodeDemoPage}[1]{%
  \clearpage
  \subsection*{Coding Demo: #1}%
  \addcontentsline{toc}{subsection}{Coding Demo: #1}%
}
\newcommand{\DomainPage}[1]{%
  \clearpage
  \subsection*{#1 (End-to-End)}%
  \addcontentsline{toc}{subsection}{#1 (End-to-End)}%
}

\begin{document}
\title{Comprehensive Study Sheet — Von Neumann and Golden--Thompson Inequalities}
\date{\today}
\maketitle
\tableofcontents
\clearpage

\section{Concept Overview}
\WHAT{
We study two fundamental matrix inequalities for complex matrices:
(1) Von Neumann trace inequality: for $A,B\in\mathbb{C}^{n\times n}$ with singular
values $\sigma_1(\cdot)\ge\cdots\ge\sigma_n(\cdot)$,
$\bigl|\mathrm{tr}(A^\ast B)\bigr|\le \sum_{i=1}^n \sigma_i(A)\sigma_i(B)$.
(2) Golden--Thompson inequality: for Hermitian $A,B$,
$\mathrm{tr}\,e^{A+B}\le \mathrm{tr}\,e^{A}e^{B}$. Domain: finite-dimensional
complex matrices; structure: unitary invariance, singular/eigenvalue orderings.
}
\WHY{
These inequalities control traces of products and exponentials in noncommutative
settings, underpinning stability bounds, perturbation theory, quantum statistical
mechanics (partition functions), and algorithmic bounds (Schatten norms, MGF).
They connect linear algebra majorization with operator convexity and product limits.
}
\HOW{
1. Establish a rearrangement/Fan inequality for Hermitian products via eigenvalue
pinching and doubly stochastic matrices.
2. Lift to general matrices via Hermitian dilation to obtain Von Neumann.
3. Prove Lie--Trotter product formula and combine with pinching/Jensen-type
arguments to obtain Golden--Thompson for traces of exponentials.
4. Interpret bounds through singular/eigenvalue alignments and equality cases.
}
\ELI{
Von Neumann says: the overlap of two matrices (via trace of their product) cannot
exceed what you would get if you perfectly lined up their strengths (singular
values) in order. Golden--Thompson says: when adding energies $A$ and $B$, the
partition function $\mathrm{tr}\,e^{A+B}$ is at most what you get if you apply
$e^{A}$ then $e^{B}$ and take the trace.
}
\SCOPE{
Finite-dimensional matrices; complex entries; $A^\ast$ denotes conjugate transpose.
Golden--Thompson requires Hermitian $A,B$. Extensions to unbounded operators need
domain care. Inequalities are sharp with known equality conditions (alignment and
commutativity). Numerical verifications use unitary-invariant norms.
}
\CONFUSIONS{
Von Neumann vs. Cauchy--Schwarz: $\lvert\mathrm{tr}(A^\ast B)\rvert\le
\|A\|_F\|B\|_F$ is weaker than von Neumann. Golden--Thompson is about traces, not
Loewner order: $e^{A+B}\nleq e^{A}e^{B}$ generally. Singular values vs.
eigenvalues: VN uses singular values; Fan inequality for Hermitian uses eigenvalues.
}
\APPLICATIONS{
\begin{bullets}
\item Spectral analysis and majorization in matrix theory.
\item Quantum mechanics: bounding partition functions and free energies.
\item Numerical linear algebra: Schatten norm dualities and stability.
\item Statistics/ML: log-det and MGF bounds for Gaussians and covariance models.
\end{bullets}
}
\textbf{ANALYTIC STRUCTURE.}
Unitary invariance, majorization, convexity/concavity of trace functionals,
operator monotonicity, product limits (Trotter).

\textbf{CANONICAL LINKS.}
Fan rearrangement $\Rightarrow$ Von Neumann (via Hermitian dilation). Lie--Trotter
product formula $\Rightarrow$ Golden--Thompson. VN supports Holder-type Schatten
bounds used in proofs and problems.

\textbf{PROBLEM-TYPE RECOGNITION HEURISTICS.}
\begin{bullets}
\item Expressions like $\mathrm{tr}(A^\ast B)$ with given singular/eigenvalues.
\item Partition function forms $\mathrm{tr}\,e^{A+B}$ with Hermitian inputs.
\item Requests for equality conditions: alignment/commutativity cues.
\item Doubly stochastic or unitary conjugations appear in setup.
\end{bullets}

\textbf{SOLUTION STRATEGY BLUEPRINT.}
\begin{bullets}
\item Diagonalize (Hermitian) or use SVD; order spectra decreasingly.
\item Convert to unitary-sandwiched diagonal forms.
\item Apply Fan/Von Neumann or Golden--Thompson appropriately.
\item Check equality via commuting or singular-vector alignment.
\item Validate by limit or norm checks (Frobenius, operator norms).
\end{bullets}

\textbf{CONCEPTUAL INVARIANTS.}
Unitary invariance of singular values and trace; convexity/monotonicity of
$\mathrm{tr}\,e^{\cdot}$ for Hermitian inputs; majorization under pinching.

\textbf{EDGE INTUITION.}
If $A,B$ commute, inequalities become equalities. For nearly commuting pairs,
gaps are $O(\|[A,B]\|^2)$. As spectra concentrate, alignment effects dominate.

\clearpage
\section{Glossary}
\glossx{Singular Values}
{Nonnegative eigenvalues of $(A^\ast A)^{1/2}$ arranged decreasingly.}
{Unitary-invariant size measures; central to Von Neumann inequality.}
{Compute SVD $A=U\Sigma V^\ast$; the diagonal of $\Sigma$ are singular values.}
{Like stretching factors of a linear map along optimal directions.}
{Pitfall: Do not confuse with absolute values of eigenvalues for nonnormal $A$.}

\glossx{Hermitian Dilation}
{Block Hermitian matrix $\mathcal{H}(A)=\begin{pmatrix}0&A\\A^\ast&0\end{pmatrix}$.}
{Turns singular-value questions into eigenvalue questions for Hermitian matrices.}
{Eigenvalues of $\mathcal{H}(A)$ are $\pm\sigma_i(A)$.}
{Think of embedding a rectangle into a symmetric shape to use symmetric tools.}
{Useful example: reduce Von Neumann to a Hermitian rearrangement inequality.}

\glossx{Pinching (Conditional Expectation)}
{Map $\Phi(X)=\sum_j P_j X P_j$ with orthogonal projections $P_j$, $\sum P_j=I$.}
{Removes off-diagonal blocks; preserves trace; increases disorder (majorization).}
{Apply to compare traces of convex functions: $\mathrm{tr}\,f(\Phi(H))\le
\mathrm{tr}\,f(H)$ for operator convex $f$.}
{Like averaging that erases cross-talk between blocks.}
{Pitfall: Inequality direction depends on operator convexity/concavity of $f$.}

\glossx{Lie--Trotter Product Formula}
{$e^{A+B}=\lim_{m\to\infty}(e^{A/m}e^{B/m})^m$ for bounded matrices.}
{Bridges addition in the exponent with multiplicative approximations.}
{Use BCH expansion to control the error, which vanishes as $m\to\infty$.}
{Like slicing time into many tiny steps: alternate $A$- and $B$-evolution.}
{Pitfall: Do not interchange limits and trace without continuity justification.}

\clearpage
\section{Symbol Ledger}
\varmapStart
\var{A,B}{Complex $n\times n$ matrices; Hermitian where stated.}
\var{A^\ast}{Conjugate transpose of $A$.}
\var{\sigma_i(A)}{Singular values of $A$, nonincreasing order.}
\var{\lambda_i(H)}{Eigenvalues of Hermitian $H$, nonincreasing order.}
\var{U,V}{Unitary matrices.}
\var{\mathrm{tr}}{Trace functional: sum of diagonal entries.}
\var{e^{X}}{Matrix exponential of $X$.}
\var{\mathcal{H}(A)}{Hermitian dilation of $A$.}
\var{n}{Matrix size; positive integer.}
\var{Z}{Partition function proxy: $Z=\mathrm{tr}\,e^{H}$.}
\var{\|\cdot\|_F}{Frobenius norm.}
\var{\|\cdot\|_2}{Operator (spectral) norm.}
\varmapEnd

\clearpage
\section{Formula Canon — One Formula Per Page}
\FormulaPage{1}{Fan Rearrangement Inequality (Hermitian Form)}
\textbf{CANONICAL MATHEMATICAL STATEMENT.}
For Hermitian $H,K\in\mathbb{C}^{n\times n}$ with eigenvalues
$\lambda_1(H)\ge\cdots\ge\lambda_n(H)$ and
$\lambda_1(K)\ge\cdots\ge\lambda_n(K)$,
\[
\mathrm{tr}(HK)\le \sum_{i=1}^n \lambda_i(H)\lambda_i(K).
\]
\WHAT{
Bounds the trace of a product of Hermitian matrices by an eigenvalue
rearrangement sum, with equality when eigenvectors are aligned.
}
\WHY{
This is the Hermitian core driving Von Neumann via dilation. It encodes
majorization from unitary mixing (pinching/doubly-stochastic averaging).
}
\FORMULA{
\[
\mathrm{tr}(HK)\le \sum_{i=1}^n \lambda_i(H)\lambda_i(K).
\]
}
\CANONICAL{
Hermitian $H,K$; eigenvalues sorted in nonincreasing order; unitary
invariance under conjugation; equality iff simultaneous eigenbasis ordering
aligns.
}
\PRECONDS{
\begin{bullets}
\item $H,K$ are Hermitian (finite-dimensional).
\item Spectral decompositions exist: $H=U\Lambda_H U^\ast$, $K=V\Lambda_K V^\ast$.
\item Eigenvalues are arranged in nonincreasing order on diagonals.
\end{bullets}
}
\textbf{SUPPORTING LEMMAS.}
\begin{lemma}
Let $D=\mathrm{diag}(d_1,\dots,d_n)$ and $E=\mathrm{diag}(e_1,\dots,e_n)$ with
$d_1\ge\cdots\ge d_n$ and $e_1\ge\cdots\ge e_n$, and let $W$ be unitary. Then
\[
\mathrm{tr}(D W E W^\ast)\le \sum_{i=1}^n d_i e_i.
\]
\end{lemma}
\begin{proof}
Compute
\[
\mathrm{tr}(D W E W^\ast)=\sum_{i=1}^n d_i (W E W^\ast)_{ii}
=\sum_{i=1}^n\sum_{j=1}^n d_i e_j |W_{ij}|^2.
\]
Let $P=[p_{ij}]$ with $p_{ij}=|W_{ij}|^2$. Then $P$ is doubly stochastic
(Birkhoff--von Neumann). Hence
\[
\mathrm{tr}(D W E W^\ast)=\sum_{i,j} d_i e_j p_{ij}
\le \max_{\Pi\ \mathrm{perm}} \sum_{i=1}^n d_i e_{\Pi(i)}
=\sum_{i=1}^n d_i e_i,
\]
where the maximum occurs by the classical rearrangement inequality when orders
are aligned. \qedhere
\end{proof}
\DERIVATION{
\begin{align*}
\text{Step 1 (Diagonalize):}\
&H=U\Lambda_H U^\ast,\quad K=V\Lambda_K V^\ast.\\
\text{Step 2 (Cyclic trace):}\
&\mathrm{tr}(HK)=\mathrm{tr}(U\Lambda_H U^\ast V\Lambda_K V^\ast)
=\mathrm{tr}(\Lambda_H W \Lambda_K W^\ast),\\
&\text{where }W=U^\ast V\text{ is unitary}.\\
\text{Step 3 (Apply Lemma):}\
&\mathrm{tr}(\Lambda_H W \Lambda_K W^\ast)\le
\sum_{i=1}^n \lambda_i(H)\lambda_i(K).\\
\text{Step 4 (Equality case):}\
&\text{Equality iff }W\text{ is a permutation aligning eigenvectors.}
\end{align*}
}
\textbf{GENERAL PROBLEM-SOLVING TEMPLATE.}
\begin{bullets}
\item Diagonalize Hermitian inputs.
\item Write trace as diagonal-unitary-diagonal-unitary adjoint form.
\item Use doubly-stochastic averaging to bound by aligned eigenvalue sum.
\item Check equality via eigenbasis alignment.
\end{bullets}
\EQUIV{
\begin{bullets}
\item Lower bound with opposite ordering:
$\mathrm{tr}(HK)\ge \sum_{i=1}^n \lambda_i(H)\lambda_{n+1-i}(K)$.
\item Ky Fan norms: $\sum_{i=1}^k \lambda_i(HK)\le
\sum_{i=1}^k \lambda_i(H)\lambda_i(K)$ for $k=1,\dots,n$ when $H,K\ge0$.
\end{bullets}
}
\LIMITS{
\begin{bullets}
\item If $H$ or $K$ is scalar multiple of identity, inequality becomes equality.
\item For commuting $H,K$, equality holds since they share eigenvectors.
\end{bullets}
}
\INPUTS{$H,K\in\mathbb{C}^{n\times n}$ Hermitian; $\lambda_i(H),\lambda_i(K)$.}
\DERIVATION{
\begin{align*}
\text{Substitute: }&H=U\Lambda_H U^\ast,\ K=V\Lambda_K V^\ast.\\
&\mathrm{tr}(HK)=\sum_{i,j}\lambda_i(H)\lambda_j(K)|W_{ij}|^2.\\
&\le \sum_{i}\lambda_i(H)\lambda_i(K)\quad(\text{rearrangement}).
\end{align*}
}
\RESULT{
$\mathrm{tr}(HK)\le \sum_i \lambda_i(H)\lambda_i(K)$ with equality iff
$U^\ast V$ is a permutation aligning orders.}
\UNITCHECK{
Trace is linear and invariant under unitary conjugations; both sides are real.
}
\PITFALLS{
\begin{bullets}
\item Sorting orders inconsistently invalidates the bound.
\item Using singular values here (instead of eigenvalues) is unnecessary.
\end{bullets}
}
\INTUITION{
Unitary mixing spreads mass via a doubly-stochastic matrix; convex ordering
ensures the aligned pairing maximizes the bilinear form in the trace.
}
\CANONICAL{
\begin{bullets}
\item $\mathrm{tr}(U\Lambda U^\ast V\Gamma V^\ast)\le \langle
\lambda^\downarrow(\Lambda),\lambda^\downarrow(\Gamma)\rangle$.
\item Doubly-stochastic mixing majorizes diagonal contributions.
\end{bullets}
}

\FormulaPage{2}{Von Neumann Trace Inequality}
\textbf{CANONICAL MATHEMATICAL STATEMENT.}
For $A,B\in\mathbb{C}^{n\times n}$ with singular values
$\sigma_1(\cdot)\ge\cdots\ge\sigma_n(\cdot)$,
\[
\bigl|\mathrm{tr}(A^\ast B)\bigr|\le \sum_{i=1}^n \sigma_i(A)\sigma_i(B).
\]
\WHAT{
Bounds the absolute trace inner product by the scalar product of singular-value
sequences, achieving equality when singular-vector bases are aligned.
}
\WHY{
It quantifies maximal overlap under unitary freedoms and is a cornerstone in
Schatten norm duality, perturbation bounds, and matrix recovery guarantees.
}
\FORMULA{
\[
\bigl|\mathrm{tr}(A^\ast B)\bigr|\le \sum_{i=1}^n \sigma_i(A)\sigma_i(B).
\]
}
\CANONICAL{
General complex matrices; singular values sorted decreasingly; equality when left
and right singular vectors of $A$ and $B$ are pairwise aligned in the same order.
}
\PRECONDS{
\begin{bullets}
\item $A,B\in\mathbb{C}^{n\times n}$.
\item Singular values are well-defined via SVD.
\item Uses Fan inequality for Hermitian and Hermitian dilations.
\end{bullets}
}
\textbf{SUPPORTING LEMMAS.}
\begin{lemma}
Let $\mathcal{H}(A)=\begin{pmatrix}0&A\\A^\ast&0\end{pmatrix}$. Then
$\lambda(\mathcal{H}(A))=\{\pm\sigma_1(A),\dots,\pm\sigma_n(A)\}$.
\end{lemma}
\begin{proof}
Compute
\[
\mathcal{H}(A)^2=\begin{pmatrix}AA^\ast&0\\0&A^\ast A\end{pmatrix}.
\]
Hence the nonzero eigenvalues of $\mathcal{H}(A)^2$ are those of $AA^\ast$ and
$A^\ast A$, i.e., $\sigma_i(A)^2$. Because $\mathcal{H}(A)$ is Hermitian,
its eigenvalues are real and occur as $\pm\sigma_i(A)$. \qedhere
\end{proof}
\DERIVATION{
\begin{align*}
\text{Step 1 (Dilate):}\
&X=\mathcal{H}(A),\ Y=\mathcal{H}(B).\\
\text{Step 2 (Trace identity):}\
&XY=\begin{pmatrix}AB^\ast&0\\0&A^\ast B\end{pmatrix},\
\mathrm{tr}(XY)=\mathrm{tr}(AB^\ast)+\mathrm{tr}(A^\ast B)\\
&=2\,\mathrm{Re}\,\mathrm{tr}(A^\ast B).\\
\text{Step 3 (Apply Fan):}\
&\mathrm{tr}(XY)\le \sum_{i=1}^{2n}\lambda_i(X)\lambda_i(Y)\\
&=2\sum_{i=1}^n \sigma_i(A)\sigma_i(B)\quad(\text{by lemma}).\\
\text{Step 4 (Phase reduction):}\
&\mathrm{Re}\,\mathrm{tr}(A^\ast B)\le \sum_i \sigma_i(A)\sigma_i(B).\\
&\text{For }\theta\in\mathbb{R},\ \mathrm{Re}\,e^{-i\theta}\mathrm{tr}(A^\ast B)
\le \sum_i \sigma_i(A)\sigma_i(B).\\
&\text{Choose }\theta=\arg\,\mathrm{tr}(A^\ast B)\Rightarrow
|\mathrm{tr}(A^\ast B)|\le \sum_i \sigma_i(A)\sigma_i(B).
\end{align*}
}
\textbf{GENERAL PROBLEM-SOLVING TEMPLATE.}
\begin{bullets}
\item Reduce general matrices to Hermitian via dilation.
\item Apply Hermitian Fan inequality to the dilations.
\item Use phase to turn real-part bound into absolute-value bound.
\item Identify alignment/equality through SVD bases.
\end{bullets}
\EQUIV{
\begin{bullets}
\item Dual form: $\sup_{U,V\ \mathrm{unitary}}\mathrm{Re}\,\mathrm{tr}(U\Sigma_A
V^\ast X \Sigma_B Y^\ast)\!=\!\sum_i\sigma_i(A)\sigma_i(B)$.
\item Schatten duality: $|\mathrm{tr}(A^\ast B)|\le \|A\|_{S_p}\|B\|_{S_q}$
for $1/p+1/q=1$ implies VN for $(p,q)=(1,\infty)$ via Ky Fan norms.
\end{bullets}
}
\LIMITS{
\begin{bullets}
\item If $A=U\Sigma V^\ast$, $B=U\Gamma V^\ast$ with $\Sigma,\Gamma\ge0$
diagonal and same singular vectors, equality holds.
\item If one matrix is zero, both sides are zero.
\end{bullets}
}
\INPUTS{$A,B\in\mathbb{C}^{n\times n}$; $\sigma_i(A),\sigma_i(B)$.}
\DERIVATION{
\begin{align*}
&\mathrm{tr}(XY)=2\,\mathrm{Re}\,\mathrm{tr}(A^\ast B),\\
&\mathrm{tr}(XY)\le 2\sum_i \sigma_i(A)\sigma_i(B)\ \Rightarrow\
\mathrm{Re}\,\mathrm{tr}(A^\ast B)\le \sum_i \sigma_i(A)\sigma_i(B).\\
&\text{Rotate phase to get absolute value bound.}
\end{align*}
}
\RESULT{
$\bigl|\mathrm{tr}(A^\ast B)\bigr|\le \sum_i \sigma_i(A)\sigma_i(B)$ with
attainability by singular-vector alignment.}
\UNITCHECK{
Both sides are unitary-invariant scalars; dimensions consistent.
}
\PITFALLS{
\begin{bullets}
\item Forgetting to sort singular values in nonincreasing order.
\item Confusing eigenvalues with singular values for nonnormal matrices.
\end{bullets}
}
\INTUITION{
Hermitian dilation converts the problem into aligned eigenvalue pairing; the
trace of the product is maximized by lining up the largest magnitudes.
}
\CANONICAL{
\begin{bullets}
\item Hermitian reduction: $\mathrm{tr}(\mathcal{H}(A)\mathcal{H}(B))
=2\mathrm{Re}\,\mathrm{tr}(A^\ast B)$.
\item Eigenvalues of $\mathcal{H}(A)$ are $\pm\sigma_i(A)$.
\end{bullets}
}

\FormulaPage{3}{Golden--Thompson Inequality}
\textbf{CANONICAL MATHEMATICAL STATEMENT.}
For Hermitian $A,B\in\mathbb{C}^{n\times n}$,
\[
\mathrm{tr}\,e^{A+B}\le \mathrm{tr}\,e^{A}e^{B}.
\]
\WHAT{
Upper bounds the partition-function-like trace of an exponential of a sum by
the trace of a product of exponentials, valid for Hermitian matrices.
}
\WHY{
Central in quantum statistical mechanics, free-energy bounds, and noncommutative
probability; it quantifies noncommutativity cost in exponentials.
}
\FORMULA{
\[
\mathrm{tr}\,e^{A+B}\le \mathrm{tr}\,(e^{A/2}e^{B}e^{A/2})
= \mathrm{tr}\,e^{A}e^{B}.
\]
}
\CANONICAL{
$A,B$ Hermitian; exponential defined by convergent power series; trace is
unitarily invariant. Equality if and only if $A$ and $B$ commute.
}
\PRECONDS{
\begin{bullets}
\item Finite-dimensional Hermitian $A,B$.
\item Continuity of trace under norm limits.
\item Product formula and pinching/Jensen-type monotonicity.
\end{bullets}
}
\textbf{SUPPORTING LEMMAS.}
\begin{lemma}[Lie--Trotter]
For bounded matrices $A,B$,
$e^{A+B}=\lim_{m\to\infty}\bigl(e^{A/m}e^{B/m}\bigr)^m$ in operator norm.
\end{lemma}
\begin{proof}
By the Baker--Campbell--Hausdorff expansion,
$e^{A/m}e^{B/m}=e^{(A+B)/m + R_m}$ where $R_m=\frac{1}{2m^2}[A,B]+O(m^{-3})$.
Then
\[
\bigl(e^{A/m}e^{B/m}\bigr)^m=e^{A+B + m R_m + o(1)}.
\]
Since $mR_m\to 0$ as $m\to\infty$, the right-hand side converges to $e^{A+B}$ in
operator norm. \qedhere
\end{proof}
\begin{lemma}[Pinching trace monotonicity]
Let $\Phi(X)=\sum_j P_j X P_j$ with orthogonal projections $P_j$ summing to $I$.
For Hermitian $H$, $\mathrm{tr}\,e^{\Phi(H)}\le \mathrm{tr}\,e^{H}$.
\end{lemma}
\begin{proof}
Write $H$ in block form according to $\{P_j\}$, so $\Phi(H)$ zeroes off-diagonal
blocks. For $t\in[0,1]$, let $H(t)=\Phi(H)+t(H-\Phi(H))$. The function
$g(t)=\mathrm{tr}\,e^{H(t)}$ satisfies
\[
g'(t)=\mathrm{tr}\Bigl(e^{H(t)}(H-\Phi(H))\Bigr).
\]
By Golden--Thompson for two infinitesimal steps (via Lie--Trotter on $H(t)$ and
$H-\Phi(H)$) and the fact that $\mathrm{tr}(P_j X)=\mathrm{tr}(X P_j)$,
we have $g'(t)\ge 0$ only when off-diagonals vanish; otherwise $g'(t)\ge 0$ is
prevented and $g$ is minimized at $t=0$. Consequently,
$g(0)=\mathrm{tr}\,e^{\Phi(H)}\le g(1)=\mathrm{tr}\,e^{H}$. A direct finite
dimensional argument is: diagonalize each block $P_j H P_j$; then
$e^{\Phi(H)}=\bigoplus_j e^{P_j H P_j}$ and
$\mathrm{tr}\,e^{\Phi(H)}=\sum_j \mathrm{tr}\,e^{P_j H P_j}\le
\mathrm{tr}\,e^{H}$ by convexity of $\mathrm{tr}\,e^{\cdot}$ under mixing of
off-diagonals, which lowers the trace due to doubly-stochastic averaging of
eigenvalues. \qedhere
\end{proof}
\DERIVATION{
\begin{align*}
\text{Step 1 (Golden form):}\
&\mathrm{tr}\,e^{A+B}=\mathrm{tr}\,\lim_{m\to\infty}C_m^m,\\
&C_m=e^{A/2m}e^{B/m}e^{A/2m}\quad\text{(Lie--Trotter symmetrized).}\\
\text{Step 2 (Continuity):}\
&\mathrm{tr}\,e^{A+B}=\lim_{m\to\infty}\mathrm{tr}\,C_m^m.\\
\text{Step 3 (Pinching bound for $m=1$):}\
&\mathrm{tr}\,e^{A+B}\le \mathrm{tr}\,C_1
=\mathrm{tr}\,e^{A/2}e^{B}e^{A/2}.\\
&\text{One route: diagonalize }B=\sum_j b_j P_j.\\
&\text{Then }C_1=e^{A/2}\Bigl(\sum_j e^{b_j}P_j\Bigr)e^{A/2},\\
&\log C_1=\log\Bigl(\sum_j e^{b_j}\,e^{A/2}P_j e^{A/2}\Bigr).\\
&\text{By pinching and operator concavity of }\log,\\
&\sum_j P_j (A+b_j) P_j \le \log C_1.\\
&\text{Apply $\mathrm{tr}\,e^{\cdot}$ (monotone) and } \mathrm{tr}\,e^{\sum_j
P_j (A+b_j) P_j}=\sum_j \mathrm{tr}\,e^{P_j A P_j + b_j}\\
&\le \mathrm{tr}\,e^{A+B}\le \mathrm{tr}\,C_1.\\
\text{Step 4 (Cyclic trace):}\
&\mathrm{tr}\,e^{A/2}e^{B}e^{A/2}=\mathrm{tr}\,e^{B}e^{A}.
\end{align*}
}
\textbf{GENERAL PROBLEM-SOLVING TEMPLATE.}
\begin{bullets}
\item Reduce to the symmetrized product $e^{A/2}e^{B}e^{A/2}$.
\item Use spectral projectors of one term to pinch and bound trace exponentials.
\item Invoke Lie--Trotter to connect product limits to the exact exponential.
\item Conclude with cyclicity of trace for the product-of-exponentials form.
\end{bullets}
\EQUIV{
\begin{bullets}
\item Golden inequality: $\mathrm{tr}\,e^{A+B}\le \mathrm{tr}\,e^{A/2}e^{B}
e^{A/2}$ (equivalent by cyclicity).
\item If $[A,B]=0$, equality: $\mathrm{tr}\,e^{A+B}=\mathrm{tr}\,e^{A}e^{B}$.
\end{bullets}
}
\LIMITS{
\begin{bullets}
\item Small commutator regime: gap is $O(\|[A,B]\|^2)$.
\item If either $A$ or $B$ is scalar, inequality becomes equality.
\end{bullets}
}
\INPUTS{$A,B$ Hermitian; spectral projectors of $B$; product formula parameter $m$.}
\DERIVATION{
\begin{align*}
&\mathrm{tr}\,e^{A+B}\stackrel{\text{Trotter}}{=}\lim_{m\to\infty}\mathrm{tr}\,
\bigl(e^{A/2m}e^{B/m}e^{A/2m}\bigr)^m\\
&\le \mathrm{tr}\,e^{A/2}e^{B}e^{A/2}
=\mathrm{tr}\,e^{A}e^{B}.
\end{align*}
}
\RESULT{
$\mathrm{tr}\,e^{A+B}\le \mathrm{tr}\,e^{A}e^{B}$ with equality iff $A$ and $B$
commute.}
\UNITCHECK{
All terms are scalars; exponentials of Hermitian are positive definite; cyclic
trace identity validates equality of two RHS expressions.
}
\PITFALLS{
\begin{bullets}
\item Confusing Loewner order with trace inequality: $e^{A+B}\nleq e^{A}e^{B}$.
\item Omitting the symmetrization $e^{A/2}$ can lead to incorrect steps.
\end{bullets}
}
\INTUITION{
Pinching removes destructive interference from off-diagonals, and many small
alternations of $A$ and $B$ approximate $e^{A+B}$; the single-step product gives
an upper bound on the trace due to convexity/majorization.
}
\CANONICAL{
\begin{bullets}
\item $\mathrm{tr}\,e^{A+B} \le \mathrm{tr}\,(e^{A/2}e^{B}e^{A/2})$.
\item Product limit: $e^{A+B}=\lim_m (e^{A/2m}e^{B/m}e^{A/2m})^m$.
\end{bullets}
}

\section{10 Exhaustive Problems and Solutions}
\ProblemPage{1}{Equality in Von Neumann via SVD Alignment}
\textbf{CANONICAL MATHEMATICAL STATEMENT.}
Show that equality in Von Neumann holds iff left/right singular vectors of $A$
and $B$ can be chosen identically with singular values in the same order.
\PROBLEM{
Given $A=U\Sigma V^\ast$, $B=X\Gamma Y^\ast$ SVDs, prove
$|\mathrm{tr}(A^\ast B)|=\sum_i \sigma_i(A)\sigma_i(B)$ iff there exist unitaries
$\tilde U,\tilde V$ such that $A=\tilde U\Sigma \tilde V^\ast$ and
$B=\tilde U\Gamma \tilde V^\ast$ with $\Sigma,\Gamma$ diagonal decreasing.
}
\MODEL{
\[
A=U\Sigma V^\ast,\quad B=X\Gamma Y^\ast,\quad
\Sigma=\mathrm{diag}(\sigma_i(A)),\ \Gamma=\mathrm{diag}(\sigma_i(B)).
\]
}
\ASSUMPTIONS{
\begin{bullets}
\item Full SVDs exist; singular values are ordered.
\item $\Sigma,\Gamma\ge 0$ diagonal; $U,V,X,Y$ unitary.
\end{bullets}
}
\varmapStart
\var{U,V,X,Y}{Unitary factors in SVDs.}
\var{\Sigma,\Gamma}{Diagonal singular-value matrices.}
\var{W,Z}{Unitary couplings $W=U^\ast X$, $Z=Y^\ast V$.}
\varmapEnd
\WHICHFORMULA{
Von Neumann inequality with equality characterized by equality in the underlying
doubly-stochastic bound (Lemma in Formula 1).
}
\GOVERN{
\[
|\mathrm{tr}(A^\ast B)|=|\mathrm{tr}(\Sigma W \Gamma Z)|\le
\sum_i \sigma_i(A)\sigma_i(B).
\]
}
\INPUTS{$\sigma_i(A)$, $\sigma_i(B)$, $U,V,X,Y$.}
\DERIVATION{
\begin{align*}
\text{Step 1: }&\mathrm{tr}(A^\ast B)=\mathrm{tr}(V\Sigma U^\ast X \Gamma Y^\ast)
=\mathrm{tr}(\Sigma W \Gamma Z),\\
&\text{with }W=U^\ast X,\ Z=Y^\ast V\ \text{unitary}.\\
\text{Step 2: }&\text{Let }Q=ZW\ \text{unitary; then }|\mathrm{tr}(\Sigma W
\Gamma Z)|=|\mathrm{tr}(Q \Gamma Z\Sigma)|.\\
\text{Step 3: }&\text{By Von Neumann, bound equals }\sum_i\sigma_i(A)\sigma_i(B).\\
\text{Step 4: }&\text{Equality in Lemma 1 requires }Z=W^\ast \text{ and }W
\text{ aligns bases},\\
&\Rightarrow U=X\Pi,\ V=Y\Pi \text{ for a common permutation }\Pi.\\
\text{Step 5: }&\tilde U=U,\ \tilde V=V \text{ realize joint alignment.}
\end{align*}
}
\RESULT{
Equality holds iff $A$ and $B$ admit a common pair of singular-vector bases with
diagonal $\Sigma,\Gamma$ in the same decreasing order.}
\UNITCHECK{
All steps use unitary invariance of trace; dimensions match.}
\EDGECASES{
\begin{bullets}
\item Repeated singular values admit nonunique bases; alignment may still hold.
\item If some singular values are zero, alignment on the support suffices.
\end{bullets}
}
\ALTERNATE{
Use variational characterization:
$\sum_i \sigma_i(A)\sigma_i(B)=\max_{U,V} \mathrm{Re}\,\mathrm{tr}(UAV^\ast B)$.}
\VALIDATION{
\begin{bullets}
\item Numerically test with random $U,V$ and fixed $\Sigma,\Gamma$.
\item Check equality when $U=X$, $V=Y$.
\end{bullets}
}
\INTUITION{
Maximizing overlap occurs when strong directions of $A$ and $B$ point together.}
\CANONICAL{
\begin{bullets}
\item Equality $\Leftrightarrow$ doubly-stochastic matrix reduces to a
permutation.
\end{bullets}
}

\ProblemPage{2}{Fan Inequality and Doubly Stochastic Matrices}
\textbf{CANONICAL MATHEMATICAL STATEMENT.}
Prove the Fan inequality using Birkhoff decomposition of a doubly-stochastic
matrix induced by a unitary.
\PROBLEM{
Given $H=U\Lambda_H U^\ast$, $K=V\Lambda_K V^\ast$, $W=U^\ast V$, show that the
coefficients $p_{ij}=|W_{ij}|^2$ form a convex combination of permutation
matrices and derive the bound on $\mathrm{tr}(HK)$.
}
\MODEL{
\[
\mathrm{tr}(HK)=\sum_{i,j}\lambda_i(H)\lambda_j(K)p_{ij},\quad
P=[p_{ij}]\in\mathcal{B}_n,
\]
where $\mathcal{B}_n$ is the Birkhoff polytope.
}
\ASSUMPTIONS{
\begin{bullets}
\item $U,V$ unitary; $W=U^\ast V$ unitary.
\item Birkhoff theorem: doubly-stochastic matrices are convex hull of permutations.
\end{bullets}
}
\varmapStart
\var{P}{Doubly-stochastic matrix with $p_{ij}=|W_{ij}|^2$.}
\var{\Pi}{Permutation matrix.}
\var{\alpha_\Pi}{Convex coefficients in Birkhoff decomposition.}
\varmapEnd
\WHICHFORMULA{
Formula 1 Lemma and rearrangement inequality.
}
\GOVERN{
\[
\mathrm{tr}(HK)=\sum_{i,j}\lambda_i(H)\lambda_j(K)p_{ij}
\le \sum_i\lambda_i(H)\lambda_i(K).
\]
}
\INPUTS{$\lambda_i(H),\lambda_i(K)$, $P\in\mathcal{B}_n$.}
\DERIVATION{
\begin{align*}
\text{Step 1: }&P=\sum_{\Pi}\alpha_\Pi \Pi,\ \alpha_\Pi\ge 0,\
\sum_\Pi \alpha_\Pi=1.\\
\text{Step 2: }&\mathrm{tr}(HK)=\sum_{\Pi}\alpha_\Pi
\sum_i \lambda_i(H)\lambda_{\Pi(i)}(K).\\
\text{Step 3: }&\max_{\Pi}\sum_i \lambda_i(H)\lambda_{\Pi(i)}(K)
=\sum_i \lambda_i(H)\lambda_i(K)\\
&\text{(rearrangement inequality for decreasing sequences).}\\
\text{Step 4: }&\Rightarrow \mathrm{tr}(HK)\le \sum_i \lambda_i(H)\lambda_i(K).
\end{align*}
}
\RESULT{
Fan inequality holds with equality for a permutation aligning eigen-orders.}
\UNITCHECK{
Both sides are real scalars; convex combination preserves bounds.}
\EDGECASES{
\begin{bullets}
\item If $P$ is a permutation, equality is attained.
\item Degeneracies: multiple permutations may maximize the sum.
\end{bullets}
}
\ALTERNATE{
Direct application of Lemma in Formula 1 without explicit Birkhoff expansion.}
\VALIDATION{
\begin{bullets}
\item Sample random unitaries $W$; compute $P$ and compare both sides.
\end{bullets}
}
\INTUITION{
Doubly-stochastic mixing cannot outperform perfectly aligned pairing.}
\CANONICAL{
\begin{bullets}
\item Mixing via $P$ convexly averages permutation outcomes.
\end{bullets}
}

\ProblemPage{3}{Golden--Thompson for Commuting Matrices}
\textbf{CANONICAL MATHEMATICAL STATEMENT.}
Show that if $[A,B]=0$ and $A,B$ Hermitian, then Golden--Thompson is an
equality: $\mathrm{tr}\,e^{A+B}=\mathrm{tr}\,e^{A}e^{B}$.
\PROBLEM{
Assume $A,B$ commute. Prove $e^{A+B}=e^{A}e^{B}$ and deduce the equality of
traces. Provide a numeric example with $2\times2$ diagonal matrices.
}
\MODEL{
\[
[A,B]=0,\quad e^{A+B}=e^{A}e^{B}=e^{B}e^{A}.
\]
}
\ASSUMPTIONS{
\begin{bullets}
\item Hermitian $A,B$; commuting implies simultaneous diagonalizability.
\end{bullets}
}
\varmapStart
\var{A,B}{Commuting Hermitian matrices.}
\var{U}{Unitary diagonalizing both: $U^\ast AU=\Lambda_A$, $U^\ast BU=\Lambda_B$.}
\varmapEnd
\WHICHFORMULA{
Golden--Thompson inequality and exponential additivity for commuting matrices.}
\GOVERN{
\[
e^{A+B}=U e^{\Lambda_A+\Lambda_B} U^\ast
=U e^{\Lambda_A} e^{\Lambda_B} U^\ast=e^{A}e^{B}.
\]
}
\INPUTS{$A=\mathrm{diag}(1,2)$, $B=\mathrm{diag}(3,4)$ (example).}
\DERIVATION{
\begin{align*}
\text{Step 1: }&A,B\ \text{commute}\Rightarrow \exists U:\ U^\ast AU=\Lambda_A,\
U^\ast BU=\Lambda_B.\\
\text{Step 2: }&e^{A+B}=U e^{\Lambda_A+\Lambda_B} U^\ast
=U e^{\Lambda_A}e^{\Lambda_B} U^\ast.\\
\text{Step 3: }&=U e^{\Lambda_A} U^\ast U e^{\Lambda_B} U^\ast=e^{A}e^{B}.\\
\text{Numeric: }&\mathrm{tr}\,e^{A+B}=\mathrm{tr}\,\mathrm{diag}(e^4,e^6)
=e^4+e^6.\\
&\mathrm{tr}\,e^{A}e^{B}=\mathrm{tr}\,\mathrm{diag}(e^1,e^2)
\mathrm{diag}(e^3,e^4)=e^4+e^6.
\end{align*}
}
\RESULT{
Equality holds for commuting pairs.}
\UNITCHECK{
Trace of similar matrices is equal; dimensions match.}
\EDGECASES{
\begin{bullets}
\item Near-commuting matrices: inequality becomes strict but close.
\end{bullets}
}
\ALTERNATE{
Use power series and $[A,B]=0$ to regroup terms and exponentiate.}
\VALIDATION{
\begin{bullets}
\item Compare both sides numerically for random commuting diagonals.
\end{bullets}
}
\INTUITION{
No noncommutative penalty when $A$ and $B$ share eigenvectors.}
\CANONICAL{
\begin{bullets}
\item Simultaneous diagonalization implies scalar equality componentwise.
\end{bullets}
}

\ProblemPage{4}{Bounding a Partition Function}
\textbf{CANONICAL MATHEMATICAL STATEMENT.}
For Hermitian $H_0,H_1$ and $\beta>0$, show
$\mathrm{tr}\,e^{-\beta(H_0+H_1)}\le \mathrm{tr}\,e^{-\beta H_0}e^{-\beta H_1}$.
\PROBLEM{
Interpret $Z(\beta)=\mathrm{tr}\,e^{-\beta(H_0+H_1)}$ as a partition function and
bound it via Golden--Thompson. Provide a $2\times 2$ numeric illustration.
}
\MODEL{
\[
Z(\beta)=\mathrm{tr}\,e^{-\beta(H_0+H_1)}.
\]
}
\ASSUMPTIONS{
\begin{bullets}
\item $H_0,H_1$ Hermitian; $\beta>0$.
\end{bullets}
}
\varmapStart
\var{H_0,H_1}{Hermitian Hamiltonians.}
\var{\beta}{Inverse temperature, positive scalar.}
\var{Z}{Partition function.}
\varmapEnd
\WHICHFORMULA{
Golden--Thompson with $A=-\beta H_0$, $B=-\beta H_1$.}
\GOVERN{
\[
\mathrm{tr}\,e^{-\beta(H_0+H_1)}\le \mathrm{tr}\,e^{-\beta H_0}e^{-\beta H_1}.
\]
}
\INPUTS{$H_0=\begin{pmatrix}1&0\\0&2\end{pmatrix}$,
$H_1=\begin{pmatrix}0&1\\1&0\end{pmatrix}$, $\beta=1$.}
\DERIVATION{
\begin{align*}
\text{Step 1: }&\mathrm{tr}\,e^{-(H_0+H_1)}\le
\mathrm{tr}\,e^{-H_0}e^{-H_1}.\\
\text{Step 2 (Compute): }&
e^{-H_0}=\mathrm{diag}(e^{-1},e^{-2}).\\
&e^{-H_1}=e^{-\begin{psmallmatrix}0&1\\1&0\end{psmallmatrix}}
=\cosh(1)I-\sinh(1)\begin{psmallmatrix}0&1\\1&0\end{psmallmatrix}.\\
\text{Step 3: }&e^{-H_0}e^{-H_1}=
\begin{psmallmatrix}
e^{-1}\cosh 1 & -e^{-1}\sinh 1\\
-e^{-2}\sinh 1 & e^{-2}\cosh 1
\end{psmallmatrix}.\\
\text{Step 4: }&\mathrm{tr}\,e^{-H_0}e^{-H_1}
=e^{-1}\cosh 1+e^{-2}\cosh 1.\\
\text{Step 5: }&\text{Numerically }e^{-1}\cosh 1+e^{-2}\cosh 1\approx 0.948.\\
&\mathrm{tr}\,e^{-(H_0+H_1)}\ \text{computed directly is smaller.}
\end{align*}
}
\RESULT{
$Z(1)\le \mathrm{tr}\,e^{-H_0}e^{-H_1}\approx 0.948$.}
\UNITCHECK{
All terms are dimensionless scalars; Hermitian exponentials are positive.}
\EDGECASES{
\begin{bullets}
\item If $[H_0,H_1]=0$, equality holds.
\item As $\beta\to 0$, both sides $\to n$.
\end{bullets}
}
\ALTERNATE{
Use Trotter splitting to approximate $e^{-(H_0+H_1)}$ and bound its trace.}
\VALIDATION{
\begin{bullets}
\item Compare eigenvalue sums of $-(H_0+H_1)$ and product side numerically.
\end{bullets}
}
\INTUITION{
Alternating evolution overestimates the partition function trace.}
\CANONICAL{
\begin{bullets}
\item Set $A=-\beta H_0$, $B=-\beta H_1$ in Golden--Thompson.
\end{bullets}
}

\ProblemPage{5}{Expectation Puzzle with Random Signs}
\textbf{CANONICAL MATHEMATICAL STATEMENT.}
Let $\varepsilon_i\in\{\pm1\}$ i.i.d. Rademacher. For fixed $A,B$ Hermitian,
show $\mathbb{E}\,\mathrm{tr}\,e^{A+\sum_i \varepsilon_i B_i}
\le \mathrm{tr}\,e^{A}\prod_i \cosh(\|B_i\|_2)$.
\PROBLEM{
Use Golden--Thompson and the scalar mgf bound for Rademacher to derive an
expectation bound on a random partition function.
}
\MODEL{
\[
Z=\mathrm{tr}\,e^{A+\sum_i \varepsilon_i B_i}.
\]
}
\ASSUMPTIONS{
\begin{bullets}
\item $A,B_i$ Hermitian; $\varepsilon_i$ independent Rademacher.
\item $\|B_i\|_2$ denotes operator norm.
\end{bullets}
}
\varmapStart
\var{A,B_i}{Hermitian matrices.}
\var{\varepsilon_i}{Rademacher random signs.}
\var{\|\cdot\|_2}{Operator norm (largest singular value).}
\varmapEnd
\WHICHFORMULA{
Golden--Thompson iteratively and scalar mgf $\mathbb{E}\,e^{t\varepsilon}=\cosh t$.
}
\GOVERN{
\[
\mathbb{E}\,\mathrm{tr}\,e^{A+\sum_i \varepsilon_i B_i}
\le \mathrm{tr}\,\prod_i \mathbb{E}\,e^{\varepsilon_i B_i} e^{A}
\le \mathrm{tr}\,e^{A}\prod_i \cosh(\|B_i\|_2).
\]
}
\INPUTS{$A,B_i$ and bounds $\|B_i\|_2$.}
\DERIVATION{
\begin{align*}
\text{Step 1: }&\mathbb{E}\,\mathrm{tr}\,e^{A+\sum_i \varepsilon_i B_i}
\le \mathbb{E}\,\mathrm{tr}\,e^{A}\prod_i e^{\varepsilon_i B_i}
\quad(\text{Golden--Thompson iteratively}).\\
\text{Step 2: }&\le \mathrm{tr}\,e^{A}\ \prod_i
\|\mathbb{E}\,e^{\varepsilon_i B_i}\|_2\\
&\le \mathrm{tr}\,e^{A}\ \prod_i \mathbb{E}\,\|e^{\varepsilon_i B_i}\|_2\\
&= \mathrm{tr}\,e^{A}\ \prod_i \max_{\pm}\|e^{\pm B_i}\|_2.\\
\text{Step 3: }&\|e^{\pm B_i}\|_2=e^{\|B_i\|_2}.\\
\text{Step 4: }&\mathbb{E}\,e^{t\varepsilon}=\cosh t\le e^{t^2/2}.\\
\text{Step 5: }&\Rightarrow\ \mathbb{E}\,\mathrm{tr}\,e^{A+\sum_i \varepsilon_i B_i}
\le \mathrm{tr}\,e^{A}\prod_i \cosh(\|B_i\|_2).
\end{align*}
}
\RESULT{
$\mathbb{E}\,\mathrm{tr}\,e^{A+\sum_i \varepsilon_i B_i}
\le \mathrm{tr}\,e^{A}\prod_i \cosh(\|B_i\|_2)$.}
\UNITCHECK{
All terms are positive scalars; operator norm exponent equals largest eigenvalue.}
\EDGECASES{
\begin{bullets}
\item If $B_i=0$, factor equals $1$.
\item If $[A,B_i]=0$ for all $i$, bound tightens and may approach equality.
\end{bullets}
}
\ALTERNATE{
Use matrix Laplace transform method with Lieb's concavity for sharper bounds.}
\VALIDATION{
\begin{bullets}
\item Monte Carlo with fixed seed; compare empirical mean and upper bound.
\end{bullets}
}
\INTUITION{
Random signs average out off-diagonals; the worst-case is governed by spectral
radii via $\cosh$.}
\CANONICAL{
\begin{bullets}
\item Iterated Golden--Thompson with scalar mgf control.
\end{bullets}
}

\ProblemPage{6}{Frobenius vs. Von Neumann}
\textbf{CANONICAL MATHEMATICAL STATEMENT.}
Compare $|\mathrm{tr}(A^\ast B)|\le \|A\|_F\|B\|_F$ and Von Neumann. Show
$\sum_i \sigma_i(A)\sigma_i(B)\le \|A\|_F\|B\|_F$ by Cauchy--Schwarz on vectors
of singular values.
\PROBLEM{
Prove the Frobenius bound and show it is weaker than Von Neumann by majorizing
the RHS via Cauchy--Schwarz applied to singular-value sequences.
}
\MODEL{
\[
\|A\|_F=(\sum_i \sigma_i(A)^2)^{1/2}.
\]
}
\ASSUMPTIONS{
\begin{bullets}
\item Singular values nonnegative; $\ell_2$ Cauchy--Schwarz applies.
\end{bullets}
}
\varmapStart
\var{\sigma_i(A)}{Singular values of $A$.}
\var{\sigma_i(B)}{Singular values of $B$.}
\varmapEnd
\WHICHFORMULA{
Von Neumann inequality and Cauchy--Schwarz in $\mathbb{R}^n$.}
\GOVERN{
\[
|\mathrm{tr}(A^\ast B)|\le \sum_i \sigma_i(A)\sigma_i(B)
\le \Bigl(\sum_i \sigma_i(A)^2\Bigr)^{1/2}
\Bigl(\sum_i \sigma_i(B)^2\Bigr)^{1/2}.
\]
}
\INPUTS{$\{\sigma_i(A)\},\{\sigma_i(B)\}$.}
\DERIVATION{
\begin{align*}
\text{Step 1: }&\text{Apply Von Neumann.}\\
\text{Step 2: }&\text{Cauchy--Schwarz: }
\sum_i x_i y_i\le (\sum_i x_i^2)^{1/2}(\sum_i y_i^2)^{1/2}.\\
\text{Step 3: }&x_i=\sigma_i(A),\ y_i=\sigma_i(B).\\
\text{Step 4: }&\Rightarrow \sum_i \sigma_i(A)\sigma_i(B)\le
\|A\|_F\|B\|_F.
\end{align*}
}
\RESULT{
Frobenius bound follows from Von Neumann and is generally weaker.}
\UNITCHECK{
All quantities are nonnegative scalars; equal when singular values proportional.}
\EDGECASES{
\begin{bullets}
\item If $A$ or $B$ has rank one, both bounds coincide.
\end{bullets}
}
\ALTERNATE{
Derive Frobenius directly: $|\mathrm{tr}(A^\ast B)|\le \|A\|_F\|B\|_F$ by
Cauchy--Schwarz in Hilbert--Schmidt inner product.}
\VALIDATION{
\begin{bullets}
\item Random matrices: check numeric inequalities and ratios.
\end{bullets}
}
\INTUITION{
Von Neumann leverages spectral ordering, sharpening the basic inner-product bound.}
\CANONICAL{
\begin{bullets}
\item Spectral majorization implies Hilbert--Schmidt inequality.
\end{bullets}
}

\ProblemPage{7}{Proof-Style: Hermitian Dilation Identity}
\textbf{CANONICAL MATHEMATICAL STATEMENT.}
Prove $\mathrm{tr}\bigl(\mathcal{H}(A)\mathcal{H}(B)\bigr)=
2\,\mathrm{Re}\,\mathrm{tr}(A^\ast B)$.
\PROBLEM{
Compute the product of dilations explicitly and take the trace; interpret the
result as twice the real part of $\mathrm{tr}(A^\ast B)$.
}
\MODEL{
\[
\mathcal{H}(A)=\begin{pmatrix}0&A\\A^\ast&0\end{pmatrix},\ 
\mathcal{H}(B)=\begin{pmatrix}0&B\\B^\ast&0\end{pmatrix}.
\]
}
\ASSUMPTIONS{
\begin{bullets}
\item Block-matrix multiplication rules.
\item Trace additivity over block-diagonals.
\end{bullets}
}
\varmapStart
\var{\mathcal{H}(A)}{Hermitian dilation of $A$.}
\var{\mathcal{H}(B)}{Hermitian dilation of $B$.}
\varmapEnd
\WHICHFORMULA{
Used in Formula 2 derivation; key identity relating traces.}
\GOVERN{
\[
\mathcal{H}(A)\mathcal{H}(B)=
\begin{pmatrix}AB^\ast&0\\0&A^\ast B\end{pmatrix}.
\]
}
\INPUTS{$A,B\in\mathbb{C}^{n\times n}$.}
\DERIVATION{
\begin{align*}
\text{Step 1: }&
\begin{pmatrix}0&A\\A^\ast&0\end{pmatrix}
\begin{pmatrix}0&B\\B^\ast&0\end{pmatrix}
=\begin{pmatrix}AB^\ast&0\\0&A^\ast B\end{pmatrix}.\\
\text{Step 2: }&\mathrm{tr}(\cdot)=\mathrm{tr}(AB^\ast)+\mathrm{tr}(A^\ast B)
=2\,\mathrm{Re}\,\mathrm{tr}(A^\ast B).
\end{align*}
}
\RESULT{
Identity holds; connects real part of $\mathrm{tr}(A^\ast B)$ to Hermitian trace.}
\UNITCHECK{
Both sides real; dimensions $2n\times 2n$ collapse to scalars via trace.}
\EDGECASES{
\begin{bullets}
\item If $B=A$, $\mathrm{tr}(\mathcal{H}(A)^2)=2\|A\|_F^2$.
\end{bullets}
}
\ALTERNATE{
Use property $\mathrm{tr}(X)+\overline{\mathrm{tr}(X)}=
2\,\mathrm{Re}\,\mathrm{tr}(X)$.}
\VALIDATION{
\begin{bullets}
\item Test with random small matrices numerically.
\end{bullets}
}
\INTUITION{
Dilation duplicates the interaction into adjoint-paired diagonal blocks.}
\CANONICAL{
\begin{bullets}
\item Bridge from general matrices to Hermitian inequalities.
\end{bullets}
}

\ProblemPage{8}{Proof-Style: Pinching Lowers $\mathrm{tr}\,e^{H}$}
\textbf{CANONICAL MATHEMATICAL STATEMENT.}
For Hermitian $H$ and orthogonal projections $\{P_j\}$ with $\sum_j P_j=I$,
prove $\mathrm{tr}\,e^{\sum_j P_j H P_j}\le \mathrm{tr}\,e^{H}$.
\PROBLEM{
Work in a basis where $\{P_j\}$ block-diagonalize the space; compare eigenvalues
of $H$ with those of the block-diagonal compression to show the trace inequality.
}
\MODEL{
\[
\Phi(H)=\sum_j P_j H P_j,\quad \mathrm{tr}\,e^{\Phi(H)}\le \mathrm{tr}\,e^{H}.
\]
}
\ASSUMPTIONS{
\begin{bullets}
\item $H$ Hermitian; spectral decomposition exists.
\item Convexity of $t\mapsto e^{t}$ and majorization under pinching.
\end{bullets}
}
\varmapStart
\var{H}{Hermitian input.}
\var{P_j}{Orthogonal projections summing to $I$.}
\var{\Phi}{Pinching map.}
\varmapEnd
\WHICHFORMULA{
Lemma in Formula 3.}
\GOVERN{
\[
\lambda(\Phi(H))\prec \lambda(H)\Rightarrow
\sum_i e^{\lambda_i(\Phi(H))}\le \sum_i e^{\lambda_i(H)}.
\]
}
\INPUTS{$H$ and projections $\{P_j\}$.}
\DERIVATION{
\begin{align*}
\text{Step 1: }&\Phi(H)\ \text{is block-diagonal in the }P_j\text{ basis}.\\
\text{Step 2: }&\lambda(\Phi(H))\prec \lambda(H)\ \text{(pinching majorization)}.\\
\text{Step 3: }&\text{Since }e^t\ \text{is convex and increasing,}\\
&\sum_i e^{\lambda_i(\Phi(H))}\le \sum_i e^{\lambda_i(H)}.\\
\text{Step 4: }&\mathrm{tr}\,e^{\Phi(H)}=\sum_i e^{\lambda_i(\Phi(H))},\
\mathrm{tr}\,e^{H}=\sum_i e^{\lambda_i(H)}.
\end{align*}
}
\RESULT{
$\mathrm{tr}\,e^{\Phi(H)}\le \mathrm{tr}\,e^{H}$.}
\UNITCHECK{
Both sides are positive scalars; eigenvalue sums coincide with traces.}
\EDGECASES{
\begin{bullets}
\item If $H$ commutes with all $P_j$, equality holds.
\end{bullets}
}
\ALTERNATE{
Direct Jensen inequality for the pinching channel using operator convexity of
the exponential.}
\VALIDATION{
\begin{bullets}
\item Numerically compare both traces for random $H$ and random projections.
\end{bullets}
}
\INTUITION{
Erasing off-diagonal interactions reduces the exponential trace by spreading
spectral mass (more mixed eigenvalues).}
\CANONICAL{
\begin{bullets}
\item Majorization under a unital completely positive map.
\end{bullets}
}

\ProblemPage{9}{Combo: Schatten Norm Holder via Von Neumann}
\textbf{CANONICAL MATHEMATICAL STATEMENT.}
For $1\le p,q\le\infty$ with $1/p+1/q=1$, prove
$|\mathrm{tr}(A^\ast B)|\le \|A\|_{S_p}\|B\|_{S_q}$ using Von Neumann and
vector Hölder.
\PROBLEM{
Reduce the inequality to singular-value sequences and apply Hölder's inequality
on $\ell_p$ norms to obtain Schatten Hölder.
}
\MODEL{
\[
\|A\|_{S_p}=\Bigl(\sum_i \sigma_i(A)^p\Bigr)^{1/p},\quad
1/p+1/q=1.
\]
}
\ASSUMPTIONS{
\begin{bullets}
\item Von Neumann inequality.
\item Hölder inequality on $\mathbb{R}^n$.
\end{bullets}
}
\varmapStart
\var{\sigma_i(A)}{Singular values of $A$.}
\var{\|A\|_{S_p}}{Schatten $p$-norm.}
\varmapEnd
\WHICHFORMULA{
Von Neumann trace inequality and Hölder inequality for sequences.}
\GOVERN{
\[
|\mathrm{tr}(A^\ast B)|\le \sum_i \sigma_i(A)\sigma_i(B)
\le \|A\|_{S_p}\|B\|_{S_q}.
\]
}
\INPUTS{$p,q$ conjugate, singular values of $A,B$.}
\DERIVATION{
\begin{align*}
\text{Step 1: }&|\mathrm{tr}(A^\ast B)|\le \sum_i \sigma_i(A)\sigma_i(B).\\
\text{Step 2: }&\text{Apply Hölder: }\sum_i x_i y_i\le
\|x\|_{\ell_p}\|y\|_{\ell_q}.\\
\text{Step 3: }&x_i=\sigma_i(A),\ y_i=\sigma_i(B)\Rightarrow
\|x\|_{\ell_p}=\|A\|_{S_p},\ \|y\|_{\ell_q}=\|B\|_{S_q}.
\end{align*}
}
\RESULT{
$|\mathrm{tr}(A^\ast B)|\le \|A\|_{S_p}\|B\|_{S_q}$.}
\UNITCHECK{
Norms and trace product are unitary invariant; dimensions consistent.}
\EDGECASES{
\begin{bullets}
\item $p=2=q$: recovers Frobenius Cauchy--Schwarz.
\item $p=1,q=\infty$: recovers Von Neumann.
\end{bullets}
}
\ALTERNATE{
Interpolation between $p=1$ and $p=2$ via Riesz--Thorin on Schatten classes.}
\VALIDATION{
\begin{bullets}
\item Compute both sides for random matrices and check inequality. 
\end{bullets}
}
\INTUITION{
Noncommutative Hölder emerges from ordering singular values optimally.}
\CANONICAL{
\begin{bullets}
\item Vector Hölder on singular-value lists yields Schatten Hölder.
\end{bullets}
}

\ProblemPage{10}{Combo: Log-Det Bound via Golden--Thompson}
\textbf{CANONICAL MATHEMATICAL STATEMENT.}
For SPD $S,T$, show $\log\mathrm{tr}\,e^{\log S + \log T}\le \log\mathrm{tr}\,ST$
and deduce $\mathrm{tr}\,e^{\log S + \log T}\le \mathrm{tr}\,ST$.
\PROBLEM{
Set $A=\log S$, $B=\log T$ in Golden--Thompson to compare the log-trace of a
sum with the trace of the product.
}
\MODEL{
\[
\mathrm{tr}\,e^{\log S + \log T}\le \mathrm{tr}\,e^{\log S}e^{\log T}
=\mathrm{tr}\,ST.
\]
}
\ASSUMPTIONS{
\begin{bullets}
\item $S,T\succ 0$ (symmetric/Hermitian positive definite).
\item Principal matrix logarithm well-defined.
\end{bullets}
}
\varmapStart
\var{S,T}{SPD matrices.}
\var{\log S,\log T}{Principal logarithms (Hermitian).}
\varmapEnd
\WHICHFORMULA{
Golden--Thompson with $A=\log S$, $B=\log T$.}
\GOVERN{
\[
\mathrm{tr}\,e^{\log S + \log T}\le \mathrm{tr}\,ST.
\]
}
\INPUTS{$S,T\succ 0$.}
\DERIVATION{
\begin{align*}
\text{Step 1: }&A=\log S,\ B=\log T\ \text{are Hermitian.}\\
\text{Step 2: }&\mathrm{tr}\,e^{A+B}\le \mathrm{tr}\,e^{A}e^{B}
=\mathrm{tr}\,ST.\\
\text{Step 3: }&\text{Take logs: }\log \mathrm{tr}\,e^{\log S + \log T}
\le \log \mathrm{tr}\,ST.
\end{align*}
}
\RESULT{
$\mathrm{tr}\,e^{\log S + \log T}\le \mathrm{tr}\,ST$.}
\UNITCHECK{
All matrices SPD; exponentials/logs well-defined; traces positive.}
\EDGECASES{
\begin{bullets}
\item If $S$ and $T$ commute, equality.
\item If $S=I$, inequality reduces to $\mathrm{tr}\,T=\mathrm{tr}\,T$.
\end{bullets}
}
\ALTERNATE{
Use Lieb's concavity: $S\mapsto \mathrm{tr}\,e^{\log S + K}$ is concave.}
\VALIDATION{
\begin{bullets}
\item Sample SPD $S,T$; verify inequality numerically. 
\end{bullets}
}
\INTUITION{
Exponentiating and adding logs corresponds to multiplying $S$ and $T$; trace
noncommutativity yields an upper bound.}
\CANONICAL{
\begin{bullets}
\item Golden--Thompson with logarithms of SPD matrices.
\end{bullets}
}

\section{Coding Demonstrations}
\CodeDemoPage{Numerical Check of Von Neumann Inequality}
\PROBLEM{
Compute $L=|\mathrm{tr}(A^\ast B)|$ and $R=\sum_i \sigma_i(A)\sigma_i(B)$ for
random matrices and assert $L\le R$ deterministically with fixed seed.}
\API{
\begin{bullets}
\item \inlinecode{def read_input(s) -> int} — parse size $n$.
\item \inlinecode{def solve_case(n) -> tuple} — compute $L,R$ and margin.
\item \inlinecode{def validate() -> None} — run assertions on random cases.
\item \inlinecode{def main() -> None} — orchestrate and print results.
\end{bullets}
}
\INPUTS{
Integer size $n\ge 1$.
}
\OUTPUTS{
Tuple $(L,R,R-L)$ as floats; nonnegative margin verifies inequality.
}
\FORMULA{
\[
L=|\mathrm{tr}(A^\ast B)|,\quad
R=\sum_{i=1}^n \sigma_i(A)\sigma_i(B),\quad L\le R.
\]
}
\textbf{SOLUTION A — From Scratch (Mathematically Explicit Implementation)}
\begin{codepy}
import numpy as np

def read_input(s):
    return int(s.strip())

def singular_values_via_eigs(M):
    # singular values = sqrt(eigs of M^* M)
    C = M.conj().T @ M
    ew = np.linalg.eigvalsh(C)
    ew = np.maximum(ew, 0.0)
    return np.sqrt(np.sort(ew)[::-1])

def solve_case(n):
    rng = np.random.default_rng(0)
    A = rng.standard_normal((n, n)) + 1j*rng.standard_normal((n, n))
    B = rng.standard_normal((n, n)) + 1j*rng.standard_normal((n, n))
    L = abs(np.trace(A.conj().T @ B))
    sA = singular_values_via_eigs(A)
    sB = singular_values_via_eigs(B)
    R = float(np.dot(sA, sB))
    return float(L), float(R), float(R - L)

def validate():
    for n in [1, 2, 4, 8]:
        L, R, m = solve_case(n)
        assert m >= -1e-10
        assert R >= 0 and L >= 0

def main():
    validate()
    L, R, m = solve_case(5)
    print("L", round(L, 6), "R", round(R, 6), "margin", round(m, 6))

if __name__ == "__main__":
    main()
\end{codepy}
\textbf{SOLUTION B — Library-Based (Validated Computational Shortcut)}
\begin{codepy}
import numpy as np

def read_input(s):
    return int(s.strip())

def solve_case(n):
    rng = np.random.default_rng(0)
    A = rng.standard_normal((n, n)) + 1j*rng.standard_normal((n, n))
    B = rng.standard_normal((n, n)) + 1j*rng.standard_normal((n, n))
    L = abs(np.trace(A.conj().T @ B))
    sA = np.linalg.svd(A, compute_uv=False)
    sB = np.linalg.svd(B, compute_uv=False)
    sA = np.sort(sA)[::-1]
    sB = np.sort(sB)[::-1]
    R = float(np.dot(sA, sB))
    return float(L), float(R), float(R - L)

def validate():
    for n in [1, 3, 6]:
        L, R, m = solve_case(n)
        assert m >= -1e-10

def main():
    validate()
    L, R, m = solve_case(7)
    print("L", round(L, 6), "R", round(R, 6), "margin", round(m, 6))

if __name__ == "__main__":
    main()
\end{codepy}
\COMPLEXITY{
Time $\mathcal{O}(n^3)$ due to SVD/eigendecomposition; space $\mathcal{O}(n^2)$.}
\FAILMODES{
\begin{bullets}
\item Numerical negatives in eigenvalues: clamp to $0$ before sqrt.
\item Ill-conditioned matrices: use Hermitian eigensolver for $M^\ast M$.
\end{bullets}
}
\STABILITY{
\begin{bullets}
\item Prefer \inlinecode{eigvalsh} on Hermitian products for stability.
\item Sorting in descending order avoids pairing mistakes.
\end{bullets}
}
\VALIDATION{
\begin{bullets}
\item Multiple sizes with seed fixed; assert nonnegative margin.
\item Cross-check A and B swapped; results unchanged.
\end{bullets}
}
\RESULT{
Both variants produce $L\le R$ with small positive margin for random inputs.}
\EXPLANATION{
Implementation maps directly to the inequality definition using trace and SVD;
sorting ensures the canonical decreasing order for singular values.}
\EXTENSION{
Vectorized batch tests; histogram of margins over many trials.}

\CodeDemoPage{Numerical Check of Golden--Thompson Inequality}
\PROBLEM{
For random Hermitian $A,B$, compute $L=\mathrm{tr}\,e^{A+B}$ and
$R=\mathrm{tr}\,e^{A}e^{B}$; assert $L\le R$ under fixed seed.}
\API{
\begin{bullets}
\item \inlinecode{def hermitian(n, rng) -> np.ndarray} — random Hermitian.
\item \inlinecode{def solve_case(n) -> tuple} — compute $L,R$ and margin.
\item \inlinecode{def validate() -> None} — run multi-size checks.
\item \inlinecode{def main() -> None} — orchestrate and print results.
\end{bullets}
}
\INPUTS{
Integer size $n\ge 1$.
}
\OUTPUTS{
Tuple $(L,R,R-L)$ as floats; nonnegative margin verifies inequality.
}
\FORMULA{
\[
L=\mathrm{tr}\,e^{A+B},\quad R=\mathrm{tr}\,e^{A}e^{B},\quad L\le R.
\]
}
\textbf{SOLUTION A — From Scratch (Mathematically Explicit Implementation)}
\begin{codepy}
import numpy as np

def hermitian(n, rng):
    M = rng.standard_normal((n, n)) + 1j*rng.standard_normal((n, n))
    return (M + M.conj().T) / 2.0

def mat_exp_via_eig(H):
    ew, V = np.linalg.eigh(H)
    return (V * np.exp(ew)) @ V.conj().T

def solve_case(n):
    rng = np.random.default_rng(1)
    A = hermitian(n, rng)
    B = hermitian(n, rng)
    eA = mat_exp_via_eig(A)
    eB = mat_exp_via_eig(B)
    L = float(np.trace(mat_exp_via_eig(A + B)).real)
    R = float(np.trace(eA @ eB).real)
    return L, R, R - L

def validate():
    for n in [1, 2, 4, 6]:
        L, R, m = solve_case(n)
        assert m >= -1e-10

def main():
    validate()
    L, R, m = solve_case(5)
    print("L", round(L, 6), "R", round(R, 6), "margin", round(m, 6))

if __name__ == "__main__":
    main()
\end{codepy}
\textbf{SOLUTION B — Library-Based (Validated Computational Shortcut)}
\begin{codepy}
import numpy as np

def hermitian(n, rng):
    M = rng.standard_normal((n, n)) + 1j*rng.standard_normal((n, n))
    return (M + M.conj().T) / 2.0

def solve_case(n):
    rng = np.random.default_rng(1)
    A = hermitian(n, rng)
    B = hermitian(n, rng)
    ewA, UA = np.linalg.eigh(A)
    ewB, UB = np.linalg.eigh(B)
    eA = (UA * np.exp(ewA)) @ UA.conj().T
    eB = (UB * np.exp(ewB)) @ UB.conj().T
    L = float(np.trace((UA * np.exp(np.linalg.eigvalsh(A + B)))
                       @ UA.conj().T).real)
    L = float(np.trace((np.linalg.eigh(A + B)[1] *
                        np.exp(np.linalg.eigvalsh(A + B)))
                       @ np.linalg.eigh(A + B)[1].conj().T).real)
    R = float(np.trace(eA @ eB).real)
    return L, R, R - L

def validate():
    for n in [2, 3, 5]:
        L, R, m = solve_case(n)
        assert m >= -1e-10

def main():
    validate()
    L, R, m = solve_case(7)
    print("L", round(L, 6), "R", round(R, 6), "margin", round(m, 6))

if __name__ == "__main__":
    main()
\end{codepy}
\COMPLEXITY{
Time $\mathcal{O}(n^3)$ via eigen-decomposition; space $\mathcal{O}(n^2)$.}
\FAILMODES{
\begin{bullets}
\item Non-Hermitian inputs: enforce Hermitian construction.
\item Numerical errors for large $\|A\|$: consider scaling and squaring if needed.
\end{bullets}
}
\STABILITY{
\begin{bullets}
\item Use Hermitian eigensolver \inlinecode{eigh} for stability.
\item Work with real parts of traces to suppress negligible imaginary drift.
\end{bullets}
}
\VALIDATION{
\begin{bullets}
\item Multiple $n$ with fixed seed; assert nonnegative margins.
\item Swap $A,B$; $R$ unchanged by trace cyclicity.
\end{bullets}
}
\RESULT{
Random Hermitian trials satisfy $L\le R$ with positive margins.}
\EXPLANATION{
Diagonalizing Hermitian matrices yields exact exponentials; traces are computed
and compared per Golden--Thompson.}

\section{Applied Domains — Detailed End-to-End Scenarios}
\DomainPage{Machine Learning}
\SCENARIO{
Bound the partition function of a quadratic energy model:
$H(\theta)=X^\top Q X + r^\top X$ with $Q\succeq 0$. Use Golden--Thompson to
upper bound $Z=\mathrm{tr}\,e^{-\beta(H_0+H_1)}$ where $H_0=X^\top Q X$ and
$H_1=r^\top X$ (discretized state space).}
\ASSUMPTIONS{
\begin{bullets}
\item Finite discrete state space $\{x_k\}_{k=1}^n$; Hermitian diagonal $H_0$.
\item $H_1$ represented as diagonal in a chosen basis or as a small perturbation.
\end{bullets}
}
\WHICHFORMULA{
Golden--Thompson with $A=-\beta H_0$, $B=-\beta H_1$ gives
$Z\le \mathrm{tr}\,e^{-\beta H_0}e^{-\beta H_1}$.}
\varmapStart
\var{H_0}{Quadratic energy (diagonal in basis of $X$).}
\var{H_1}{Linear term as Hermitian operator.}
\var{\beta}{Inverse temperature parameter.}
\var{Z}{Partition function $\mathrm{tr}\,e^{-\beta(H_0+H_1)}$.}
\varmapEnd
\PIPELINE{
\begin{bullets}
\item Build $H_0,H_1$ on an $n$-point grid.
\item Compute $Z$ and its GT upper bound $Z_{\mathrm{GT}}$.
\item Compare numerically; study gap vs. $\|[H_0,H_1]\|$.
\end{bullets}
}
\textbf{Implementation (From Scratch)}
\begin{codepy}
import numpy as np

def build_H(n=50, beta=0.5):
    xs = np.linspace(-2, 2, n)
    Q = 1.5
    r = -0.7
    H0 = np.diag(Q*xs**2)
    H1 = np.diag(r*xs)
    A = -beta*H0
    B = -beta*H1
    return A, B

def trace_exp(H):
    ew = np.linalg.eigvalsh(H)
    return float(np.sum(np.exp(ew)))

def main():
    np.random.seed(0)
    A, B = build_H()
    Z = trace_exp(A + B)
    Zgt = np.trace((np.linalg.eigh(A)[1]*np.exp(np.linalg.eigvalsh(A))
                    ) @ np.linalg.eigh(A)[1].conj().T @
                   (np.linalg.eigh(B)[1]*np.exp(np.linalg.eigvalsh(B))
                    ) @ np.linalg.eigh(B)[1].conj().T).real
    print("Z", round(Z, 6), "Z_GT", round(float(Zgt), 6))

if __name__ == "__main__":
    main()
\end{codepy}
\textbf{Implementation (Library Version)}
\begin{codepy}
import numpy as np

def build_H(n=50, beta=0.5):
    xs = np.linspace(-2, 2, n)
    H0 = np.diag(1.5*xs**2)
    H1 = np.diag(-0.7*xs)
    return -beta*H0, -beta*H1

def expm_from_eigh(H):
    ew, U = np.linalg.eigh(H)
    return (U * np.exp(ew)) @ U.conj().T

def main():
    A, B = build_H()
    Z = float(np.sum(np.exp(np.linalg.eigvalsh(A + B))))
    Zgt = float(np.trace(expm_from_eigh(A) @ expm_from_eigh(B)).real)
    print("Z", round(Z, 6), "Z_GT", round(Zgt, 6))

if __name__ == "__main__":
    main()
\end{codepy}
\METRICS{Report $Z$ and $Z_{\mathrm{GT}}$; check $Z\le Z_{\mathrm{GT}}$.}
\INTERPRET{GT gives a computable upper bound on the partition function.}
\NEXTSTEPS{Use Trotter splitting to tighten the bound with multiple steps.}

\DomainPage{Quantitative Finance}
\SCENARIO{
For a Gaussian factor model with precision $J=J_0+J_1$ (both Hermitian
positive), bound the log-partition function via
$\log Z=\log\mathrm{tr}\,e^{-\frac12 X^\top J X}\le
\log\mathrm{tr}\,e^{-\frac12 X^\top J_0 X}e^{-\frac12 X^\top J_1 X}$.}
\ASSUMPTIONS{
\begin{bullets}
\item Discretized Gaussian state; $J_0,J_1\succeq 0$.
\item Finite grid for numerical illustration.
\end{bullets}
}
\WHICHFORMULA{
Golden--Thompson on $A=-\frac12 X^\top J_0 X$, $B=-\frac12 X^\top J_1 X$.}
\varmapStart
\var{J_0,J_1}{Precision components (PSD).}
\var{Z}{Partition sum on a grid.}
\var{\beta}{Implicit scaling via $\frac12$ factor.}
\varmapEnd
\PIPELINE{
\begin{bullets}
\item Build $J_0,J_1$ on a grid basis as diagonal operators.
\item Compute $Z$ and GT bound; study sensitivity to coupling strength.
\end{bullets}
}
\textbf{Implementation (Full Pipeline)}
\begin{codepy}
import numpy as np

def build_prec(n=60, kappa=0.3):
    xs = np.linspace(-3, 3, n)
    J0 = np.diag(0.8 + 0.2*xs**2)
    J1 = np.diag(kappa*(1 + xs**2))
    return J0, J1

def Z_and_bound(J0, J1):
    A = -0.5*J0
    B = -0.5*J1
    ew_sum = np.linalg.eigvalsh(A + B)
    Z = float(np.sum(np.exp(ew_sum)))
    expA = (np.linalg.eigh(A)[1]*np.exp(np.linalg.eigvalsh(A))
            ) @ np.linalg.eigh(A)[1].conj().T
    expB = (np.linalg.eigh(B)[1]*np.exp(np.linalg.eigvalsh(B))
            ) @ np.linalg.eigh(B)[1].conj().T
    Zgt = float(np.trace(expA @ expB).real)
    return Z, Zgt

def main():
    J0, J1 = build_prec(kappa=0.5)
    Z, Zgt = Z_and_bound(J0, J1)
    print("Z", round(Z, 6), "Z_GT", round(Zgt, 6))

if __name__ == "__main__":
    main()
\end{codepy}
\METRICS{Compare $Z$ vs. $Z_{\mathrm{GT}}$; verify inequality.}
\INTERPRET{Noncommutativity inflates the bound; stronger coupling increases gap.}
\NEXTSTEPS{Extend to low-rank $J_1$; use Trotter to refine bounds.}

\DomainPage{Deep Learning}
\SCENARIO{
Analyze a stability surrogate: for weight matrices $W_1,W_2$ (Hermitian parts),
bound $\mathrm{tr}\,e^{\alpha(\mathrm{sym}(W_1)+\mathrm{sym}(W_2))}$ by GT to
control an exponential Lipschitz proxy.}
\ASSUMPTIONS{
\begin{bullets}
\item Use symmetric parts $\mathrm{sym}(W)=\tfrac12(W+W^\top)$ as Hermitian.
\item $\alpha>0$ small scaling parameter.
\end{bullets}
}
\WHICHFORMULA{
Golden--Thompson with $A=\alpha\,\mathrm{sym}(W_1)$,
$B=\alpha\,\mathrm{sym}(W_2)$.}
\PIPELINE{
\begin{bullets}
\item Generate random weight matrices.
\item Form symmetric parts; compute LHS and GT RHS.
\item Inspect dependence on $\alpha$.
\end{bullets}
}
\textbf{Implementation (End-to-End)}
\begin{codepy}
import numpy as np

def sym(M):
    return 0.5*(M + M.T)

def expm_from_sym(H):
    ew, U = np.linalg.eigh(H)
    return (U * np.exp(ew)) @ U.T

def main():
    rng = np.random.default_rng(3)
    W1 = rng.standard_normal((50, 50))
    W2 = rng.standard_normal((50, 50))
    A = 0.1*sym(W1)
    B = 0.1*sym(W2)
    L = float(np.sum(np.exp(np.linalg.eigvalsh(A + B))))
    eA = expm_from_sym(A)
    eB = expm_from_sym(B)
    R = float(np.trace(eA @ eB))
    print("L", round(L, 6), "R", round(R, 6), "gap", round(R - L, 6))

if __name__ == "__main__":
    main()
\end{codepy}
\METRICS{Report $L,R$ and gap across trials; gap should be nonnegative.}
\INTERPRET{GT bounds an exponential stability proxy arising in residual stacks.}
\NEXTSTEPS{Apply multi-split Trotter to tighten bounds for larger $\alpha$.}

\DomainPage{Kaggle / Data Analytics}
\SCENARIO{
Given covariance-like symmetric matrices $C_0,C_1\succeq 0$, bound
$Z=\mathrm{tr}\,e^{-(C_0+C_1)}$ via GT and compare with the product bound on a
synthetic dataset.}
\ASSUMPTIONS{
\begin{bullets}
\item $C_0,C_1$ SPD; data features standardized.
\item Use eigendecompositions for exponentials.
\end{bullets}
}
\WHICHFORMULA{
Golden--Thompson with $A=-C_0$, $B=-C_1$.}
\PIPELINE{
\begin{bullets}
\item Create SPD $C_0,C_1$ from synthetic correlations.
\item Compute $Z$ and $Z_{\mathrm{GT}}$; verify inequality.
\item Analyze sensitivity to correlation strength.
\end{bullets}
}
\textbf{Implementation (Complete EDA Pipeline)}
\begin{codepy}
import numpy as np

def spd_from_corr(n=40, rho=0.6, seed=4):
    rng = np.random.default_rng(seed)
    M = rng.standard_normal((n, n))
    Q, _ = np.linalg.qr(M)
    D = np.diag(np.linspace(1.0, 2.0, n))
    C0 = Q @ D @ Q.T
    C1 = (1 - rho)*np.eye(n) + rho*np.ones((n, n))/n
    return C0, C1

def tr_exp(H):
    ew = np.linalg.eigvalsh(H)
    return float(np.sum(np.exp(ew)))

def main():
    C0, C1 = spd_from_corr()
    L = tr_exp(-(C0 + C1))
    eC0 = (np.linalg.eigh(-C0)[1]*np.exp(np.linalg.eigvalsh(-C0))
           ) @ np.linalg.eigh(-C0)[1].T
    eC1 = (np.linalg.eigh(-C1)[1]*np.exp(np.linalg.eigvalsh(-C1))
           ) @ np.linalg.eigh(-C1)[1].T
    R = float(np.trace(eC0 @ eC1))
    print("Z", round(L, 6), "Z_GT", round(R, 6), "gap", round(R - L, 6))

if __name__ == "__main__":
    main()
\end{codepy}
\METRICS{Compare $Z$ and $Z_{\mathrm{GT}}$; ensure nonnegative gap.}
\INTERPRET{GT gives a safe upper bound for exponential-of-covariance traces.}
\NEXTSTEPS{Use PCA basis to reduce dimensionality before exponentiation.}

\end{document}