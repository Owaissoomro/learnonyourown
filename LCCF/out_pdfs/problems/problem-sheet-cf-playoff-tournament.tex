% !TeX program = xelatex
\documentclass[11pt]{article}
\usepackage[margin=1in]{geometry}
\usepackage{microtype}
\usepackage{setspace}\setstretch{1.05}
\usepackage{fontspec}
\setmainfont{Latin Modern Roman}
\setmonofont{Latin Modern Mono}[BoldFont={Latin Modern Mono},ItalicFont={Latin Modern Mono}]
\usepackage{amsmath,mathtools,amsthm}
\newtheorem{lemma}{Lemma}
\newtheorem{theorem}{Theorem}
\usepackage{unicode-math}\setmathfont{Latin Modern Math}\allowdisplaybreaks[4]
% --- overflow and alignment slack ---
\setlength{\jot}{7pt}
\sloppy\emergencystretch=8em\hfuzz=1pt\vfuzz=2pt\raggedbottom
% --- breakable math helpers ---
\newenvironment{BreakableEquation}{\begin{equation}\begin{aligned}}{\end{aligned}\end{equation}}
\newenvironment{BreakableEquation*}{\begin{equation*}\begin{aligned}}{\end{aligned}\end{equation*}}
\newenvironment{tightalign}{\begingroup\small\allowdisplaybreaks\begin{align}}{\end{align}\endgroup}

\usepackage{xcolor}
\usepackage{xurl} % better URL wrapping
\usepackage[colorlinks=true,linkcolor=black,urlcolor=blue]{hyperref}
\usepackage{fancyhdr}
\pagestyle{fancy}\fancyhf{}\lhead{\nouppercase{\leftmark}}\rhead{\thepage}\setlength{\headheight}{26pt}
\usepackage{enumitem,booktabs,tabularx}
\newcolumntype{L}{>{\raggedright\arraybackslash}X}
\renewcommand{\arraystretch}{1.1}
\setlength{\parindent}{0pt}\setlength{\parskip}{8pt plus 2pt minus 1pt}

% Listings + upquote (no shell-escape needed)
\usepackage{listings}
\usepackage{upquote}
\lstdefinestyle{crisp}{
  basicstyle=\ttfamily\footnotesize,
  frame=single,
  breaklines=true,
  breakatwhitespace=false, % allow breaks inside long tokens
  tabsize=4,
  showstringspaces=false,
  numbers=left,
  numberstyle=\tiny,
  numbersep=8pt,
  keepspaces=true,
  columns=fullflexible,
  backgroundcolor=\color{black!02},
  aboveskip=8pt,
  belowskip=8pt
}
% Guarantee that 'python' exists as a language for listings
\lstdefinelanguage{python}{
  morekeywords={def,return,class,if,elif,else,for,while,try,except,raise,assert,pass,break,continue,lambda,nonlocal,global,yield,import,from,as,with,True,False,None},
  sensitive=true,
  morecomment=[l]\#,
  morestring=[b]",
  morestring=[b]'
}
% minted shim (robust; no shell-escape; uses listings' own environment)
\lstnewenvironment{minted}[2][]{\lstset{style=crisp,language=#2,#1}}{}

\usepackage[most]{tcolorbox}
\tcbset{colback=white,colframe=black!15,boxrule=0.4pt,arc=2pt,left=6pt,right=6pt,top=6pt,bottom=6pt,before skip=10pt,after skip=10pt,breakable}
\usepackage{titlesec}
\titleformat{\section}{\large\bfseries}{\thesection}{0.6em}{}
\titlespacing*{\section}{0pt}{*2.0}{*1.0}

% ===== CONSISTENCY TOOLKIT (macros) =====
\newlist{ledger}{enumerate}{1}
\setlist[ledger]{label=\textbullet,leftmargin=2em,itemsep=2pt,topsep=6pt}
\newcommand{\LEDGER}[1]{\textbf{ARITHMETIC LEDGER:}\par\begin{ledger}#1\end{ledger}}

\newlist{algosteps}{enumerate}{1}
\setlist[algosteps]{label=\arabic*.,leftmargin=2em,itemsep=2pt,topsep=6pt}

\newlist{bullets}{itemize}{1}
\setlist[bullets]{label=--,leftmargin=1.6em,itemsep=2pt,topsep=4pt}

\newcommand{\varmapStart}{\textbf{VARIABLE MAPPING:}\par\begin{bullets}}
\newcommand{\var}[2]{\item $#1$ — #2}
\newcommand{\varmapEnd}{\end{bullets}}

\newcommand{\LINE}[2]{\par\noindent\textbf{#1:}~#2\par}
\newcommand{\WHAT}[1]{\LINE{WHAT}{#1}}
\newcommand{\WHY}[1]{\LINE{WHY}{#1}}
\newcommand{\HOW}[1]{\LINE{HOW}{#1}}
\newcommand{\ELI}[1]{\LINE{ELI5}{#1}}
\newcommand{\STATEMENT}[1]{\LINE{STATEMENT}{#1}}
\newcommand{\BREAKDOWN}[1]{\LINE{PROBLEM BREAKDOWN}{#1}}
\newcommand{\MODEL}[1]{\LINE{CANONICAL MATHEMATICAL MODEL}{#1}}
\newcommand{\ASSUMPTIONS}[1]{\LINE{ASSUMPTIONS}{#1}}
\newcommand{\INVARIANTS}[1]{\LINE{INVARIANTS}{#1}}
\newcommand{\WHICHFORMULA}[1]{\LINE{WHICH FORMULA \& WHY}{#1}}
\newcommand{\GOVERN}[1]{\LINE{GOVERNING EQUATION(S)}{#1}}
\newcommand{\INPUTS}[1]{\LINE{INPUTS}{#1}}
\newcommand{\OUTPUTS}[1]{\LINE{OUTPUTS}{#1}}
\newcommand{\SAMPLES}[1]{\LINE{SAMPLES}{#1}}
\newcommand{\RESULT}[1]{\LINE{RESULT}{#1}}
\newcommand{\COMPLEXITY}[1]{\LINE{TIME/SPACE COMPLEXITY}{#1}}
\newcommand{\MEMORY}[1]{\LINE{MEMORY FOOTPRINT}{#1}}
\newcommand{\CORRECTNESS}[1]{\LINE{CORRECTNESS SKETCH}{#1}}
\newcommand{\OPTIMALITY}[1]{\LINE{OPTIMALITY}{#1}}
\newcommand{\FAILMODES}[1]{\LINE{FAILURE MODES}{#1}}
\newcommand{\VALIDATION}[1]{\LINE{VALIDATION}{#1}}
\newcommand{\UNITCHECK}[1]{\LINE{UNIT CHECK}{#1}}
\newcommand{\EDGECASES}[1]{\LINE{EDGE CASES}{#1}}
\newcommand{\CHECKLIST}[1]{\LINE{CHECKLIST}{#1}}
\newcommand{\DERIV}[1]{\LINE{DERIVATION}{#1}}
\newcommand{\LEMMAHEAD}[1]{\LINE{SUPPORTING LEMMA}{#1}}
\newcommand{\PITFALLS}[1]{\LINE{PITFALLS}{#1}}

\usepackage{etoolbox}\pretocmd{\section}{\clearpage}{}{}

\newcommand{\FormulaPage}[2]{%
  \clearpage
  \section*{Formula #1 — #2}%
  \addcontentsline{toc}{section}{Formula #1 — #2}%
}
\newcommand{\ApproachPage}[2]{%
  \clearpage
  \subsection*{Approach #1 — #2}%
  \addcontentsline{toc}{subsection}{Approach #1 — #2}%
}

\begin{document}
\title{Interview Problem Sheet — Playoff Tournament}
\date{\today}
\author{}
\maketitle
\tableofcontents
\clearpage

% === Notes macro (BODY-ONLY; do not alter the preamble) ===
% Prints exactly N blank pages with empty headers/footers, then leaves you on the last blank page.
\newcommand{\NotePages}[1]{%
  \clearpage
  \begingroup
  \newcount\NPi \newcount\NPn
  \NPi=0 \NPn=#1
  \loop\ifnum\NPi<\NPn
    \advance\NPi by 1
    \mbox{}\thispagestyle{empty}%
    \ifnum\NPi<\NPn\clearpage\fi
  \repeat
  \endgroup
}

% ============ 1. Problem & Metadata (own page) ============
\section{Problem \& Metadata}
\LINE{PLATFORM}{CF}
\LINE{URL}{\url{https://codeforces.com/problemset/problem/1535/D}}
\LINE{DIFFICULTY / RATING}{1800}
\STATEMENT{You have $2^k$ teams in a single-elimination playoff. There are $2^k-1$ games. In the first round, teams are paired as $(1,2)$, $(3,4)$, $\ldots$ in this order, so $2^{k-1}$ games are played. The loser of each game is eliminated. If more than one team remains, the winners continue: first, the winner of $(1\text{ vs }2)$ plays the winner of $(3\text{ vs }4)$, then the winner of $(5\text{ vs }6)$ plays the winner of $(7\text{ vs }8)$, and so on, until one champion remains.

Let a string $s$ of length $2^k-1$ describe the chronological results of all games:
\begin{bullets}
\item if $s_i$ is $0$, then in the $i$-th game the team with lower index wins;
\item if $s_i$ is $1$, then in the $i$-th game the team with greater index wins;
\item if $s_i$ is ?, the result of the $i$-th game is unknown (either team could win).
\end{bullets}
Define $f(s)$ as the number of possible champions over all replacements of each ? by $0$ or $1$. A team $i$ is a possible champion if there exists a completion of $s$ that makes team $i$ the champion.

You are given the initial $s$ and must process $q$ queries:
\begin{bullets}
\item $p~c$ — replace $s_p$ by character $c$ (where $c \in \{?,0,1\}$), then output $f(s)$.
\end{bullets}

Input:
\begin{bullets}
\item The first line contains an integer $k$ ($1 \le k \le 18$).
\item The second line contains a string of length $2^k-1$ with characters in $\{?,0,1\}$ — the initial $s$.
\item The third line contains an integer $q$ ($1 \le q \le 2\cdot 10^5$).
\item Each of the next $q$ lines contains $p$ and $c$ ($1 \le p \le 2^k-1$, $c \in \{?,0,1\}$).
\end{bullets}
Output: For each query, print a single integer — $f(s)$.}
\BREAKDOWN{Model the tournament as a perfect binary tree of internal matches and leaf teams. Maintain, under point updates to the match-result string, the number of possible winners for the entire tree.}
\ELI{Each match node contributes the number of champions coming from its left and right, depending on whether its result is fixed or unknown; we update only on the path to the root.}
\NotePages{3}

% ============ 2. IO Contract (own page) ============
\section{IO Contract}
\INPUTS{Single test:
\begin{bullets}
\item Integer $k$ with $1 \le k \le 18$.
\item String $s$ of length $2^k-1$, characters in $\{?,0,1\}$.
\item Integer $q$ with $1 \le q \le 2\cdot 10^5$.
\item Then $q$ lines, each with integer $p$ ($1 \le p \le 2^k-1$) and character $c \in \{?,0,1\}$.
\end{bullets}}
\OUTPUTS{For each query, after applying the update to $s$, print on its own line the integer $f(s)$ — the number of possible champions.}
\SAMPLES{
Example 1:
\begin{bullets}
\item Input:
\begin{bullets}
\item $k=1$
\item $s=\texttt{?}$
\item $q=2$
\item Queries: $(1,0)$, $(1,?)$
\end{bullets}
\item Output:
\begin{bullets}
\item After $(1,0)$: $1$
\item After $(1,?)$: $2$
\end{bullets}
\end{bullets}
Example 2:
\begin{bullets}
\item Input:
\begin{bullets}
\item $k=2$
\item $s=\texttt{???}$
\item $q=3$
\item Queries: $(1,0)$, $(2,1)$, $(3,0)$
\end{bullets}
\item Output:
\begin{bullets}
\item After $(1,0)$: $3$
\item After $(2,1)$: $2$
\item After $(3,0)$: $1$
\end{bullets}
\end{bullets}
}
\NotePages{3}

% ============ 3. Canonical Mathematical Model (own page) ============
\section{Canonical Mathematical Model}
\MODEL{Let $n=2^k$. Consider a perfect binary tree with $n$ leaves (teams $1,\ldots,n$ in-order) and $n-1$ internal nodes (matches). The string $s$ of length $n-1$ is chronological from first-round left-to-right games to the final. Define for each internal node $v$ a value $dp[v]$ equal to the number of possible champions arising from the subtree rooted at $v$ given the partial information in $s$.}
\varmapStart
\var{k}{number of rounds; $n=2^k$ teams}
\var{n}{number of teams}
\var{s}{string of length $n-1$ describing match outcomes chronologically}
\var{dp[v]}{number of possible winners in subtree of node $v$}
\var{L(v),R(v)}{left and right children of internal node $v$}
\var{m}{number of internal nodes, $m=n-1$}
\var{r}{the root internal node (the final)}
\var{\pi}{mapping from chronological index to internal-node index; practically implemented by reversing $s$}
\varmapEnd
\GOVERN{
\[
\begin{aligned}
&\text{Base (leaves): } \forall\ \text{leaf }\ell,\quad \#\text{winners from }\ell = 1.\\
&\text{Internal node }v\text{ with character }c=\text{label}(v)\in\{0,1,?\}:\\
&\quad dp[v] =
\begin{cases}
dp[L(v)] + dp[R(v)], & c = ?,\\
dp[L(v)], & c = 0,\\
dp[R(v)], & c = 1.
\end{cases}\\
&\text{Champion count: } f(s) = dp[r].
\end{aligned}
\]
}
\ASSUMPTIONS{Leaves correspond to teams in increasing index from left to right. The mapping from $s$ positions to internal nodes is the reverse of chronology (root is last), so operationally we reverse $s$ once and index internal nodes in binary-heap order $1,\ldots,n-1$.}
\INVARIANTS{
\begin{bullets}
\item Each subtree's $dp$ equals the number of possible winners of that sub-bracket consistent with current information.
\item Updates only affect the path from a modified node to the root; all other subtrees remain unchanged.
\end{bullets}
}
\NotePages{3}

% ============ 4. Approach A — Baseline (own page) ============
\section{Approach A — Baseline}
\ApproachPage{A}{Brute Force / Direct}
\WHICHFORMULA{Enumerate all completions of ? in $s$ and simulate the tournament chronologically, counting distinct champions.}
\ASSUMPTIONS{Feasible only for tiny $k$ because the number of completions is exponential in the number of ? characters.}
\textbf{Algorithm Steps}
\begin{algosteps}
\item Identify the indices of ? in $s$; for each binary assignment to them, create a fully-specified result string.
\item Simulate the tournament round-by-round in chronological order, using the fixed results to pick winners by lower/higher index.
\item Record the unique champion; at the end, output the number of distinct champions across all completions.
\end{algosteps}
\COMPLEXITY{Let $u$ be the number of ? characters. Simulation per completion costs $O(n)$, hence $T(n)=O(n\cdot 2^u)$ and $S(n)=O(n)$.}
\[
\begin{aligned}
T(n) &= \text{number of completions}\times \text{cost per simulation} \\
     &= 2^u \times O(n) \\
     &= O(n\cdot 2^u).
\end{aligned}
\]
\CORRECTNESS{Every completion yields exactly one champion, and we count unique champions over all completions; this matches the definition of $f(s)$.}
\EDGECASES{No ? in $s$ (then $f(s)=1$). Single round $k=1$. All games unknown.}
\textbf{Code (Baseline)}
\begin{minted}{python}
import sys
from itertools import product
from typing import List, Tuple

def simulate_chronological(s: str, k: int) -> int:
    """Return the champion for a fully specified s (no '?')."""
    n = 1 << k
    teams = list(range(1, n + 1))
    idx = 0
    cur = teams
    while len(cur) > 1:
        nxt = []
        for i in range(0, len(cur), 2):
            a, b = cur[i], cur[i + 1]
            c = s[idx]
            idx += 1
            if c == '0':
                nxt.append(a if a < b else b)
            elif c == '1':
                nxt.append(a if a > b else b)
            else:
                raise ValueError("simulate_chronological expects no '?'")
        cur = nxt
    return cur[0]

def f_bruteforce(s: str, k: int) -> int:
    """Count number of possible champions by enumerating all '?' completions."""
    qpos = [i for i, ch in enumerate(s) if ch == '?']
    if not qpos:
        return 1  # fully specified tournament has a unique champion
    champions = set()
    for bits in product('01', repeat=len(qpos)):
        s_list = list(s)
        for i, bit in zip(qpos, bits):
            s_list[i] = bit
        champ = simulate_chronological(''.join(s_list), k)
        champions.add(champ)
    return len(champions)

def read_input() -> Tuple[int, str, int, List[Tuple[int, str]]]:
    data = sys.stdin.read().strip().split()
    it = iter(data)
    k = int(next(it))
    s = next(it).strip()
    q = int(next(it))
    queries = []
    for _ in range(q):
        p = int(next(it)); c = next(it)
        queries.append((p, c))
    return k, s, q, queries

def solve_all():
    k, s, q, queries = read_input()
    # Brute force per query is infeasible generally; here we just show the API.
    # We recompute f(s) via brute force after each update (works only for tiny inputs).
    s_list = list(s)
    out = []
    for p, c in queries:
        s_list[p - 1] = c
        out.append(str(f_bruteforce(''.join(s_list), k)))
    sys.stdout.write("\n".join(out))

def _tiny_test():
    # k=1, initial '?', queries set to 0 then ?
    inp = "1\n?\n2\n1 0\n1 ?\n"
    exp = "1\n2\n"
    from io import StringIO
    backup_stdin, backup_stdout = sys.stdin, sys.stdout
    try:
        sys.stdin = StringIO(inp)
        sys.stdout = StringIO()
        solve_all()
        out = sys.stdout.getvalue()
        assert out == exp
    finally:
        sys.stdin, sys.stdout = backup_stdin, backup_stdout

if __name__ == "__main__":
    _tiny_test()
    # main disabled for baseline due to exponential behavior on large cases
    # Uncomment to run on tiny inputs:
    # solve_all()
\end{minted}
\VALIDATION{Checked on $k=1$ tiny case; additional checks recommended only for very small $k$ due to exponential growth.}
\NotePages{3}

% ============ 5. Approach B — Improved (own page) ============
\section{Approach B — Improved}
\ApproachPage{B}{Rebuild DP after each update}
\WHICHFORMULA{Use a bottom-up DP on the implicit complete binary tree. After each update to $s$, rebuild the entire DP in $O(n)$ using the recurrence.}
\ASSUMPTIONS{Maintain leaves as $1$. Reverse $s$ so that internal nodes are indexed in heap order: node $i$ corresponds to character $s_{\text{rev}}[i-1]$.}
\textbf{Algorithm Steps}
\begin{algosteps}
\item Precompute $n=2^k$, $m=n-1$, and store $s_{\text{rev}}=s$ reversed.
\item For each query $(p,c)$, set $s_{\text{rev}}[m-p]=c$.
\item Rebuild $dp[i]$ for $i=n-1$ down to $1$:
\begin{bullets}
\item If $s_{\text{rev}}[i-1]=?$ then $dp[i]=dp[2i]+dp[2i+1]$.
\item If $0$ then $dp[i]=dp[2i]$ or $dp[2i+1]$ depending on mapping; either consistent mapping yields correct count. We use $dp[i]=dp[2i+1]$ for $0$ and $dp[i]=dp[2i]$ for $1$ after reversal.
\end{bullets}
\item Output $dp[1]$.
\end{algosteps}
\COMPLEXITY{Each rebuild is $O(n)$ time and $O(n)$ space, so total $O(qn)$ time. This is too slow for the largest constraints but demonstrates the DP structure.}
\[
\begin{aligned}
T(n) &= O(q \cdot n),\quad S(n) = O(n).
\end{aligned}
\]
\CORRECTNESS{By construction, each node aggregates the number of possible winners from its children according to its match outcome; thus the root yields $f(s)$.}
\textbf{Code (Improved)}
\begin{minted}{python}
import sys
from typing import List, Tuple

def build_dp_from_reversed(sr: List[str], k: int) -> int:
    """Given reversed s as list of chars, compute f(s) in O(n)."""
    n = 1 << k
    dp = [0] * (2 * n)
    for i in range(n, 2 * n):
        dp[i] = 1
    for i in range(n - 1, 0, -1):
        ch = sr[i - 1]
        if ch == '?':
            dp[i] = dp[i * 2] + dp[i * 2 + 1]
        elif ch == '0':
            dp[i] = dp[i * 2 + 1]
        else:
            dp[i] = dp[i * 2]
    return dp[1]

def read_input() -> Tuple[int, str, int, List[Tuple[int, str]]]:
    data = sys.stdin.read().strip().split()
    it = iter(data)
    k = int(next(it))
    s = next(it).strip()
    q = int(next(it))
    queries = []
    for _ in range(q):
        p = int(next(it)); c = next(it)
        queries.append((p, c))
    return k, s, q, queries

def solve_all():
    k, s, q, queries = read_input()
    n = 1 << k
    m = n - 1
    sr = list(s[::-1])
    out = []
    for p, c in queries:
        sr[m - p] = c
        out.append(str(build_dp_from_reversed(sr, k)))
    sys.stdout.write("\n".join(out))

def _tests():
    # Tiny sanity checks
    # k=1: s="?" -> 2; after setting to 0 -> 1
    inp = "1\n?\n1\n1 0\n"
    from io import StringIO
    backup_stdin, backup_stdout = sys.stdin, sys.stdout
    try:
        sys.stdin = StringIO(inp)
        sys.stdout = StringIO()
        solve_all()
        assert sys.stdout.getvalue() == "1\n"
    finally:
        sys.stdin, sys.stdout = backup_stdin, backup_stdout

if __name__ == "__main__":
    _tests()
    # solve_all()
\end{minted}
\VALIDATION{Validated on tiny cases; maps to known recurrence. Note: Mapping uses reversed $s$ and the convention $0\mapsto$ right child, $1\mapsto$ left child after reversal.}
\NotePages{3}

% ============ 6. Approach C — Optimal (own page) ============
\section{Approach C — Optimal}
\ApproachPage{C}{Point-Update Segment DP in $O(\log n)$}
\WHICHFORMULA{Maintain the bottom-up DP array and update only along the path from the modified internal node to the root. This yields $O(\log n)$ per query.}
\ASSUMPTIONS{Store $s_{\text{rev}}=s^{\text{rev}}$ so that internal nodes are in heap order: node $i$ corresponds to position $i-1$ in $s_{\text{rev}}$. Leaves carry value $1$.}
\textbf{Algorithm Steps}
\begin{algosteps}
\item Initialize $n=2^k$, $m=n-1$, $s_{\text{rev}} = s$ reversed, and $dp[n..2n-1]=1$.
\item Build $dp[i]$ for $i=n-1$ down to $1$ using:
\begin{bullets}
\item if $s_{\text{rev}}[i-1]=?$ then $dp[i]=dp[2i]+dp[2i+1]$;
\item if $s_{\text{rev}}[i-1]=0$ then $dp[i]=dp[2i+1]$;
\item if $s_{\text{rev}}[i-1]=1$ then $dp[i]=dp[2i]$.
\end{bullets}
\item For an update $(p,c)$: set $s_{\text{rev}}[m-p]=c$ and let $i=(m-p)+1$. While $i\ge 1$, recompute $dp[i]$ from its children and set $i\gets \lfloor i/2\rfloor$.
\item After each update, output $dp[1]$.
\end{algosteps}
\OPTIMALITY{Any correct algorithm must at least propagate the effect of an update through affected ancestors, which is a path of length $\Theta(\log n)$ in a balanced binary tree. Thus the $O(\log n)$ bound is tight up to constants.}
\COMPLEXITY{Initialization $O(n)$; each query $O(\log n)$ time, $O(1)$ extra space (beyond $O(n)$ storage).}
\[
\begin{aligned}
T(n) &= O(n) + q\cdot O(\log n),\quad S(n)=O(n).
\end{aligned}
\]
\textbf{Code (Final Submission)}
\begin{minted}{python}
import sys
from typing import List, Tuple

def read_input() -> Tuple[int, str, int, List[Tuple[int, str]]]:
    data = sys.stdin.read().strip().split()
    it = iter(data)
    k = int(next(it))
    s = next(it).strip()
    q = int(next(it))
    queries = []
    for _ in range(q):
        p = int(next(it)); c = next(it)
        queries.append((p, c))
    return k, s, q, queries

def build_initial(k: int, s: str):
    n = 1 << k
    m = n - 1
    sr = list(s[::-1])  # reversed mapping to heap order
    dp = [0] * (2 * n)
    for i in range(n, 2 * n):
        dp[i] = 1
    for i in range(n - 1, 0, -1):
        ch = sr[i - 1]
        if ch == '?':
            dp[i] = dp[i * 2] + dp[i * 2 + 1]
        elif ch == '0':
            dp[i] = dp[i * 2 + 1]
        else:  # '1'
            dp[i] = dp[i * 2]
    return n, m, sr, dp

def recompute_node(i: int, sr: List[str], dp: List[int]):
    ch = sr[i - 1]
    if ch == '?':
        dp[i] = dp[i * 2] + dp[i * 2 + 1]
    elif ch == '0':
        dp[i] = dp[i * 2 + 1]
    else:
        dp[i] = dp[i * 2]

def solve_all():
    k, s, q, queries = read_input()
    n, m, sr, dp = build_initial(k, s)
    out_lines = []
    for p, c in queries:
        pos = m - p  # position in reversed string (0-based)
        if sr[pos] != c:
            sr[pos] = c
            i = pos + 1  # internal node index
            while i >= 1:
                recompute_node(i, sr, dp)
                i //= 2
        out_lines.append(str(dp[1]))
    sys.stdout.write("\n".join(out_lines))

def _asserts():
    # 1) k=1 basic transitions
    inp = "1\n?\n3\n1 0\n1 1\n1 ?\n"
    exp = "1\n1\n2\n"
    from io import StringIO
    backup_stdin, backup_stdout = sys.stdin, sys.stdout
    try:
        sys.stdin = StringIO(inp)
        sys.stdout = StringIO()
        solve_all()
        assert sys.stdout.getvalue() == exp
    finally:
        sys.stdin, sys.stdout = backup_stdin, backup_stdout
    # 2) k=2, all unknown then fix final only
    inp = "2\n???\n2\n3 0\n3 1\n"
    # Setting only final to 0 or 1 leaves two semifinals unknown: each side contributes 1 or 1, so f=2 both times
    exp = "2\n2\n"
    try:
        sys.stdin = StringIO(inp)
        sys.stdout = StringIO()
        solve_all()
        assert sys.stdout.getvalue() == exp
    finally:
        sys.stdin, sys.stdout = backup_stdin, backup_stdout
    # 3) Random small cross-check with rebuild
    def oracle(k, s, queries):
        n = 1 << k
        m = n - 1
        sr = list(s[::-1])
        def full():
            dp = [0]*(2*n)
            for i in range(n, 2*n):
                dp[i]=1
            for i in range(n-1,0,-1):
                ch = sr[i-1]
                if ch=='?':
                    dp[i]=dp[i*2]+dp[i*2+1]
                elif ch=='0':
                    dp[i]=dp[i*2+1]
                else:
                    dp[i]=dp[i*2]
            return dp[1]
        ans=[]
        for p,c in queries:
            sr[m-p]=c
            ans.append(full())
        return ans
    import random
    random.seed(0)
    k = 3
    s = "?" * ((1 << k) - 1)
    qs = [(random.randint(1, (1<<k)-1), random.choice(['0','1','?'])) for _ in range(10)]
    # Run solver
    from io import StringIO
    inp = str(k) + "\n" + s + "\n" + str(len(qs)) + "\n" + "\n".join(f"{p} {c}" for p,c in qs) + "\n"
    backup_stdin, backup_stdout = sys.stdin, sys.stdout
    try:
        sys.stdin = StringIO(inp); sys.stdout = StringIO()
        solve_all()
        out = list(map(int, sys.stdout.getvalue().strip().split()))
    finally:
        sys.stdin, sys.stdout = backup_stdin, backup_stdout
    assert out == oracle(k, s, qs)

if __name__ == "__main__":
    _asserts()
    solve_all()
\end{minted}
\VALIDATION{Three asserts:
\begin{bullets}
\item $k=1$ basic transitions over queries.
\item $k=2$ final-only updates keep $f(s)=2$.
\item Random small cross-check vs an $O(n)$ rebuild oracle.
\end{bullets}}
\RESULT{After each point update, output the total number of possible champions $f(s)=dp[\text{root}]$.}
\NotePages{3}

% ============ 7. Testing & Final Reference Implementation (own page) ============
\section{Testing \& Final Reference Implementation}
\LINE{TEST PLAN}{Unit tests on smallest sizes, structured cases fixing only finals or only first-round matches, and random fuzzing for small $k$ by comparing $O(\log n)$ updates vs $O(n)$ rebuilds.}
\LINE{CROSS-CHECKS}{Compare Approach C outputs with Approach B (full rebuild) on random sequences for small $k \le 4$.}
\LINE{EDGE-CASE GENERATOR}{Generate patterns like all '?', alternating 0/1, and repeated updates toggling the same position.}
\begin{minted}{python}
# Deterministic generators for boundaries, degenerates, adversarials
import random

def gen_all_unknown(k: int, q: int):
    n = 1 << k
    m = n - 1
    s = "?" * m
    queries = []
    for i in range(q):
        p = (i % m) + 1
        c = random.choice(['0', '1', '?'])
        queries.append((p, c))
    return k, s, queries

def gen_alternating(k: int, q: int):
    n = 1 << k
    m = n - 1
    s = "".join('0' if i % 2 == 0 else '1' for i in range(m))
    queries = []
    for i in range(q):
        p = random.randint(1, m)
        c = '?' if i % 3 == 0 else ('0' if i % 2 == 0 else '1')
        queries.append((p, c))
    return k, s, queries
\end{minted}
\textbf{Reference Code (Ready to Submit)}
\begin{minted}{python}
import sys
from typing import List, Tuple

def read_input() -> Tuple[int, str, int, List[Tuple[int, str]]]:
    data = sys.stdin.read().strip().split()
    it = iter(data)
    k = int(next(it))
    s = next(it).strip()
    q = int(next(it))
    queries = []
    for _ in range(q):
        p = int(next(it)); c = next(it)
        queries.append((p, c))
    return k, s, q, queries

def build_initial(k: int, s: str):
    n = 1 << k
    m = n - 1
    sr = list(s[::-1])
    dp = [0] * (2 * n)
    for i in range(n, 2 * n):
        dp[i] = 1
    for i in range(n - 1, 0, -1):
        ch = sr[i - 1]
        if ch == '?':
            dp[i] = dp[i * 2] + dp[i * 2 + 1]
        elif ch == '0':
            dp[i] = dp[i * 2 + 1]
        else:
            dp[i] = dp[i * 2]
    return n, m, sr, dp

def recompute_node(i: int, sr: List[str], dp: List[int]):
    ch = sr[i - 1]
    if ch == '?':
        dp[i] = dp[i * 2] + dp[i * 2 + 1]
    elif ch == '0':
        dp[i] = dp[i * 2 + 1]
    else:
        dp[i] = dp[i * 2]

def solve_all():
    k, s, q, queries = read_input()
    n, m, sr, dp = build_initial(k, s)
    out_lines = []
    for p, c in queries:
        pos = m - p
        if sr[pos] != c:
            sr[pos] = c
            i = pos + 1
            while i >= 1:
                recompute_node(i, sr, dp)
                i //= 2
        out_lines.append(str(dp[1]))
    sys.stdout.write("\n".join(out_lines))

def _sanity():
    # Minimal regression checks
    from io import StringIO
    backup_stdin, backup_stdout = sys.stdin, sys.stdout
    try:
        # k=1 toggles
        inp = "1\n?\n3\n1 0\n1 1\n1 ?\n"
        exp = "1\n1\n2\n"
        sys.stdin = StringIO(inp); sys.stdout = StringIO()
        solve_all()
        assert sys.stdout.getvalue() == exp
        # k=2 final-only updates
        inp = "2\n???\n2\n3 0\n3 1\n"
        exp = "2\n2\n"
        sys.stdin = StringIO(inp); sys.stdout = StringIO()
        solve_all()
        assert sys.stdout.getvalue() == exp
    finally:
        sys.stdin, sys.stdout = backup_stdin, backup_stdout

if __name__ == "__main__":
    _sanity()
    solve_all()
\end{minted}
\NotePages{3}

% ============ 8. Review & Pitfalls (own page) ============
\section{Review \& Pitfalls}
\WHAT{Maintain a DP on a perfect binary tree where each internal node aggregates possible champions based on its outcome; process updates along the path to the root.}
\WHY{Common interview pattern: dynamic updates on trees/segment-tree-like structures with string-index remapping.}
\CHECKLIST{
\begin{bullets}
\item Compute $n=2^k$, $m=n-1$.
\item Reverse $s$ to align chronology with heap indexing.
\item Initialize leaves to $1$.
\item Build $dp$ bottom-up once.
\item For each update, map $p$ to $m-p$ in reversed string.
\item Recompute up the ancestor chain to the root.
\item Output $dp[1]$ after each update.
\end{bullets}
}
\EDGECASES{
\begin{bullets}
\item $k=1$ (single match).
\item No ? in $s$ (always output $1$).
\item All ? in $s$ (initially $n$ possible champions).
\item Repeated updates to the same position.
\item Long runs of the same character.
\item Large $q$ with $k=18$ (ensure $O(\log n)$ per update).
\end{bullets}
}
\PITFALLS{
\begin{bullets}
\item Forgetting to reverse $s$ so that indices match heap order.
\item Off-by-one when mapping $p$ to reversed index: it is $m-p$ in $0$-based.
\item Recomputing leaves: ensure you never write to $dp[i]$ for $i\ge n$.
\item Mixing left/right mapping for characters after reversal; stay consistent.
\item Not using fast I/O for large $q$.
\item Accidentally rebuilding full DP per query.
\end{bullets}
}
\FAILMODES{A naive rebuild per query times out. Incorrect mapping between $s$ and nodes gives wrong counts. The presented method survives by using a consistent reverse-indexing and local path updates.}
\ELI{Think of each match as saying “winners can come from left, right, or both”. Add them up from the bottom. When a match result changes, only its ancestors’ counts change, so fix just that path quickly.}
\NotePages{3}

\end{document}