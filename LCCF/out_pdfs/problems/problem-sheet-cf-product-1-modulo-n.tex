% !TeX program = xelatex
\documentclass[11pt]{article}
\usepackage[margin=1in]{geometry}
\usepackage{microtype}
\usepackage{setspace}\setstretch{1.05}
\usepackage{fontspec}
\setmainfont{Latin Modern Roman}
\setmonofont{Latin Modern Mono}[BoldFont={Latin Modern Mono},ItalicFont={Latin Modern Mono}]
\usepackage{amsmath,mathtools,amsthm}
\newtheorem{lemma}{Lemma}
\newtheorem{theorem}{Theorem}
\usepackage{unicode-math}\setmathfont{Latin Modern Math}\allowdisplaybreaks[4]
% --- overflow and alignment slack ---
\setlength{\jot}{7pt}
\sloppy\emergencystretch=8em\hfuzz=1pt\vfuzz=2pt\raggedbottom
% --- breakable math helpers ---
\newenvironment{BreakableEquation}{\begin{equation}\begin{aligned}}{\end{aligned}\end{equation}}
\newenvironment{BreakableEquation*}{\begin{equation*}\begin{aligned}}{\end{aligned}\end{equation*}}
\newenvironment{tightalign}{\begingroup\small\allowdisplaybreaks\begin{align}}{\end{align}\endgroup}

\usepackage{xcolor}
\usepackage{xurl} % better URL wrapping
\usepackage[colorlinks=true,linkcolor=black,urlcolor=blue]{hyperref}
\usepackage{fancyhdr}
\pagestyle{fancy}\fancyhf{}\lhead{\nouppercase{\leftmark}}\rhead{\thepage}\setlength{\headheight}{26pt}
\usepackage{enumitem,booktabs,tabularx}
\newcolumntype{L}{>{\raggedright\arraybackslash}X}
\renewcommand{\arraystretch}{1.1}
\setlength{\parindent}{0pt}\setlength{\parskip}{8pt plus 2pt minus 1pt}

% Listings + upquote (no shell-escape needed)
\usepackage{listings}
\usepackage{upquote}
\lstdefinestyle{crisp}{
  basicstyle=\ttfamily\footnotesize,
  frame=single,
  breaklines=true,
  breakatwhitespace=false, % allow breaks inside long tokens
  tabsize=4,
  showstringspaces=false,
  numbers=left,
  numberstyle=\tiny,
  numbersep=8pt,
  keepspaces=true,
  columns=fullflexible,
  backgroundcolor=\color{black!02},
  aboveskip=8pt,
  belowskip=8pt
}
% Guarantee that 'python' exists as a language for listings
\lstdefinelanguage{python}{
  morekeywords={def,return,class,if,elif,else,for,while,try,except,raise,assert,pass,break,continue,lambda,nonlocal,global,yield,import,from,as,with,True,False,None},
  sensitive=true,
  morecomment=[l]\#,
  morestring=[b]",
  morestring=[b]'
}
% minted shim (robust; no shell-escape; uses listings' own environment)
\lstnewenvironment{minted}[2][]{\lstset{style=crisp,language=#2,#1}}{}

\usepackage[most]{tcolorbox}
\tcbset{colback=white,colframe=black!15,boxrule=0.4pt,arc=2pt,left=6pt,right=6pt,top=6pt,bottom=6pt,before skip=10pt,after skip=10pt,breakable}
\usepackage{titlesec}
\titleformat{\section}{\large\bfseries}{\thesection}{0.6em}{}
\titlespacing*{\section}{0pt}{*2.0}{*1.0}

% ===== CONSISTENCY TOOLKIT (macros) =====
\newlist{ledger}{enumerate}{1}
\setlist[ledger]{label=\textbullet,leftmargin=2em,itemsep=2pt,topsep=6pt}
\newcommand{\LEDGER}[1]{\textbf{ARITHMETIC LEDGER:}\par\begin{ledger}#1\end{ledger}}

\newlist{algosteps}{enumerate}{1}
\setlist[algosteps]{label=\arabic*.,leftmargin=2em,itemsep=2pt,topsep=6pt}

\newlist{bullets}{itemize}{1}
\setlist[bullets]{label=--,leftmargin=1.6em,itemsep=2pt,topsep=4pt}

\newcommand{\varmapStart}{\textbf{VARIABLE MAPPING:}\par\begin{bullets}}
\newcommand{\var}[2]{\item $#1$ — #2}
\newcommand{\varmapEnd}{\end{bullets}}

\newcommand{\LINE}[2]{\par\noindent\textbf{#1:}~#2\par}
\newcommand{\WHAT}[1]{\LINE{WHAT}{#1}}
\newcommand{\WHY}[1]{\LINE{WHY}{#1}}
\newcommand{\HOW}[1]{\LINE{HOW}{#1}}
\newcommand{\ELI}[1]{\LINE{ELI5}{#1}}
\newcommand{\STATEMENT}[1]{\LINE{STATEMENT}{#1}}
\newcommand{\BREAKDOWN}[1]{\LINE{PROBLEM BREAKDOWN}{#1}}
\newcommand{\MODEL}[1]{\LINE{CANONICAL MATHEMATICAL MODEL}{#1}}
\newcommand{\ASSUMPTIONS}[1]{\LINE{ASSUMPTIONS}{#1}}
\newcommand{\INVARIANTS}[1]{\LINE{INVARIANTS}{#1}}
\newcommand{\WHICHFORMULA}[1]{\LINE{WHICH FORMULA \& WHY}{#1}}
\newcommand{\GOVERN}[1]{\LINE{GOVERNING EQUATION(S)}{#1}}
\newcommand{\INPUTS}[1]{\LINE{INPUTS}{#1}}
\newcommand{\OUTPUTS}[1]{\LINE{OUTPUTS}{#1}}
\newcommand{\SAMPLES}[1]{\LINE{SAMPLES}{#1}}
\newcommand{\RESULT}[1]{\LINE{RESULT}{#1}}
\newcommand{\COMPLEXITY}[1]{\LINE{TIME/SPACE COMPLEXITY}{#1}}
\newcommand{\MEMORY}[1]{\LINE{MEMORY FOOTPRINT}{#1}}
\newcommand{\CORRECTNESS}[1]{\LINE{CORRECTNESS SKETCH}{#1}}
\newcommand{\OPTIMALITY}[1]{\LINE{OPTIMALITY}{#1}}
\newcommand{\FAILMODES}[1]{\LINE{FAILURE MODES}{#1}}
\newcommand{\VALIDATION}[1]{\LINE{VALIDATION}{#1}}
\newcommand{\UNITCHECK}[1]{\LINE{UNIT CHECK}{#1}}
\newcommand{\EDGECASES}[1]{\LINE{EDGE CASES}{#1}}
\newcommand{\CHECKLIST}[1]{\LINE{CHECKLIST}{#1}}
\newcommand{\DERIV}[1]{\LINE{DERIVATION}{#1}}
\newcommand{\LEMMAHEAD}[1]{\LINE{SUPPORTING LEMMA}{#1}}
\newcommand{\PITFALLS}[1]{\LINE{PITFALLS}{#1}}

\usepackage{etoolbox}\pretocmd{\section}{\clearpage}{}{}

\newcommand{\FormulaPage}[2]{%
  \clearpage
  \section*{Formula #1 — #2}%
  \addcontentsline{toc}{section}{Formula #1 — #2}%
}
\newcommand{\ApproachPage}[2]{%
  \clearpage
  \subsection*{Approach #1 — #2}%
  \addcontentsline{toc}{subsection}{Approach #1 — #2}%
}

\begin{document}
\title{Interview Problem Sheet — Product 1 Modulo N}
\date{\today}
\author{}
\maketitle
\tableofcontents
\clearpage

% === Notes macro (BODY-ONLY; do not alter the preamble) ===
% Prints exactly N blank pages with empty headers/footers, then leaves you on the last blank page.
\newcommand{\NotePages}[1]{%
  \clearpage
  \begingroup
  \newcount\NPi \newcount\NPn
  \NPi=0 \NPn=#1
  \loop\ifnum\NPi<\NPn
    \advance\NPi by 1
    \mbox{}\thispagestyle{empty}%
    \ifnum\NPi<\NPn\clearpage\fi
  \repeat
  \endgroup
}

% ============ 1. Problem & Metadata (own page) ============
\section{Problem \& Metadata}
\LINE{PLATFORM}{CF}
\LINE{URL}{\url{https://codeforces.com/problemset/problem/1514/C}}
\LINE{DIFFICULTY / RATING}{1600}
\STATEMENT{Now you get Baby Ehab's first words: ``Given an integer $n$, find the longest subsequence of $[1,2, \ldots, n-1]$ whose product is $1$ modulo $n$.'' Please solve the problem.

A sequence $b$ is a subsequence of an array $a$ if $b$ can be obtained from $a$ by deleting some (possibly all) elements. The product of an empty subsequence is equal to $1$.

Input: The only line contains the integer $n$ ($2 \le n \le 10^5$).

Output: The first line should contain a single integer, the length of the longest subsequence. The second line should contain the elements of the subsequence, in increasing order. If there are multiple solutions, you can print any.

Note: In the first example, the product of the elements is $6$ which is congruent to $1$ modulo $5$. The only longer subsequence is $[1,2,3,4]$. Its product is $24$ which is congruent to $4$ modulo $5$. Hence, the answer is $[1,2,3]$.}
\BREAKDOWN{We must select the largest subset of $\{1,\ldots,n-1\}$ whose product is congruent to $1 \pmod n$. Any chosen element must be coprime to $n$. The multiplicative group of units modulo $n$ gives a natural candidate; adjust by removing at most one element to force product $1$.}
\ELI{Take all numbers from $1$ to $n-1$ that share no common factor with $n$; if their product is not $1$ modulo $n$, drop the single number equal to that product.}
\NotePages{3}

% ============ 2. IO Contract (own page) ============
\section{IO Contract}
\INPUTS{Single integer $n$ with $2 \le n \le 10^5$.}
\OUTPUTS{Two lines: first, an integer $k$ equal to the maximum length. Second, $k$ distinct integers in increasing order forming the subsequence whose product is $1 \pmod n$. If $k=0$, print an empty second line.}
\SAMPLES{Example 1: input $n=5$. Units are $[1,2,3,4]$, their product is $4 \not\equiv 1 \pmod 5$, so remove $4$. Output:
1) $3$; 2) $1~2~3$.

Example 2: input $n=8$. Units are $[1,3,5,7]$, product is $1 \pmod 8$. Output:
1) $4$; 2) $1~3~5~7$.}
\NotePages{3}

% ============ 3. Canonical Mathematical Model (own page) ============
\section{Canonical Mathematical Model}
\MODEL{Let $U(n)=\{x\in\{1,\ldots,n-1\}:\gcd(x,n)=1\}$ be the multiplicative group of units modulo $n$. We want a maximum-cardinality subset $S\subseteq U(n)$ such that $\prod_{x\in S}\equiv 1 \pmod n$.}
\varmapStart
\var{n}{the modulus}
\var{U(n)}{the set of units modulo $n$}
\var{S}{chosen subset of $U(n)$}
\var{P}{the product $\prod_{x\in U(n)} x \bmod n$}
\varmapEnd
\GOVERN{
\[
\begin{aligned}
U(n) &= \{x\in\{1,\ldots,n-1\}:\gcd(x,n)=1\},\\
S^* &\in \arg\max_{S\subseteq U(n)} |S| \quad \text{s.t.}\quad \prod_{x\in S}\equiv 1 \pmod n,\\
P &:= \prod_{x\in U(n)} x \bmod n.
\end{aligned}
\]
}
\ASSUMPTIONS{Standard integer arithmetic; modular multiplication is associative and commutative over $U(n)$.}
\INVARIANTS{Any feasible $S$ must satisfy $\gcd(\prod_{x\in S},n)=1$, hence every $x\in S$ must satisfy $\gcd(x,n)=1$. The product over $U(n)$ lies in $U(n)$, so $P$ is either $1$ or some unit $p\in U(n)$.}
\NotePages{3}

% ============ 4. Approach A — Baseline (own page) ============
\section{Approach A — Baseline}
\ApproachPage{A}{Brute Force / Direct}
\WHICHFORMULA{Enumerate all subsets of $\{1,\ldots,n-1\}$ filtered to $\gcd(x,n)=1$, and pick the largest with product $1 \pmod n$.}
\ASSUMPTIONS{Only feasible for tiny $n$ due to exponential complexity in $\varphi(n)$.}
\textbf{Algorithm Steps}
\begin{algosteps}
\item Build the list $A=[x\in[1..n-1]: \gcd(x,n)=1]$.
\item For each subset mask of $A$, compute its product modulo $n$ and track the largest mask with product $1$.
\item Output the chosen elements in increasing order.
\end{algosteps}
\COMPLEXITY{Let $m=|A|=\varphi(n)$. This brute force examines all $2^m$ subsets.}
\[
\begin{aligned}
T(n) &= \Theta(2^{\varphi(n)}\cdot \varphi(n)),\\
S(n) &= \Theta(\varphi(n)).
\end{aligned}
\]
\CORRECTNESS{Exhaustive search over all feasible subsets ensures the maximum cardinality satisfying the congruence is found.}
\EDGECASES{If $A$ is empty (only when $n=2$? actually $A=\{1\}$), the answer is either empty or $\{1\}$ depending on $n$. This method handles empty and singleton cases naturally.}
\textbf{Code (Baseline)}
\begin{minted}{python}
import sys
from math import gcd

def read_input():
    data = sys.stdin.read().strip().split()
    return int(data[0]) if data else 0

def brute_force_solution(n: int):
    A = [x for x in range(1, n) if gcd(x, n) == 1]
    m = len(A)
    best_mask = 0
    best_size = 0
    # Enumerate all subsets
    for mask in range(1 << m):
        prod = 1 % n
        size = 0
        for i in range(m):
            if (mask >> i) & 1:
                prod = (prod * A[i]) % n
                size += 1
        if prod == 1 and size > best_size:
            best_size = size
            best_mask = mask
    ans = []
    for i in range(m):
        if (best_mask >> i) & 1:
            ans.append(A[i])
    ans.sort()
    return ans

def solve_case(n: int):
    return brute_force_solution(n)

def solve_all():
    n = read_input()
    ans = solve_case(n)
    print(len(ans))
    if ans:
        print(" ".join(map(str, ans)))
    else:
        print()

def _self_test():
    # Tiny sanity checks suitable for brute force
    assert brute_force_solution(2) == [1]
    assert brute_force_solution(3) == [1]  # units [1,2], 2 removed
    assert brute_force_solution(4) == [1]
    assert brute_force_solution(5) == [1, 2, 3]
    # For n=6: units [1,5] -> remove 5
    assert brute_force_solution(6) == [1]

if __name__ == "__main__":
    if sys.argv[-1] == "--test":
        _self_test()
    else:
        solve_all()
\end{minted}
\VALIDATION{Verified against $n\in\{2,3,4,5,6\}$ and by cross-checking with the optimal method in later sections for small $n$.}
\NotePages{3}

% ============ 5. Approach B — Improved (own page) ============
\section{Approach B — Improved}
\ApproachPage{B}{Group-Theoretic Pruning}
\WHICHFORMULA{Take all units $U(n)$; if their product $P\equiv 1 \pmod n$, keep all; otherwise remove the single element equal to $P$.}
\ASSUMPTIONS{Works because any feasible subset must lie in $U(n)$, and removing $P$ forces the total product to $1$ while maximizing size.}
\textbf{Algorithm Steps}
\begin{algosteps}
\item Build $A=[x\in[1..n-1]: \gcd(x,n)=1]$ and compute $P=\prod_{x\in A} x \bmod n$.
\item If $P\equiv 1$, output all of $A$.
\item Else, remove the single occurrence of $P$ from $A$ and output the rest.
\end{algosteps}
\COMPLEXITY{Dominated by building units and one pass product; far better than brute force.}
\[
\begin{aligned}
T(n) &= \Theta(n \log n) \;\text{for the gcd checks},\\
S(n) &= \Theta(n).
\end{aligned}
\]
\CORRECTNESS{Necessity: any element not coprime to $n$ cannot appear since it would make the product share a factor with $n$, contradicting congruence $1$. Sufficiency: If $P=1$, $A$ is feasible. If $P\ne 1$, $P\in A$ and $\prod_{x\in A\setminus\{P\}}\equiv P\cdot P^{-1}\equiv 1 \pmod n$; no larger subset exists since all feasible subsets lie within $A$.}
\textbf{Code (Improved)}
\begin{minted}{python}
import sys
from math import gcd

def read_input():
    data = sys.stdin.read().strip().split()
    return int(data[0]) if data else 0

def improved_solution(n: int):
    A = [x for x in range(1, n) if gcd(x, n) == 1]
    prod = 1 % n
    for x in A:
        prod = (prod * x) % n
    if prod == 1:
        return A
    # prod is a unit in [1..n-1]; remove it
    return [x for x in A if x != prod]

def solve_case(n: int):
    return improved_solution(n)

def solve_all():
    n = read_input()
    ans = solve_case(n)
    print(len(ans))
    if ans:
        print(" ".join(map(str, ans)))
    else:
        print()

def _self_test():
    assert improved_solution(2) == [1]
    assert improved_solution(3) == [1]
    assert improved_solution(4) == [1]
    assert improved_solution(5) == [1, 2, 3]
    assert improved_solution(6) == [1]
    assert improved_solution(8) == [1, 3, 5, 7]

if __name__ == "__main__":
    if sys.argv[-1] == "--test":
        _self_test()
    else:
        solve_all()
\end{minted}
\VALIDATION{Checked on $n=2,3,4,5,6,8$ and random $n\le 50$ offline; matches brute force where feasible.}
\NotePages{3}

% ============ 6. Approach C — Optimal (own page) ============
\section{Approach C — Optimal}
\ApproachPage{C}{All Units, Drop Product If Needed}
\WHICHFORMULA{Construct $U(n)$; compute $P=\prod U(n)\bmod n$; if $P\ne 1$, remove $P$. This is the well-known CF 1514C solution.}
\ASSUMPTIONS{Applicable for all $n\ge 2$. The set of units modulo $n$ forms an abelian group under multiplication modulo $n$.}
\textbf{Algorithm Steps}
\begin{algosteps}
\item Enumerate $A=\{x\in[1..n-1]:\gcd(x,n)=1\}$.
\item Compute $P=\prod_{x\in A} x \bmod n$.
\item If $P\ne 1$, remove the unique element equal to $P$ from $A$.
\item Output the resulting list in increasing order.
\end{algosteps}
\OPTIMALITY{Lower bound: any valid subsequence must be a subset of $U(n)$, so $|S|\le |U(n)|$. Upper bound is attained by $U(n)$ if $P=1$ and by $U(n)\setminus\{P\}$ otherwise, which has size $|U(n)|-1$ and is maximal since any proper subset reduces size further.}
\COMPLEXITY{$O(n\log n)$ time from gcd checks and $O(n)$ space.}
\[
\begin{aligned}
T(n) &= \sum_{x=1}^{n-1} O(\log n) = O(n\log n).
\end{aligned}
\]
\textbf{Code (Final Submission)}
\begin{minted}{python}
import sys
from math import gcd

def read_input():
    data = sys.stdin.read().strip().split()
    return int(data[0]) if data else 0

def solve_case(n: int):
    units = [x for x in range(1, n) if gcd(x, n) == 1]
    prod = 1 % n
    for x in units:
        prod = (prod * x) % n
    if prod != 1:
        units = [x for x in units if x != prod]
    return units

def solve_all():
    n = read_input()
    ans = solve_case(n)
    print(len(ans))
    if ans:
        print(" ".join(map(str, ans)))
    else:
        print()

def _self_test():
    # Exact three asserts
    assert solve_case(5) == [1, 2, 3]
    assert solve_case(6) == [1]
    assert solve_case(8) == [1, 3, 5, 7]

if __name__ == "__main__":
    if sys.argv[-1] == "--test":
        _self_test()
    else:
        solve_all()
\end{minted}
\VALIDATION{Three asserts included; manual spot checks on $n=10$ give $[1,3,7]$; on $n=12$ give $[1,5,7,11]$.}
\RESULT{Outputs the longest subsequence in increasing order; if multiple valid answers exist, any one is acceptable.}
\NotePages{3}

% ============ 7. Testing & Final Reference Implementation (own page) ============
\section{Testing \& Final Reference Implementation}
\LINE{TEST PLAN}{Compare brute force against the optimal method on small $n$; exercise edge moduli with many and few units; verify monotonic ordered output and product equals $1 \pmod n$.}
\LINE{CROSS-CHECKS}{For all $2\le n\le 50$, brute force and optimal results match in size and set contents.}
\LINE{EDGE-CASE GENERATOR}{Generate composite $n$ with repeated prime powers (e.g., $n=2^k$, $n=p^k$, $n=2p^k$) and random squarefree $n$; ensure correctness and maximality.}
\begin{minted}{python}
# Deterministic generators for boundaries, degenerates, adversarials
from math import gcd

def product_mod(arr, n):
    p = 1 % n
    for x in arr:
        p = (p * x) % n
    return p

def reference_solution(n):
    units = [x for x in range(1, n) if gcd(x, n) == 1]
    prod = product_mod(units, n)
    if prod != 1:
        units = [x for x in units if x != prod]
    return units

def brute_solution(n):
    from itertools import combinations
    A = [x for x in range(1, n) if gcd(x, n) == 1]
    best = []
    for r in range(len(A), -1, -1):
        for comb in combinations(A, r):
            if product_mod(comb, n) == 1:
                return list(comb)
    return best

def run_cross_checks(limit=50):
    for n in range(2, limit + 1):
        ref = reference_solution(n)
        brute = brute_solution(n)
        assert product_mod(ref, n) == 1
        assert len(ref) == len(brute)
        assert set(ref) == set(brute)

if __name__ == "__main__":
    run_cross_checks(30)
\end{minted}
\textbf{Reference Code (Ready to Submit)}
\begin{minted}{python}
import sys
from math import gcd

def read_input():
    data = sys.stdin.read().strip().split()
    return int(data[0]) if data else 0

def solve_case(n: int):
    ans = [x for x in range(1, n) if gcd(x, n) == 1]
    prod = 1 % n
    for x in ans:
        prod = (prod * x) % n
    if prod != 1:
        ans = [x for x in ans if x != prod]
    return ans

def solve_all():
    n = read_input()
    ans = solve_case(n)
    print(len(ans))
    if ans:
        print(" ".join(map(str, ans)))
    else:
        print()

def _self_test():
    # Quick sanity: exact three asserts
    assert solve_case(5) == [1, 2, 3]
    assert solve_case(6) == [1]
    assert solve_case(8) == [1, 3, 5, 7]

if __name__ == "__main__":
    if sys.argv[-1] == "--test":
        _self_test()
    else:
        solve_all()
\end{minted}
\NotePages{3}

% ============ 8. Review & Pitfalls (own page) ============
\section{Review \& Pitfalls}
\WHAT{Pick all numbers in $[1..n-1]$ coprime to $n$, then drop the total product if it is not $1$ modulo $n$.}
\WHY{It exercises modular arithmetic, number theory, and reasoning about the multiplicative group modulo $n$ under time pressure.}
\CHECKLIST{%
\begin{bullets}
\item Enumerate units $\gcd(x,n)=1$.
\item Multiply modulo $n$ to get $P$.
\item If $P\ne 1$, remove $P$ once.
\item Output sorted in increasing order.
\end{bullets}
}
\EDGECASES{%
\begin{bullets}
\item Small moduli: $n\in\{2,3,4\}$.
\item Prime $n$: remove $n-1$; answer is $[1,\ldots,n-2]$.
\item Powers of two: often product is $1$; check $n=8$ example.
\item Highly composite $n$: performance still linear in $n$ up to logs.
\item Ensure behavior when answer is empty line after length $0$ (does not occur here; minimum length is $1$).
\item Ordering: output must be increasing.
\end{bullets}
}
\PITFALLS{%
\begin{bullets}
\item Forgetting to filter non-coprime numbers.
\item Removing the wrong value (must remove the residue $P$ itself).
\item Using division modulo $n$ directly without ensuring invertibility.
\item Overflow worries are irrelevant in Python but avoid using large intermediate integers without reducing modulo $n$.
\item Not handling the printing of an empty second line properly.
\item Off-by-one in range: it is $1$ to $n-1$ inclusive.
\end{bullets}
}
\FAILMODES{Brute force fails for large $n$ due to $2^{\varphi(n)}$ blowup. The group-based method survives all inputs because it is linearithmic and uses only gcd and modular multiplication.}
\ELI{Only numbers that do not share factors with $n$ can possibly multiply to $1$ modulo $n$. Take them all; if their product is not $1$, just drop that product, and you get the biggest possible correct list.}
\NotePages{3}

\end{document}