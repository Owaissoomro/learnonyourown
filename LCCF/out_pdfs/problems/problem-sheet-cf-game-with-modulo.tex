% !TeX program = xelatex
\documentclass[12pt]{article}
\usepackage[margin=1in]{geometry}
\usepackage{microtype}
\usepackage{setspace}\setstretch{1.05}
\usepackage{fontspec}
\setmainfont{Latin Modern Roman}
\setmonofont{Latin Modern Mono}[BoldFont={Latin Modern Mono},ItalicFont={Latin Modern Mono}]
\usepackage{amsmath,mathtools,amsthm}
\newtheorem{lemma}{Lemma}
\newtheorem{theorem}{Theorem}
\usepackage{unicode-math}\setmathfont{Latin Modern Math}\allowdisplaybreaks[4]
% --- overflow and alignment slack ---
\setlength{\jot}{7pt}
\sloppy\emergencystretch=8em\hfuzz=1pt\vfuzz=2pt\raggedbottom
% --- breakable math helpers ---
\newenvironment{BreakableEquation}{\begin{equation}\begin{aligned}}{\end{aligned}\end{equation}}
\newenvironment{BreakableEquation*}{\begin{equation*}\begin{aligned}}{\end{aligned}\end{equation*}}
\newenvironment{tightalign}{\begingroup\small\allowdisplaybreaks\begin{align}}{\end{align}\endgroup}

\usepackage{xcolor}
\usepackage{xurl} % better URL wrapping
\usepackage[colorlinks=true,linkcolor=black,urlcolor=blue]{hyperref}
\usepackage{fancyhdr}
\pagestyle{fancy}\fancyhf{}\lhead{\nouppercase{\leftmark}}\rhead{\thepage}\setlength{\headheight}{26pt}
\usepackage{enumitem,booktabs,tabularx}
\newcolumntype{L}{>{\raggedright\arraybackslash}X}
\renewcommand{\arraystretch}{1.1}
\setlength{\parindent}{0pt}\setlength{\parskip}{8pt plus 2pt minus 1pt}

% Listings + upquote (no shell-escape needed)
\usepackage{listings}
\usepackage{upquote}
\lstdefinestyle{crisp}{
  basicstyle=\ttfamily\footnotesize,
  frame=single,
  breaklines=true,
  breakatwhitespace=false, % allow breaks inside long tokens
  tabsize=4,
  showstringspaces=false,
  numbers=left,
  numberstyle=\tiny,
  numbersep=8pt,
  keepspaces=true,
  columns=fullflexible,
  backgroundcolor=\color{black!02},
  aboveskip=8pt,
  belowskip=8pt
}
% Guarantee that 'python' exists as a language for listings
\lstdefinelanguage{python}{
  morekeywords={def,return,class,if,elif,else,for,while,try,except,raise,assert,pass,break,continue,lambda,nonlocal,global,yield,import,from,as,with,True,False,None},
  sensitive=true,
  morecomment=[l]\#,
  morestring=[b]",
  morestring=[b]'
}
% minted shim (robust; no shell-escape; uses listings' own environment)
\lstnewenvironment{minted}[2][]{\lstset{style=crisp,language=#2,#1}}{}

\usepackage[most]{tcolorbox}
\tcbset{colback=white,colframe=black!15,boxrule=0.4pt,arc=2pt,left=6pt,right=6pt,top=6pt,bottom=6pt,before skip=10pt,after skip=10pt,breakable}
\usepackage{titlesec}
\titleformat{\section}{\large\bfseries}{\thesection}{0.6em}{}
\titlespacing*{\section}{0pt}{*2.0}{*1.0}

% ===== CONSISTENCY TOOLKIT (macros) =====
\newlist{ledger}{enumerate}{1}
\setlist[ledger]{label=\textbullet,leftmargin=2em,itemsep=2pt,topsep=6pt}
\newcommand{\LEDGER}[1]{\textbf{ARITHMETIC LEDGER:}\par\begin{ledger}#1\end{ledger}}

\newlist{algosteps}{enumerate}{1}
\setlist[algosteps]{label=\arabic*.,leftmargin=2em,itemsep=2pt,topsep=6pt}

\newlist{bullets}{itemize}{1}
\setlist[bullets]{label=--,leftmargin=1.6em,itemsep=2pt,topsep=4pt}

\newcommand{\varmapStart}{\textbf{VARIABLE MAPPING:}\par\begin{bullets}}
\newcommand{\var}[2]{\item $#1$ — #2}
\newcommand{\varmapEnd}{\end{bullets}}

\newcommand{\LINE}[2]{\par\noindent\textbf{#1:}~#2\par}
\newcommand{\WHAT}[1]{\LINE{WHAT}{#1}}
\newcommand{\WHY}[1]{\LINE{WHY}{#1}}
\newcommand{\HOW}[1]{\LINE{HOW}{#1}}
\newcommand{\ELI}[1]{\LINE{ELI5}{#1}}
\newcommand{\STATEMENT}[1]{\LINE{STATEMENT}{#1}}
\newcommand{\BREAKDOWN}[1]{\LINE{PROBLEM BREAKDOWN}{#1}}
\newcommand{\MODEL}[1]{\LINE{CANONICAL MATHEMATICAL MODEL}{#1}}
\newcommand{\ASSUMPTIONS}[1]{\LINE{ASSUMPTIONS}{#1}}
\newcommand{\INVARIANTS}[1]{\LINE{INVARIANTS}{#1}}
\newcommand{\WHICHFORMULA}[1]{\LINE{WHICH FORMULA \& WHY}{#1}}
\newcommand{\GOVERN}[1]{\LINE{GOVERNING EQUATION(S)}{#1}}
\newcommand{\INPUTS}[1]{\LINE{INPUTS}{#1}}
\newcommand{\OUTPUTS}[1]{\LINE{OUTPUTS}{#1}}
\newcommand{\SAMPLES}[1]{\LINE{SAMPLES}{#1}}
\newcommand{\RESULT}[1]{\LINE{RESULT}{#1}}
\newcommand{\COMPLEXITY}[1]{\LINE{TIME/SPACE COMPLEXITY}{#1}}
\newcommand{\MEMORY}[1]{\LINE{MEMORY FOOTPRINT}{#1}}
\newcommand{\CORRECTNESS}[1]{\LINE{CORRECTNESS SKETCH}{#1}}
\newcommand{\OPTIMALITY}[1]{\LINE{OPTIMALITY}{#1}}
\newcommand{\FAILMODES}[1]{\LINE{FAILURE MODES}{#1}}
\newcommand{\VALIDATION}[1]{\LINE{VALIDATION}{#1}}
\newcommand{\UNITCHECK}[1]{\LINE{UNIT CHECK}{#1}}
\newcommand{\EDGECASES}[1]{\LINE{EDGE CASES}{#1}}
\newcommand{\CHECKLIST}[1]{\LINE{CHECKLIST}{#1}}
\newcommand{\DERIV}[1]{\LINE{DERIVATION}{#1}}
\newcommand{\LEMMAHEAD}[1]{\LINE{SUPPORTING LEMMA}{#1}}
\newcommand{\PITFALLS}[1]{\LINE{PITFALLS}{#1}}

\usepackage{etoolbox}\pretocmd{\section}{\clearpage}{}{}

\newcommand{\FormulaPage}[2]{%
  \clearpage
  \section*{Formula #1 — #2}%
  \addcontentsline{toc}{section}{Formula #1 — #2}%
}
\newcommand{\ApproachPage}[2]{%
  \clearpage
  \subsection*{Approach #1 — #2}%
  \addcontentsline{toc}{subsection}{Approach #1 — #2}%
}

\begin{document}
\title{Interview Problem Sheet — Game with modulo}
\date{\today}
\author{}
\maketitle
\tableofcontents
\clearpage

% === Notes macro (BODY-ONLY; do not alter the preamble) ===
% Prints exactly N blank pages with empty headers/footers, then leaves you on the last blank page.
\newcommand{\NotePages}[1]{%
  \clearpage
  \begingroup
  \newcount\NPi \newcount\NPn
  \NPi=0 \NPn=#1
  \loop\ifnum\NPi<\NPn
    \advance\NPi by 1
    \mbox{}\thispagestyle{empty}%
    \ifnum\NPi<\NPn\clearpage\fi
  \repeat
  \endgroup
}

% ============ 1. Problem & Metadata (own page) ============
\section{Problem \& Metadata}
\LINE{PLATFORM}{CF}
\LINE{URL}{\url{https://codeforces.com/problemset/problem/1103/B}}
\LINE{DIFFICULTY / RATING}{2000}
\STATEMENT{This is an interactive problem.

Vasya and Petya are going to play the following game: Petya has some positive integer number $a$. After that Vasya should guess this number using the following questions. He can say a pair of non-negative integer numbers $(x, y)$. Petya will answer him:
\begin{bullets}
\item ``x'', if $(x \bmod a) \ge (y \bmod a)$.
\item ``y'', if $(x \bmod a) < (y \bmod a)$.
\end{bullets}
We define $(x \bmod a)$ as a remainder of division $x$ by $a$.

Vasya should guess the number $a$ using no more than $60$ questions.

It is guaranteed that Petya has a number that satisfies the inequality $1 \le a \le 10^9$.

Help Vasya playing this game and write a program that will guess the number $a$.

Note: In the first test, you should play $3$ games with Petya's numbers $1$, $2$ and $3$.

In the first game, Petya will answer ``x'' to any question, because $(x \bmod 1) = 0$ for any integer $x$.

In the second game, if you will ask pair $(0, 0)$, the answer will be ``x'', because $(0 \bmod 2) \ge (0 \bmod 2)$. But if you will ask pair $(2, 5)$, the answer will be ``y'', because $(2 \bmod 2) < (5 \bmod 2)$, because $(2 \bmod 2) = 0$ and $(5 \bmod 2) = 1$.}
\BREAKDOWN{We need an adaptive querying strategy to determine the hidden modulus $a$ within $60$ comparisons of the form $(x \bmod a) \ge (y \bmod a)$. First, detect whether $a=1$. Otherwise, bracket $a$ between consecutive powers of two using $O(\log 10^9)$ queries, then determine the exact value using a monotone predicate enabling binary search.}
\ELI{Find the power-of-two bucket that contains $a$, then ask cleverly crafted comparisons that act like ``is $a \le$ threshold?'' to binary search $a$.}
\NotePages{3}

% ============ 2. IO Contract (own page) ============
\section{IO Contract}
\INPUTS{Since the original task is interactive, we present a deterministic offline harness for practice and testing:
\begin{bullets}
\item Either a single integer $a$ ($1 \le a \le 10^9$), or
\item An integer $T$ followed by $T$ integers $a_i$ ($1 \le a_i \le 10^9$).
\end{bullets}
Each $a$ represents Petya's hidden number.}
\OUTPUTS{For each input $a$, output a single line with the guessed value of $a$.}
\SAMPLES{Examples (offline harness):
\begin{bullets}
\item Input: \texttt{1} \quad Output: \texttt{1}
\item Input: \texttt{3 1 2 3} \quad Output:
\begin{verbatim}
1
2
3
\end{verbatim}
\end{bullets}}
\NotePages{3}

% ============ 3. Canonical Mathematical Model (own page) ============
\section{Canonical Mathematical Model}
\MODEL{We can query a comparator $\mathcal{O}_a(x,y)$ that returns $\text{``x''}$ iff $x \bmod a \ge y \bmod a$, otherwise $\text{``y''}$. Our goal is to determine $a \in \{1,\ldots,10^9\}$ with at most $60$ queries.}
\varmapStart
\var{a}{unknown integer modulus to recover}
\var{x,y}{non-negative integers we choose for each query}
\var{k}{integer with $2^k < a \le 2^{k+1}$}
\var{L}{lower power-of-two bound $L=2^k$}
\var{t}{offset used in the second binary search}
\var{s}{gap $s=a-L$}
\varmapEnd
\GOVERN{
\[
\mathcal{O}_a(x,y)=
\begin{cases}
\text{``x''}, & (x \bmod a) \ge (y \bmod a),\\
\text{``y''}, & \text{otherwise.}
\end{cases}
\]
}
\ASSUMPTIONS{All queries must respect $x,y \ge 0$. The oracle behaves deterministically according to true modulo arithmetic. We are allowed up to $60$ queries.}
\INVARIANTS{
\begin{bullets}
\item For $k$ increasing from $0$, $\mathcal{O}_a(2^k,2^{k+1})$ is ``y'' while $2^{k+1} \le a$ and becomes ``x'' once $2^k < a \le 2^{k+1}$. Hence we can bracket $a$.
\item With $L=2^k$ and $a \in (L,2L]$, the predicate $P(t):=\big(\mathcal{O}_a(L,L+t)=\text{``x''}\big)$ is monotone in $t$ and satisfies
\[
P(t)=
\begin{cases}
\text{false}, & a>L+t,\\
\text{true}, & a\le L+t,
\end{cases}
\]
so $t^\star=a-L$ can be found by binary search.
\end{bullets}
}
\NotePages{3}

% ============ 4. Approach A — Baseline (own page) ============
\section{Approach A — Baseline}
\ApproachPage{A}{Brute Force / Direct}
\WHICHFORMULA{Use power-of-two bracketing to find $L$, then linearly scan $t=1,2,\ldots,L$ to locate the first $t$ such that $\mathcal{O}_a(L,L+t)$ flips to ``x''.}
\ASSUMPTIONS{Only for demonstration on small inputs, as linear scan may exceed the query budget for large $L$.}
\textbf{Algorithm Steps}
\begin{algosteps}
\item If $\mathcal{O}_a(0,1)$ is ``x'', return $a=1$.
\item Find the smallest $k$ with $\mathcal{O}_a(2^k,2^{k+1})$ = ``x''. Set $L=2^k$.
\item For $t$ from $1$ to $L$, query $\mathcal{O}_a(L,L+t)$. The first ``x'' occurs at $t=a-L$. Return $a=L+t$.
\end{algosteps}
\COMPLEXITY{Worst-case queries: $1$ (check $a=1$) $+~30$ (find $k$) $+~L$ (scan), which is infeasible for large $L$. Time and space are otherwise $O(L)$ and $O(1)$.}
\[
\begin{aligned}
T_{\text{queries}}(a) &\le 1 + 30 + L, \quad L \le 2^{29}.\\
\end{aligned}
\]
\CORRECTNESS{From the invariants: the bracketing step yields $L < a \le 2L$. The scan over $t$ uses a monotone predicate in $t$ so the first ``x'' indeed corresponds to $t=a-L$.}
\EDGECASES{Handle $a=1$ upfront. When $a=2^m$, the scan must reach $t=L$; still correct.}
\textbf{Code (Baseline)}
\begin{minted}{python}
# CF TEMPLATE (baseline; linear scan after bracketing)
from typing import List, Tuple

def ask_factory(a: int):
    count = {'q': 0}
    def ask(x: int, y: int) -> str:
        count['q'] += 1
        return 'x' if (x % a) >= (y % a) else 'y'
    return ask, count

def guess_a_baseline(a: int) -> Tuple[int, int]:
    ask, cnt = ask_factory(a)
    # Special case a == 1
    if ask(0, 1) == 'x':
        return 1, cnt['q']
    # Find smallest k with ask(2^k, 2^(k+1)) == 'x'
    k = 0
    while True:
        x, y = 1 << k, 1 << (k + 1)
        if ask(x, y) == 'x':
            break
        k += 1
    L = 1 << k
    # Linear scan for t in [1..L]
    for t in range(1, L + 1):
        if ask(L, L + t) == 'x':
            return L + t, cnt['q']
    # Should never reach here
    raise AssertionError("Failed to determine a")

def read_input() -> List[int]:
    import sys
    data = sys.stdin.read().strip().split()
    if not data:
        return []
    nums = list(map(int, data))
    if len(nums) == 1:
        return nums
    if nums[0] == len(nums) - 1:
        return nums[1:]
    return nums

def solve_case(a: int) -> str:
    val, q = guess_a_baseline(a)
    # Baseline may exceed 60 for large a; we do not assert here.
    assert val == a
    return str(val)

def solve_all(cases: List[int]) -> str:
    return "\n".join(solve_case(a) for a in cases)

def main():
    cases = read_input()
    if not cases:
        # Tiny self-checks only (to avoid linear scan blow-up)
        for a in [1, 2, 3, 5, 7, 8, 13, 31]:
            val, q = guess_a_baseline(a)
            assert val == a
        print("", end="")
        return
    out = solve_all(cases)
    print(out)

if __name__ == "__main__":
    # Basic asserts
    v, q = guess_a_baseline(1)
    assert v == 1
    v, q = guess_a_baseline(3)
    assert v == 3
    main()
\end{minted}
\VALIDATION{The baseline code includes asserts on small values where linear scan is safe. For large $a$, do not rely on this approach due to query budget.}
\NotePages{3}

% ============ 5. Approach B — Improved (own page) ============
\section{Approach B — Improved}
\ApproachPage{B}{Optimized Data Structure / Pruning / DP}
\WHICHFORMULA{Keep the same bracketing, but replace the linear scan by a binary search over $t \in [1,L]$ using the monotone predicate $P(t)=\big(\mathcal{O}_a(L,L+t)=\text{``x''}\big)$, which holds iff $a \le L+t$.}
\ASSUMPTIONS{We maintain $L=2^k$ with $L<a\le 2L$, ensuring $t^\star=a-L\in[1,L]$ and the predicate is monotone in $t$.}
\textbf{Algorithm Steps}
\begin{algosteps}
\item If $\mathcal{O}_a(0,1)$ is ``x'', return $a=1$.
\item Find the smallest $k$ with $\mathcal{O}_a(2^k,2^{k+1})$ = ``x''. Set $L=2^k$.
\item Binary search the smallest $t \in [1,L]$ with $\mathcal{O}_a(L,L+t)$ = ``x''. Return $a=L+t$.
\end{algosteps}
\COMPLEXITY{Query complexity is bounded by $1 + 30 + \lceil \log_2 L \rceil \le 60$, since $L \le 2^{29}$. Time is $O(\log L)$; space is $O(1)$.}
\[
\begin{aligned}
T_{\text{queries}}(a) &\le 1 + 30 + \lceil \log_2 L \rceil \le 60.
\end{aligned}
\]
\CORRECTNESS{From the invariant, $P(t)$ is false when $a>L+t$ and true when $a\le L+t$. Thus the minimal $t$ with $P(t)$ true is exactly $t^\star=a-L$, giving the correct $a$.}
\textbf{Code (Improved)}
\begin{minted}{python}
# CF TEMPLATE (improved; binary search after bracketing)
from typing import List, Tuple

def ask_factory(a: int):
    count = {'q': 0}
    def ask(x: int, y: int) -> str:
        count['q'] += 1
        return 'x' if (x % a) >= (y % a) else 'y'
    return ask, count

def guess_a_improved(a: int) -> Tuple[int, int]:
    ask, cnt = ask_factory(a)
    # Special case a == 1
    if ask(0, 1) == 'x':
        return 1, cnt['q']
    # Find smallest k with ask(2^k, 2^(k+1)) == 'x'
    k = 0
    while True:
        x, y = 1 << k, 1 << (k + 1)
        if ask(x, y) == 'x':
            break
        k += 1
    L = 1 << k
    # Binary search t in [1, L]
    lo, hi = 1, L
    while lo < hi:
        mid = (lo + hi) // 2
        if ask(L, L + mid) == 'x':
            hi = mid
        else:
            lo = mid + 1
    t = lo
    return L + t, cnt['q']

def read_input() -> List[int]:
    import sys
    data = sys.stdin.read().strip().split()
    if not data:
        return []
    nums = list(map(int, data))
    if len(nums) == 1:
        return nums
    if nums[0] == len(nums) - 1:
        return nums[1:]
    return nums

def solve_case(a: int) -> str:
    val, q = guess_a_improved(a)
    assert val == a
    # Must meet the 60-queries constraint
    assert q <= 60, f"used {q} queries for a={a}"
    return str(val)

def solve_all(cases: List[int]) -> str:
    return "\n".join(solve_case(a) for a in cases)

def main():
    cases = read_input()
    if not cases:
        # Self-checks
        for a in [1, 2, 3, 4, 5, 6, 7, 8, 31, 32, 33, 10**9, 999_999_937]:
            val, q = guess_a_improved(a)
            assert val == a
            assert q <= 60
        print("", end="")
        return
    out = solve_all(cases)
    print(out)

if __name__ == "__main__":
    # Deterministic asserts
    v, q = guess_a_improved(1)
    assert v == 1 and q <= 1
    v, q = guess_a_improved(2)
    assert v == 2 and q <= 60
    main()
\end{minted}
\VALIDATION{We assert both correctness and the query budget $\le 60$ on diverse cases including boundaries and a large prime near $10^9$.}
\NotePages{3}

% ============ 6. Approach C — Optimal (own page) ============
\section{Approach C — Optimal}
\ApproachPage{C}{Provably Optimal Method}
\WHICHFORMULA{Two-phase strategy:
\begin{bullets}
\item Find $k$ with $\mathcal{O}_a(2^k,2^{k+1})=$``x'' to get $L=2^k < a \le 2^{k+1}$.
\item Binary search smallest $t \in [1,L]$ with $\mathcal{O}_a(L,L+t)=$``x'', which is equivalent to testing predicate $a \le L+t$.
\end{bullets}}
\ASSUMPTIONS{We rely on $1 \le a \le 10^9$, hence $k \le 29$, and the comparator behaves exactly as defined.}
\textbf{Algorithm Steps}
\begin{algosteps}
\item Query $(0,1)$. If answer is ``x'', return $a=1$.
\item For $k=0,1,\ldots$, query $(2^k, 2^{k+1})$ until the answer is ``x''. Set $L=2^k$.
\item Binary search the smallest $t \in [1,L]$ with query $(L, L+t)$ returning ``x''. Return $a=L+t$.
\end{algosteps}
\OPTIMALITY{This achieves at most $1 + 30 + 29 = 60$ queries in the worst case, matching the allowed limit and within logarithmic factors of the information-theoretic bound for the search over $[1,10^9]$.}
\COMPLEXITY{At most $60$ oracle queries; arithmetic operations are $O(1)$ per query; memory is $O(1)$.}
\[
\begin{aligned}
T_{\max} &\le 1 + 30 + 29 = 60.
\end{aligned}
\]
\textbf{Code (Final Submission)}
\begin{minted}{python}
# CF: read_input(), solve_case()/solve_all(), main()+guard + asserts
from typing import List, Tuple

def ask_factory(a: int):
    """
    Offline oracle for (x mod a) >= (y mod a) ? 'x' : 'y', with query counter.
    """
    count = {'q': 0}
    def ask(x: int, y: int) -> str:
        count['q'] += 1
        return 'x' if (x % a) >= (y % a) else 'y'
    return ask, count

def guess_a(a: int) -> Tuple[int, int]:
    """
    Optimal strategy:
      - Detect a==1 by (0,1).
      - Find k with ask(2^k, 2^(k+1)) == 'x' => L=2^k in (a/2, a].
      - Binary search minimal t in [1, L] with ask(L, L+t) == 'x' (equiv a <= L+t).
    Returns (guessed_a, queries_used).
    """
    ask, cnt = ask_factory(a)
    # Step 0: a == 1?
    if ask(0, 1) == 'x':
        return 1, cnt['q']
    # Step 1: bracket a between (2^k, 2^(k+1)]
    k = 0
    while True:
        x, y = 1 << k, 1 << (k + 1)
        if ask(x, y) == 'x':
            break
        k += 1
    L = 1 << k
    # Step 2: binary search threshold t* = a - L in [1..L]
    lo, hi = 1, L
    while lo < hi:
        mid = (lo + hi) // 2
        if ask(L, L + mid) == 'x':
            hi = mid
        else:
            lo = mid + 1
    t = lo
    return L + t, cnt['q']

def read_input() -> List[int]:
    import sys
    data = sys.stdin.read().strip().split()
    if not data:
        return []
    nums = list(map(int, data))
    if len(nums) == 1:
        return nums
    if nums[0] == len(nums) - 1:
        return nums[1:]
    return nums

def solve_case(a: int) -> str:
    val, q = guess_a(a)
    assert val == a
    assert q <= 60, f"Exceeded query budget: {q}"
    return str(val)

def solve_all(cases: List[int]) -> str:
    return "\n".join(solve_case(a) for a in cases)

def main():
    cases = read_input()
    if not cases:
        # Deterministic self-tests
        for a in [1, 2, 3, 4, 7, 8, 15, 16, 17, 31, 32, 33, 10**9, 999_999_937]:
            val, q = guess_a(a)
            assert val == a
            assert q <= 60
        print("", end="")
        return
    out = solve_all(cases)
    print(out)

if __name__ == "__main__":
    # Exactly 3 asserts or I/O mini-tests
    v, q = guess_a(1)
    assert v == 1 and q <= 1
    v, q = guess_a(2)
    assert v == 2 and q <= 60
    v, q = guess_a(1_000_000_000)
    assert v == 1_000_000_000 and q <= 60
    main()
\end{minted}
\VALIDATION{Asserts cover small, power-of-two, successor-of-power-of-two, and large prime/maximum cases, enforcing the $\le 60$ query bound.}
\RESULT{The algorithm deterministically returns the exact hidden $a$ with at most $60$ oracle queries.}
\NotePages{3}

% ============ 7. Testing & Final Reference Implementation (own page) ============
\section{Testing \& Final Reference Implementation}
\LINE{TEST PLAN}{Unit-test the offline oracle-based solver across:
\begin{bullets}
\item Small sweep: $a \in [1,200]$.
\item Structured cases: powers of two and their neighbors.
\item Random sampling near boundaries (close to $10^9$).
\end{bullets}
Assert correctness and query count $\le 60$.}
\LINE{CROSS-CHECKS}{Compare outputs and query counts of Approach B and C on shared test sets; they should match in outputs, with identical or better query counts for C.}
\LINE{EDGE-CASE GENERATOR}{Generate adversarial $a$ where $a$ is just above $2^k$ (hardest binary search edge) and $a=2^{k+1}$ (threshold at end of range).}
\begin{minted}{python}
# Deterministic generators for boundaries, degenerates, adversarials
from typing import List, Tuple
import random

def hard_cases(limit_pow: int = 29) -> List[int]:
    cases = [1]
    for k in range(limit_pow + 1):
        L = 1 << k
        R = 1 << (k + 1)
        # Only include if within bound
        if L <= 10**9:
            if L >= 1:
                cases.append(max(1, L))
            if L + 1 <= 10**9:
                cases.append(L + 1)
        if R <= 10**9:
            cases.append(R)
    # Deduplicate and sort
    return sorted(set(cases))

def random_near_bounds(seed: int = 0, n: int = 50) -> List[int]:
    random.seed(seed)
    vals = []
    for _ in range(n):
        if random.random() < 0.5:
            k = random.randint(0, 29)
            L = 1 << k
            delta = random.randint(0, min(1000, max(0, 10**9 - L)))
            vals.append(min(10**9, L + delta))
        else:
            vals.append(random.randint(1, 10**9))
    return vals

# Reference check using the final solver
def ask_factory(a: int):
    c = {'q': 0}
    def ask(x: int, y: int) -> str:
        c['q'] += 1
        return 'x' if (x % a) >= (y % a) else 'y'
    return ask, c

def guess_a(a: int) -> Tuple[int, int]:
    ask, cnt = ask_factory(a)
    if ask(0, 1) == 'x':
        return 1, cnt['q']
    k = 0
    while True:
        if ask(1 << k, 1 << (k + 1)) == 'x':
            break
        k += 1
    L = 1 << k
    lo, hi = 1, L
    while lo < hi:
        mid = (lo + hi) // 2
        if ask(L, L + mid) == 'x':
            hi = mid
        else:
            lo = mid + 1
    return L + lo, cnt['q']

def run_tests():
    # Small sweep
    for a in range(1, 201):
        val, q = guess_a(a)
        assert val == a and q <= 60
    # Hard structured
    for a in hard_cases():
        val, q = guess_a(a)
        assert val == a and q <= 60
    # Randoms
    for a in random_near_bounds(42, 100):
        val, q = guess_a(a)
        assert val == a and q <= 60

if __name__ == "__main__":
    run_tests()
\end{minted}
\textbf{Reference Code (Ready to Submit)}
\begin{minted}{python}
# Single, final reference solution with the required API (CF/LC) + asserts
from typing import List, Tuple

def ask_factory(a: int):
    cnt = {'q': 0}
    def ask(x: int, y: int) -> str:
        cnt['q'] += 1
        return 'x' if (x % a) >= (y % a) else 'y'
    return ask, cnt

def guess_a(a: int) -> Tuple[int, int]:
    ask, cnt = ask_factory(a)
    if ask(0, 1) == 'x':
        return 1, cnt['q']
    k = 0
    while True:
        if ask(1 << k, 1 << (k + 1)) == 'x':
            break
        k += 1
    L = 1 << k
    lo, hi = 1, L
    while lo < hi:
        mid = (lo + hi) // 2
        if ask(L, L + mid) == 'x':
            hi = mid
        else:
            lo = mid + 1
    return L + lo, cnt['q']

def read_input() -> List[int]:
    import sys
    data = sys.stdin.read().strip().split()
    if not data:
        return []
    nums = list(map(int, data))
    if len(nums) == 1:
        return nums
    if nums[0] == len(nums) - 1:
        return nums[1:]
    return nums

def solve_case(a: int) -> str:
    val, q = guess_a(a)
    assert val == a
    assert q <= 60
    return str(val)

def solve_all(cases: List[int]) -> str:
    return "\n".join(solve_case(a) for a in cases)

def main():
    cases = read_input()
    if not cases:
        # Minimal sanity checks
        for a in [1, 2, 3, 4, 8, 9, 31, 32, 10**9]:
            v, q = guess_a(a)
            assert v == a and q <= 60
        print("", end="")
        return
    print(solve_all(cases))

if __name__ == "__main__":
    # Quick asserts
    v, q = guess_a(1)
    assert v == 1 and q <= 1
    v, q = guess_a(32)
    assert v == 32 and q <= 60
    main()
\end{minted}
\NotePages{3}

% ============ 8. Review & Pitfalls (own page) ============
\section{Review \& Pitfalls}
\WHAT{Bracket $a$ between consecutive powers of two, then binary search on a monotone predicate derived from modulo comparisons.}
\WHY{Interactive comparison-style problems often hide a monotone predicate suitable for binary search; recognizing and proving monotonicity is the key interview skill.}
\CHECKLIST{
\begin{bullets}
\item Check $a=1$ via $(0,1)$.
\item Find smallest $k$ with $\mathcal{O}_a(2^k,2^{k+1})$ = ``x''.
\item Set $L=2^k$.
\item Binary search minimal $t \in [1,L]$ s.t. $\mathcal{O}_a(L,L+t)$ = ``x''.
\item Return $L+t$; verify query count $\le 60$.
\end{bullets}
}
\EDGECASES{
\begin{bullets}
\item $a=1$ (all remainders are $0$).
\item $a=2$ (early crossing at $k=0$).
\item $a=2^m$ (threshold at $t=L$).
\item $a=2^m+1$ (threshold at $t=1$).
\item $a$ near $10^9$ (largest $k$, tight budget).
\item Non-divisibility of $2^k$ by $a$ (does not affect the monotone predicate $P(t)$).
\end{bullets}
}
\PITFALLS{
\begin{bullets}
\item Using $(0,2^k)$ to detect crossing is not monotone; avoid.
\item Binary searching on $(L+m,2L+m)$ leads to a non-monotone response pattern; use $(L,L+t)$ instead.
\item Off-by-one in binary search for minimal $t$; ensure lower\_bound semantics.
\item Forgetting that the answer ``x'' includes equality $(\ge)$.
\item Exceeding the query budget by scanning $t$ linearly.
\item Mishandling the $a=1$ case (must be checked first).
\end{bullets}
}
\FAILMODES{Naive scans or non-monotone predicates cause too many queries or incorrect convergence. The proposed method maintains monotonicity and achieves $\le 60$ queries always.}
\ELI{First, find the power-of-two bucket that contains $a$. Then ask questions of the form ``is $a \le L+t$?'' for different $t$ to pin down $a$ by binary search. This guarantees we find $a$ quickly.}
\NotePages{3}

\end{document}